
\begin{itemize}
\item Social Security
\item Environmental regulations
\item Minimum wage/taxi regulation
\item Military/police
\item Rent control/farm subsidies
\item Macroeconomic policies
\end{itemize}



Market Failure



Government Failure

Officials may not want to maximize social welfare
Inefficiencies caused by taxes
Price controls
But getting rid of price controls doesn't lead to a Pareto improvement -> Cost-Benefit Analysis!


Cost-Benefit Analysis







\begin{itemize}
\item Cost-benefit analysis
\item Taxes, general
\item Social Security
\item Health care (e.g., Medicare and Medicaid)
\item National defense
\item Transportation
\item Education (e.g., school vouchers)
\item Welfare programs (e.g., Earned Income Tax Credit)
\item Price controls (e.g., rent control, minimum wage)
\item Unions
\item Global warming, natural resources, and other environmental issues
\item Bad things (illegal drugs, prostitution, alcohol, cigarettes\ldots)
\item Competition policy (e.g., monopoly\index{monopoly}, antitrust, patents)
\item Monetary policy (e.g., Alan Greenspan)
\end{itemize}

\end{document}
\clearpage
(Actions that prevent or restrict or inhibit trading)




Sulphur Dioxide trading


Taxes





Illegal drugs
Prostitution
Cigarettes


Traffic congestion


Social security 


Education
School vouchers


National defense (public goods)


Public goods and public bads
Information problems
Health insurance


Competition Policy
Antitrust
Patents and Copyrights
Property Rights
Rules of the Game












OK, but there's more: what happens if the government ignores these limits and attempts to intervene in the market? Inefficiencies result. Here are a bunch of examples:

Example: Price ceiling (e.g., rent control, gasoline price controls during the 1970s). With a price ceiling, the government mandates that prices cannot go above a certain price. Consider Figure 1 (on the last pages of this document). We have a supply curve and a demand curve, and without government intervention we would expect the market price to be \$3.50 per gallon. If the government imposes a price ceiling (say of \$2 per gallon), then we get the following:

\begin{itemize}
\item First, at a price of \$2, sellers are only willing to sell 2 million gallons. In the absence of a black market (see below), then, only two million gallons will be traded. (See Figure 2.) This creates an inefficiency: there is a seller out there willing to sell an additional gallon for any price above \$2 (this is the marginal cost\index{marginal!cost} of the 2,000,001st gallon), and a buyer willing to buy an additional gallon for any price below \$5 (this is the marginal value\index{marginal!value} of the 2,000,001st gallon), but they are not allowed to trade with each other because of the price ceiling. The resulting inefficiency, called a deadweight loss, is the shaded area in Figure 3. This is the sum of the consumer surplus and producer surplus lost because of the price ceiling. 

\item But the deadweight loss shown in Figure 3 is actually the best case scenario. It assumes that the few units that are for sale are sold to the buyers with the highest values. This assumption may not hold: at a price of \$2, there are only two million gallons for sale, and buyers want to buy 5 million gallons. So the lucky buyers may end up being those with relatively low valuations of the good, e.g., those willing to pay \$3 or \$4 per gallon rather than those willing to pay \$6 or \$7 per gallon. In this worst case scenario, the gains from trade are shown in Figure 4, and in comparison with Figure 1 we can see that the deadweight loss is much larger. (Note: one should expect some attempts at resale here, as the successful buyers with low values attempt to sell to the unsuccessful buyers with high values. Such scalping is actually a kind of black market-see below.)

\item Since only two million gallons are for sale, there is an imbalance between supply and demand. Normally, the price would adjust to balance the two, but the price ceiling prevents this adjustment. As a result, some other mechanism will be used to balance out supply and demand. For example, in addition to paying with money buyers might be required to pay with time, e.g., by waiting in line. (This is what happened during the days of gasoline price controls.) The gains from trade here are limited to those shown in Figure 4: the consumer surplus is reduced because consumers pay with time as well as money. Alternatively, sellers might use other mechanisms to distinguish between buyers; in cities with rent controls, for example, apartment owners might use race or gender or age to discriminate for or against certain buyers.

\item Price ceilings have a tendency to lead to black markets in which illegal transactions occur at prices higher than the price ceiling. This is simply because individuals see that further gains from trade are possible-for example, that there are some sellers and some buyers who would happily trade at \$3.00 but are prevented from doing so by the price ceiling, or that some of the lucky buyers at \$2.00 would rather resell their gasoline at the black market price of \$4.00 than make use of it themselves. In order to maintain the price ceiling, then, the government must take steps to limit or eliminate black market transactions or other attempts to get around the price ceiling. (Another way to get around price ceilings: during the gasoline crisis in the 1970s, some gas stations used tactics such as this: they would list the price of gasoline as \$2 per gallon, in accordance with the government mandate, but not offer any gasoline for sale at that price. Instead, in order to buy gasoline you also had to buy a tune-up or an oil-change, and by artificially inflating the price of the tune-up or oil-change the station could essentially charge more for gasoline.)   
\end{itemize}

Ex: Price floor (e.g., agricultural products, minimum wage): prices cannot go below a price X. (Show inefficiency.)

Example: Price floor (e.g., agricultural products, minimum wage). With a price floor (also called a price support), the government mandates that prices cannot go below a certain price. Using the same market as above (the market in Figure 1, with an unregulated market price of \$3.50 per gallon), we get the following if the government imposes a price floor of \$5 per gallon:

\begin{itemize}
\item First, at a price of \$5, sellers are willing to sell 5 million gallons, but buyers are only willing to buy 2 million gallons. So only 2 million gallons will be traded. What happens to those other 3 million gallons? Well, in order to avoid a black market (in which transactions occur at prices below \$5 per gallon), the government has to purchase those undesired 3 million gallons. What can the government do with those 3 million gallons? Well, the best thing the government can do is give those items to the buyers with the highest values. But this creates an inefficiency: there is a seller out there who produced an additional unit of gasoline at a cost of \$4, and that government gives that gallon to someone who only values it at less than \$3.50! The resulting inefficiency or deadweight loss is the shaded area in Figure 6. This is the difference between total cost and total value for the extra goods produced because of the price floor. 

\item As in the price ceiling case, the deadweight loss shown in Figure 6 is actually the best case scenario. It assumes that the extra units that are produced find their ways into the hands of the buyers with the highest values. This assumption may not hold: the government may simply throw the extra 3 million gallons of gasoline on the ground (or, to use a more cogent example, allow surplus wheat to rot in grain silos). In this worst-case scenario, the resulting inefficiency is the same area as before (D2 in Figure 7) plus another area representing the cost of throwing that extra 3 million gallons on the ground instead of giving it to buyers (areas D1 and D3).

\item Of course, it seems quite inefficient to pay farmers to grow crops and then let the crops rot in government silos. It would be better to pay farmers to grow crops and then let them rot on the farm (i.e., without incurring harvest or transportation costs), or-heck-just to pay farmers to not grow the crops in the first place. This is what has happened at various points in time with U.S. agricultural policy, and it is an attempt to reduce the inefficiencies that result from agricultural price supports. 
\end{itemize}

Example: Taxes. If the government imposes a \$3 per gallon tax on the sellers of gasoline, the new market equilibrium will be that shown in Figure 8. (Question: what would happen if the tax were on the buyers instead?) Area A is the new consumer surplus, area B is the new producer surplus, and area T is the tax revenue that goes to the government. Note that this looks very similar to the case of a price ceiling (as in Figure 2). As in the price ceiling, then, the tax creates an inefficiency. Also note that the tax revenue itself (area T) is not a source of inefficiency: the government could use this money to pay for police or other goods, or they could simply give the money back to the individuals in the market. The inefficiency caused by taxes is caused not by the tax revenue but by the transactions that are not  consummated because of the tax: with the tax, someone is willing to sell for \$3 and someone is willing to buy for \$4, but that transaction does not take place, and the buyer, seller, and government are all worse off as a result. 

Some interesting observations about taxes:
\begin{itemize}
\item You can think about the economic effects of prohibitions (e.g., against the selling of sex or illegal drugs) as identical to those for an extremely high tax: ``Yes, you can sell crack, but only if you pay a tax of \$1 billion per vial." 
\item As in the case of price ceilings, a black market may arise to try to get around the tax. For example, in some countries you can (if you have the right connections) purchase black market cigarettes that are not subject to cigarette taxes. The incentive for this black market is that (1) as in the case of price ceilings, some transactions are attractive to both buyer and seller but cannot take place because of the tax, and (2) avoiding the tax allows the buyer and seller to divide between them the amount of the tax, which would otherwise go to the government.
\item Just as we saw how total revenue changed depending on where the market price is, the total amount of tax collection changes depending on how steep the tax rate is. The essential trade-off for the government is the same one that a monopolist faces when determining what price to charge: increasing the tax rate brings in more tax revenue for those transactions that do take place, but decreases the number of transactions that take place. In other words, at some level the tax rate is so high that raising the rate further would actually decrease revenue, and lowering the tax rate would actually increase revenue. This ``Laffer curve" result was one of the underpinnings of the tax cuts during the Reagan administration. 
\item Figure 13 shows the deadweight loss (i.e., the inefficiency) caused by a \$1 per gallon tax. Note that the shaded area is equivalent to one of the little squares on the graph. Now, Figure 14 shows the deadweight loss caused by a \$2 per gallon tax. The area here is equivalent to 4 squares. If you look at the deadweight loss of a \$3 per gallon tax (Figure 9) you will see that the area is equivalent to 9 squares. In other words, the deadweight loss is increasing in proportion to the square of the tax rate: a \$1 tax results in a loss of 12=1, a \$2 tax results in a loss of 22=4, a \$3 tax results in a loss of 32=9. The lesson here is that broad taxes (e.g., a 1\% sales tax on everything) result in less inefficiency than narrow taxes (e.g., a 20\% sales tax on food). You can study results like this in a course on public finance.   
\end{itemize}

Example: Subsidies. The effect of a subsidy is similar to that of a price floor. The inefficiency comes because of overproduction.

OK, where are we? We are here: Government intervention in the form of price ceilings or floors or taxes or subsidies leads to inefficiencies. This suggests that we should have a minimalist government that doesn't intervene in markets. In fact, it's not clear why governments should intervene at all. Hopefully we can clear this up in the next lectures, when we make the case in favor of government intervention. 

But the key lesson here is this: Given a choice between a philosophy supporting government intervention unless conditions A, B, or C are met and a philosophy opposing government intervention unless conditions A, B, or C are met, a strong case can be made for the latter: In many cases ``the market" will take care of it, and we should be careful when considering government intervention.









%\part{The Role of Government}


\chapter{Equity}


We've basically already studied this topic when we did the Second Welfare Theorem: if you have equity concerns, an initial redistribution can address those concerns without requiring further intervention in the market (e.g., price controls or price ceilings or subsidies or etc.). One pertinent example comes from the work on famines by Amartya Sen, winner of the 1998 Nobel Prize in Economics (see http://www.nobel.se/economics/laureates/1998/press.html). Sen studied various 20th century famines and concluded that many of them were caused not by a shortage of food (food production in many famine years was on par with that in other years, and in some famines there were actually exports of food) but by a shortage of money: Certain segments of the population experienced income losses, and without income they were unable to purchase food. One implication of Sen's work is that donations of money might have been just as (or perhaps more) effective as food in famine-fighting. 

Another example is the argument of Milton Friedman (winner of the 1976 Nobel Prize in Economics) that the U.S. government should get rid of food stamps and subsidized housing and other direct forms of government assistance and instead institute a negative income tax: Individuals with incomes below a certain level would pay a negative income tax, i.e., would receive a check from the government to boost their incomes. With higher incomes, these individuals would be able to go out and buy food and housing \&etc in the market; the inefficiencies stemming from government involvement in the market would disappear. Although a negative income tax was never adopted, it was considered, and some of our current welfare programs have the same flavor. For example, the Earned Income Tax Credit, a major welfare program, boosts the earnings of low-income workers by sending them ``tax credits" for taxes that they never paid! (My understanding is that this program is wildly popular pretty much across the political spectrum. Many economists like it, too, seeing it as a much more targeted way of getting money to the poor than an increase in the minimum wage.)

Other government programs (e.g., Social Security) also have an element of redistribution about them, since they take money from certain individuals (e.g., current workers) and hand the cash over to other individuals (e.g., current retirees). The federal income tax is also sometimes seen as a form of redistribution since those with higher incomes generally pay a higher percentage of their income in taxes. (A tax with this structure is called progressive. If the poor pay a higher percentage of their incomes in taxes than the rich then it is called regressive. Note that this criteria features percentages of incomes; if someone earning \$100,000 per year pays \$10,000 in tax and someone earning \$10,000 per year pays \$1,500, this is still called a regressive tax system because the high income person pays 10\% of their income in taxes and the low-income person pays 15\%. You can learn more about all this stuff in a public finance class.) 

Question: Consider the following quote about rent controls. Can you explain (preferably with an example or other story) why rent controls are inefficient? Also, if rent control is inefficient, does getting rid of rent control result in a Pareto improvement? 
 
``There is a popular and misguided view that all economic changes represent nothing more than redistributions. Gains to one only subtract from another. Rent control is one example. In this view, the only effect of rent control is redistribution-landlords receive less, and are worse off, by the same amount that their tenants' rents are reduced (and the tenants are better off). In some countries, unions have expressed similar views, and see wage increases as having no further consequences than redistributing income to workers from those who own or who manage firms. This view is mistaken, because in each of these instances, there are consequences beyond the redistribution. Rent control that keeps rents below the level that clears the rental housing market results in inefficiencies. For those concerned about renters who cannot afford the going rate, there are better approaches that make the renters as well as the landlords better off than under rent control. Thus, with rent control, the economy is not Pareto efficient."

From Stiglitz, Principles of Microeconomics, 2nd Ed. (p. 320).


Answer: Rent controls create two kinds of inefficiency. First, there is a dynamic inefficiency: by suppressing prices, rent controls reduce the incentive that potential landlords have to create new apartment buildings; this is likely to cause trouble in the long run. Second, there is an immediate inefficiency. Say that there's an old man living in an apartment, paying the government-establish rent of \$800 per month. He values the apartment at \$1,000 per month (and so has a consumer surplus of \$200 per month), but some young internet executive values that same apartment at \$2,000. This is inefficient because it's possible to make someone better off without making anybody else worse off. Here's how: Have the young executive pay the old man \$300 per month, and let the executive rent the apartment for \$800 per month. Under this new situation, the apartment owner is not worse off (she's still receiving \$800 per month), the old man is better off (he now has a consumer surplus of \$300 instead of \$200, and can use the \$800 he's no longer paying in rent to rent a different place), and the young executive is better off (she is only paying \$800 + \$300 = \$1,100 for an apartment that she values at \$2,000, so she's gaining a consumer surplus of \$900). 













\section{A Rogues' Gallery of Market Failure}

In what follows, keep your eyes on three things:
\begin{description}
\item [Impediments to bargain] There are always incentives to eliminate inefficiencies: if redistributing resources can make somebody better off without making anybody else worse off, there is a natural incentive for the potentially better-off person(s) to work on bringing about that redistribution by bargaining with others. It is only in combination with impediments to bargaining that other difficulties create problems and inefficiencies. 
\item [Incentives] One reason why many markets function so well is that the incentives are all in the right place. (Recall how the demand curve is like a carrot, providing incentives for suppliers to supply more, and how the supply curve is like a stick, providing incentives for buyers to buy less.) Sometimes, however, incentives are not in the right places---for example, if property rights are not well defined---and in these instances markets won't function well. 
\item [Information] Good information is crucial to well-functioning markets. Some instances of market failure stem from lack of information, or from lopsided information (e.g., one person knowing something that somebody else doesn't know).
\end{description}

Recall again the distinction between positive and normative economics:
� Positive economics attempts to predict. ``Taxing internet commerce will reduce the volume of e-commerce" is a positive statement. The use of words like ``will" generally identify positive statements.
� Normative economics attempts to value. ``We should not tax internet commerce" is a normative statement. The use of words like ``should" generally identify normative statements.

Last time we made the case for limited government involvement. We had the First Welfare Theorem tell us that competition yields efficient outcomes, and the Second Welfare Theorem tell us that equity concerns can be addressed separately, e.g., through an initial reallocation of resources (taxing some people and giving the money to other people).

So: now we're going to look at the case for more active government involvement. This case focuses on inequity, integrity, and inefficiency. 

\begin{itemize}
\item A case can be made that government action is necessary or desirable to address equity concerns or other issues unrelated to efficiency. 
\item A case can be made that that government action is necessary to preserve the integrity of the market, i.e., to establish the rule of law, prevent firms from conspiring to fix prices, etc. 
\item A case can be made that government action is necessary to correct market failure. A market failure occurs when ``the market" does not yield a Pareto efficient outcome. 
\end{itemize}

So, let's go through and look at these issues:


We are now going to look at market failure, instances in which markets fail to produce efficient outcomes. Unlike the last discussion (about antitrust), market failure actually is my area of expertise, so you should expect a lot from me here.

OK: First we're going to look at a bunch of examples of market failure. Your job, as you read through these, is to think about identification, explanation, and correlation. In other words, (1)  Why do we consider these things to be market failure? Where is the inefficiency? (2) Why are these things market failure? What is the source of the inefficiency? And (3) What connections (if any) exist between different examples? Which of the examples can you group together as related?


A Rogues' Gallery of Market Failure

Monopoly\index{monopoly}. When a monopoly\index{monopoly} cannot perfectly price discriminate (e.g., when it must sell to everybody at the same price), we get market failure.

College admissions. A recent article in the New York Times (``Ease Up, Top Universities Tell Stressed Applicants", 12/7/00) reported that the college application process ``has become such a high-stress exercise in resume padding that students are arriving at their campuses on the brink of burnout\ldots At Harvard, the admissions office has written a paper lamenting that students `seem like dazed survivors of some bewildering lifelong boot camp.'" 

Discrimination. Despite some economic intuition to the contrary (more on this later), discrimination appears to be more than a figment of the imagination. Consider, to take a recent example, a story in the New York Times titled ``Diploma at Hand, Japanese Women Find Glass Ceiling Reinforced With Iron" (1/1/01):

When Yoko Hayakawa, a student at one of Japan's elite universities, began a round of job interviews with technology companies here recently, it was not long before she noticed a pattern that struck her as very strange.
Although she had been getting excellent grades, spoke English fluently and was bursting with professional ambition, all the recruiters were asking trivial questions about her social life or how she would feel about accepting a clerical position.
``One company asked if I would work as a secretary, instead of in a technical job," said the 23-year-old Keio University senior\ldots 

Later on in the article another female student comes to the following conclusion: ``What I've learned\ldots is that if a company has a choice between a man and a woman, they will choose the man, even if he is of lesser ability." If she is correct, this is a clear example of market failure: the labor market is failing to allocate jobs efficiently.

Environmental problems. Pollution, overfishing, and other environmental problems are symptoms of market failure. In the case of overfishing, for example, why doesn't ``the market" prevent inefficient harvest practices?

Drug use in professional athletics. Without regulations banning the use of steroids and other performance-enhancing drugs, many (all?) athletes would use them. Why do athletes ignore the dangerous medical side effects of these drugs, and why is it necessary for regulatory agencies to ban the use of such drugs?   

Patent races. In a patent race, multiple companies or scientists race to be the first to develop a new drug or a new product or whatever. This can create inefficiencies: For example, two companies racing to develop a new drug may both work their employees extra hard, incurring huge (and socially undesired) costs in order to beat out the competition.

Insurance markets (I). Car insurance companies often force their policyholders to take a deductible (i.e., to be liable for the first \$x in damages). Since the whole point of insurance is to be insured against damage, this seems quite inefficient.

Insurance markets (II). In some instances (e.g., health insurance in Washington State), it is simply not possible for an individual to buy insurance at any price. The market failure here is that the market simply does not exist.

``Free" printing at UW. When I arrived on campus in 1998, printing at the UWired computer labs on campus was completely free. Sounds great, right? Well, not quite: students printed 35,000 pages each day, thousands of  which went unclaimed, and boxes of printed pages were dumped in the recycling bin each day. The University's bill for paper and toner alone reached \$25,000 per month. Not a very efficient way to run things.

Class feedback sheets. Past experience suggests that a significant number of students fail to take those end-of-quarter comment sheets seriously: they leave them blank or write silly comments or otherwise fail to put time and effort into them. But this creates a huge inefficiency: by failing to give me good feedback, students block me from improving my class, and hundreds (thousands?) of future students may be doomed to an inefficiently lousy educational experience.

Health care and education. ``Free printing" and feedback sheets are only two examples of market failure in education, which is plagued with inefficiencies. (Think back to our discussion about national student testing.) Health care faces similar problems: HMOs fail to provide people with good coverage, health insurance is sometimes unavailable (see above), preventive care (``an ounce of prevention is worth a pound of cure") is routinely ignored, etc. What is it about these two markets that make them so troublesome?

Traffic congestion. Recall the example on one of your quizzes where congestion created a two-hour commute, yet nobody would use the bus system that if universally adopted would reduce everybody's commute to 40 minutes. The free-market outcome of everybody driving their own car does not produce efficiency.

Cell phones in class (or at movies or during meetings). Having cell phones go off anywhere and everywhere is not efficient. Arrrgggh! 

Group housing. I love my housemates dearly, but they put my expensive knives in the dishwasher, they use up all the hot water in the morning, and they seem to delight in stranding me in the bathroom without any toilet paper. Bad bad bad.

OK. So what's going on in these examples? Well, there are three things I'd like you to keep an eye on:

Armed with these ideas, let's go back and take a look at our rogues' gallery: 







\section{Integrity of the Market}

This is basically the government playing referee. Most generally, having a court system and laws \&etc takes us out of the Wild West and into a world where people can trade (or not trade) without fear of personal injury, theft, etc. Almost everybody agrees that this is a good thing for government to do.

Governments can also play a more specific role by getting involved in antitrust activities such as the case against Microsoft. Firms know that they can make more money by cooperating than by competing (recall that competition resulted in a Prisoners-Dilemma situation for producers), so there is an incentive for firms to get together to avoid competition. (This incentive is especially evident in cases where there are only a few firms-a limited number of firms makes bargaining easier and makes deals easier to make and enforce.) Adam Smith identified this incentive in The Wealth of Nations: ``People of the same trade seldom meet together, even for merriment and diversion, but the conversation ends in a conspiracy against the public, or in some contrivance to raise prices." (Cited in Stiglitz as Book 1, Chapter 10, Part II.)

Antitrust economics (which you can study more if you take a class in industrial organization) is an interesting blend of economics and law. This blend is interesting because economics is (or at least likes to think of itself as) a field that is forward-looking: if I come up with a new theory today that does better than the old theory, the new theory is in and the old theory goes out. Law, with its concern for legal precedent, is much more backwards-looking. The result is antitrust litigation that combines modern economic arguments with legal precedents that go back 50 or more years. 

Legally, the important laws are the Sherman Act of 1890 (which made it illegal to ``monopolize" a market or to collude to fix prices), the Clayton Act of 1914 (which outlaws mergers and acquisitions likely to ``substantially lessen competition"), and the Robinson-Patman Act of 1936 (which makes it illegal to price discriminate if the result is to ``effectively lessen competition").

What is the connection between monopolies and inefficiency? Curiously, it's more complicated than you might think. We should expect monopolies to charge higher prices, but this is not a source of inefficiency: the only effect here is to redistribute income from buyers to the seller. The inefficiency from monopoly\index{monopoly} comes only if the monopoly\index{monopoly} cannot price discriminate: in this case the monopolist will not produce some units that it should have produced. But a perfectly price discriminating monopolist is not inefficient; it captures all of the consumer surplus, but it does not create inefficiencies! 

Perhaps the most important take-home message here is that it is not illegal to have a monopoly\index{monopoly}; it is only illegal to get a monopoly\index{monopoly} in an illegal manner. What does this mean? Well, let's say you invent a new kind of technology for printing; your invention is so great that everybody abandons their old printers and buys your printer. You now have a monopoly\index{monopoly} of the printer market, but you acquired that monopoly\index{monopoly} in a perfectly legal manner: your product was superior. So there's nothing wrong with what you did, and the government will not come after you for violating antitrust laws. What would be illegal is if you got your monopoly\index{monopoly} by buying up all the other printer makers (the Clayton Act outlaws such behavior) or if you got your monopoly\index{monopoly} by extending some other monopoly\index{monopoly} you had previously (e.g., if you already had a monopoly\index{monopoly} on computers and purposefully designed them to work only with your printers). It was this type of illegal monopolization that the government accused Microsoft of: Microsoft had (according to the government) perfectly legally acquired a monopoly\index{monopoly} in operating systems, but then illegally attempted to monopolize the browser market.

There's lots of disagreement among economists (and others) about how active the government needs to be in enforcing antitrust laws. For example, there is disagreement over what market share constitutes a monopoly\index{monopoly}, over the harm from oligopoly (e.g., when only a few firms control the market), over the ability of firms to use predatory pricing to drive away competition, and over the stability of price-fixing cartels. Confession: Antitrust economics is not my specialty and this is just a very brief sketch. If you're interested in reading more about this topic I'd recommend section 12.3 of Silberberg's \emph{Principles} textbook.



Monopoly\index{monopoly}. Competitive markets result in the outcome shown in Figure 1 (towards the back of this file), with a market price of p*; the resulting consumer and producer surplus are areas A and B. If the monopoly\index{monopoly} cannot price discriminate we get the result in Figure 2: the monopolist will set a price pM to maximize its profits (i.e., its producer surplus); this price will be higher than the competitive market price; and we get an inefficiency of area D, the deadweight loss. (We also get a reduction in the consumer surplus, area A; this may displease your sense of equity, but it is not a source of inefficiency.) 

Are the incentives in the right place here? Yes. The monopoly\index{monopoly} wants to make as much profit as it can, and it would love to serve those consumers in area D. (In fact, the monopolist's ideal is to perfectly price discriminate, in which case it gets the producer surplus shown in Figure 3; note that there is no inefficiency in this case.) The problem here is due to inability to bargain-the monopolist may not be able to strike side-deals at lower prices with those consumers in area D-and perhaps due to lack of information as well. (If the monopolist could identify consumers with high and low valuations, it might be able to effectively price discriminate and thereby eliminate the inefficiency.)


















\section{Incentive Problems}


Last time we looked at monopoly\index{monopoly} and traced the inefficiency to impediments and information problems (e.g., the monopolist doesn't have enough information to price discriminate). The next character in the Rogues' Gallery is college admissions:

College admissions. A recent article in the New York Times (``Ease Up, Top Universities Tell Stressed Applicants", 12/7/00) reported that the college application process ``has become such a high-stress exercise in resume padding that students are arriving at their campuses on the brink of burnout\ldots At Harvard, the admissions office has written a paper lamenting that students 'seem like dazed survivors of some bewildering lifelong boot camp.'" 

The college admissions problem is in fact quite similar to the Prisoners' Dilemma problem. Consider, for example, two high school students vying for the last spot in the incoming class at UW. They can either both spend \$1,000 on an SAT preparation course or not. If they're both equally matched, they are both likely to take the SAT prep course, a result that benefits neither of them: both would be better off to agree to not take the SAT course, just like both prisoners would be better off to agree to not confess. 

As in the case of the college admissions problem, a fundamental relationship to the prisoners' dilemma problem exists for plenty of other items in the rogues' gallery: patent races, steroid use, environmental problems, the UW printing policy, traffic congestion, cell phones, group housing, class feedback\ldots  All these problems share the same root as the prisoners' dilemma problem. 

Before we discuss what that root \emph{is}, let's discuss what it \emph{is not}: it has nothing to do with information. In the case of the prisoners, it makes no difference if they know what the other prisoner is going to do: confessing is a dominant strategy, so information is irrelevant. Similar results hold in all these other cases; for example, if you know that your competitor is going to take the SAT prep course, you should definitely take it too; but if know your competitor is not going to take the prep course, you have even more reason to take it! So information is irrelevant, and the idea that we can completely solve these problems with information is wrong. Telling high school students that they're burning themselves (and their peers) out, or telling car owners that they're contributing to congestion, or telling consumers that the products they buy pollute the environment or contribute to overfishing, may do some good (especially if those individuals have consciences), but it ignores the underlying problem.

That underlying problem is about \textbf{incentives}. We all have plenty of incentives to look after ourselves, but we often don't have a strong incentive to look after others. In the prisoners' dilemma, I'm concerned about myself, and that concern leads me to confess. I'm not concerned about you, and in particular I'm not concerned that my confession is going to increase your jail time by 5 years. Of course, you're not concerned about me either, and in particular you're not concerned that your confession is going to increase my jail time by 5 years. The end result is a long jail sentence for both of us.

The same issue can be seen in many of the other examples in the Rogues' Gallery. Drivers in the morning or afternoon rush hour don't take into account the effect they have on other drivers. (Each additional car slows down all the ones behind it; the cumulative impact can be an hour of delay or more!) Consumers don't take into account the environmental impacts of their consumption on others. UW students who printed while the printing was free didn't take account of the impact of their actions on the UW budget or on other students. Cell phone users don't take into account the impact of their phones on others.

Such ``external" effects are called (cleverly) externalities. A \textbf{negative externality} occurs when one person's actions negatively impact another person (e.g., pollution). A \textbf{positive externality} occurs when one person's actions positively impact another person (e.g., putting time and effort into class feedback creates a positive benefit for future students). The incentive structure is such that self-interested individuals don't take these externalities into account, so we are likely to see the underprovision of goods associated with positive externalities (e.g., too few people putting time and energy into class feedback) and overprovision of goods associated with negative externalities (e.g., too many drivers on the road, too much pollution, too much fishing). 

Note that externalities are strongly connected with lack of well-defined property rights. In open-access fisheries, for example, anybody who wants to can come fish, and nobody has an incentive to care about the impact of their fishing on other people either now or in the future. The result is akin to an open-access bank account: anybody can withdraw money, so it's not surprising that the bank account gets depleted or that nobody has much incentive to look after its long-term health.

Compare this result with that in a privately-owned fishery, e.g., a lake owned by one person. When that person fishes, he knows that his fishing has an impact on other people in the future, but he has a strong incentive to care about that impact because the person it affects in the future is him. (Recall the section we did at the beginning of the quarter on fish-as-capital and optimal individual behavior.) So there is in fact no externality, and we should not expect to find inefficiencies in this fishery.

This suggests one possible solution to the problem of externalities: define property rights. For example, privatizing fisheries (getting them out of public hands and into private hands) is one way to internalize the relevant externalities.   

Of course, defining property rights is not always possible. For example, the excessive incentives created in patent races can be tied to a lack of property rights: we should not expect there to be inefficiencies if somebody owned the patent and then auctioned it off. But before the fact, nobody owns the patent. But it seems impossible to establish a property right for something that has not yet been invented.

We'll talk more about government intervention later. First: Below are some analyses of other Rogues' Gallery market failures that involve incentive problems. Enjoy!

Environmental problems. Externalities are central to pollution, overfishing, and many other environmental problems. In all of these cases there is a Prisoners' Dilemma situation, which is often called The Tragedy of the Commons after a famous article by Garrett Hardin. That article is in your reading packet, and is required reading for the next quiz\ldots  

Patent races. Consider the following game: I offer \$1 to the person in class who first gives the correct answer to the following question: You're a shopkeeper in a strange world that has only \$.04 coins and \$.07 coins. So, for example, you can give somebody \$.08 or \$.11 or \$4 in change, but you cannot make \$.05 or \$.06 or \$.09 in change. The question: What is the largest amount for which you cannot make change?

Or consider the following game: I offer \$1 to the person in class who has their hand on the doorknob of the class at the end of 2 minutes. 

Both of these games are likely to induce a lot of effort, with people trying to figure out the coin problem or trying to get their hand on the doorknob. These games are in fact similar to a patent race, in which multiple companies or scientists are racing to be the first to develop a new drug or a new product or whatever. In both the silly games and the patent races, the are ample incentives for the participants; in fact, there can sometimes be too much incentive. For example, two companies racing to develop a new drug may both work their employees extra hard, incurring huge costs in order to beat out the competition; these costs may not be in the best interests of society, which doesn't particularly care which of the companies develops the drug. These costs, then, are a source of inefficiency. Like the classroom games, the patent race induces an inefficiently high amount of effort.

The inefficiency here can be traced to an incentive problem, in this case an excessive amount of incentive: what society cares about is that the drug get developed, but the incentive for the company is to develop the drug first. There is also an element of an information problem: if everybody knew that company X was way ahead in the patent race and was going to develop the drug first, nobody else would compete and there wouldn't be an inefficiently large amount of investment. And, as usual, there's an impediment to bargaining: if the various companies could get together and agree to share the proceeds from the new drug, they could jointly develop the drug in a way that eliminated the patent race inefficiencies. (Unfortunately, or perhaps fortunately, such bargaining is easier said than done.)


``Free" printing at UW. When I arrived on campus in 1998, printing at the UWired computer labs on campus was completely free. During my first year, the University instituted a charge system involving a fee of \$.07 or \$.08 per page. After a letter appeared in Ruckus (the lefty campus newspaper) criticizing this policy, I did some research and wrote a Letter to the Editor, which Ruckus published in November 1998:

Last year, during the glory days of free printing, UWired went though 35,000 pieces of paper each day. Thousands of those pages went unclaimed, proof that students were unwittingly internalizing an insidious economics lesson about the value (or lack thereof) of paper and printing. That lesson carried a heavy price (\$25,000 per month for paper and toner alone) for the University, for the students and families that pay its bills, and for the environment, which was destined to suffer once students left the pristine (chlorine-free recycled paper) confines of campus. 

Thanks to the new printing charges, we are now learning to think before we print. Paper use has dropped to a quarter of its previous level; the torrent of unclaimed output has slowed to a trickle. What students should fight for now is a fair share of the resulting financial gains. For example, the revenue from the printing charges (and the savings from reduced paper and toner use) could be ``recycled" into a \$5 or \$10 credit on students' Husky Cards at the beginning of each quarter. Properly designed, such a credit could preserve the paper-saving incentives created by the new policy while actually reducing the overall financial burden on the average student. 

The problem with free printing (in my opinion) is that it creates the wrong incentives: students have no reason to conserve paper (e.g., by printing doublesided), and consequently do not conserve paper. Since paper and printing are costly to the University (in fact, the cost is about \$.07 or \$.08 per page), the result was an inefficiency: students printed out pages that they valued at  \$.00 or \$.01 per page, and the University had to bear the cost; other projects that had value (e.g., additional computers for students) were not funded because of the high cost of ``free" printing.

The impediments to bargaining here are clear: if all the students could get together and agree to limit printing, they could reach an efficient outcome. But there are simply too many students, and the incentive to free-ride is too strong.


Traffic congestion. There is good information about traffic congestion, so the problem here involves incentives: when you get on the road at 8:00 in the morning, you are actually slowing down other drivers! One car may not seem like a big deal, but the extra slowdown that you create may slow down all future drivers during that commute; the cumulative impact of your participation in the commute on other drivers can easily reach 60 minutes or more! Of course, you're not thinking about your impact on other drivers; you're just thinking about getting to wherever it is you're going. So there's an incentive problem. There's also an impediment to bargaining: if all those other drivers could get together and each chip in \$.01, you might very well accept that sum and agree to take the bus or carpool or drive later or otherwise get out of their way. But such a deal is unlikely because of the vast numbers of people involved and the high costs of communication. 


Cell phones in class (or during movies or meetings). This is mostly an incentive problem: people with cell phones don't take into account the annoyance it causes others when their phone goes off. The result is an inefficiently high number of cell phone interruptions. The large number of people involved makes bargaining impossible: you can't just go to the opera and make a deal with the other 1,000 people there. (Question: How do we deal with this issue in our class? Answer: we internalize the externality by fining people whose cell phones go off in class.)


Class comment sheets. What incentive do you have to put time and energy into the feedback that you give me at the end of the quarter? Clearly it would be efficient for you to put a reasonable amount of effort into it: the things you say might affect the way I teach in the future, and if I end up in a teaching career that could influence thousands of future students. Unfortunately, it doesn't really affect you: you probably won't take my course again, so there's no good incentive for you to provide me with good feedback. (That good feedback, then, is an example of a positive externality: good feedback yields a benefit to future students.) If bargaining was easy, you would have an incentive: students who are taking my class in the future would come to you and pay you to put time and energy into your feedback. Unfortunately, such bargaining is not easy, and the result is likely to be an inefficiently poor educational experience for my future students. (Of course, the same incentive structure will be to your detriment in your future classes-you'll receive inefficiently poor educational experiences because of the lack of incentives for previous students in those classes-but the effort that you put into my feedback is irrelevant in terms of that result and your putting time and energy into feedback for my class will not change the incentive structure for others.)


Group housing. Lots of incentive problems. In my house, for example, I bought some nice knives, and my housemates keep putting them in the dishwasher even though they're not dishwasher-safe. Another example: we share food in my house (everybody buys whatever they want, and we divide the receipts up evenly amongst the five of us), which creates an incentive to go out and buy expensive treats: When I buy Ben and Jerry's for \$5, it actually only costs me \$1 because my housemates are splitting the bill; so the incentives for me to be caution in my purchasing decisions are wrong. These and other negative externalities from group living situations are nicely summed up by the common experience of finding that your nice housemate has used the last sheets of toilet paper and left you without. There is a negative externality involved here-your housemate doesn't have much of an incentive to save some toilet paper for you-but there are also lots of interesting strategic behaviors here. For example, the root of the toilet paper problem is that nobody has gone to the store to buy toilet paper. Why? Well, maybe it's because you thought that your housemate would go, and she thought that you would go, and so it becomes a game of chicken to see who's going to give in first and go buy toilet paper. 


Charitable givings. Probably the scariest thing about teaching an introductory economics course is some evidence that such courses are likely to make students give less money to charity. (Researchers asked students before the beginning of the quarter and after the end of the quarter to give money to charity, and the experience of taking econ reduced charitable giving.) This is perhaps due to discussions of externalities: charitable giving is a positive externality in that there is an external benefit (e.g., food) for somebody else. So we can conclude that charity alone is unlikely to provide an efficient level of public goods, e.g., we can't just fund schools and parks and police with charitable donations. But this doesn't mean that you shouldn't give money to charity, any more than our discussion about voting means that you shouldn't vote! I think you should vote, and I think you should give to charity, and I mean it! 












\section{Green Taxes}


Last time we discussed various examples of market failures arising from incentive problems, e.g., environmental issues, traffic congestion, patent races. Today's topic is how to deal with those problems. So: We did some brainstorming a few Thursday's back on how to deal with traffic congestion, and came up with a bunch of ideas: build more roads, expand public transportation, encourage carpooling and telecommuting, raise the gas tax or impose tolls on roads, limit car use (e.g., the Mexico City solution: cars with license plates ending in even-numbered digits are only allowed to drive on even-numbered days; those ending in odd-numbered digits are only allowed to drive on odd-numbered days), etc.

So let's do the same thing for another problem, e.g., some environmental problem like global warming or overfishing. What ideas can we brainstorm?

Global Warming
Force every country to reduce its emissions by 10\% or 20\%
Ban coal
Encourage the use of renewable energy 
Encourage the use of nuclear energy
Impose a carbon tax 
Tighten CAFE standards (these Corporate Average Fuel Economy standards determine the minimum gas mileage for cars in the U.S. One reason for the proliferation of SUVs is that they are exempt from CAFE standards)
Adopt a tradable permit system for carbon
Mandate the use of electric cars 


Overfishing
Limit the fishing season
Restrict equipment (e.g., ban nets or sonar or outboard motors)
Impose a tax on catching fish
Adopt a tradable permit system (in fisheries, these are called ITQs, Individual Transferable Quotas)


OK, so: how do we decide which of these policies to adopt? Well, economists like to advocate for the use of economic instruments, namely taxes or tradable permits: the government can either impose a tax (say, on carbon; when used to deal with an environmental problem, such a tax is often called a green tax) or use a tradable permit system (in which a certain number of  ``licenses to pollute" or ``licenses to fish" are auctioned off or otherwise distributed, with the proviso that owners of these licenses can either use them-i.e., emit one ton of pollution, or catch one ton of fish-or sell them to someone else to use). Economists contrast these policies with command-and-control policies through which the government specifies the use (or bans the use) of certain items (e.g., restricts equipment use, limits the fishing season, etc.). Why?

Intuitively, the reason is this: Recall that incentives are the root of market failure in the case of pollution or overfishing: because of externalities, individuals don't have the right incentives to limit pollution or to limit their fishing. Economic instruments aim to change those incentives directly; many command-and-control policies take a less direct approach, and often leave the underlying incentive problem unchanged. A good analogy here is to compare medical treatments focused on symptoms and those focused on the disease: command-and-control policies target the symptoms of market failure; economic instruments target the disease.

Moving beyond intuition, we can identify four ways in which economic instruments are superior to command-and-control approaches:


\begin{description}

\item [Innovation] Putting a tax on gasoline encourages firms and individuals to innovate and find ways to use less gasoline; in effect, they are engaging in a socially beneficial form of tax evasion. Command-and-control approaches (for example, banning SUVs) fail to provide any incentive for long-term innovation. This highlights the power of market forces in stimulating innovation. ``Build a better mousetrap, and the world will beat a path to your door" is an amazingly simply two-step operation: first you build a better mousetrap, and then the world (guided by the invisible hand) beats a path to your door. Under a command-and-control approach, however, building a better mousetrap (or a better pollution-control device) is just the beginning: after you build it, you need to convince some government bureaucrats that it really is a better mousetrap; get legislation introduced in Congress mandating the adoption of your mousetrap; battle for the passage of that legislation against well-funded opponents who claim that adopting your mousetrap will drive them out of business and, besides, there is no mouse problem in the first place; and then gear up for a protracted legal battle. The alternative is, in the words of Alan Durning, to make ``prices tell the ecological truth" and then let market forces do their thing.

\item [Efficiency] By correcting the incentive problem underlying the market failure, economic instruments eliminate the market failure, returning us to all the Pareto efficiency results we've seen before. In contrast, command-and-control policies intended to reduce the inefficiencies caused by market failure often create inefficiencies of their own. For example, environmental policies sometimes require firms to spend millions of dollars to reduce pollution by X amount when the firms themselves can identify alternative ways to get the same amount of pollution reduction at much lower cost. This is a clear source of inefficiency (by allowing the firm to adopt the alternatives, they are better off, and nobody else is worse off). 

\item [Shades of gray] Economic instruments are sometimes better able to handle issues that aren't black and white. For example, current EPA regulations on the pesticide malathion (used on apples grown here in Washington State and elsewhere) specify a maximum of 8 parts per million (ppm) of malathion on apples in order to protect the health and safety of consumers. The effect of this regulation is to establish a black-and-white dividing line---8 ppm of malathion is completely harmless, but 9 ppm is so dangerous as to be illegal---where no such dividing line exists. In contrast, economic instruments provide a constant incentive for farmers to use less: using a tax, perhaps in addition to a maximum standard, would help provide the right incentives.

\item [Revenue] Economic instruments can generate revenue: taxes bring in money, and tradable permits can be auctioned off (like the frequency spectrum) to raise funds. That money does not disappear into a hole in the ground. One option is to use the revenue to fund various programs-for example, carbon tax revenue is sometimes mentioned as a possible source of ``transition" funds for privatizing Social Security. Another option is to use the revenue to reduce existing taxes, e.g., to use carbon tax revenue to reduce payroll taxes. This sort of policy has been adopted in various European countries (e.g., Germany, which is fighting high unemployment caused in part by high payroll taxes), and was the basis for the book I worked on at Northwest Environment Watch; called Tax Shift, the book made the case for a revenue-neutral tax shift, i.e., one that used revenues from green taxes to make dollar-for-dollar reductions in existing taxes. Such a tax shift promises a ``double dividend": one dividend comes from ``taxing the bad stuff" and thereby correcting the environmental problem (which, like other market failures, creates inefficiency); the second dividend comes from reducing existing taxes and thereby eliminating a government-created market failure (taxing ``the good stuff"-income, payroll, savings, and other things we'd like more of-creates an inefficiency, so reducing these taxes increases efficiency). 

\end{description}



Let's finish off by looking at some pictures. Figure 1 on the last page shows our familiar supply and demand curve, except that there is an externality associated with this good which creates a difference between private marginal costs\index{marginal!cost} and social marginal costs\index{marginal!cost}. In other words, the marginal cost\index{marginal!cost} of an additional unit is different for the individual and for society: there is some external cost which the individual doesn't take into account. For example, we could be looking at the market for gasoline, with pollution creating an externality of \$.40 for each gallon. Or we could be looking at the market for fish, with an externality of \$.40 per fish created by fact that each fish you catch this year limits reproduction and reduces the number of fish that will be available next year. 

Next: If individuals behave in a self-interested manner, we expect them to ignore the externalities they impose on others. As a result, we expect to get the equilibrium shown in Figure 2, with an equilibrium price of \$.70 and an equilibrium quantity of 7. This equilibrium is inefficient, and the inefficiency adds up to area D shown on the graph. The source of the inefficiency is overproduction: at a market price of \$.70, a buyer with a marginal value\index{marginal!value} of \$.80 is buying, and a seller with a (private) marginal cost\index{marginal!cost} of \$.60 is producing. The problem here is that nobody is taking the externality into account: between them, the consumer surplus for this individual consumer (\$.80 - \$.70 = \$.10) and the producer surplus for this individual producer (\$.70 - \$.60 = \$.10) add up to \$.20, which gets outweighed by the \$.40 externality: social welfare goes down rather than up!

To correct this problem we can impose a tax of \$.40 on the sellers of this product. The result of this tax is to shift the supply curve up by \$.40, and we see in Figure 3 that this aligns the private marginal cost\index{marginal!cost} curve with the social marginal cost\index{marginal!cost} curve, thereby eliminating the inefficiency. The tax functions as a surrogate for the externality: individuals still don't care about the external effects of their actions, but the tax makes them act as if they did. The tax brings individual incentives back in line.

Another way to correct this problem is to notice that the efficient level of production is 5 (say, 5,000 tons of fish), and use a tradable permit system. The government creates 5,000 permits, and requires anybody catching fish to turn in one license for each ton of fish. The licenses are either auctioned off (raising money for the government) or otherwise allocated, and we get an efficiency allocation because potential fishermen are able to buy and sell licenses, thereby exhausting all possible gains from trade.

In the end, a tax of \$.40 has the same effect as auctioning off 5,000 permits: with a tax of \$.40, the equilibrium quantity will be 5,000, and when the government auctions off the 5,000 permits the auction price will be \$.40.












\section{Information Problems}

Good information is crucial to well-functioning markets. Some instances of market failure stem from lack of information, or from lopsided information (e.g., one person knowing something that somebody else doesn't know). Our first example is from a famous paper about the used car market called ``The Market for Lemons". Its author, George Akerlof, won the 2001 Nobel Prize in Economics along with Joseph Stiglitz and Michael Spence; all three made early contributions to the economic theory of information.

\section{The Market for Lemons}

The world contains good cars (i.e., ``peaches") and bad cars (``lemons"), and we might expect the market for used cars to feature both types of cars. One salient feature of this market, however, is that there is an asymmetry of information: the sellers know whether their car is good or bad, but the buyers often do not know (and may not be able to find out). Akerlof showed that this asymmetric information can have significant ramifications, as shown in the following numerical example.

Imagine that there are equal numbers of four types of used cars. These types are listed in table~\ref{lemons1}, along with the values of those cars to their current owners (each of whom would be willing to sell their car for any price greater than their value) and to their potential owners (each of whom would be willing to buy a car for any price less than their value). A well-functioning market would exhaust all possible gains from trade: since the potential buyers value the cars more than their current owners ($140>110,\ 100>70,\ 60>50,\ 20>10$), we would expect all types of cars to be traded.

\begin{table}
\begin{center}
\begin{tabular}{|ccc|}
\hline
Car Type & Current Owner's Value & Potential Buyer's Value \\
Excellent & 110 &   140 \\
Good    & 70 & 100 \\
Average & 50 & 60 \\
Lemon & 10 & 20 \\ \hline
\end{tabular}
\end{center}
\caption{Types of used cars and their values to their current owners and potential buyers}
\label{lemons1}
\end{table}

Now we introduce an information problem into this market: imagine that sellers know what kind of car they have, but that potential buyers cannot distinguish between the different types, i.e., do not know whether they are buying a good car or a lemon. One ramification is that all used cars must sell for the same price: since all used cars look the same to the buyers, they have no reason to pay more for one than for another; since sellers know that all used cars look the same to the buyers, they have no reason to charge less for one than for another. 

Is it possible that all four types of cars will still be traded? The answer is ``No". To see why, imagine that all four types of cars are traded; we will show that this leads to a contradiction. Since buyers don't know if they're buying a good car or a lemon, they should be willing to pay no more than the \textbf{expected value} of a used car, namely 
\[
\frac{1}{4}(140) + \frac{1}{4}(100) + \frac{1}{4}(60) +\frac{1}{4}(20) = 80.
\]
So the market price cannot be higher than \$80. But this market price is too low to induce all of the owners to sell their cars! In particular, owners of excellent cars value them at \$110, so they will keep their cars rather than put them up for sale. This is the first effect of the information asymmetry: owners of excellent cars will not put them up for sale, resulting in an inefficiency. Gains from trade are not exhausted because the excellent cars are not traded. 

Now, is it possible that the remaining three types of cars (good, average, and lemons) will be traded? Again, the answer is ``No". Buyers should \emph{know} that excellent cars won't be sold, so with only three types of cars on the market, they should be willing to pay no more than the expected value of a used car, which is now
\[
\frac{1}{3}(100) + \frac{1}{3}(60) +\frac{1}{3}(20) = 60.
\]
So the market price cannot be higher than \$60. But this market price is too low to induce all of the relevant owners to sell their cars! In particular, owners of good cars value them at \$70, so they will keep their cars rather than put them up for sale. This leads to a second inefficiency: owners of good cars will not put them up for sale, leading to additional unrealized gains from trade.

Similar reasoning shows that it is not possible for only average cars and lemons to be traded. In this case, buyers would be willing to pay no more than the expected value of a used car, namely $\displaystyle \frac{1}{2}(60) + \frac{1}{2}(20) = 40.$ But owners of average cars are unwilling to sell their cars at such a low price, so they will not put them up for sale, resulting in even more unrealized gains from trade.

Ultimately, only lemons will be sold (and the market price will be somewhere between \$10 and \$20). The information asymmetry causes the market to \textbf{unravel}, and leads to market failure: there are potential gains from trade involving excellent, good, and average cars, but these cars are not traded. Note that the difficulties here stem not from incentive problems but from information problems. Specifically, this is an example of an \textbf{adverse selection} (or \textbf{hidden information}) problem.

Curiously (and crucially), the problem stems not so much from the \emph{lack} of information as from the \emph{asymmetry} of information. The market would work fine if buyers knew what kind of car they were buying: all four types would be sold, and there would be four different market prices. But the market would also work fine if the sellers didn't know what kind of car they were selling. If the sellers were as much in the dark as the buyers, the sellers would do an expected value calculation and conclude that the expected value of keeping the car was 
\[
\frac{1}{4}(110) + \frac{1}{4}(70) + \frac{1}{4}(50) +\frac{1}{4}(10) = 60.
\]
So the sellers would be willing to sell for anything more than \$60, the buyers would be willing to buy for anything less than \$80, and we would expect an efficient outcome to be reached.

Adverse selection problems can arise in any situation in which there is asymmetric information, i.e., one party has knowledge that the other party does not have. One important example is medical care: patients often know much more about their health condition than hospitals do, so the patients that sign up for health insurance often require more medical care than average. This can lead to the unraveling of the insurance market, as was the case recently in Washington State: in some areas of the state, there were no insurance companies willing to sell health insurance to individuals.




\section{Moral Hazard}

Our next example also involves the insurance industry. Consider car insurance. There is a fundamental problem with car insurance: when the company gives you an insurance policy, you lose a good portion of the incentive you had to drive carefully! This is called a \textbf{moral hazard} or \textbf{hidden action} problem, and the same sort of problem arises in many other situations: since the FDIC insures bank account balances up to \$10,000, depositors have no reason to investigate the financial soundness of the banks they put their money in; if the IMF intervenes in currency crises to bail out foreign investors who made bad investments in emerging markets, investors may not have much of an incentive to avoid bad investments in the first place; if companies pay salespersons by the hour instead of on commission, those salespeople have less reason to work hard; etc. 

The moral hazard problem is partly an incentive problem, but it is fundamentally different than the Prisoner's Dilemma-type problems we examined earlier. Recall that information was irrelevant in the Prisoner's Dilemma problem: regardless of what the other player did, your best strategy was to confess. In moral hazard problems, information is not irrelevant: if the insurance company could tell whether or not you were driving carefully, they could rewrite your policy to cover you only if you were driving carefully; if companies could tell which salespeople were slacking off and which were working hard, they could compensate them accordingly. Perfect information, then, would solve the moral hazard problem, and it is the importance of information that distinguishes these problems from Prisoner's Dilemma-type problems.\footnote{The feature that distinguishes moral hazard problems from adverse selection problems is that moral hazard problems deal with issues that individuals have some control over---Should I drive carefully? Should I work hard?---whereas adverse selection problems deal with issues over which individuals have no control---Do I have a good car or a lemon? Do genetic factors make me more likely to need medical care?}

Insurance companies \&etc address the moral hazard problem by sharing risk: they put a deductible on your car insurance (e.g., make you liable for the first \$500 in damage) in order to give you a greater incentive to drive carefully. To understand why such policies result in inefficiencies, we must first look at why it is that people buy insurance in the first place. So, imagine that your million-dollar house sits at the edge of a cliff and everybody knows it's going to fall down next year. Can you benefit from going out and buying house insurance? No: no insurance company will insure you for less than \$1 million, and for \$1 million you can insure yourself!

The point of insurance, then, is not to protect you from certainty but to protect you from uncertainty or risk: if there's a 1 in a thousand chance that your million-dollar house will burn down in any given year, your expected loss in any given year is  . If you are risk-averse, however (we discussed this way back at the beginning of the quarter), you are willing to pay to avoid risk, i.e., you might be willing to pay \$1,100 each year to avoid risk. The insurance company then makes money off the \$100 that you pay beyond your expected loss.

So: the point of buying insurance is to avoid risk. Without any information problems, you would be able to transfer all of your risk to insurance companies, and you would exhaust all possible gains from trade. Because of the moral hazard problem, however, the insurance company forces you to bear part of the risk (e.g., with a deductible), and this is the source of inefficiency.  

\section{Other Information Problems}

One final example:

Discrimination. The assumption of profit-maximizing firm behavior is pretty hard to square with the concept of discrimination: if firms are profit-maximizing, why would they hire a less qualified person when they could hire a more qualified person? From this perspective, competition and markets create an incentive to eliminate discrimination. 

Nonetheless, discrimination appears to be more than a figment of the imagination. Consider, to take a recent example, a story in the New York Times titled ``Diploma at Hand, Japanese Women Find Glass Ceiling Reinforced With Iron" (1/1/01):

When Yoko Hayakawa, a student at one of Japan's elite universities, began a round of job interviews with technology companies here recently, it was not long before she noticed a pattern that struck her as very strange.

Although she had been getting excellent grades, spoke English fluently and was bursting with professional ambition, all the recruiters were asking trivial questions about her social life or how she would feel about accepting a clerical position.

``One company asked if I would work as a secretary, instead of in a technical job," said the 23-year-old Keio University senior\ldots 

Later on in the article another female student comes to the following conclusion: ``What I've learned\ldots is that if a company has a choice between a man and a woman, they will choose the man, even if he is of lesser ability." 

So: How do we integrate what appears to be discrimination in practice with the economics that we learn in theory?

Well, one approach is to abandon the practical result: we could simply say that theory tells us not to believe in discrimination, so the complaints of these women must be unfounded. Such an approach is perhaps not so implausible: although there are plenty of studies claiming to show that women in the U.S. are discriminated against (e.g., receiving less pay for the same work), there are also studies disputing these findings (which, if you're interested, you can find from hunting around the websites of the Cato Institute or other libertarian thinktanks). 

Another approach is to abandon the theoretical result: we could simply say that companies are not profit-maximizing machines, and that social and cultural factors play a role in their decisions. Such an approach is also perhaps not so implausible.

Yet another approach (championed by Glenn Loury, an economist at Cornell) is to look at potential market failures that explain what we see in practice without violating what we believe in theory. Here's what Loury might say about the discrimination against women in Japan:

1) According to the article, business managers in Japan (who are mostly male) believe that women will leave their jobs after only a few years in order to have children. 
2) Training engineers and other technical employees is costly, so businesses can make higher profits by hiring employees that they think will stick around for a while. Given the beliefs identified in (1) above (women are likely to leave their jobs to have children), it makes sense for profit-maximizing businesses to offer women jobs as secretaries but not as engineers.
3) Assume (as seems likely) that women in low-paying secretarial jobs are much more likely to quit their jobs to have families and take care of those families than women in high-paying technical jobs. Given the behavior identified in (2) above (firms offer only secretarial jobs to women), we can guess that many Japanese women will abandon their jobs to have and raise families.
4) Given the behavior in (3) above (many Japanese women leave their jobs to have children), it make sense for businesses to rationally believe that women are likely to leave their jobs after only a few years in order to have children.

In short, we end up with a self-reinforcing set of beliefs and actions that are logically consistent but nonetheless sub-optimal! The firms are profit-maximizing, and it is that profit-maximization that leads them to discriminate against women.

This analysis is part of a more advanced kind of Nash equilibrium that involves beliefs as well as strategies (recall from game theory that in a Nash equilibrium the players strategies are best responses to each other): given their beliefs that women will quit to have kids, firms won't hire women engineers; given that firms won't hire women engineers, education is unlikely to be a profitable investment for women, and rational women would avoid studying engineering and instead go out and take dead-end secretarial jobs which they are likely to quit in order to have kids; given that women out there in the world actually do quit their jobs in order to have kids, it is rational for firms to believe that women will quit their jobs in order to have kids.

The problem, then, if you believe Loury's analysis, has nothing to do with a bad incentive structure or non-profit-maximizing behavior. Instead, it has to do with lack of information: firms cannot identify which young women will or will not leave their jobs to have kids, and the young women cannot identify themselves: even those that do want to leave to have kids have an incentive to say otherwise in order to get a couple of years of high salary prior to leaving to have kids. These informational problems lie at the heart of Loury's analysis. As for impediments to bargaining, well, in an extreme case there will never be an opportunity to bargain: women will not pursue professional careers, and managers will never see them. In the actual cases discussed in the article, one might see opportunities for bargaining, e.g., the women could offer to sign contracts that would fine them heavily for leaving the company after only a few years, or could agree to be paid mostly in stock options that won't vest for many years. It is not clear (at least not to me) why we do not see such contracts actually written.



Conclusion about information problems: Some instances of market failure are fundamentally about informational problems rather than incentive problems. We've looked at three such examples, and if you take a class in game theory you can study many more.









\section{Individual Failure}

\begin{quotation}\index{jokes!construction workers}
\noindent Three construction workers---a Mexican guy, an Irish guy, and a redneck---are taking their lunch break on the 120$^{th}$ floor of an unfinished office tower. The Mexican guy opens his lunch box and exclaims, ``Burritos---again. If I get burritos for lunch one more time, I'm going to jump." The Irish guy opens his lunch box: ``Corned beef---again. If I get corned beef one more time, I'm jumping with you. The redneck opens his lunch box: ``Bologna---again. If I get bologna one more time, I'm with you, guys."

The next day at lunch time the three men sit down with their lunch boxes. The Mexican guy opens his, shakes his head, and jumps out of the building. The Irish guy looks inside his lunch box, sighs, and jumps after Jose. The redneck hesitantly peers inside his lunch box. ``Wait for me, guys!" he exclaims, and jumps after them. All three fall to their deaths.

The next week at the funeral, the Mexican guy's wife is in tears: ``If only he'd said something, I could have made enchiladas, or tacos, or\ldots " She breaks down in tears, and the Irish guy's wife joins in: ``How was I to know he didn't like corned beef? All he had to do was tell me, and I would have made him something else\ldots " She begins sobbing. The redneck's wife looks around: ``Hey, don't blame me---that dumb-ass packed his own lunch!"
\end{quotation}

We've spent some time looking at market failure caused by impediments to bargaining, information problems, and incentive problems. But let's say that there are no information problems and no incentive problems. Can we still get market failure?

Yes, because there's one more potential source of market failure: \textbf{individual failure}. Recall the fundamental assumption of economics: \textbf{decisions are made by optimizing individuals}. If this assumption does not hold---in particular, if individuals do not optimize---then inefficient allocations of resources may result even if there are no other problems. This is not at all surprising: if individuals don't optimize, then letting everybody do whatever they want no longer seems like such a good idea!

Of all the various sources of market failure, individual failure is of the least interest to economists: it violates the fundamental assumption of economics and so is incompatible with economic analysis. But individual failure may be of great interest to those who are not wedded to the economic assumption, and it brings that assumption into sharp focus.

For example, consider Social Security, or occupational health and safety legislation, or teachers who require you to turn in rough drafts of papers, These policies are undeniably paternalistic: some authority figure is forcing you to save money for your retirement, or to avoid jobs that carry certain risks, or to write a rough draft. But if you're a rational individual, acting optimally, such paternalism is unnecessary. You can save for your retirement on your own, make your own decision about whether or not to accept a risky job, or decide for yourself if you need to write a rough draft. If individuals optimize, these paternalistic policies serve only to limit options and therefore reduce welfare. 


\end{document}














One final comment: If you're ever in an argument with an economist and are in desperate need of victory, just remember these magic words: ``What about the General Theory of the Second Best?" The Theory of the Second Best relates to instances in which there is more than one source of market failure, and basically it says that problems in one market can spill over into other markets and cause problems. As a result, second-best worlds are rife with counterintuitive results, and economists can say very little about these worlds. For example:
� We generally say that coupons are a source of inefficiency, but if they're being used by a monopolist to price discriminate, they may lead to Pareto improvements.
� Free trade is usually good, but if there is an externality (e.g., pollution), then free trade may harm society by increasing the level of the externality.
�  Price controls are usually bad, but if a monopoly\index{monopoly} is charging inefficiently high prices then price controls can produce efficiency.

The general idea here is this: any move away from perfect markets throws a wrench into the spinning wheel of economic theory. So: take away from this what you wish. 

Theory of the Second Best (What about second best problems?) Coupons; monopoly\index{monopoly} polluter; 








\section{Inefficiency}

We've seen that governments can play a role in satisfying equity (or other non-efficiency-related) concerns and in playing referee by promoting and preserving competition in the market. Our next topic is market failure-inefficient market outcomes-and government response to market failure. 



\section{Social Welfare Criteria}

We begin now on the final part of our course, Part V: Buyers, Sellers, Regulators: The Role of Government. Recall the story so far: We began by studying the behavior of individuals (e.g., present value and discounting, expected value), then studied the interactions of two or more individuals (game theory), then studied the interactions of one seller and many buyers (e.g., auctions, monopoly\index{monopoly}), then studied the interactions of many buyers and many sellers (competitive markets). Now we are going to introduce another agent: the government. Our goal here is to try to understand the role of government, i.e., when, why, and how the government should intervene.

Before we delve into this topic, it's important to make a distinction between positive and normative economics:
\begin{description}
\item[Positive economics] attempts to predict. ``Taxing internet commerce will reduce the volume of e-commerce" is a positive statement. The use of words like ``will" generally identify positive statements.
\item[Normative economics] attempts to value. ``We should not tax internet commerce" is a normative statement. The use of words like ``should" generally identify normative statements.
\end{description}

Often times you will see positive and normative statements together, such as ``We should not tax internet commerce because such taxes would hamper the growth of e-commerce." This attempts to draw a normative conclusion from a positive observation: that taxes would hamper the growth of e-commerce is a positive claim; that such hampering is bad is a normative claim.

One final note on this: To remember which is which between positive and normative, it may help to connect ``normative" with ``norm", which dictionary.com defines as ``a standard, model, or pattern regarded as typical: the current middle-class norm of two children per family."  


OK, onward to the role of government. First we're going to make the case for limiting the role of government, and then later we'll make the case for government intervention. So, here's the case for limiting the role of government:


� Government failure. We've spent some time discussing the role of government in correcting externalities \&etc. But it's important to remember that governments are not benign dictatorships: governments are made up of individuals (e.g., politicians) and those individuals have choices and incentives and are self-interested just like the rest of us. Government failure occurs when those incentives and self-interests don't align with the interests of society, e.g., when governments pass laws restricting competition or otherwise benefiting some segment of society at the expense of society at large. For more on this, read Mancur Olson's\index{Olson, Mancur} book The Logic of Collective Action. 