\documentclass[twoside]{article}
\usepackage{pstricks, pst-node, pst-tree, pst-plot, pst-text}

%\usepackage[dvips, pdfnewwindow=true]{hyperref}

\usepackage{version} %Allows version control; also \begin{comment} and \end{comment}
\includeversion{EXAM}\excludeversion{KEY}
%\includeversion{KEY}\excludeversion{EXAM}

\newcommand{\mybigskip}{\vspace{1in}}

\begin{document}

\pagestyle{empty}
\thispagestyle{empty}

\vspace*{-1cm}
\enlargethispage{4\baselineskip}
\begin{center}
\Large Exam \#1 \begin{EXAM}(75 points)\end{EXAM} \begin{KEY} Answer Key \end{KEY}
\end{center}
\normalsize
\bigskip

\begin{EXAM}

\begin{itemize}

\item Other than this cheat sheet (which you should tear off), all you are allowed to use for help are the basic functions on a calculator.

\item The space provided below each question should be sufficient for your answer, but you can use additional paper if needed.

\item \emph{Show your work for partial credit.} It is very difficult to give partial credit if the only thing on your page is ``$x=3$".

\begin{comment}
\item Take the exam during an \emph{uninterrupted period of no more than 2 hours}. (It should not take that long.) The space provided below each question should be sufficient for your answer, but you can use additional paper if needed. \emph{You are encouraged to show your work for partial credit.} It is very difficult to give partial credit if the only thing on your page is ``$x=3$".

\item \emph{Other than this cheat sheet, all you are allowed to use for help are the basic functions on a calculator.} Partial translation: no books, no notes, no websites, no talking to other people, and no advanced calculator features like programmable functions or present value formulas.

\item People who have taken the exam can talk to each other all they want, and people who have not taken the exam can talk to each other all they want, but communication between the two groups about class should be limited to three phrases: ``Yes", ``No", and ``Have you taken the exam?"

\item For questions or other emergencies, call me at x5124 or 206-351-5719.
\end{comment}

\item \textbf{Expected value} is given by summing likelihood times value over all possible outcomes:
\[
\mbox{Expected Value}\ \ \  = \ \ \ \sum_{\mbox{Outcomes \emph{i}}} \mbox{Probability(\emph{i})} \cdot \mbox{Value(\emph{i})}.
\]


\item A \textbf{fair bet} is a bet with an expected value of zero.

\item The \textbf{future value of a lump sum payment} of $\$x$ invested for $n$ years at interest rate $s$ is $\displaystyle \mbox{FV} = x(1+s)^{n}$. The \textbf{present value of a lump sum payment} of $\$x$ after $n$ years at interest rate $s$ is $\displaystyle \mbox{PV} = \frac{x}{(1+s)^{n}}.$ (Note that this formula also works for values of $n$ that are negative or zero.)

\item The present value of an \textbf{annuity} paying $\$x$ at the end of each year for $n$ year at interest rate $s$ is
\[
\mbox{PV}=x\left[ \frac{1 - \displaystyle\frac{1}{(1+s)^n}}{s}\right].
\]
The present value of the related \textbf{perpetuity} (with annual payments forever) is
\[
\mbox{PV}=\frac{x}{s}.
\]

\item The \textbf{inflation rate}, $i$, is the rate at which prices rise. The \textbf{nominal interest rate}, $n$, is the interest rate in terms of dollars. The \textbf{real interest rate}, $r$, is the interest rate in terms of purchasing power. These are related by
\[
1+r=\frac{1+n}{1+i}.
\]
When the inflation rate is small, we can approximate this as
\[
r \approx n-i.
\]


\end{itemize}




\clearpage\
\clearpage
\end{EXAM}

\begin{EXAM}

\vspace*{-2cm}

\begin{flushright}
(5 points) Name: \hspace*{1in}

\medskip
%Student Number: \hspace*{1in}
\end{flushright}

\bigskip

\end{EXAM}


\begin{enumerate}


%This problem is in qa1intro
\item \begin{EXAM} A pharmaceutical company comes out with a new pill that prevents baldness. When asked why the drug costs so much, the company spokesman replies that the company needs to recoup the \$1 billion it spent on research and development (R\&D). \end{EXAM}

    \begin{enumerate}

    \item \begin{EXAM} (5 points) Will a profit-maximizing firm pay attention to R\&D costs when determining its pricing? Yes \ \   No  (Circle one and explain briefly.)     \mybigskip \end{EXAM}

\begin{KEY}
No: the R\&D expenditure is a sunk cost. If it spent twice as much or half as much to discover the drug, it should still charge the same price, because that's the price that maximizes profit.
\end{KEY}

    \item \begin{EXAM} (5 points)
        \begin{description}
        \item [If you said ``Yes" above:] Do you think the company would have charged less for the drug if it had discovered it after spending only \$5 million instead of \$1 billion? Yes  No  (Circle one and explain briefly.)
        \item [If you said ``No" above:] Do R\&D costs affect the company's behavior \emph{before} they decide whether or not to invest in the R\&D, \emph{after} they invest in the R\&D, both before and after, or neither?
         \end{description}
    \vspace{1in} \end{EXAM}

\begin{KEY}
The only time that R\&D costs affect the company's behavior is \emph{before} they're sunk: when the company is thinking about spending money on R\&D, it has to determine whether or not it's going to be profitable to make that investment given their estimate of how much they'll be able to charge for the pill. Once they do the R\&D, however, it's a sunk cost and will no longer influence their profit-maximizing decisions.
\end{KEY}

    \end{enumerate}







\item \begin{EXAM} Imagine that you park in front of a parking meter and face the following choice: you can either put \$1 in the meter, or not put anything in the meter and risk getting a ticket. If you don't feed the meter, there's a 90\% chance that you'll get away with it (and hence pay nothing) and a 10\% chance that a meter maid will catch you and give you a \$20 ticket. \end{EXAM}

    \begin{enumerate}

    \item \begin{EXAM} (5 points) The expected amount of money you'll pay if you feed the meter is, obviously, \$1. What is the expected amount of money you'll pay if you try to risk it? \clearpage \end{EXAM}

\begin{KEY}
The expected cost is $(.90)(0)+(.10)(\$20) = \$2$.
\end{KEY}

    \item \begin{EXAM} (5 points) Imagine that the city managers want to save money by firing 90\% of the meter maids, meaning that the chance of getting caught will be only 1\%. Use your expected value calculation above to suggest a way to do this without changing the expected value (and hence the attractiveness) of cheating. Show your work. \vspace{1in} \end{EXAM}

\begin{KEY}
The city could increase the amount of the ticket from \$20 to \$200. This would mean that the expected value of risking it is still \$2: $(.99)(0) + (.01)(\$200) = \$2$.
\end{KEY}

    \end{enumerate}












\item \begin{EXAM} Imagine that you own a lake and that you're trying to maximize the present value of catching fish from the lake, which currently has 1000 fish in it. The population growth function of the fish is described in Figure~\ref{fig:fishqa1}.

\bigskip

\psset{unit=.5cm}
\begin{figure}[h]
\begin{center}
\begin{pspicture}(0,0)(21,12)
    \psplot{0}{20}{.2 x 10 mul mul .001 x 10 mul 2 exp mul sub}
    \psline[linestyle=dashed](10,0)(10,10)(0,10)
    \psline[linestyle=dashed](6,0)(6,8.4)(0,8.4)
    \psaxes[labels=none, ticks=none, tickstyle=bottom, showorigin=false, dx=5cm, Dx=1000, dy=5cm, Dy=100](21,12)
    \rput[t](10,-.2){$1000$}
    \rput[t](20,-.2){$2000$}
    \rput[t](6,-.2){$600$}
    \rput[r](-.2,10){$100$}
    \rput[r](-.2,8.4){$84$}
\rput[lt](.2,12){Growth $G(s)$}
\rput[b](16, .2){Stock Size $s$}
%\rput[l]{90}(-2.5,8){\small{Dollars}}
\end{pspicture}
\end{center}
\caption{A population growth function for fish.} %$G(s)=.2s - .0001s^2$.
\label{fig:fishqa1} %
\end{figure}
\bigskip
\end{EXAM}


    \begin{enumerate}

    \item \begin{EXAM} (5 points) The maximum sustainable yield policy is to catch 100 fish at the end of this year, 100 fish at the end of the next year, and so on, forever. If the price of fish is always $\$1$ per fish, what is the resulting present value at a 5\% interest rate? \clearpage \end{EXAM}

\begin{KEY}
Plug \$100 and 5\% into the perpetuity formula to get a present value of \$2000.
\end{KEY}


    \item \begin{EXAM} (5 points) An alternative policy is to catch 400 fish \emph{today} (so that 600 remain in the lake), and then catch 84 fish at the end of this year, 84 fish at the end of the next year, and so on, forever. What is the resulting present value? (Assume as above a price of \$1 per fish and an interest rate of 5\%.) Is it higher or lower than the present value of the maximum sustainable yield policy? \vspace{3in} \end{EXAM}

\begin{KEY}
Plug \$84 and 5\% into the perpetuity formula to get a present value of \$1680. Adding this to the \$400 you get from catching 400 fish today and you get a present value of \$2080, which is higher than the present value of the maximum sustainable yield policy.
\end{KEY}

    \item \begin{EXAM} (5 points) Explain (as if to a non-economist) the phrase ``fish are capital", or otherwise explain the importance of the interest rate at the Bank of America in management decisions regarding natural resources such as fish. For full credit, connect this explanation with the somewhat surprising result that you (hopefully) got above, namely that the alternative policy has a \emph{higher} present value than the maximum sustainable yield policy. \clearpage \end{EXAM}

\begin{KEY}
To maximize your present value you need to compare the return you'll get from ``investing in the fish" (or the trees, or the oil) with the return you'll get from investing in the bank. Investing in the bank means catching the fish and putting the proceeds in the bank; investing in the fish means letting the fish grow and reproduce so there will be more fish next year. It turns out that maximum sustainable yield (MSY) is not the economically optimal policy because the \emph{marginal} interest rate you get from investing in the fish is zero at MSY!
\end{KEY}

    \end{enumerate}

\begin{comment}
%This problem is in qa1time
\item \begin{EXAM}(5 points) Explain (as if to a non-economist) the phrases ``fish are capital," ``trees are capital," and/or ``oil is capital," or otherwise explain the importance of the interest rate at the Bank of America in management decisions regarding natural resources such as fish, trees, and oil.
\clearpage\end{EXAM}

\begin{KEY}
To maximize your present value you need to compare the return you'll get from ``investing in the fish" (or the trees, or the oil) with the return you'll get from investing in the bank. Investing in the bank means catching the fish, cutting down the trees, or selling the oil and putting the proceeds in the bank. Investing in the fish means letting the fish grow and reproduce so there will be more fish next year; investing in the trees means letting the trees grow so there will be more lumber next year; investing in the oil means keeping the oil in the hopes that the price will go up next year.
\end{KEY}
\end{comment}





\item \begin{EXAM} Just about everybody agrees that the Social Security system faces financial troubles down the road: after the Baby Boomers retire, the money coming into the system through payroll taxes is not expected to be enough to finance the benefits that the system promises to retirees. \end{EXAM}

    \begin{enumerate}

    \item \begin{EXAM} (5 points) The Social Security Board of Trustees has examined the system's deficit over the next 75 years. If you use an interest rate of 6\%, it turns out that you can divide that 75-year deficit into annual payments of \$225 billion each year for the next 75 years. In present value terms, what is the system's 75-year deficit? (Note: If all the zeroes confuse you or your calculator, use \$225 instead of \$225 billion. Or keep in mind that a billion is a one with 9 zeroes after it. Go down to 6 zeroes and you get a million; go up to 12 zeroes and you get a trillion.) \vspace{2in} \end{EXAM}

\begin{KEY}
Plug \$225 billion, $.06$, and 75 years into the annuity formula to get a present value of about \$3.7 trillion.
\end{KEY}


    \item \begin{EXAM} (5 points) \emph{If you didn't get an answer above or don't feel comfortable with it, use \$4 trillion for the following question.} Your answer above is the present value of the 75-year deficit of the Social Security system. Explain what this means \emph{in English}, i.e., as if to a non-economist. \clearpage \end{EXAM}

\begin{KEY}
If we put \$3.7 trillion in the bank today at 6\% interest, we can make the Social Security system solvent for the next 75 years.
\end{KEY}

    \item \begin{EXAM} (5 points) In a recent op-ed piece in the \emph{Wall Street Journal} (``One Thing We Can All Agree On", January 25, 2005), Treasury Secretary John W.\ Snow claimed that each year we stick with the existing Social Security system adds \$600 billion to the system's long-term deficit. If we stick with the existing system \emph{forever}, he said, the present value of the resulting deficit is \$10 trillion. Show that these two claims agree with each other if he's using an interest rate of 6\%. (Note: if these numbers confuse you or your calculator then use \$600 instead of \$600 billion and \$10,000 instead of \$10 trillion.)  \vspace{1in} \end{EXAM}

\begin{KEY}
Plugging \$600 billion and $.06$ into the present value of a perpetuity formula, we can see that the present value of an infinite stream of \$600 billion payments is \$10 trillion at a 6\% interest rate.
\end{KEY}

\begin{comment}
    \item \begin{EXAM} (5 points) There are many different Social Security reform proposals, including one that President Bush is working on. Assume (accurately enough) that all these reform proposals leave benefits unchanged for workers currently over 55. How (if at all) do the benefits promised these workers affect the merits of the various reform proposals? \vspace{2in} \end{EXAM}

\begin{KEY}
These are sunk costs (or, if you prefer, sunk benefits), so they cannot be the sole reason for choosing one reform proposal over another. However, they can still affect the merits of the various reform proposals. For example, if paying off the benefits promised to workers currently over 55 will increase the government debt level, the impact that will have on the financial health of the U.S.\ government may increase or decrease the attractiveness of various reform proposals. (See problem 2 in Chapter 1 for a similar situation.)
\end{KEY}
\end{comment}

    \end{enumerate}




\item \begin{EXAM} Social Security benefits are adjusted for inflation, meaning that payments to retirees increase at the rate of inflation. \end{EXAM}

    \begin{enumerate}

    \item \begin{EXAM} (5 points) How much will Grammy be paid in one year, in two years, and in three years if her current benefit is \$1000 and inflation is 3\%? \vspace{1.5in} \end{EXAM}

\begin{KEY}
Plug the inflation rate $(.03)$ into the future value formula to get a payment of \$1030 in one year, \$1060.90 in two years, and 1092.73 in three years.
\end{KEY}

    \item \begin{EXAM} (5 points) One way to calculate the present value of these three payments is to use brute force:  determine the present value of each payment separately and then add them together. Go ahead and do this; in doing so you need to choose between the nominal interest rate (which you should assume to be 5\%) and the real interest rate (which you should assume to be 2\%). \clearpage \end{EXAM}

\begin{KEY}
Use the nominal interest rate and the lump sum formula to get a present value of approximately
\[ \$980.95 + \$962.27 + \$943.94 = \$2887.16. \]
\end{KEY}

    \item \begin{EXAM} (5 points) If the nominal interest rate is 5\% and the inflation rate is 3\%, the rule of thumb tells us that the real interest rate is indeed about 2\%. What is a more accurate estimate of the real interest rate? \vspace{2in} \end{EXAM}

\begin{KEY}
Plugging $.05$ and $.03$ into the true formula shows that the actual real interest rate is about 1.94\%.
\end{KEY}

    \item \begin{EXAM} (5 points) Calculate the present value of receiving \$1000 at the end of each year for 3 years if the relevant interest rate is 2\%. \vspace{2in} \end{EXAM}

\begin{KEY} Plug \$1000, $.02$, and 3 years into the annuity formula to get a present value of \$2883.88. This is very close to the answer from above! (If you use $1.94\%$ as a better estimate of the real interest rate, the annuity formula gives you a present value of \$2887.26, which is very close indeed to the \$2887.16 figure above.) The punch line here is that you can use the real interest rate to determine the present value of inflation-adjusted annuities.
\end{KEY}

    \end{enumerate}




\end{enumerate}
\end{document}










\item \begin{EXAM} Consider a choice between receiving a lump sum of \$100 today and receiving an annuity of \$20 every year for 10 years. \emph{As always, banks are standing by to accept deposits and/or make loans at the nominal interest rate.} \end{EXAM}

    \begin{enumerate}

        \item \begin{EXAM} (5 points) One issue that might affect your choice is the interest rate. Compared to a ``low" interest rate (say, 3\%), does a ``high" interest rate (say, 7\%) favor the lump sum or the annuity? (Although it will almost certainly help to do a numerical example with these numbers, this question is really about a more general issue: do higher higher interest rates favor ``money today" or ``money tomorrow"?)  Support your answer with a brief, intuitive (i.e., non-mathematical) explanation.
    \vspace{1.5in}\end{EXAM}

\begin{KEY}
Higher interest rates favor the lump sum payment (``money today") because higher interest rates make the future less important: you need to put less money in the bank today in order to get \$20 in 10 years if the interest rate goes up from 7\% to 10\%.
\end{KEY}

    \item \begin{EXAM} (5 points) Another issue that might affect your choice is your preference for ``money today" versus ``money tomorrow"; for example, you might \emph{really} want money today so that you can buy a new computer. Does this mean you should choose the \$100 lump sum even if the annuity has a higher present value? Circle one (Yes\ \ \ No) and explain \emph{briefly} why or why not.
    \vspace{1.5in}\end{EXAM}

\begin{KEY}
No: you can use the bank to transfer money between time periods. If the annuity has a higher value, you should choose the annuity and then borrow against it (or sell it) in order to have access to money today.
\end{KEY}

    \end{enumerate}











\item \begin{EXAM} You win a \$100 lump sum payment in the lottery! You decide to put your money in a 40-year Certificate of Deposit (CD) paying 6\% annually. The inflation rate is 4\% annually.\end{EXAM}

    \begin{enumerate}
    \item \label{itsitmoney} \begin{EXAM} (5 points) How much money will be in your bank account at the end of 40 years?
    \mybigskip\end{EXAM}

\begin{KEY}
Put \$100 and 6\% in the future value formula to get about \$1028.57.
\end{KEY}


    \item \begin{EXAM} (5 points) Assume that after 40 years you'll have 10 times more money (i.e., \$1000). Does this mean you'll be able to buy 10 times more stuff? Circle (Yes\ \ \ No) and \emph{briefly} explain.
    \mybigskip\end{EXAM}

\begin{KEY}
No: inflation means that you'll have 10 times more money, but not 10 times more purchasing power.
\end{KEY}

    \item \label{itsit} \begin{EXAM} (5 points) Assume that ``It's It" ice cream bars cost \$1 today, and that their price increases at the rate of inflation. How much will an It's It bar cost in 40 years? How many will you be able to buy with the money you'll have in 40 years? (Note: If you didn't get an answer to question~\ref{itsitmoney}, use \$1000 for the amount of money you'll have in 40 years.)
    \mybigskip\mybigskip\end{EXAM}

\begin{KEY}
Plug \$1 and 4\% into the future value formula to get a price of about \$4.80. With \$1028.57, you'll be able to buy about 214 ice cream bars.
\end{KEY}

    \item \begin{EXAM}(5 points) Calculate the real interest rate using \emph{both} the ``rule of thumb" and the true formula.
    \mybigskip\mybigskip\end{EXAM}

\begin{KEY}
The rule of thumb says that the real interest rate is approximately $6-4=2\%$. The true formula gives us $r=\frac{1+n}{1+i}-1 = \frac{1.06}{1.04}-1\approx .019$, i.e., about 1.9\%.
\end{KEY}

    \item \begin{EXAM}(5 points) Assume that the real interest rate is 1.92\%. Use this interest rate to calculate the future value of your \$100 lump sum if you let it gain interest for 40 years. How does your answer compare with your answer from question~\ref{itsit}? \end{EXAM}

\begin{KEY}
Plug \$100 and 1.92\% into the future value formula to get a future value of about \$214. This equals the answer from question~\ref{itsit}.
\end{KEY}

    \end{enumerate}





\end{enumerate}


\end{document}

















\begin{comment}
\item For the sake of simplicity, textbooks often assume that interest is calculated and payments are made once a year. In reality, the relevant time frames can be months (as in monthly car payments), days (as with bank interest payments) or even ``continuous compounding", a method that uses infinitely small time frames and is often used by credit card companies. So:
    \begin{enumerate}
    \item If you put \$100 in the bank at a 12\% interest rate that is calculated annually, how much will you have at the end of one year?
    \item A reasonable \emph{monthly} interest rate can be derived from a \emph{yearly} interest rate simply by dividing by 12, which in this case yields an estimates 1\% monthly interest rate. If you put \$100 in the bank for 12 months at a 1\% interest rate that is calculated monthly, how much will you have at the end of one year? [Hint: Just use the same process you would use for figuring out how much you'd have if you put \$100 in the bank for 12 years at a 1\% interest rate that is calculated yearly.]
    \item (Challenge!) Hopefully you found that you have more money at the end of the year when you get 1\% interest monthly instead of 12\% interest annually. Can you explain (as if to a non-economist) why that is?
    \end{enumerate}
\end{comment}









\begin{comment}
\item ``Comparable investments should have comparable expected rates of return."
    \begin{enumerate}
    \item (5 points) Explain (as if to a non-economist) why this should be true.
    \begin{EXAM}\mybigskip\mybigskip\end{EXAM}

\begin{KEY}
If comparable investments didn't have comparable expected rates of return, who would invest in the asset with the lower rate of return? This is like the traffic analogy: different lanes should have comparable expected travel times because otherwise individuals would get out of the slow lane and into the fast lane.
\end{KEY}

    \item (5 points) Explain the importance of the word ``expected" in the above phrase. (Note that ``expected" is used here in the technical sense of ``expected value.") It may help to give an example.

\begin{KEY}
The importance of the word ``expected" is that \emph{expected} rates of return may differ from \emph{actual} rates of return. Microsoft and Enron might reasonably have had comparable expected rates of return in 1998, but as it turned out Microsoft is doing okay and Enron is bankrupt. By analogy: just because you think different lanes are going to travel at about the same speed doesn't mean that they actually will: there may be an accident up ahead that significantly slows down one of the lanes.
\end{KEY}

    \end{enumerate}
\end{comment}

\begin{EXAM}\mybigskip\mybigskip\end{EXAM}







\begin{comment}
\item (5 points) Consider choosing between an annuity paying \$100 at the end of every year for 250 years and a perpetuity paying \$100 at the end of every year forever. The \emph{difference between these two options} is, well, it's an infinite number of \$100 payments beginning at the end of year 251. The \emph{difference between the \emph{present values} of these two options} is, well, at an interest rate of 5\% it's about \$.01 (or, even more precisely, about \$.0100857). Your job is to try to reconcile these two (rather different) perspectives. It may help to play around with the numbers; it will almost certainly help to remember that the key idea behind present values is figuring out how much money you need to put in the bank today to finance a stream of payments in the future.

\begin{KEY}
The difference between the two options is an infinite number of \$100 payments beginning at the end of year 251, so the present value of this difference is the difference between the present values of the two options. But the present value of this difference is only about \$.01: if you put \$.0100857 in the bank today, at the end of 250 years you'll have about \$2000 and can then ``live off the interest", getting interest payments of \$100 every year forever.
\end{KEY}
\end{comment}

\end{enumerate}

\end{document}












\item The nominal interest rate is 10\% per year. The rate of inflation is 5\% per year. Round all your answers as appropriate.
    \begin{enumerate}
    \item (5 points) You put \$100 in the bank today. How much will be in your account after 10 years?
    \begin{EXAM}\mybigskip\end{EXAM}
    \item (5 points) You can buy an apple fritter (a type of donut) for \$1 today. The price of donuts goes up at the rate of inflation. How much will an apple fritter cost after 10 years?
    \begin{EXAM}\mybigskip\end{EXAM}
    \item (5 points) Calculate $x$, the number of apple fritters you could buy for \$100 today. Then calculate $y$, the number of apple fritters you could buy after ten years if you put that \$100 in the bank. Finally, calculate $\displaystyle z=100\cdot \frac{y-x}{x}$. (The deal with $z$ is that you can say, ``If I put my money in the bank, then after ten years I will be able to buy $z\%$ more apple fritters.") \label{fritter}
    \begin{EXAM}\mybigskip\end{EXAM}
    \item (5 points) Given the nominal interest rate and inflation rate above, calculate the real interest rate to two significant digits (e.g., ``3.81\%"). Check your answer with the ``rule of thumb" approximation. \label{real}
    \begin{EXAM}\mybigskip\end{EXAM}
    \item (5 points) Calculate how much money you'd have after 10 years if you put \$100 in the bank today \emph{at the real interest rate you calculated in the previous question (\ref{real}).} Compare your answer here with the result from question~\ref{fritter}. (YORAM: Fix this $z$ thing; it doesn't really work!)
    \begin{EXAM}\mybigskip\end{EXAM}
    \end{enumerate}












\item Companies and governments use \textbf{bonds} to borrow money. If you buy a ten-year U.S. government bond with a \textbf{face value} of \$100 and a \textbf{coupon} of 5\%, it means that Uncle Sam will pay you \$5 at the end of each year for 10 years, plus \$100 at the end of the tenth year.

    \begin{enumerate}
    \item (5 points) To calculate the present value of this bond, should you use the real interest rate or the nominal interest rate? \newline Circle one: Real \ \  Nominal
    \medskip
    \item (5 points) Calculate the present value of the bond described above if the appropriate interest rate is 4\%. You should get an answer of about \$108, meaning that if you pay \$100 for this bond, your rate of return over ten years is about 8\%. (Note: Since I'm giving you the answer, you must show your work to get credit.)
    \begin{EXAM}\mybigskip\end{EXAM}
    \item (5 points) Instead of buying the U.S. government bond, you buy a bond from a private company that has a 12\% chance of bankruptcy this year. So with probability $.88$ you get all of the promised payments, and with probability $.12$ you get nothing. What is the \emph{expected} present value of this bond? You should get an answer of about \$95, meaning that if you pay \$85 for this bond, your expected rate of return over ten years is about 12\%. (Again, you must show your work to get credit.)

    \begin{EXAM}\mybigskip\end{EXAM}
    \item (5 points) What is the risk premium associated with the private bond? (Recall that the risk premium is the difference between the expected rate of return of a risky investment and the rate of return of a risk-free investment such as U.S. government bonds.)
    \begin{EXAM}\mybigskip\end{EXAM}
    \end{enumerate}



























\begin{comment}
\item In ``The Tragedy of the Commons," Garrett Hardin contrasts ``appeals to conscience" with ``mutual coercion mutually agreed upon."

    \begin{enumerate}
    \item (5 points) Give one example of an ``appeal to conscience" and one example of ``mutual coercion mutually agreed upon," or otherwise define what these terms means.
    \begin{EXAM}\mybigskip\end{EXAM} \begin{EXAM}\mybigskip\end{EXAM} \begin{EXAM}\mybigskip\end{EXAM}
    \item (5 points) Which approach does Hardin recommend for dealing with population growth or other situations featuring a ``tragedy of the commons"? (Circle one: Appeals to conscience \ \ \ Mutual coercion)
    \begin{EXAM}\mybigskip\end{EXAM}
    \end{enumerate}
\end{comment}







%A similar problem is in qa1uncertainty
\item \begin{EXAM} Imagine that you are taking a multiple-guess exam. There are \emph{six} choices for each question; a correct answer is worth 1 point, and an incorrect answer is worth 0 points. You are on Problem \#23, and it just so happens that the question and possible answers for Problem \#23 are in Hungarian. (When you ask your teacher, she claims that the class learned Hungarian on Tuesday\ldots.)\end{EXAM}

    \begin{enumerate}
    \item \begin{EXAM} (5 points) You missed class on Tuesday, so you don't understand any Hungarian. What is the expected value of guessing randomly on this problem? (Fractions and decimal answers are both fine.)
    \mybigskip\end{EXAM}

\begin{KEY}
\noindent The expected value of guessing randomly is $\frac{1}{6}(1) + \frac{5}{6}(0) = \frac{1}{6}.$
\end{KEY}



    \item \begin{EXAM} (5 points) Now imagine that your teacher wants to discourage random guessing by people like you. To do this, she changes the scoring system, so that a blank answer is worth 0 points and an incorrect answer is worth $x$, e.g., $x=-\frac{1}{2}$. What should $x$ be in order to make random guessing among six answers a fair bet (i.e., one with an expected value of 0)?
    \vspace{2in}\end{EXAM}

\begin{KEY}
If an incorrect answer is worth $x$, the expected value from guessing randomly is $\frac{1}{6}(1) + \frac{5}{6}(x) = \frac{1+5x}{6}.$ If the teacher wants this expected value to equal zero, she must set $x=-\frac{1}{5}.$
\end{KEY}



    \item \begin{EXAM} (5 points) Your teacher ends up choosing $x=-\frac{1}{4}$, i.e., penalizing people one quarter of a point for marking an incorrect answer. How much Hungarian will you need to remember from your childhood in order to make guessing a better-than-fair bet? In other words, how many answers will you need to eliminate so that guessing among the remaining answers yields an expected value strictly greater than 0?
    \clearpage\end{EXAM}

\begin{KEY}
If you can't eliminate any answers, the expected value of guessing randomly is $\frac{1}{6}(1) + \frac{5}{6}\left(-\frac{1}{4}\right) = -\frac{1}{24}.$ If you can eliminate one answer, you have a 1 in 5 chance of getting the right answer if you guess randomly, so your expected value if you can eliminate one answer is $\frac{1}{5}(1) + \frac{4}{5}\left(-\frac{1}{4}\right) = 0.$ If you can eliminate two answers, you have a 1 in 4 chance of getting the right answer if you guess randomly, so your expected value if you can eliminate two answers is $\frac{1}{4}(1) + \frac{3}{4}\left(-\frac{1}{4}\right) = \frac{1}{16}.$ So you need to eliminate at least two answers in order to make random guessing yield an expected value greater than zero.
\end{KEY}

    \end{enumerate}





\item \label{MBA} \begin{EXAM} The Whitman economics department webpage says the following about the benefits of getting an MBA (Masters in Business Administration): ``The typical MBA in the class of 2004 made \$56,499 before earning the MBA degree and expects a post-MBA salary of \$77,147. That's a 35\% increase, and an immediate return on the MBA investment." Let's look at this a little more closely; assume that you can use banks to save or borrow money at an 8\% nominal interest rate. \end{EXAM}

    \begin{enumerate}

    \item \label{MBAannuity} \begin{EXAM} Mr.\ Undergrad graduated from Whitman two years ago. He went straight to work: one year ago he was paid \$56,499, today he was paid another \$56,499, and at the end of every year from now on (i.e., forever) he will be paid \$56,499. Calculate the present value of his income stream. [Hint: split the calculation up into three parts---the amount he was paid last year, the amount he's paid today, and the amount he'll be paid in the future---and add them up at the end. Or, if you're looking for a challenge, think of a more elegant way to do this in two steps instead of four.] \vspace{2in} \end{EXAM}

\begin{KEY}
The present value of \$56,499 one year ago is
\[
(\$56,499)(1.08)=\$61,018.92.
\]
The present value of \$56,499 today is simply that: \$56,499. And the perpetuity formula tells us that the present value of the payments in the future is
\[
\frac{\$56,499}{.08}=\$706,237.50.
\]
Add them together and you get \$823,755.42. The elegant alternative is to put yourself two years in the past, calculate the present value of the forthcoming stream of payments \emph{from that perspective} to be \$706,237.50, and then translate this into today's money using the future value formula:
\[
(\$706,237.50)(1.08)^2 = \$823,755.42.
\]
\end{KEY}


    \item \begin{EXAM} Ms.\ MBA also graduated from Whitman two years ago. She went to business school instead of working, so one year ago she \emph{paid} \$30,000 in tuition and today (graduation day) she paid \emph{another} \$30,000 in tuition. The good news is that at the end of every year from now on (i.e., forever) she will be paid \$77,147. Calculate the present value of her income stream (which includes the tuition payments as well as her salary). [Hint: again, split up the calculation into three parts and then add them up at the end; this time, there is no elegant short-cut.] \clearpage \end{EXAM}

\begin{KEY}
The present value of $-\$30,000$ one year ago is
\[
(-\$30,000)(1.08)=-\$32,400.
\]
The present value of $-\$30,000$ today is simply that: $-\$30,000$. And the perpetuity formula tells us that the present value of the payments in the future is
\[
\frac{\$77,147}{.08}=\$964,337.50.
\]
Add them together and you get \$901,937.50.
\end{KEY}

    \item \label{MBAperpetuity} \begin{EXAM} Of course, the assumption that these individuals live and work forever is an approximation made for the sake of mathematical convenience. So let's figure out how bad of an approximation we get by making that assumption: By how much would the present value of Ms.\ MBA's income stream fall if she was paid \$77,147 at the end of every year for a limited time of 40 years instead of forever? First take a wild guess (which is not worth any points) and then see how well your intuition matches up with the actual answer. [Note: The straightforward approach to this problem is fine, but if you have the time and interest you might hunt for an elegant alternative.] \vspace{2in} \end{EXAM}

\begin{KEY}
The perpetuity formula tells us that the present value of the infinite stream of payments is \$964,337.50. The annuity formula tells us that the present value of 40 years' worth of payments is
\[
\$77,147 \left[ \frac{1 - \displaystyle\frac{1}{(1.08)^{40}}}{.08}\right] \approx \$919,948.14.
\]
Subtracting one from the other gives us a difference of \$44,389.36, which isn't really all that much. The elegant alternative approach, incidentally, is to notice that the difference between payments lasting forever and payments lasting 40 years is payments lasting forever starting at the end of year 40; we have already calculated the present value of these payments \emph{from the perspective of year 40} to be \$964,337.50, so all we have to do is discount this amount back to today by using the lump sum formula:
\[
\frac{\$964,337.50}{(1.08)^{40}}\approx \$44,389.36.
\]
\end{KEY}


    \item  \begin{EXAM} Explain (as if to a non-economist) why it makes sense for \$964,337.50 to be the present value of receiving \$77,147 at the end of every year forever when the interest rate is 8\%. \vspace{2in} \end{EXAM}

\begin{KEY}
Put that amount of money in the bank at 8\% interest and at the end of every year you can ``live off the interest", an amount that equals $(\$964,337.50)(.08)=\$77,147.$
\end{KEY}


    \item \begin{EXAM} How much do you have to put in the bank today to get \$964,337.50 at the end of 40 years? Compare your answer here with that for question~\ref{MBAperpetuity} above. \vspace{2in} \end{EXAM}

\begin{KEY}
This is calculated above to be approximately \$44,389.36.
\end{KEY}


    \item \begin{EXAM} Imagine that Ms.\ MBA actually faces an uncertain future once she gets her MBA: there is a 70\% probability that she will get a business management job paying \$100,000 a year and a 30\% probability that she will fall in love with a non-profit management job paying\ldots somewhat less. How much does the non-profit management job have to pay in order for her to ``expect [in the sense of expected value] a post-MBA salary of \$77,147"? \vspace{2in} \end{EXAM}

\begin{KEY}
We need $x$ such that $(.7)(\$100,000)+(.3)(\$x)=\$77,147$. Solving for $x$ gives $x\approx 23,823.33$.
\end{KEY}

\item \begin{EXAM} \label{MBA2} \emph{Extra credit bonus problem!} Assume that \$56,499 was the amount that Mr.\ Undergrad was paid one year ago, but that thereafter his wage rose (and continues to rise) at 4\%, the rate of inflation. Recalculate the present value from question~(\ref{MBAannuity}). \end{EXAM}

\begin{KEY}
The present value of \$56,499 one year ago is still
\[
(\$56,499)(1.08)=\$61,018.92.
\]
Adjusting today's payment for inflation yields
\[
(\$56,499)(1.04)=\$58,758.96,
\]
and its present value is exactly that: \$58,758.96. The present value of future payments can be calculated using \$58,758.96 and the \emph{real interest rate}, which turns out to be $\frac{1}{26}$, or about 3.846\%. Plugging this into the perpetuity formula gives us $\$1,527,732.96\approx\frac{\$58,758.96}{.03846}.$ (The number to the left of the equal sign is actually the precise number using $\frac{1}{26}$ rather than $.03846$.) Add them together and you get \$1,647,510.84. The elegant alternative is to put yourself two years in the past, when the inflation-adjusted wage was $\frac{\$56,499}{1.04}=\$54,325.96154$; use the \emph{real} interest rate to calculate the present value of the forthcoming stream of payments \emph{from that perspective} to be $\$1,412,475.00\approx\frac{\$54,325.96}{.03846}$; and then translate this into today's money using the future value formula and the \emph{nominal} interest rate: $(\$1,412,475.00)(1.08)^2=\$1,647,510.84.$
\end{KEY}

    \end{enumerate} 