
\documentclass{article}
%%%%%%%%%%%%%%%%%%%%%%%%%%%%%%%%%%%%%%%%%%%%%%%%%%%%%%%%%%%%%%%%%%%%%%%%%%%%%%%%%%%%%%%%%%%%%%%%%%%%%%%%%%%%%%%%%%%%%%%%%%%%%%%%%%%%%%%%%%%%%%%%%%%%%%%%%%%%%%%%%%%%%%%%%%%%%%%%%%%%%%%%%%%%%%%%%%%%%%%%%%%%%%%%%%%%%%%%%%%%%%%%%%%%%%%%%%%%%%%%%%%%%%%%%%%%
%TCIDATA{OutputFilter=LATEX.DLL}
%TCIDATA{Version=5.00.0.2552}
%TCIDATA{<META NAME="SaveForMode" CONTENT="1">}
%TCIDATA{Created=Monday, November 20, 2006 19:29:01}
%TCIDATA{LastRevised=Friday, November 30, 2007 13:36:32}
%TCIDATA{<META NAME="GraphicsSave" CONTENT="32">}
%TCIDATA{<META NAME="DocumentShell" CONTENT="Standard LaTeX\Blank - Standard LaTeX Article">}
%TCIDATA{CSTFile=40 LaTeX article.cst}

\newtheorem{theorem}{Theorem}
\newtheorem{acknowledgement}[theorem]{Acknowledgement}
\newtheorem{algorithm}[theorem]{Algorithm}
\newtheorem{axiom}[theorem]{Axiom}
\newtheorem{case}[theorem]{Case}
\newtheorem{claim}[theorem]{Claim}
\newtheorem{conclusion}[theorem]{Conclusion}
\newtheorem{condition}[theorem]{Condition}
\newtheorem{conjecture}[theorem]{Conjecture}
\newtheorem{corollary}[theorem]{Corollary}
\newtheorem{criterion}[theorem]{Criterion}
\newtheorem{definition}[theorem]{Definition}
\newtheorem{example}[theorem]{Example}
\newtheorem{exercise}[theorem]{Exercise}
\newtheorem{lemma}[theorem]{Lemma}
\newtheorem{notation}[theorem]{Notation}
\newtheorem{problem}[theorem]{Problem}
\newtheorem{proposition}[theorem]{Proposition}
\newtheorem{remark}[theorem]{Remark}
\newtheorem{solution}[theorem]{Solution}
\newtheorem{summary}[theorem]{Summary}
\newenvironment{proof}[1][Proof]{\noindent\textbf{#1.} }{\ \rule{0.5em}{0.5em}}
\input{tcilatex}

\begin{document}


\subsection{EC322; David Reinstein; 2007}

\subsection{\textbf{Asymmetic Information Cournot}\protect\bigskip }

\textbf{Note:}\ Watson's text covers a numeric example of this, more general
game.

\subsection{\textbf{Motivation}:}

Two firms choose quantities at the same time.

One firm does not know whether the other firm is high or low cost, both know
their own costs.

Demand is linear, marginal costs are constant\bigskip 

\textit{Q: How would you model this?}

\begin{eqnarray*}
P(Q) &=&a-Q=a-q_{1}-q_{2} \\
c_{1}(q_{1}) &=&cq_{1} \\
c_{2}(q_{2}) &=&c_{2}\cdot q_{2} \\
\text{\textit{Where}} &\text{: }& \\
c_{2} &=&C_{H}\text{ with probability }\theta \\
c_{2} &=&C_{L}\text{ with probability }1-\theta
\end{eqnarray*}

\bigskip

\textbf{Players: }Firms 1 and 2

\bigskip

\textbf{Actions:\ }Choose a quantity ($q_{1}$ and $q_{2}$)

\bigskip

Actions by types: $1$ chooses $q_{1}$

Player $2$, type $c_{H}$ chooses $q_{2}(c_{H})$

Player $2$, type $c_{L}$ chooses $q_{2}(c_{L})$

\bigskip

\textbf{Payoffs: }%
\[
\pi _{k}=q_{k}\cdot p(Q)=q_{k}\cdot (a-q_{1}-q_{2}-c_{k})\text{ for }k\in
\{1,2\} 
\]

\bigskip

\textbf{States: }Firm 2's marginal cost $c_{2}\in \{C_{H,}C_{L}\}$ where $%
C_{L}<C_{H}$

\bigskip

\textbf{Information structure }

$2$ knows $c_{2}$ 

$1$ assigns $P(c_{2}=C_{H})=\theta $, $P(c_{2}=C_{L})=1-\theta $

\bigskip

\textit{Note}:\ In these games, beliefs always correspond to the correct
(correctly computed)\ probabilities given the information.

\bigskip 

\textbf{Best Responses }

Note:\ These will yield a NE\ strategy profile, a BNE outcome where these
intersect,

Firm 2:

$q_{2}^{\ast }(c_{t})$ solves 
\begin{eqnarray*}
\max_{q_{2}}\text{ }[(a-q_{1}^{\ast }-q_{2})-C_{t}]q_{2}\text{ for }t &\in
&\{H,L\} \\
s.t.\text{ }q_{2} &>&0
\end{eqnarray*}

\bigskip

\textit{Q:\ }\bigskip \textit{Explain intuitively what firm 1 will do as a
`best response' ...}

\bigskip

\textit{A: Firm 1 will set quantity to optimize its expected profit, knowing
that there is some probability it will face a high-cost competitor, and some
probability it will face a low cost competitor. \ The optimal quantity will
be somewhere in between, in this case, in fact the weighted average of the
optimal quantities in each case.}

\bigskip

Firm 1:

$q_{1}^{\ast }$ solves 
\begin{eqnarray*}
&&\max_{q_{1}}\text{ }\left\{ 
\begin{array}{c}
\theta \lbrack a-q_{1}-q_{2}^{\ast }(C_{H})-c]q_{1}\text{ } \\ 
+(1-\theta )[a-q_{1}-q_{2}^{\ast }(C_{L})-c]q_{1}%
\end{array}%
\right\}  \\
s.t.q_{1} &>&0
\end{eqnarray*}

\bigskip

\textit{Q: How to solve these for BR's?}\bigskip

\textit{A: Solve for first order-conditions (set first derivative equal to
zero). \ This is a necessary condition for an interior optimum, though not
sufficient. We want the `choice variable' on the left hand side only, as a
function of the parameters and the other firm's choice.}

\bigskip

FOC's:

\begin{equation}
q_{2}^{\ast }(C_{t})=\frac{a-q_{1}^{\ast }-C_{t}}{2}\text{ for }t=L,H 
\tag{1 \& 2}
\end{equation}

\begin{equation}
q_{1}^{\ast }=\frac{\theta \lbrack a-q_{2}^{\ast }(C_{H})-c]+(1-\theta
)[a-q_{2}^{\ast }(C_{L})-c]}{2}  \tag{3}
\end{equation}

\bigskip

\textbf{Note}\textit{:\ }This assumes the parameters are such that both
quantities are positive. \ Actually, we \textit{do }have to worry about
`corner solutions' here -- the high cost firm may not produce -- but we
ignore this for now.

\bigskip

\textit{Q: How to solve these for BNE quantities?}

\bigskip

\textit{A: We need to solve this system of equations, substituting one
player-type's optimum into the others to find a `fixed point.'}

\bigskip

\textit{Q: What will these BNE quantities be a function of?}

\bigskip

\textit{A:They will be a function of the parameters (}$\theta ,c,C_{H},C_{L}$%
\textit{). \ They will *not* be a function of the other player-type's
quantities!}

\bigskip 

We solve this by substituting equation 3 ($q_{1}^{\ast }$) into equations 1
\& 2, and then substituting each of these ($q_{2}^{\ast }(C_{L})$ and $%
q_{2}^{\ast }(C_{H})$) into equation 3. \ 

\bigskip

This yields:%
\begin{eqnarray*}
q_{2}^{\ast }(C_{H}) &=&\frac{a-2C_{H}+C}{3}+\frac{1-\theta }{6}(C_{H}-C_{L})
\\
q_{2}^{\ast }(C_{L}) &=&\frac{a-2C_{L}+C}{3}-\frac{\theta }{6}(C_{H}-C_{L})
\end{eqnarray*}

\bigskip

\textit{Question: What is interesting about these BNE\ outcomes (outputs)?}

\bigskip

\begin{itemize}
\item The output for the high-cost type is \textit{higher} than the
high-cost type's output in the complete information case (i.e., in the case
where player 1 knew 2's cost). \ Remember, this complete information output
was $q_{i}^{\ast }=\frac{a-2c_{i}+c_{j}}{3}$ where $c_{i}$ was own and $%
c_{j} $ was the other firm's cost.

\item Similarly, the output for the low-cost type is \textit{lower } than
the low-cost type's output in the complete information case.

\item For both types the BNE\ output involves not only the cost of the
player's type, but the cost of the player's other type! \ Note that this was 
\textit{not} true for the BR functions! \ 
\end{itemize}

\bigskip

\textit{Why?}

\bigskip

A: Your optimal output depends not only on your own costs but on your
expectation of the other player's output, and thus on how the other player
will react to what it \textit{thinks} are your costs! \ 

Firm $2$ will produce less when its costs are high than when they are low,
ceterus paribus. \ However, if firm 1 is not sure firm 2's costs are high,
firm $1$ doesn't know how much firm 2 will limit its output and thus 1 will
not increase its output as much as it would if it \textit{knew }firm 2's
costs were high. \ Firm 2 knows this, and thus does not limit its output as
much. \ This second effect -- the response to firm $1$'s lack of
information, is described by the second additive terms in the above...

\bigskip

\textit{Q:\ What did we forget to compute?}

\bigskip

Firm 1's BNE strategy (action):%
\[
q_{1}^{\ast }=\frac{a-2c+\theta C_{H}+(1-\theta )C_{L}}{3} 
\]

\bigskip

Note, this is like a response based on `average' $c_{2}$, i.e., $\theta
C_{H}+(1-\theta )C_{L}$. \ This equivalence does not always hold.

\end{document}
