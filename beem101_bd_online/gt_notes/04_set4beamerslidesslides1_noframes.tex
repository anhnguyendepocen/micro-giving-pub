\begin{document}

\title{EC322, Fourth Set: Dynamic (``Extensive'') Games of Complete and
Perfect Information: Notation, Backwards Induction, Normal Form,
Subgame-perfect Nash Equilibrium)}



    \textbf{EC322, Fourth Set: Dynamic (``Extensive'') Games of Complete and Perfect Information: Notation,
    Backwards Induction, Normal Form, Subgame-perfect Nash Equilibrium)}



Reading

    Osborne: Ch 5; Gibbons: 2.1-2.2, 2.4



\section{Dynamic (Extensive) Games / Extensive form}

\textbf{Extensive form game}
An \textbf{extensive form game }is a description of the sequential structure
of the decision problems by players in strategic situations.

It specifies:

\begin{itemize}
\item
\begin{itemize}
\item the players in the game

\item the order of moves (when each player has to move)

\item what the players' choices are when they move

\item what each player knows when she makes her moves

\item the payoff received by each player for each combination of moves that
could be chosen by the players

\item All these are common knowledge among the players.
\end{itemize}
\end{itemize}




There is \textbf{perfect information in an extensive game} if:

\begin{itemize}
\item
\begin{itemize}
\item All previous moves are observed before the next move is chosen

\item A Player knows Who has moved What before she makes a decision\bigskip
\end{itemize}
\end{itemize}




Osborne:\ Players, sequences, player function, preferences/payoffs

-- Formal notation allows infinite games and formal proofs

\textbf{\textquotedblleft Game tree\textquotedblright }: an
\textquotedblleft arborescence\textquotedblright

Example :\FRAME{ftbphF}{5.3167in}{4.0066in}{0pt}{}{}{Figure}{\special%
{language "Scientific Word";type "GRAPHIC";display "USEDEF";valid_file
"T";width 5.3167in;height 4.0066in;depth 0pt;original-width
4.5438in;original-height 2.1843in;cropleft "0";croptop "1";cropright
"1";cropbottom "0";tempfilename 'J8D62H00.bmp';tempfile-properties "XPR";}}



 \textbf{Example 2}:\ Prisoner's dillemma with sequential moves, perfect
information \FRAME{ftbphF}{5in}{4in}{0in}{}{}{Figure}{\special{language
"Scientific Word";type "GRAPHIC";maintain-aspect-ratio TRUE;display
"USEDEF";valid_file "T";width 5in;height 4in;depth 0in;original-width
4.2798in;original-height 2.6799in;cropleft "0";croptop "1";cropright
"1";cropbottom "0";tempfilename 'J8BMM901.wmf';tempfile-properties "XPR";}}

Now the 2nd player can observe the 1st player's move before acting. \

Note that this means the 2nd player now has 2 decisions to make -- one at
each `information set' (defined later), while the first player still has
only one.


\textbf{\ Decision Nodes}

\begin{itemize}
\item
\begin{itemize}
\item Each node has at least one arrow pointing out from it

\item and at most one arrow pointing into it

\item Not the same as a position of the board in chess
\end{itemize}
\end{itemize}


\textbf{Backward Induction}

To some extent this is the all of finite extensive games

Yields no prediction with indifference,

no prediction with an infinite/indefinite horizon

unclear what it predicts with imperfect information.

\bigskip




\textbf{BWI is similar to an iterated dominance prediction},

where a player only plays actions that are dominant in each subgame

\begin{itemize}
\item
\begin{itemize}
\item But in what sense is does it yield an equilibrium?
\end{itemize}
\end{itemize}



\textbf{Strategies}

Need complete plan of action \textquotedblleft even for histories that, if
the strategy is followed, do not occur\textquotedblright


\textquotedblleft Imagine player wishes to leave explicit instructions to
some agent\textquotedblright\ concerning how to play on his/her behalf.

-- Note: Not the same as commitment!



\textbf{Strategies (Example)}:

\FRAME{ftbphF}{4.983in}{3.3781in}{0pt}{}{}{Figure}{\special{language
"Scientific Word";type "GRAPHIC";display "USEDEF";valid_file "T";width
4.983in;height 3.3781in;depth 0pt;original-width 4.5438in;original-height
2.1843in;cropleft "0";croptop "1";cropright "1";cropbottom "0";tempfilename
'J8D62H00.bmp';tempfile-properties "XPR";}}
In the Entry Game, the incumbent's strategies are \{Accomodate, Fight\}, and
the challenger's strategies are \{In, Out\}.\bigskip



\textbf{Example 2 (sequential PD): }

\FRAME{ftbphF}{3.6671in}{2.9332in}{0pt}{}{}{Figure}{\special{language
"Scientific Word";type "GRAPHIC";maintain-aspect-ratio TRUE;display
"USEDEF";valid_file "T";width 3.6671in;height 2.9332in;depth
0pt;original-width 4.2798in;original-height 2.6799in;cropleft "0";croptop
"1";cropright "1";cropbottom "0";tempfilename
'J8D62H03.wmf';tempfile-properties "XPR";}}

\textit{Strategies:}

Player 1: \{Clam, Rat\}

Player 2:\ \{(Clam,Clam),(Clam,Rat),(Rat,Clam),(Rat,Rat)\}\

\textit{Where, for player 2, the 1st element of a strategy specifies the
action if player 1 clams, and the second element specifies the action if
player 1 rats.}



\textbf{Strategy Profile }

We can think of the players as simultaneously choosing strategies!

Can express in strategic (e.g., matrix) form

-- `Outcome':\ Not the same as a `strategy profile' -- A strategy profile
determines an outcome


\section{Strategic form of extensive game}

\textbf{Strategic form of extensive game}

Specify set of \textbf{strategies} for each player (pure strategies only,
for now)


\begin{quote}
\textit{`To every extensive form game there is a corresponding strategic
form game, where we think of players simultaneously choosing strategies they
will implement. }


\textit{But a given strategic form game can, in general, correspond to
several extensive form games.'\bigskip }
\end{quote}



\textbf{Strategies (Example II)}:

\FRAME{ftbphF}{4.117in}{3.2935in}{0pt}{}{}{Figure}{\special{language
"Scientific Word";type "GRAPHIC";maintain-aspect-ratio TRUE;display
"USEDEF";valid_file "T";width 4.117in;height 3.2935in;depth
0pt;original-width 4.4317in;original-height 2.6243in;cropleft "0";croptop
"1";cropright "1";cropbottom "0";tempfilename
'J8D62H04.bmp';tempfile-properties "XPR";}}

\bigskip

\textit{What are the possible strategies for each player?:}

\bigskip

%Player 2:\ \{(l$_{1},$l$_{2}$),(l$_{1},$r$_{2}$),(r_{1}$,$l$_{2}$),(r$_{1},$r$_{2}$)\}

\bigskip

Player 1:\ \{(L,l),(L,r),(R,l),(R,r)\}

Interpretation of L,l:\ `I intend to choose L, but if I make a mistake and
choose R instead, I will subsequently choose l'




\begin{tabular}{ccccc}
& \textbf{l}$_{1}$\textbf{,l}$_{2}$ & \textbf{l}$_{1}$\textbf{,r}$_{2}$ &
\textbf{r}$_{1}$\textbf{,l}$_{2}$ & \textbf{r}$_{1},$\textbf{r}$_{2}$ \\
\textbf{L,l} & 3,2 & 3,2 &  &  \\
\textbf{L,r} & 3,2 &  &  &  \\
\textbf{R,l} & 1,4 &  &  & 2,5 \\
\textbf{R,r} &  &  &  &
\end{tabular}

\textbf{Exercise}:\ fill in the rest of the payoffs, find NE.\bigskip


\section{NE of extensive game}

\textbf{NE of extensive game}
We can find the Nash equilibrium of an extensive form by constructing an
equivalent normal form game and solving that game.



\textbf{Example (entry game again):}

\FRAME{ftbphF}{3.858in}{2.616in}{0pt}{}{}{Figure}{\special{language
"Scientific Word";type "GRAPHIC";display "USEDEF";valid_file "T";width
3.858in;height 2.616in;depth 0pt;original-width 4.5438in;original-height
2.1843in;cropleft "0";croptop "1";cropright "1";cropbottom "0";tempfilename
'J8D62H00.bmp';tempfile-properties "XPR";}}

\begin{tabular}{ccc}
& \textbf{Accommodate} & \textbf{Fight} \\
\textbf{In} & 2,1 & -3,-1 \\
\textbf{Out} & 0,2 & 0,2%
\end{tabular}

\bigskip

\begin{tabular}{ccc}
& \textbf{Accommodate} & \textbf{Fight} \\
\textbf{In} & \underline{2,1} & -3,-1 \\
\textbf{Out} & 0,2 & \underline{0,2}%
\end{tabular}




Notice that there are two Nash equilibria: (In, Accommodate) and (Out, Fight).

\textit{Which one seems more reasonable? \ How do we interpret }(Out, Fight)?

How do we interpret this NE?

How do they learn of other players' strategies (especially off the
equilibrium path)?

--Can consider a `slightly perturbed steady state' with rare deviations from
equilibrium

$\mathbf{\Longrightarrow }$\textquotedblleft Non-robust\textquotedblright\
steady states

-- \textquotedblleft On those (rare) occasions when the challenger enters,
the subsequent behavior of the incumbent to fight



\textbf{Non-credible threats}

(e.g., grenade game)

For the threat to be credible it has to be a best response (or tie) `ex
post' -- i.e., \textquotedblleft when we get there,\textquotedblright\ i.e.,
in every `subgame.'



\textbf{Example II. (sequential PD):}

\FRAME{ftbphF}{5in}{4in}{0in}{}{}{Figure}{\special{language "Scientific
Word";type "GRAPHIC";maintain-aspect-ratio TRUE;display "USEDEF";valid_file
"T";width 5in;height 4in;depth 0in;original-width 4.1444in;original-height
3.0403in;cropleft "0";croptop "1";cropright "1";cropbottom "0";tempfilename
'J8D62H03.wmf';tempfile-properties "XPR";}}

How to draw the payoff matrix / strategic form? Depict each of the
strategies.



\begin{tabular}{ccccc}
& \textbf{Clam,Clam,} & \textbf{Clam,Rat} & \textbf{Rat,Clam} & \textbf{%
Rat,Rat} \\
\textbf{Clam} &  &  &  &  \\
\textbf{Rat} &  &  &  &
\end{tabular}

Fill in the payoffs.

\begin{tabular}{ccccc}
& \textbf{Clam,Clam,} & \textbf{Clam,Rat} & \textbf{Rat,Clam} & \textbf{%
Rat,Rat} \\
\textbf{Clam} & 2,2 & 2,2 & 0,3 & 0,0 \\
\textbf{Rat} & 3,0 & 0,0 & 3,0 & 1,1%
\end{tabular}



What is/are the NE?

\begin{tabular}{ccccc}
& \textbf{Clam,Clam} & \textbf{Clam,Rat} & \textbf{Rat,Clam} & \textbf{%
Rat,Rat} \\
\textbf{Clam} & 2,2 & \underline{2},2 & 0,\underline{3} & 0,\underline{3} \\
\textbf{Rat} & \underline{3},0 & 0,0 & \underline{3},0 & \underline{1,1}%
\end{tabular}

The NE \textit{profile} is \{Rat, (Rat,Rat)\}

The NE \textit{outcome} is Rat, Rat, as in the static game.

The resulting payoffs are 1,1.



\textbf{Sequential RPS:}

\FRAME{ftbphF}{5.0004in}{2.4417in}{0pt}{}{}{Figure}{\special{language
"Scientific Word";type "GRAPHIC";display "USEDEF";valid_file "T";width
5.0004in;height 2.4417in;depth 0pt;original-width 5.1996in;original-height
2.2399in;cropleft "0";croptop "1";cropright "1";cropbottom "0";tempfilename
'J8C1L510.wmf';tempfile-properties "XPR";}}

What does BWI predict?

For any choice of player 1, player 2 will crush, wrap, or slice it. \ Player
2's best strategy is (paper, scissors, rock).


Thus payoffs will be -1,1 in any case. \

But this leaves 1 with no clear choice -- BWI doesn't yield a clear
prediction here.

Player 1 has 3 strategies that are equally good (bad), as is any combination
of mixed strategies.



 So, we have:%
\begin{eqnarray*}
&&[\{p_{1},p_{2},1-p_{1}-p_{2}\},\{paper|rock,scissors|paper,rock|scissors\}]
\\
\text{for any }p_{1},p_{2}\text{ s.t. }0 &\leq &p_{1}\leq 1,\text{ }0\leq
p_{2}\leq 1,~p_{1}+p_{2}\leq 1
\end{eqnarray*}

\textit{where the first 3 arguments refer to the probability player 1
assigns to rock, paper and scissors, respectively}

are all equally reasonable as predictions (and will all turn out to be NE).
\ You can check this if you like.


\section{Subgame Perfect Nash Equilibrium}

\textbf{Information set }

An \textbf{information set} for a player is a collection of decision nodes
satisfying:


i. `The player has the move at every node in an information set'


ii.`When the play of the game reaches a node in the information set, the
player with the move does not know what node in the information set has (or
has not)\ been reached'


-- The Player can not distinguish between nodes connected by an information set

-- Because he/she doesn't know what decisions were taken previously (by
another player or by "nature")

-- Player can't play different strategy at nodes connected by an information
set


\textbf{Example}: Prisoner's dillemma in extensive form:

\FRAME{ftbphF}{3.6837in}{2.9464in}{0pt}{}{}{Figure}{\special{language
"Scientific Word";type "GRAPHIC";maintain-aspect-ratio TRUE;display
"USEDEF";valid_file "T";width 3.6837in;height 2.9464in;depth
0pt;original-width 4.0556in;original-height 3.1199in;cropleft "0";croptop
"1";cropright "1";cropbottom "0";tempfilename
'J8C1L511.wmf';tempfile-properties "XPR";}}

Note:\ the order here doesn't matter -- we could change the `player
function' and have 2 play `first' and then `1' without consequence.


\textbf{Example: `Rocks/Paper/Scissor' in extensive
form, with variations.}

\textbf{Traditional RPS:\bigskip }

\FRAME{ftbphF}{5.0004in}{2.3246in}{0pt}{}{}{Figure}{\special{language
"Scientific Word";type "GRAPHIC";display "USEDEF";valid_file "T";width
5.0004in;height 2.3246in;depth 0pt;original-width 5.1839in;original-height
2.0714in;cropleft "0";croptop "1";cropright "1";cropbottom "0";tempfilename
'J8C1L512.wmf';tempfile-properties "XPR";}} \textit{Intuition}: what are the
NE?

This is a simultaneous `strictly competitive' game. \ This is an
`anti-coordination' game.

There can be no pure strategy NE.

If player 1 puts $\frac{1}{3}$ probability on each player 2 is neutral
between each action.. \ If player 2 puts $\frac{1}{3}$ probability on each
(of R, P, S) player 1 is neutral between each action. \ Thus this is a NE,
although there may other NE as well (I don't think there are, but it is
worth checking), using the techniques for simultaneous games we learned
previously.


\textbf{RPS where player 1 is very slow at making a fist:}

\FRAME{ftbphF}{5.0004in}{2.4417in}{0pt}{}{}{Figure}{\special{language
"Scientific Word";type "GRAPHIC";display "USEDEF";valid_file "T";width
5.0004in;height 2.4417in;depth 0pt;original-width 5.2079in;original-height
2.0398in;cropleft "0";croptop "1";cropright "1";cropbottom "0";tempfilename
'J8C1L513.wmf';tempfile-properties "XPR";}}

\textit{Intuition}: What will occur?

If player 1 plays rock 2 will play paper and win, yielding payoffs of 0,1.

So player 1 probably won't want to play rock.

But what will player 1 want to play? \ Paper or scissors?

If player 1 plays scissors or paper player 2 will not know which is played,
and might play scissors (`if not rock'), so as to either win or tie (but
never lose).

But knowing this, player 1 might play scissors -- but then 2 would want to
play rock, which would mean 1 would want to play paper, which would mean 2
would want to play scissors, which would mean 1 would want to play scissors
... etc.

There would seem to be no pure strategy prediction here. \ \

But there may be a mixed strategy NE in this `\textit{subgame'.}



\textbf{Perfect Information: }Every information set is a singleton.

\textbf{Imperfect Information: }At least one nonsingleton information set.

\textbf{Proper subgame :}

i. Begins at a decision node $n$ that is a singleton information set (and
not the first decision node)

ii. Includes all the decision and terminal nodes following $n$ in the game
(but no nodes that do not follow $n$)

iii. Does not cut any information sets


\textbf{Subgame Perfect Nash Equilibrium}
\textbf{Subgame Perfect Nash Equilibrium (SPNE) }\textquotedblleft A
strategy profile s* [where] ... in no subgame can any player i do better by
choosing a strategy different from s$_{i}$*, given that every other player j
adheres to s$_{j}$*\textquotedblright

-- I.e., the strategy profile generates a NE in every subgame

`A NE is subgame-perfect if it does not involve a noncredible threat'

For finite horizon perfect information games, the SPNE (where unique) turns
out to yield the same outcome as the BWI outcome.

Every \emph{generic} game has a unique SPNE.


\textbf{Example (entry game again):}

\FRAME{ftbphF}{4.4832in}{3.0394in}{0pt}{}{}{Figure}{\special{language
"Scientific Word";type "GRAPHIC";display "USEDEF";valid_file "T";width
4.4832in;height 3.0394in;depth 0pt;original-width 4.5438in;original-height
2.1843in;cropleft "0";croptop "1";cropright "1";cropbottom "0";tempfilename
'J8D62H00.bmp';tempfile-properties "XPR";}}


\begin{tabular}{ccc}
& \textbf{Accomodate} & \textbf{Fight} \\
\textbf{In} & \underline{2,1} & -3,-1 \\
\textbf{Out} & 0,2 & \underline{0,2}%
\end{tabular}

What are the NE? \ There are two Nash equilibria: (In, Accomodate) and (Out,
Fight).

What is the BWI outcome? \ (Out, Fight).

\bigskip



What are the subgames?

Strictly speaking, there is only one subgame. \ \ This one:\FRAME{ftbphF}{%
2.0581in}{1.6463in}{0pt}{}{}{Figure}{\special{language "Scientific
Word";type "GRAPHIC";maintain-aspect-ratio TRUE;display "USEDEF";valid_file
"T";width 2.0581in;height 1.6463in;depth 0pt;original-width
2.44in;original-height 1.1042in;cropleft "0";croptop "1";cropright
"1";cropbottom "0";tempfilename 'J8C1L515.wmf';tempfile-properties "XPR";}}

In strategic (matrix) form:

\begin{tabular}{cc}
\textbf{Accomodate} & \textbf{Fight} \\
2,1 & -3,-1 \\
0,2 & \underline{0,2}%
\end{tabular}

The entrant has only one rational strategy in this subgame (and thus only
one NE): fight!

This does not form part of the NE profile (In, Accomodate). \

Thus, while there are 2 NE there is only one \textit{subgame perfect }NE
strategy profile: \{Out, Fight\}.



\textit{Subgame perfection }is a \textbf{refinement }of NE: all SPNE\ are
NE, but not all NE\ are SPNE.

SPNE is harder to solve for in games with \textquotedblleft
ties\textquotedblright : -- Example: Fig 172.1 from Osborne (good exercise)

Be able to write the strategic form, and be able to identify NE and SPNE in
such games


\textbf{Sequential RPS again:}
\FRAME{ftbphF}{5.0004in}{2.4417in}{0pt}{}{}{Figure}{\special{language
"Scientific Word";type "GRAPHIC";display "USEDEF";valid_file "T";width
5.0004in;height 2.4417in;depth 0pt;original-width 5.1996in;original-height
2.2399in;cropleft "0";croptop "1";cropright "1";cropbottom "0";tempfilename
'J8C1L616.wmf';tempfile-properties "XPR";}}\bigskip

\begin{tabular}{cccccccc}
& R,R,R & R,R,P & R,R,S & R,P,R & R,P,P & R,P,S & Nuts! There are too many
of these! \\
R &  &  &  &  &  &  &  \\
P &  &  &  &  &  &  &  \\
S &  &  &  &  &  &  &
\end{tabular}%
\bigskip


\textbf{The three-stage example again}:

\FRAME{ftbphF}{4.117in}{3.2935in}{0pt}{}{}{Figure}{\special{language
"Scientific Word";type "GRAPHIC";maintain-aspect-ratio TRUE;display
"USEDEF";valid_file "T";width 4.117in;height 3.2935in;depth
0pt;original-width 4.4317in;original-height 2.6243in;cropleft "0";croptop
"1";cropright "1";cropbottom "0";tempfilename
'J8D62H04.bmp';tempfile-properties "XPR";}}

What is the BWI outcome?


L,r$_{1}$ with payoffs 2,3\bigskip




What are the NE (strategy profiles)?

\begin{tabular}{ccccc}
& \textbf{l}$_{1}$\textbf{,l}$_{2}$ & \textbf{l}$_{1}$\textbf{,r}$_{2}$ &
\textbf{r}$_{1}$\textbf{,l}$_{2}$ & \textbf{r}$_{1},$\textbf{r}$_{2}$ \\
\textbf{L,l} & \underline{3},2 & \underline{3},2 & \underline{2},\underline{3%
} & 2,\underline{3} \\
\textbf{L,r} & \underline{3},2 & \underline{3},2 & \underline{2},\underline{3%
} & 2,\underline{3} \\
\textbf{R,l} & 1,4 & 2,\underline{5} & 1,4 & 2,\underline{5} \\
\textbf{R,r} & 1,\underline{4} & \underline{3},3 & 1,\underline{4} &
\underline{3},3%
\end{tabular}

\{(L,l),(r$_{1},$l$_{2}$)\} \textit{and} \{(L,r),(r$_{1},$l$_{2}$)\}

... both of which generate the outcome L,r$_{1}$ with payoffs 2,3\bigskip as
in BWI.

The only difference in moves is `off the equilibrium path' so it doesn't
matter to the outcome.


 Which of these are \textit{subgame perfect}?

\bigskip

i. What are the subgames? (Hint:\ there are 3 of them).


Here's one:

\FRAME{ftbphF}{2.2167in}{1.7733in}{0pt}{}{}{Figure}{\special{language
"Scientific Word";type "GRAPHIC";maintain-aspect-ratio TRUE;display
"USEDEF";valid_file "T";width 2.2167in;height 1.7733in;depth
0pt;original-width 1.3516in;original-height 0.9921in;cropleft "0";croptop
"1";cropright "1";cropbottom "0";tempfilename
'J8C1L618.wmf';tempfile-properties "XPR";}}

\begin{tabular}{cc}
\textbf{l} & 2,5 \\
\textbf{r} & \underline{3},3%
\end{tabular}




Here's another:

\FRAME{ftbphF}{2.5496in}{2.0398in}{0pt}{}{}{Figure}{\special{language
"Scientific Word";type "GRAPHIC";maintain-aspect-ratio TRUE;display
"USEDEF";valid_file "T";width 2.5496in;height 2.0398in;depth
0pt;original-width 1.7916in;original-height 1.7127in;cropleft "0";croptop
"1";cropright "1";cropbottom "0";tempfilename
'J8C1L619.wmf';tempfile-properties "XPR";}}

\begin{tabular}{ccc}
& \textbf{l}$_{2}$ & \textbf{r}$_{2}$ \\
\textbf{l} & \underline{1},4 & 2,\underline{5} \\
\textbf{r} & \underline{1},\underline{4} & \underline{3},3%
\end{tabular}

\bigskip


 And here's the third:

\FRAME{ftbphF}{1.692in}{1.3533in}{0pt}{}{}{Figure}{\special{language
"Scientific Word";type "GRAPHIC";maintain-aspect-ratio TRUE;display
"USEDEF";valid_file "T";width 1.692in;height 1.3533in;depth
0pt;original-width 1.4645in;original-height 1.0328in;cropleft "0";croptop
"1";cropright "1";cropbottom "0";tempfilename
'J8C1L61A.wmf';tempfile-properties "XPR";}}

\begin{tabular}{cc}
l$_{1}$ & \textbf{r}$_{1}$ \\
3,2 & 2,\underline{3}%
\end{tabular}



ii. Which of these subgames are part of the NE under consideration?

\bigskip

Only the first and second ones.

\bigskip

iii. What are the NE in these subgames:

\{r\} in the first

and \{r,l$_{2}$\} in the second.

\bigskip



iv. Which of our NE includes these NE strategies in each subgame?

Only \{(L,r),(r$_{1},$l$_{2}$)\} \textit{and} \textbf{not }\{(L,l),(r$_{1},$l%
$_{2}$)\} . \

$\mathbf{\Longrightarrow }$So only \{(L,r),(r$_{1},$l$_{2}$)\} is a \textit{%
SP}NE.

\bigskip



\textbf{Why do we care}, since both of these lead to the outcome L,r$_{1}$
with payoffs 2,3?

\begin{itemize}
\item
\begin{itemize}
\item I don't.

\item You do, because it could be on the test.
\end{itemize}
\end{itemize}

\bigskip

You could have come up with this quicker by just looking at the moves `off
the equilibrium path' that you ruled out with BWI. \ I just wanted to do it
the slow, painful way.


\section{\textbf{Non-discrete extensive form games, BWI}}

\textbf{Non-discrete extensive form games, BWI}
Let's say we have a 2-stage game; player 1 chooses $a_{1}\in A_{1}$, then
player 2 observes this and chooses $a_{2}\in A_{2}$.

We use BWI.

We first solve for 2's choice as a function of 1's choice:

\[
\max_{a_{2}\in A_{2}}u_{2}(a_{1},a_{2})
\]

Assuming a \textit{unique} best response we have a reaction \textit{function
}$R_{2}(a_{1})$.

Next, we assume player 1 has also solved this problem, and thus also knows
player 2's BR function. \

Taking this into account, 1 will solve the problem:
\[
\max_{a_{1}\in A_{1}}u_{1}(a_{1},R_{2}(a_{1}))
\]

Assuming a unique solution, we call
\[
(a_{1}^{\ast },R_{2}(a_{1}^{\ast }))=(a_{1}^{\ast },a_{2}^{\ast })
\]

the BWI outcome of this game.



Since it is a finite, perfect information game with a reaction function
(rather than correspondence),

the strategy: $(a_{1}^{\ast },R_{2}(a_{1}))$  that yields the BWI$\ $outcome
($a_{1}^{\ast },a_{2}^{\ast }$) is a SPNE strategy profile.

We know this is subgame perfect:\ $R_{2}(a_{1}^{\ast })$ is the NE each of
the subgame(s) that begin after any of 1's choices, i.e.,  at the second
decision node.


\textbf{Examples: }

-- Ultimatum game

Players: 1 and 2

Terminal Histories: (x,Z) 0

-- Stackleberg game (go to PDF)

-- Entry/location game (Salop model)

-- Wages and unemployment in a unionized firm


\section{SPNE in Games with some imperfect information}

\textbf{SPNE in Games with some imperfect information}
I.e., some previous moves are observed, some are not.


Where a player (B) in the second period does not observe the move of the
player (A) in the first period, this is akin (in theory) to A and B's moves
being simultaneous. \ Thus imperfect information allows for some
simultaneity in an extensive game.

--It is also akin to B moving first (unobserved by the A), followed by A's
move.

Can depict (if discrete) as game tree with connected information sets

-- or with mix of tree and matrices



Before, BWI `did the job,' so we kicked SPNE to the curb.

Now, we need \textit{subgame perfection }again, so let's hope it still
answers our phone calls.

By the way, let's try not to tee it off this time because we will need it
again for infinite/indefinite games, ok?





SPNE: a NE in every subgame

--Solve simultaneous games for NE as usual

-- I.e., where information sets are connected, solve for NE of static
(sub)game before proceeding backwards



Example: \textbf{RPS where player 1 is very slow at making a fist:}

\FRAME{ftbphF}{5.0004in}{2.4417in}{0pt}{}{}{Figure}{\special{language
"Scientific Word";type "GRAPHIC";display "USEDEF";valid_file "T";width
5.0004in;height 2.4417in;depth 0pt;original-width 5.2079in;original-height
2.0398in;cropleft "0";croptop "1";cropright "1";cropbottom "0";tempfilename
'J8C1L00U.wmf';tempfile-properties "XPR";}}

How many information sets does each player have?:

\bigskip

Player 1 has 1, player 2 has 2.\bigskip

What are the strategies?:

\bigskip

Player 1: R,P,S

Player 2: (R,R), (R,P),(R,S),(P,R), (P,P),(P,S),(S,R), (S,P),(S,S)

\textit{Where }(for player 2) the first element denotes the strategy at the
information set to the left,

and the second element denotes the strategy at the information set to the
right.

\textit{Note} a strategy does \textit{not} specify different actions to take
at decision nodes that are in the same information set. \




Thus, the strategic form, in matrix notation, is:

\begin{tabular}{cccccccccc}
& \textbf{R,R} & \textbf{R,P} & \textbf{R,S} & \textbf{P,R} & \textbf{P,P} &
\textbf{P,S} & \textbf{S,R} & \textbf{S,P} & \textbf{S,S} \\
\textbf{R} & 0,0 & 0,0 & 0,0 & -1,1 & -1,1 & -1,1 & 1,-1 & 1,-1 & 1,-1 \\
\textbf{P} & 1,-1 & 0,0 & 1,-1 & 1,-1 & 0,0 & -1,1 & 1,-1 & 0,0 & -1,1 \\
\textbf{S} & -1,1 & 1,-1 & 0,0 & -1,1 & 1,-1 & 0,0 & -1,1 & 1,-1 & 0,0%
\end{tabular}




Finding the best responses :

\begin{tabular}{cccccccccc}
& \textbf{R,R} & \textbf{R,P} & \textbf{R,S} & \textbf{P,R} & \textbf{P,P} &
\textbf{P,S} & \textbf{S,R} & \textbf{S,P} & \textbf{S,S} \\
\textbf{R} & 0,0 & 0,0 & 0,0 & -1,\underline{1} & -1,\underline{1} & -1,%
\underline{1} & \underline{1},-1 & \underline{1},-1 & \underline{1},-1 \\
\textbf{P} & \underline{1},-1 & 0,0 & \underline{1},-1 & \underline{1},-1 &
0,0 & -1,\underline{1} & \underline{1},-1 & 0,0 & -1,\underline{1} \\
\textbf{S} & -1,\underline{1} & \underline{1},-1 & 0,0 & -1,\underline{1} &
\underline{1},-1 & \underline{0},0 & -1,\underline{1} & \underline{1},-1 &
0,0%
\end{tabular}




What are the proper subgames?

This one:

\FRAME{ftbphF}{2.3503in}{1.8796in}{0pt}{}{}{Figure}{\special{language
"Scientific Word";type "GRAPHIC";maintain-aspect-ratio TRUE;display
"USEDEF";valid_file "T";width 2.3503in;height 1.8796in;depth
0pt;original-width 1.7443in;original-height 1.3117in;cropleft "0";croptop
"1";cropright "1";cropbottom "0";tempfilename
'J8C1L61B.wmf';tempfile-properties "XPR";}}

And this one:

\FRAME{ftbphF}{3.408in}{1.3732in}{0pt}{}{}{Figure}{\special{language
"Scientific Word";type "GRAPHIC";display "USEDEF";valid_file "T";width
3.408in;height 1.3732in;depth 0pt;original-width 3.3997in;original-height
1.4155in;cropleft "0";croptop "1";croright "1";cropbottom
"0";tempfilename
'J8C1L71C.wmf';tempfile-properties "XPR";}}




\end{document}


