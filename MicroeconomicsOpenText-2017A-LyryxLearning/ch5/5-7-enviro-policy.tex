\section{Environmental policy and climate change}\label{sec:ch5sec7}

The 2007 recipients for the Nobel Peace Prize were the United Nation's Intergovernmental Panel on Climate Change (IPCC), and Al Gore, former vice president of the United States. The Nobel committee cited the winners ``for their efforts to build up and disseminate greater knowledge about man-made climate change, and to lay the foundations for the measures that are needed to counteract such change.'' While Al Gore is best known for his efforts to bring awareness of climate change to the world, through his book and associated movie (\textit{An Inconvenient Truth}), the IPCC is composed of a large, international group of scientists that has worked for many years in developing a greater understanding of the role of human activity in global warming. Reports on the extent and causes of the externality that we call global warming are now plentiful. The IPCC has produced several reports at this point; a major study was undertaken in the UK under the leadership of former World Bank Chief Economist Sir Nicholas Stern. Countless scientific papers have been published on the subject.

\subsection*{Greenhouse gases}

The emission of \terminology{greenhouse gases} (GHGs) is associated with a wide variety of economic activities such as coal-based power generation, oil-burning motors, wood-burning stoves, etc. The most common GHG is carbon dioxide. The gases, upon emission, circulate in the earth's atmosphere and, if their build-up is excessive, prevent sufficient radiant heat from escaping. The result is a slow warming of the earth's surface and air temperatures. It is envisaged that such temperature increases will, in the long term, increase water temperatures, possibly cause glacial melting, with the result that water levels worldwide may rise. In addition to the possibility of higher water levels (which the IPCC estimates will be about one foot by the end of the 21\textsuperscript{st} century), oceans may become more acidic, weather patterns may change and weather events may become more variable and severe. The changes will be latitude-specific and vary by economy and continent, and ultimately will impact the agricultural production abilities of certain economies.

\begin{DefBox}
\textbf{Greenhouse gases} that accumulate excessively in the earth's atmosphere prevent heat from escaping and lead to \textbf{global warming}.
\end{DefBox}

While most scientific findings and predictions are subject to a degree of uncertainty, there is little disagreement in the scientific community on the very long-term impact of increasing GHGs in the atmosphere. There is some skepticism as to whether the generally higher temperatures experienced in recent decades are completely attributable to anthropogenic activity since the industrial revolution, or whether they also  reflect a natural cycle in the earth's temperature. But scientists agree that a continuance of the recent rate of GHG emissions will ultimately lead to serious climatic problems. And since GHG emissions are strongly correlated with economic growth, the very high rate of economic growth in many large-population economies such as China and India mean that GHGs could accumulate at a faster rate than considered likely in the 1990s.

This is an area where economic, atmospheric and environmental \textit{models} are used to make predictions. We have just one earth and humankind has never witnessed current GHG emission patterns and trends. Consequently the methodology of this science is strongly model based. Scientists attempt to infer something about the relationship between temperature and climate on the one hand and carbon dioxide concentrations in the atmosphere on the other, using historical data. Data values are inferred by examining ice cores and tree rings from eons past. Accordingly, there is a degree of uncertainty regarding the precise impact of GHG concentrations on water levels, temperatures, and extreme weather events.

The consensus is that, in the presence of such uncertainty, a wise strategy would involve controls on the further buildup of gases, unless the cost of such a policy was prohibitive.

\subsection*{GHGs as a common property}

A critical characteristic of GHGs is that they are what we call in economics a `common property': every citizen in the world `owns' them, every citizen has equal access to them, and it matters little where these GHGs originate. Consequently, if economy A reduces its GHG emissions, economy B may simply increase their emissions rather than incur the cost of reducing its emissions also. Hence, economy A's behaviour goes unrewarded. This is the crux of international agreements -- or disagreements. Since GHGs are a common property, in order for A to have the incentive to reduce emissions, it needs to know that B will act correspondingly.

\subsection*{The Kyoto Protocol}

The world's first major response to climate concerns came in the form of the United Nations--sponsored Earth Summit in Rio de Janeiro in 1992. This was followed by the signing of the Kyoto Protocol in 1997, in which a group of countries committed themselves to reducing their GHG emissions relative to their 1990 emissions levels by the year 2012. Canada's Parliament subsequently ratified the Kyoto Protocol, and thereby agreed to meet Canada's target of a 6 percent reduction in GHGs relative to the amount emitted in 1990. 

On a per-capita basis, Canada is one of the world's largest contributors to global warming, even though Canada's percentage of the total is just 2 percent. Many of the world's major economies refrained from signing the Protocol---most notably China, the United States, and India. Canada's emissions in 1990 amounted to approximately 600 giga tonnes (Gt) of carbon dioxide; but by the time we ratified the treaty in 2002, emissions were about 25\% above that level. Hence the signing was somewhat meaningless, in that Canada had virtually a zero possibility of attaining its target.

The target date of 2012 has come and gone; and the leaders of the world economy, at their meeting in Copenhagen failed to come up with a new agreement that would have greater force. In 2012 the Rio+20 summit was held -- in Rio once again, with the objective of devising a means of reducing GHG emissions.

The central challenge in this area is that developed economies are those primarily responsible for the buildup of GHGs in the post industrial revolution era. Developing economies, however, do not accept that the developed economies should be free to continue to emit GHGs at current levels, while the developing economies should be required to limit theirs at a much lower level.

To compound difficulties, there exists strong skepticism in some economies regarding the urgency to implement limits on the growth in emissions.

\subsection*{Canada's GHG emissions}

An excellent summary source of data on Canada's emissions and performance during the period 1990-2010 is available on Environment Canada's web site. See:

\href{http://www.ec.gc.ca/Publications/A91164E0-7CEB-4D61-841C-BEA8BAA223F9/Executive-Summary-2012\_WEB-v3.pdf}{Environment Canada -- National Inventory Report -- GHG sources and sinks in Canada 1990-2010.}

Canada, like many economies, has become more efficient in its use of energy (the main source of GHGs) in recent decades---its \textit{use of energy per unit of total output} has declined steadily. On a \textit{per capita} basis Canada's emissions amounted to 23.5 tonnes in 2005, and dropped to 20.3 by 2010. This improvement in efficiency means that Canada's GDP is now \textit{less energy intensive}. The quest for increased efficiency is endless, if economic growth is to continue at rates that will satisfy the world's citizens and more broadly the impoverished world. The critical challenge is to produce more output while using not just less energy per unit of output, but to use less energy in total

While Canada's energy intensity (GHGs per unit of output) has dropped by a very substantial amount -- 27\% between 1990 and 2010 -- overall emissions increased by almost 20\%. Furthermore, while developed economies have increased their efficiency, it is the \textit{world's} efficiency that is ultimately critical. By outsourcing much our its manufacturing sector to China, Canada and the West have offloaded some of their most GHG-intensive activities. But GHGs are a common property resource.

Canada's GHG emissions also have a regional aspect: the \textit{production} of oil and gas, which has created considerable wealth for all Canadians (and contributed to the appreciation of the Canadian dollar in the last decade), is both energy intensive and concentrated in a limited number of provinces (Alberta, Saskatchewan and more recently Newfoundland and Labrador).

\subsection*{GHG Measurement}

GHG atmospheric concentrations are measured in parts per million (ppm). Current levels in the atmosphere are below 400 ppm, and long-term levels above 500 could lead to serious economic and social disruption. In the immediate pre-industrial revolution era concentrations were in the 250 ppm range. Hence 500 ppm represents the `doubling' factor that is so frequently discussed in the media. 

GHGs are augmented by the annual additions to the stock already in the atmosphere, and at the same time they decay---though very slowly. GHG-reduction strategies that propose an immediate reduction in emissions are more costly than those aimed at a more gradual reduction. For example, a slower investment strategy would permit in-place production and transportation equipment to reach the end of its economic life rather than be scrapped and replaced `prematurely'. Policies that focus upon longer term replacement are therefore less costly. 

While not all economists and policy makers agree on the time scale for attacking the problem, most agree that, the longer major GHG reduction is postponed, the greater the efforts will have to be in the long term---because GHGs will build up more rapidly in the near term. 

A critical question in controlling GHG emissions relates to the cost of their control: how much of annual growth might need to be sacrificed in order to get emissions onto a sustainable path? Again estimates vary. The Stern Review proposed that, with an increase in technological capabilities, a strategy that focuses on the relative near-term implementation of GHG reduction measures might cost ``only'' a few percentage points of the value of world output.  If correct, this may not be an inordinate price to pay for risk avoidance in the longer term.

Nonetheless, such a reduction will require particular economic policies, and specific sectors will be impacted more than others. 

\subsection*{Economic policies for climate change}

There are three main ways in which polluters can be controlled. One involves issuing direct controls; the other two involve incentives---in the form of pollution taxes, or on tradable ``permits'' to pollute.

To see how these different policies operate, consider first Figure~\ref{fig:optimalpollution}. It is a standard diagram in environmental economics, and is somewhat similar to our supply and demand curves. On the horizontal axis is measured the quantity of environmental damage or pollution, and on the vertical axis its dollar value or cost. The upward-sloping damage curve represents the cost to society of each additional unit of pollution or gas, and it is therefore called a \terminology{marginal damage curve}. It is positively sloped to reflect the reality that, at low levels of emissions, the damage of one more unit is less than at higher levels. In terms of our earlier discussion, this means that an increase in GHGs of 10 ppm when concentrations are at 300 ppm may be less damaging than a corresponding increase when concentrations are at 500 ppm.

% Figure 5.7
\begin{FigureBox}{1.25}{1.75}{25em}{The optimal quantity of pollution \label{fig:optimalpollution}}{$Q^*$ represents the optimal amount of pollution. More than this would involve additional social costs because damages exceed abatement costs. Coversely, less than $Q^*$ would require an abatement cost that exceeds the reduction in damage.}
% supply line
\draw [supplycolour,ultra thick,domain=0.4:5,name path=S] plot (\x, {1.5*(0.25*\x-0.1)}) node [black,mynode,above] {Marginal\\damage};
% demand line
\draw [demandcolour,ultra thick,domain=6:0.35,name path=D] plot (\x, {1/\x}) node [mynode,right,black] {Marginal\\abatement cost};
% axes
\draw [thick, -] (0,3) node (yaxis) [mynode1,above] {Pollution\\cost} |- (6,0) node (xaxis) [mynode1,right] {Pollution\\quantity};
% intersection of S and D
\draw [name intersections={of=S and D, by=E}]
	[dotted,thick] (E) -- (xaxis -| E) node [mynode,below] {$Q^{*}$};
\end{FigureBox}

\begin{DefBox}
The \textbf{marginal damage curve} reflects the cost to society of an additional unit of pollution.
\end{DefBox}

The second curve is the abatement curve. It reflects the cost of reducing emissions by one unit, and is therefore called a \terminology{marginal abatement curve}. This curve has a negative slope indicating that, as we reduce the total quantity of pollution produced, the cost of further unit reductions rises. This shape corresponds to reality. For example, halving the emissions of pollutants and gases from automobiles may be achieved by adding a catalytic converter and reducing the amount of lead in gasoline. But reducing those emissions all the way to zero requires the development of major new technologies such as electric cars---an enormously more costly undertaking.

\begin{DefBox}
The \textbf{marginal abatement curve} reflects the cost to society of reducing the quantity of pollution by one unit.
\end{DefBox}

If producers are unconstrained in the amount of pollution they produce, they may produce more than what we will show is the optimal amount -- corresponding to $Q^*$. This amount is optimal in the sense that at levels greater than $Q^*$ the damage exceeds the cost of reducing the emissions. However, reducing emissions by one unit below $Q^*$ would mean incurring a cost per unit reduction that exceeds the benefit of that reduction. Another way of illustrating this is to observe that at a level of pollution above $Q^*$ the cost of reducing it is less than the damage it inflicts, and therefore a net gain accrues to society as a result of the reduction. But to reduce pollution below $Q^*$ would involve an abatement cost greater than the reduction in pollution damage and therefore no net gain to society. This constitutes a first rule in optimal pollution policy.

\textit{An optimal quantity of pollution occurs when the marginal cost of abatement equals the marginal damage.}

A second guiding principle emerges by considering a situation in which some firms are relative `clean' and others are `dirty'. More specifically, a clean firm A may have already invested in new equipment that uses less energy per unit of output produced, or emits fewer pollutants per unit of output. In contrast the dirty firm B uses older dirtier technology. Suppose furthermore that these two firms form a particular sector of the economy and that the government sets a limit on total pollution from this sector, and that this limit is less than what the two firms are currently producing. What is the least costly method to meet the target?

The intuitive answer to this question goes as follows: in order to reduce pollution at least cost to the sector, calculate what it would cost each firm to reduce pollution from its present level. Then implement a system so that the firm with the least cost of reduction is the first to act. In this case the `dirty' firm will likely have a lower cost of abatement since it has not yet upgraded its physical plant. This leads to a second rule in pollution policy:

\textit{With many polluters, the least cost policy to society requires producers with the lowest abatement costs to act first.}

This principle implies that policies which impose the same emission limits on firms may not be the least costly manner of achieving a target level of pollution. Let us now consider the use of \terminology{tradable permits} and \terminology{corrective/carbon taxes} as policy instruments. These are market-based systems aimed at reducing GHGs.

\begin{DefBox}
\textbf{Tradable permits} and \textbf{corrective/carbon taxes} are market-based systems aimed at reducing GHGs.
\end{DefBox}

\subsection*{Incentive mechanism I: tradable permits}

A system of tradable permits is frequently called a `cap and trade' system, because it limits or caps the total permissible emissions, while at the same time allows a market to develop in permits. For illustrative purposes, consider the hypothetical two-firm sector we developed above, composed of firms A and B. Firm A has invested in clean technology, firm B has not. Thus it is less costly for B to reduce emissions than A if further reductions are required. Next suppose that each firm is allocated by the government a specific number of `GHG emission permits'; and that the total of such permits is less than the amount of emissions at present, and that each firm is emitting more than its permits allow. How can these firms achieve the target set for this sector of the economy?

The answer is that they should be able to engage in mutually beneficial trade: If firm B has a lower cost of reducing emissions than A, then it may be in A's interest to pay B to reduce B's emissions heavily. This would free up some of B's emission permits. A in essence is thus buying B's emission permits from B. 

This solution may be efficient from a resource use perspective: having A reduce emissions might involve a heavy investment cost for A. But having B reduce emissions might involve a more modest cost -- one that he can more than afford by selling his emission permits to A.

The largest system of tradable permits currently operates in the European Union. It covers more than 10,000 large energy-using installations. Trading began in 2005. A detailed description of its operation is contained in Wikipedia. California introduced a similar scheme in November 2012.

See: \href{http://en.wikipedia.org/wiki/European\_Union\_Emission\_Trading\_Scheme}{Wikipedia -- European Union Emission Trading Scheme}

\subsection*{Incentive mechanism II: taxes}

Corrective taxes are frequently called \textit{Pigovian} taxes, after the economist Arthur Pigou. He advocated taxing activities that cause negative externalities. These taxes have been examined above in Section~\ref{sec:taxsureff}. Corrective taxes of this type can be implemented as part of a tax package \textit{reform}. For example, taxpayers are frequently reluctant to see governments take `yet more' of their money, in the form of new taxes. Such concerns can be addressed by reducing taxes in other sectors of the economy, in such a way that the package of tax changes maintains a `revenue neutral' impact. 

\subsection*{Policy in practice -- international}

In an ideal world, permits would be traded internationally, and such a system might be of benefit to developing economies: if the cost of reducing pollution is relatively low in developing economies because they have few controls in place, then developed economies, for whom the cost of GHG reduction is high could induce firms in the developing world to undertake cost reductions.  Such a trade would be mutually beneficial. For example, if a developed-economy firm must expend \$30 to reduce GHGs by one tonne, and this can be achieved at a cost of \$10 in the developing economy, then the firm in the developed world could pay up to \$20 to the firm in the developing world to reduce GHGs by one tonne. Both would obviously gain from such an arrangement. This gain arises because of the common property nature of the gases -- it matters not where they originate.

This process is evidently just an extension of the domestic cap-and-trade system described above under `incentive mechanism I' to the international market. The advantage of internationalizing the system is that the differences in the cost of reducing emissions may be very large internationally, and the scope for gains correspondingly larger.

\subsection*{Policy in practice -- domestic large final emitters}

Governments frequently focus upon quantities emitted by individual firms, sometimes because governments are reluctant to introduce carbon taxes or a system of tradable permits. Specifically the focus is upon firms called \textit{large final emitters} (LFEs). Frequently, a relatively small number of producers are responsible for a disproportionate amount of an economy's total pollution, and limits are placed on those firms in the belief that significant economy-wide reductions can be achieved in this manner. A further reason for concentrating on these LFEs is that the monitoring costs are relatively small compared to the costs associated with monitoring \textit{all} firms in the economy. It must be kept in mind that pollution permits may be a legal requirement in some jurisdictions, but monitoring is still required, because firms could choose to risk polluting without owning a permit.

\subsection*{Revenues from taxes and permits}

Taxes and tradable permits differ in that taxes generate revenue for the government from polluting producers, whereas permits may not generate revenue, or may generate less revenue. If the government simply \textit{allocates} permits initially to all polluters, free of charge, and allows a market to develop, such a process generates no revenue to the government. While economists may advocate an \textit{auction} of permits in the start-up phase of a tradable permits market, such a mechanism may run into political objections.

Setting taxes at the appropriate level requires knowledge of the cost and damage functions associated with GHGs.

Despite the monitoring costs and the incomplete information that governments typically have about pollution activities, there exist a number of fruitful tools for reducing pollutants and GHGs. Permits and taxes are market based and are efficient when sufficient information is available. In contrast, direct controls may be fruitful in specific instances. In formulating pollution policy it must be kept in mind that governments rarely have every bit of the information they require; pollution policy is no exception. 