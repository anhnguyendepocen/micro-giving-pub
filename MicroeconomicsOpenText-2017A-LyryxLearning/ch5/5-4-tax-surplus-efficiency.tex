\section{Taxation, surplus and efficiency} \label{sec:taxsureff}

Despite enormous public interest in taxation and its impact on the economy, it is one of the least understood areas of public policy. In this section we will show how an understanding of two fundamental tools of analysis---elasticities and economic surplus---provides powerful insights into the field of taxation.

We begin with the simplest of cases, the federal government's goods and services tax (GST) or the provincial governments' sales taxes (PST). These taxes combined vary by province, but we suppose that a typical rate is 13 percent. Note that this is a \textit{percentage}, or \textit{ad valorem}, tax, not a \textit{specific} tax of so many dollars per unit traded. Figure~\ref{fig:efficiencycosttax} illustrates the supply and demand curves for some commodity. In the absence of taxes, the equilibrium $E_0$ is defined by the combination $(P_0, Q_0)$.

% Figure 5.3
\begin{FigureBox}{1.25}{1.75}{25em}{The efficiency cost of taxation \label{fig:efficiencycosttax}}{The tax shifts $S$ to $S_t$ and reduces the quantity traded from $Q_0$ to $Q_t$. At $Q_t$ the demand value placed on an additional unit exceeds the supply valuation by $E_t$A. Since the tax keeps output at this lower level, the economy cannot take advantage of the additional potential surplus between $Q_t$ and $Q_0$. Excess burden = deadweight loss = A$E_tE_0$.}
% supply lines
\draw [supplycolour,ultra thick,name path=S] (0,0.5) node [mynode,below right,black] {F} -- (4,1.5) node [mynode,right,black] {$S$};
\draw [supplycolour,ultra thick,name path=St] (0,0.75)  -- (4,2.25) node [mynode,right,black] {$S_t$};
% demand line
\draw [demandcolour,ultra thick,domain=0:4,name path=D] (0,2) node [mynode,above right,black] {B} -- (4,0) node [black,mynode,above right,black] {$D$};
% axes
\draw [thick, -] (0,3) node (yaxis) [above] {Price} |- (5,0) node (xaxis) [right] {Quantity};
% intersection of demand and supply lines
\draw [name intersections={of=S and D, by=E0},name intersections={of=St and D, by=Et}]
	[dotted,thick] (yaxis |- E0) node [mynode,left] {$P_0$} -- (E0) node [mynode,above] {$E_0$} -- (xaxis -| E0) node [mynode,below] {$Q_0$}
	[dotted,thick] (yaxis |- Et) node [mynode,left] {$P_t$} -- (Et) node [mynode,above] {$E_t$} -- (xaxis -| Et) node [mynode,below] {$Q_t$};
% path to create dotted line from P_ts to A
\path [name path=PtsA] (xaxis -| Et) -- +(0,3);
% intersection of S with PtsA
\draw [name intersections={of=S and PtsA, by=A}]
	[dotted,thick] (yaxis |- A) node [mynode,left] {$P_{ts}$} -- (A) node [mynode,below right] {A};
% path to create tax wedge
\path [name path=taxwedge] (3.5,0) -- (3.5,3);
% intersection of taxwedge with supply lines
\draw [name intersections={of=S and taxwedge, by=notax},name intersections={of=St and taxwedge, by=withtax}]
	[<->,thick,shorten >=1mm,shorten <=1mm] (notax) -- node [mynode,right,midway] {Tax\\wedge} (withtax);
\end{FigureBox}

A 13-percent tax is now imposed, and the new supply curve $S_t$ lies 13 percent above the no-tax supply $S$. A \terminology{tax wedge} is therefore imposed between the price the consumer must pay and the price that the supplier receives. The new equilibrium is $E_t$, and the new market price is at $P_t$. The price received by the supplier is lower than that paid by the buyer by the amount of the tax wedge. The post-tax supply price is denoted by $P_{ts}$.

There are two \textit{burdens} associated with this tax. The first is the \terminology{revenue burden}, the amount of tax revenue paid by the market participants and received by the government. On each of the $Q_t$ units sold, the government receives the amount $(P_t-P_{ts})$. Therefore, tax revenue is the amount $P_tE_t$A$P_{ts}$. As illustrated in Chapter~\ref{chap:elasticities}, the degree to which the market price $P_t$ rises above the no-tax price $P_0$ depends on the supply and demand elasticities.

\begin{DefBox}
A \textbf{tax wedge} is the difference between the consumer and producer prices.

The \textbf{revenue burden} is the amount of tax revenue raised by a tax.
\end{DefBox}

The second burden of the tax is called the \textit{excess burden}. The concepts of consumer and producer surpluses help us comprehend this. The effect of the tax has been to reduce consumer surplus by $P_tE_tE_0P_0$. This is the reduction in the pre-tax surplus given by the triangle $P_0$B$E_0$. By the same reasoning, supplier surplus is reduced by the amount $P_0E_0$A$P_{ts}$; prior to the tax it was $P_0E_0$F. Consumers and suppliers have therefore seen a reduction in their well-being that is measured by these dollar amounts. Nonetheless, the government has additional revenues amounting to $P_tE_t$A$P_{ts}$, and this tax imposition therefore represents a \textit{transfer} from the consumers and suppliers in the marketplace to the government. Ultimately, the citizens should benefit from this revenue when it is used by the government, and it is therefore not considered to be a net loss of surplus.

However, there remains a part of the surplus loss that is not transferred, the triangular area $E_tE_0$A. This component is called the \terminology{excess burden}, for the reason that it represents the component of the economic surplus that is not transferred to the government in the form of tax revenue. It is also called the \terminology{deadweight loss}, DWL.

\begin{DefBox}
The \textbf{excess burden}, or \textbf{deadweight loss}, of a tax is the component of consumer and producer surpluses forming a net loss to the whole economy.
\end{DefBox}

The intuition behind this concept is not difficult. At the output $Q_t$, the value placed by consumers on the last unit supplied is $P_t$ (=$E_t$), while the production cost of that last unit is $P_{ts}$ (= A). But the potential surplus ($P_t-P_{ts}$) associated with producing an additional unit cannot be realized, because the tax dictates that the production equilibrium is at $Q_t$ rather than any higher output. Thus, if output could be increased from $Q_t$ to $Q_0$, a surplus of value over cost would be realized on every additional unit equal to the vertical distance between the demand and supply functions $D$ and $S$. Therefore, the loss associated with the tax is the area $E_tE_0$A.

In public policy debates, this excess burden is rarely discussed. The reason is that notions of consumer and producer surpluses are not well understood by non-economists, despite the fact that the value of lost surpluses can be very large. Numerous studies have attempted to estimate the excess burden associated with raising an additional dollar from the tax system. They rarely find that the excess burden is less than 25 percent. This is a sobering finding. It tells us that if the government wished to implement a new program by raising additional tax revenue, the benefits of the new program should be 25 percent greater than the amount expended on it!

The impact of taxes and other influences that result in an inefficient use of the economy's resources are frequently called \terminology{distortions}. The examples we have developed in this chapter indicate that distortions can describe either an inefficient output being produced, as in the taxation example, or an inefficient allocation of a given output, as in the case of apartments being allocated by lottery.

\begin{DefBox}
A \textbf{distortion} in resource allocation means that production is not at an efficient output, or a given output is not efficiently allocated.
\end{DefBox}

\subsection*{Elasticities and the excess burden}

We suggested above that elasticities are important in determining the size of the deadweight loss of a tax. Going back to Figure~\ref{fig:efficiencycosttax}, suppose that the demand curve through $E_0$ were more elastic (with the same supply curve, for simplicity). The post-tax equilibrium $E_t$ would now yield a lower $Q_t$ value and a price between $P_t$ and $P_0$. The resulting tax revenue raised and the magnitude of the excess burden would differ because of the new elasticity.

\subsection*{A wage tax}

A final example will illustrate how the concerns of economists over the magnitude of the DWL are distinct from the concerns expressed in much of the public debate over taxes. Figure~\ref{fig:taxationlaboursupply} illustrates the demand and supply for a certain type of labour. On the demand side, the analysis is simplified by assuming that the demand for labour is horizontal, indicating that the gross wage rate is fixed, regardless of the employment level. On the supply side, the upward slope indicates that individuals supply more labour if the wage is higher. The equilibrium $E_0$ reflects that $L_0$ units of labour are supplied at the gross, that is, pre-tax wage $W_0$.

% Figure 5.4
\begin{FigureBox}{0.4}{0.5}{25em}{Taxation and labour supply \label{fig:taxationlaboursupply}}{The demand for labour is horizontal at $W_0$. A tax on labour reduces the wage paid to $W_t$. The loss in supplier surplus is the area $W_0E_0E_tW_t$. The government takes $W_0$B$E_tW_t$ in tax revenue, leaving B$E_0E_t$ as the DWL of the wage tax.}
% supply line
\draw [supplycolour,ultra thick,domain=0.3333:14,name path=S] plot (\x, {0.666*\x+0.333}) node [black,mynode,right] {$S$};
% demand lines
\draw [demandcolour,ultra thick,name path=Dt] (0,5) node [mynode,left,black] {$W_t$} -- (15,5) node [black,mynode,right] {$D_t$};
\draw [demandcolour,ultra thick,name path=D] (0,7) node (W0) [mynode,left,black] {$W_0$} -- (15,7) node [black,mynode,right] {$D$};
% axes
\draw [thick, -] (0,10) node (yaxis) [above] {Wage} |- (15,0) node (xaxis) [right] {Labour};
% intersection of demand and supply line
\draw [name intersections={of=S and D, by=E0},name intersections={of=S and Dt, by=Et}]
	[dotted,thick] (E0) node [mynode,above left] {$E_0$} -- (xaxis -| E0) node [mynode,below] {$L_0$}
	[dotted,thick] (W0 -| Et) node (B) [mynode,above] {B} -- (Et) node [mynode,above left] {$E_t$} -- (xaxis -| Et) node [mynode,below] {$L_t$};
% path to create arrow for wage tax
\path [name path=wagetax] (1,0) -- +(0,10);
% intersection of wagetax line with demand lines
\draw [name intersections={of=wagetax and Dt, by=withtax},name intersections={of=wagetax and D, by=notax}]
	[<->,thick,shorten >=0.5mm,shorten <=0.5mm] (withtax) -- node [mynode,right,midway] {Wage tax} (notax);
% arrow for excess burden=deadweight loss=BE_0E_t
\draw [<-,thick,shorten <=3.5mm] (B) -- +(4,-3) node [mynode,below right] {Excess burden\\= deadweight loss\\= B$E_0E_t$};
\end{FigureBox}

An income tax is now imposed. If this is, say, 20 percent, then the net wage falls to 80 percent of the gross wage in this example, given the horizontal demand curve. The new equilibrium $E_t$ is defined by the combination $(W_t,L_t)$. Less labour is supplied because the net wage is lower. The government generates tax revenue of ($W_0-W_t$) on each of the $L_t$ units of labour now supplied, and this is the area $W_0$B$E_tW_t$. The loss in surplus to the suppliers is $W_0E_0E_tW_t$, and therefore the DWL is the triangle B$E_0E_t$. Clearly the magnitude of the DWL depends upon the supply elasticity.

Whereas the DWL consequence of the wage tax is important for economists, public debate is more often focused on the reduction in labour supply and production. Of course, these two issues are not independent. A larger reduction in labour supply is generally accompanied by a bigger excess burden.