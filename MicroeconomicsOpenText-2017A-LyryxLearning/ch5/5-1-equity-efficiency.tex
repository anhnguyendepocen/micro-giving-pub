\section{Equity and efficiency}\label{sec:ch5sec1}

In modern mixed economies, markets and governments together determine the output produced and also who benefits from that output. In this chapter we explore a very broad question that forms the core of welfare economics: Are markets a good way to allocate scarce resources in view of the fact that they not only give rise to inequality and poverty, but also fail to capture the impacts of productive activity on non-market participants? Mining impacts the environment, traffic results in road fatalities, alcohol and tobacco cause premature deaths and prescription pills are abused. These products all generate secondary impacts beyond their stated objective. We frequently call these external effects. The analysis of markets in this larger sense involves not just positive economics; appropriate policy is additionally a normative issue because policies can impact the various participants in different ways and to different degrees. \terminology{Welfare economics}, therefore, deals with both normative and positive issues.

\begin{DefBox}
\textbf{Welfare economics} assesses how well the economy allocates its scarce resources in accordance with the goals of efficiency and equity.
\end{DefBox}

Political parties on the left and right disagree on how well a market economy works. Canada's New Democratic Party emphasizes the market's failings and the need for government intervention, while the Progressive Conservative Party believes, broadly, that the market fosters choice, incentives, and efficiency. What lies behind this disagreement? The two principal factors are \terminology{efficiency} and \terminology{equity}. Efficiency addresses the question of how well the economy's resources are used and allocated. In contrast, equity deals with how society's goods and rewards are, and should be, distributed among its different members, and how the associated costs should be apportioned.

\begin{DefBox}
\textbf{Equity} deals with how society's goods and rewards are, and should be, distributed among its different members, and how the associated costs should be apportioned.

\textbf{Efficiency} addresses the question of how well the economy's resources are used and allocated.
\end{DefBox}

Equity is also concerned with how different generations share an economy's productive capabilities: more investment today makes for a more productive economy tomorrow, but more greenhouse gases today will reduce environmental quality tomorrow. These are inter-generational questions. 

Climate change caused by global warming forms one of the biggest challenges for humankind at the present time. As we shall see in this chapter, economics has much to say about appropriate policies to combat warming. Whether pollution-abatement policies should be implemented today or twenty years from now involves considerations of equity between generations. Our first task is to develop an analytical tool which will prove vital in assessing and computing welfare benefits and costs -- economic surplus.