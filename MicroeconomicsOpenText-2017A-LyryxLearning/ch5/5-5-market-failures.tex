\section{Market failures -- externalities}\label{sec:ch5sec5}

The consumer and producer surplus concepts we have developed are extremely powerful tools of analysis, but the world is not always quite as straightforward as simple models indicate. For example, many suppliers generate pollutants that adversely affect the health of the population, or damage the environment, or both. The term \terminology{externality} is used to denote such impacts. Externalities impact individuals who are not participants in the market in question, and the effects of the externalities may not be captured in the market price. For example, electricity-generating plants that use coal reduce air quality, which, in turn, adversely impacts individuals who suffer from asthma or other lung ailments. While this is an example of a negative externality, externalities can also be positive.

\begin{DefBox}
An \textbf{externality} is a benefit or cost falling on people other than those involved in the activity's market. It can create a difference between private costs or values and social costs or values.
\end{DefBox}

We will now show why markets characterized by externalities are not efficient, and also show how these externalities might be corrected or reduced. The essence of an externality is that it creates a divergence between private costs/benefits and social costs/benefits. If a steel producer pollutes the air, and the steel buyer pays only the costs incurred by the producer, then the buyer is not paying the full ``social'' cost of the product. The problem is illustrated in Figure~\ref{fig:negextineff}.

% Figure 5.5
\begin{FigureBox}{1.25}{1.75}{25em}{Negative externalities and inefficiency \label{fig:negextineff}}{A negative externality is associated with this good. $S$ measures private costs, whereas $S_f$ measures the full social cost. The socially optimal output is $Q^*$, not the market outcome $Q_0$. Beyond $Q^*$ the real cost exceeds the demand value; therefore $Q_0$ is not an efficient output. A tax that increases $P$ to $P^*$ and reduces output is one solution to the externality.}
% supply lines
\draw [supplycolour,ultra thick,name path=Sp] (0,0.5) node [mynode,left,black] {R}-- (4,1.5) node [black,mynode,right] {$S$ (Private supply cost)};
\draw [supplycolour,ultra thick,name path=Sf] (0,0.75) node [mynode,left,black] {K} -- (4,2.25) node [black,mynode,right] {$S_f$ (Full social supply cost)};
% demand line
\draw [demandcolour,ultra thick,name path=D] (0,2) node [mynode,above right,black] {U} -- (4,0) node [black,mynode,above right] {$D$};
% axes
\draw [thick, -] (0,3) node (yaxis) [above] {Price} |- (5,0) node [mynode1,right] (xaxis) {Quantity of\\electricity};
% intersection of supply and demand
\draw [name intersections={of=Sp and D, by=E0},name intersections={of=Sf and D, by=Estar}]
	[dotted,thick] (yaxis |- E0) node [mynode,left] {$P_0$} -- (E0) node [mynode,above right] {$E_0$} -- (xaxis -| E0) node [mynode,below] {$Q_0$}
	[dotted,thick] (yaxis |- Estar) node [mynode,left] {$P^{*}$} -- (Estar) node [mynode,above] {$E^{*}$};
% path to create dotted line from E_0 to A
\path [name path=E0Aline] (E0) -- +(0,1.5);
% intersection of E0Aline with Sf
\draw [name intersections={of=E0Aline and Sf, by=A}]
	[dotted,thick] (E0) -- (A) node [mynode,above] {A};
% path to create V on Sp line
\path [name path=Vline] (xaxis -| Estar) -- (Estar);
% intersection of Vline with Sp line
\draw [name intersections={of=Sp and Vline, by=V}]
	[dotted,thick] (Estar) -- (V) node [mynode,below right] {V} -- (xaxis -| Estar) node [mynode,below] {$Q^{*}$};
\end{FigureBox}

\subsection*{Negative externalities}

In Figure~\ref{fig:negextineff}, the supply curve $S$ represents the cost to the supplier, whereas $S_f$ (the \textit{full} cost) reflects, in addition, the cost of bad air to the population. Of course, we are assuming that this external cost is ascertainable, in order to be able to characterize $S_f$ accurately. Note also that this illustration assumes that, as power output increases, the external cost \textit{per unit} rises, because the difference between the two supply curves increases with output. This implies that low levels of pollution do less damage: Perhaps the population has a natural tolerance for low levels, but higher levels cannot be tolerated easily and so the cost is greater.

Despite the externality, \textit{an efficient level of production can still be defined}. It is given by $Q^*$, not $Q_0$. To see why, consider the impact of reducing output by one unit from $Q_0$. At $Q_0$ the willingness of buyers to pay for the marginal unit supplied is $E_0$. The (private) supply cost is also $E_0$. But from a societal standpoint there is a pollution/health cost of A$E_0$ associated with that unit of production. The full cost, as represented by $S_f$, exceeds the buyer's valuation. Accordingly, if the last unit of output produced is cut, society gains by the amount A$E_0$, because the cut in output reduces the excess of true cost over value.

Applying this logic to each unit of output between $Q_0$ and $Q^*$, it is evident that society can increase its well-being by the dollar amount equal to the area $E^*$A$E_0$, as a result of reducing production. 

Next, consider the consequences of reducing output further from $Q^*$. Note that pollution is being created here, and environmentalists frequently advocate that pollution should be reduced to zero. However, an efficient outcome may not involve a zero level of pollution! If the production of power were reduced below $Q^*$, the loss in value to buyers, as a result of not being able to purchase the good, would exceed the full cost of its production.

If the government decreed that, instead of producing $Q^*$, no pollution would be tolerated, then society would forgo the possibility of earning the total real surplus equal to the area U$E^*$K. Economists do not advocate such a zero-pollution policy; rather, we advocate a policy that permits a ``tolerable'' pollution level -- one that still results in net benefits to society. In this particular example, the total cost of the tolerated pollution equals the area between the private and full supply functions, K$E^*$VR.

As a matter of policy, how is this market influenced to produce the amount $Q^*$ rather than $Q_0$? One option would be for the government to intervene directly with production quotas for each firm. An alternative would be to impose a \terminology{corrective tax} on the good whose production causes the externality: With an appropriate increase in the price, consumers will demand a reduced quantity. In Figure~\ref{fig:negextineff} a tax equal to the dollar value V$E^*$ would shift the supply curve upward by that amount and result in the quantity $Q^*$ being traded.

\begin{DefBox}
A \textbf{corrective tax} seeks to direct the market towards a more efficient output.
\end{DefBox}

We are now venturing into the field of environmental policy, and this is explored in the following section. The key conclusion of the foregoing analysis is that an efficient working of the market continues to have meaning in the presence of externalities. An efficient output level still maximizes economic surplus where surplus is correctly defined.

\subsection*{Positive externalities}

Externalities of the \textit{positive} kind enable individuals or producers to get a type of `free ride' on the efforts of others. Real world examples abound: When a large segment of the population is inoculated against disease, the remaining individuals benefit on account of the reduced probability of transmission.

A less well recognized example is the benefit derived by many Canadian firms from research and development (R\&D) undertaken in the United States. Professor Dan Treffler of the University of Toronto has documented the positive spillover effects in detail. Canadian firms, and firms in many other economies, learn from the research efforts of U.S. firms that invest heavily in R\&D. In the same vein, universities and research institutes open up new fields of knowledge, with the result that society at large, and sometimes the corporate sector, gain from this enhanced understanding of science, the environment, or social behaviours.

The free market may not cope any better with these positive externalities than it does with negative externalities, and government intervention may be beneficial. For example, firms that invest heavily in research and development would not undertake such investment if competitors could have a complete free ride and appropriate the fruits. This is why \textit{patent laws} exist, as we shall see later in discussing Canada's competition policy. These laws prevent competitors from copying the product development of firms that invest in R\&D. If such protection were not in place, firms would not allocate sufficient resources to R\&D, which is a real engine of economic growth. In essence, the economy's research-directed resources would not be appropriately rewarded, and thus too little research would take place.

While patent protection is one form of corrective action, subsidies are another. We illustrated above that an appropriately formulated tax on a good that creates negative externalities can reduce demand for that good, and thereby reduce pollution. A subsidy can be thought of as a negative tax. Consider the example in Figure~\ref{fig:positiveext}.

% Figure 5.6
\begin{FigureBox}{1.25}{1.75}{25em}{Positive externalities - the market for flu shots \label{fig:positiveext}}{The value to society of vaccinations exceeds the value to individuals: the greater the number of individuals vaccinated, the lower is the probability of others contracting the virus. $D_f$ reflects this additional value. Consequently, the social optimum is $Q^*$ which exceeds $Q_0$.}
% supply line
\draw [supplycolour,ultra thick,domain=0:5,name path=S] plot (\x, {1.5*(0.25*\x+0.5)}) node [black,mynode,above] {$S$};
% demand lines
\draw [demandcolour,ultra thick,domain=0:4,name path=Dp] plot (\x, {-0.5*\x+2}) node [black,mynode,above right] {$D$ (Private value)};
\draw [demandcolour,ultra thick,domain=0:5,name path=Df] plot (\x, {-0.25*\x+2.5}) node [black,mynode,above right] {$D_f$ (Full social value)};
% axes
\draw [thick, -] (0,3) node (yaxis) [above] {Price} |- (6,0) node (xaxis) [mynode1,right] {Quantity};
% intersection of demand and supply
\draw [name intersections={of=S and Dp, by=P0Q0},name intersections={of=S and Df, by=Qstar}]
	[dotted,thick] (yaxis |- P0Q0) node [mynode,left] {$P_0$} -| (xaxis -| P0Q0) node [mynode,below] {$Q_0$}
	[dotted,thick] (Qstar) -- (xaxis -| Qstar) node [mynode,below] {$Q^{*}$};
% path to create PstarQstar point of Dp line
\path [name path=starline] (xaxis -| Qstar) -- (Qstar);
% intersection of starline with Dp
\draw [name intersections={of=starline and Dp, by=PstarQstar}]
	[dotted,thick] (yaxis |- PstarQstar) node [mynode,left] {$P^{*}$} -- (PstarQstar);
\end{FigureBox}

Individuals have a demand for flu shots given by $D$. This reflects their private valuation -- their personal willingness to pay. But the social value of flu shots is greater. When a given number of individuals are inoculated, the probability that others will be infected falls. Additionally, with higher rates of inoculation, the health system will incur fewer costs in treating the infected. Therefore, the value to society of any quantity of flu shots is greater than the sum of the values that individuals place on them. 

Let $D_f$ reflects the full social value of any quantity of flu shots. If $S$ is the supply curve, the socially optimal, efficient, market outcome is $Q^*$. How can we influence the market to move from $Q_0$ to $Q^*$? One solution is a subsidy that would reduce the price from $P_0$ to $P^*$. Rather than shifting the supply curve upwards, as a tax does, the subsidy would shift the supply \textit{downward}, sufficiently to intersect $D$ at the output $Q^*$. In some real world examples, the value of the positive externality is so great that the government may decide to drive the price to zero, and thereby provide the inoculation at a zero price. For example, children typically get their MMR shots (measles, mumps, and rubella) free of charge.