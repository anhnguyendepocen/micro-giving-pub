\begin{FigureBox}{1}{1}{25em}{Gini index and Lorenz curve \label{fig:ginilorenzcurve}}{The more equal are the income shares, the closer is the Lorenz curve to the diagonal line of equality. The Gini index is the ratio of the area A to the area (A+B). The Lorenz curve plots the cumulative percentage of total income against the cumulative percentage of the population.}
\begin{axis}[
	axis line style=thick,
	every tick label/.append style={font=\footnotesize},
	every node near coord/.append style={font=\scriptsize},
	xticklabel=\pgfmathparse{100*\tick}\pgfmathprintnumber{\pgfmathresult}\,\%,
	xticklabel style={anchor=north,/pgf/number format/1000 sep=},
	scaled y ticks=false,
	x=1cm/0.1,
	yticklabel=\pgfmathparse{100*\tick}\pgfmathprintnumber{\pgfmathresult}\,\%,
	yticklabel style={/pgf/number format/fixed,/pgf/number format/1000 sep = \thinspace},
	xmin=0,xmax=1,ymin=0,ymax=1,
	xlabel={Cumulative share of population},
	ylabel={Cumulative share of total income},
]
\addplot[ultra thick,black,mark=none] coordinates { % absolute equality
	(0,0)
	(1,1)
} node [black,mynode,pos=0.6,above left] {Line of\\absolute\\equality};
\addplot[ultra thick,datasetcolourthree,mark=none] coordinates { % Lorenz curve
	(0,0)
	(0.2,0.041)
	(0.4,0.138)
	(0.6,.294)
	(0.8,.531)
	(1,1)
} node [black,mynode,pos=0.6,below right] {Lorenz\\curve};
\addplot[mark=none] coordinates {
	(0.7,0.1)
} node [mynode,above] {B};
\addplot[mark=none] coordinates {
	(0.5,0.3)
} node [mynode,above] {A};
\end{axis}
\end{FigureBox}