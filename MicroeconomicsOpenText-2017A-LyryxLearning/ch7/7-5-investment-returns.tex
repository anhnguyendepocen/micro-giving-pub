\section{Real returns to investment}\label{sec:ch7sec5}

When an individual investor considers investing in a firm, he will examine the returns normally obtained from the available alternatives---cash, government bonds, gold, housing, etc. To understand how he makes a choice, let us focus on just two of these possibilities: stocks and government bonds. A bond results from borrowing. Suppose an individual lends \$100 to the government; this loan is accompanied by a promise from the government to make an annual payment of \$4 and pay back the loan in 5 years. The agreement is essentially a bond - a promise to pay. Bonds are also issued by private corporations. The annual return on this bond is 4\%, because it yields \$4 per \$100 lent. The \$4 payment is also called the `coupon'. This 4\% is the nominal return on the bond. But if the inflation rate in the economy is 1.5\% then the \terminology{real return} is just 2.5\% -- the nominal return minus the inflation rate. These concepts were developed in Chapter~\ref{chap:tmd}.

\begin{DefBox}
\textbf{Real return}: the nominal return minus the rate of inflation.
\end{DefBox}

Since we typically have a pretty good idea of the likely rate of inflation for short horizons, short-term bonds carry very little risk. In contrast to such low-risk bonds, an individual can invest in a risky company. Over the course of a year, the individual may get a dividend from the firm, and may find that his investment has appreciated (or depreciated) to yield a capital gain (or loss). So the return to investing in a corporation stock or share is the sum of the dividend and the capital gain. The real return is this nominal return, adjusted for inflation.

\begin{DefBox}
\textbf{Real return on corporate stock}: the sum of dividend plus capital gain, adjusted for inflation.
\end{DefBox}

In addition to being riskier than government bonds, company shares tend to yield a higher return on average over long time periods. If stocks are more risky individuals would not invest in them unless they were compensated for the higher risk. This long-term higher real return to stocks than bonds is a pattern that economists have observed in virtually all economies for which there are sufficiently long data series available on corporate returns.  

The major risk component of corporate stocks comes in the capital-value variation in the stocks. While dividend changes are usually relatively small, capital gains and losses can be enormous. Start-up firms can go from zero value to high value in a short period of time. Likewise established firms can collapse quickly. Thus, it is very difficult to predict the real return from investing in a company (rather than safe bonds) and so individuals need to be compensated in the form of a higher expected return from such investment if they are to undertake it.

% Application box 7.2
\begin{ApplicationBox}{The value of a financial advisor \label{app:finadvisor}}
The modern economy has thousands of highly-trained financial advisors. The successful ones earn huge salaries. But there is a puzzle: Why do such advisors exist? Can they predict the behaviour of the market any better than an uninformed advisor? Two insights help us answer the question.

\bigskip
First, Burton Malkiel wrote a best seller called A Random Walk down Wall Street. He provided ample evidence that a portfolio chosen on the basis of a monkey throwing darts at a list of stocks would do just as well as the average portfolio constructed by your friendly financial advisor.

\bigskip
Second, there are costs of transacting: An investor who builds a portfolio must devote time to the undertaking, and incur the associated financial trading cost. In recognizing this, investors may choose to invest in what they call mutual funds -- a diversified collection of stocks -- or may choose to employ a financial advisor who will essentially perform the same task of building a diversified portfolio. But, on average, financial advisors cannot beat the market, even though many individual investors would like to believe otherwise.

\bigskip
The principal service supplied by financial advisors is one of management: advisors develop an investment strategy and prevent investors from chasing a dream, or putting all of their eggs in one basket. Sensible strategies rather than picking winners are the key services provided.
\end{ApplicationBox}