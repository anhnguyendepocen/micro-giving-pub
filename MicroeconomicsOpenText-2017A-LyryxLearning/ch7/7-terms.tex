\newpage
\markboth{Key Terms}{Key Terms}
	\addcontentsline{toc}{section}{Key terms}
	\section*{\textsc{Key Terms}}
\begin{keyterms}
\textbf{Sole proprietor} is the single owner of a business and is responsible for all profits and losses.

\textbf{Partnership}: a business owned jointly by two or more individuals, who share in the profits and are jointly responsible for losses.

\textbf{Corporation or company} is an organization with a legal identity separate from its owners that produces and trades.

\textbf{Shareholders} invest in corporations and therefore are the owners. They have limited liability personally if the firm incurs losses.

\textbf{Dividends} are payments made from after-tax profits to company shareholders.

\textbf{Capital gains (losses)} arise from the ownership of a corporation when an individual sells a share at a price higher (lower) than when the share was purchased.

\textbf{Limited liability} means that the liability of the company is limited to the value of the company's assets.

\textbf{Retained earnings} are the profits retained by a company for reinvestment and not distributed as dividends.

\textbf{Principal or owner}: delegates decisions to an agent, or manager.

\textbf{Agent}: usually a manager who works in a corporation and is directed to follow the corporation's interests.

\textbf{Principal-agent problem}: arises when the principal cannot easily monitor the actions of the agent, who therefore may not act in the best interests of the principal.

\textbf{Stock option}: an option to buy the stock of the company at a future date for a fixed, predetermined price.

\textbf{Fair gamble}: one where the gain or loss will be zero if played a large number of times.

\textbf{Risk}: the risk associated with an investment can be measured by the dispersion in possible outcomes. A greater dispersion in outcomes implies more risk.

\textbf{Risk-averse} person will refuse a fair gamble, regardless of the dispersion in outcomes.

\textbf{Risk-neutral} person is interested only in whether the odds yield a profit on average, and ignores the dispersion in possible outcomes.

\textbf{Risk pooling}: a means of reducing risk and increasing utility by aggregating or pooling multiple independent risks.

\textbf{Risk spreading}: spreads the risk of a venture among multiple sub insurers.

\textbf{Real return}: the nominal return minus the rate of inflation.

\textbf{Real return on corporate stock}: the sum of dividend plus capital gain, adjusted for inflation. 

\textbf{Capital market}: a set of financial institutions that funnels financing from investors into bonds and stocks.

\textbf{Portfolio}: a combination of assets that is designed to secure an income from investing and to reduce risk.

\textbf{Diversification} reduces the total risk of a portfolio by pooling risks across several different assets whose individual returns behave independently.

\textbf{Variance} is the weighted sum of the deviations between all possible outcomes and the mean, squared.
\end{keyterms}