\section{Risk and the investor}\label{sec:ch7sec3}

Firms cannot grow without investors. A successful firm's founder always arrives at a point where more investment is required if her enterprise is to expand. Frequently, she will not be able to secure a sufficiently large loan for such growth, and therefore must induce outsiders to buy shares in her firm. She may also realize that expansion carries risk, and she may want others to share in this risk. Risk plays a central role in the life of the firm and the investor. Most investors prefer to avoid risk, but are prepared to assume a limited amount of it if the anticipated rewards are sufficiently attractive. Let us define precisely what we mean by risk in the field of financial analysis.

Consider a gamble with a 50\% chance of making \$10 and a 50\% chance of losing \$10; on average you break even. This is what we call a `fair' gamble. In contrast, a 60\% chance of winning and a 40\% chance of losing, or a 20\% chance of winning plus an 80\% chance of losing are not `fair' gambles, because on average the gambler would not break even if he took the same gamble repeatedly. 

\begin{DefBox}
\textbf{Fair gamble}: one where the gain or loss will be zero if played a large number of times.
\end{DefBox}

Next, let us suppose that the stakes are higher in the fair game. Instead of \$10, a gain or loss of \$20 is at stake, with the probabilities remaining at 50/50. This is still a fair game, but it carries more risk, because the range of possible outcomes is greater. We also say that the dispersion or spread in the possible outcomes is greater. In Section~\ref{sec:diversification} we will introduce a statistical measure of the possible spreads in outcomes. But for the moment let's think about how investors view risk.

Individuals can be classified in accordance with their \textit{attitude towards risk}. A \terminology{risk-averse} person prefers to avoid risk, but may choose to bet or gamble if the odds are sufficiently in their favour, despite their inherent dislike of risk. A \terminology{risk-neutral} person is interested only in whether the odds yield a profit on average, and ignores the dispersion in possible outcomes.

\begin{DefBox}
The \textbf{risk} associated with an investment can be measured by the dispersion in possible outcomes. A greater dispersion in outcomes implies more risk.

A \textbf{risk-averse} person will refuse a fair gamble, regardless of the dispersion in outcomes.

A \textbf{risk-neutral} person is interested only in whether the odds yield a profit on average, and ignores the dispersion in possible outcomes.
\end{DefBox}

For completeness we note that there also exist individuals who are risk lovers -- people who actually prefer more risk to less. While such individuals certainly exist (in the form of gambling addicts), they are relatively rare and are considered frequently to exhibit pathological behaviour. Accordingly, we focus on the much more frequently-encountered risk-averse group.

We observe that individuals tend to buy insurance where significant losses might be incurred. Insurance is the opposite of gambling. Most people who own a house that has even a small probability of burning, or being burgled, purchase insurance. By doing so they are avoiding risk. But how much are they willing to pay for such insurance? If the house is worth \$250,000 and the probability of its burning down is 1 percent, what can we say about someone who is willing to pay more than 1 percent in an insurance premium? If the premium is 1.5 percent, it is not `fair', but a risk-averse person may be willing to pay this amount in order to avoid the risk. What utility behaviour underlies the risk-averse person's decision?

\begin{ApplicationBox}{The `Sub-Prime' mortgage crisis: a principal-agent problem \label{app:subprime}}
With a decline in interest and mortgage rates in the early part of the twenty first century, many individuals believed they could afford to buy a house because the borrowing costs were lower than before. Employees and managers of lending companies believed likewise, and they structured loans in such a way as to provide an incentive to low-income individuals to borrow. These mortgage loans frequently enabled purchasers to buy a house with only a 5\% down payment, in some cases even less, coupled with a repayment schedule that saw low repayments initially but higher repayments subsequently. The initial interest cost was so low in many of these mortgages that it was even lower than the `prime' rate -- the rate banks charge to their most prized customers. 

\bigskip
The crisis that resulted became known as the `sub-prime' mortgage crisis. In many cases lenders got bonuses based on the total value of loans they oversaw, regardless of the quality or risk associated with the loan. The consequence was that they had the incentive to make loans to customers to whom they would not have lent, had these employees and managers been lending their own money, or had they been remunerated differently. The outcomes were disastrous for numerous lending institutions. When interest rates climbed, borrowers could not repay their loans. The construction industry produced a flood of houses that, combined with the sale of houses that buyers could no longer afford, sent housing prices through the floor. This in turn meant that recent house purchasers were left with negative value in their homes -- the value of their property was less than what they paid for it. Many such `owners' simply returned the keys to their bank, declared bankruptcy and walked away. Some lenders went bankrupt; some were bailed out by the government, others bought by surviving firms. This is a perfect example of the principal agent problem -- the managers of the lending institutions did not have the incentive to act in the interest of the owners of those institutions.

\bigskip
The broader consequence of this lending practice was a financial collapse greater than any since the Depression of the nineteen thirties. Assets of the world's commercial and investment banks plummeted in value. Their assets included massive loans and investments both directly and indirectly to the real estate market, and when real estate values fell, so inevitably did the value of the assets based on this sector. Governments around the world had to buy up bad financial assets from financial institutions, or invest massive amounts of taxpayer money in these same institutions. Otherwise the world's financial system might have collapsed, with unknowable consequences.

\bigskip
Taxpayers and shareholders together bore the burden of this disastrous investment policy. Shareholders in many banks saw their shares drop in value to just a few percent of what they had been worth a year or two prior to the collapse. 
\end{ApplicationBox}