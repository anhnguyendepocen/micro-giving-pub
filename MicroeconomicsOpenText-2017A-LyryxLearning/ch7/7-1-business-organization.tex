\section{Business organization}\label{sec:ch7sec1}

Many different types of suppliers provide goods and services to the marketplace. Some are small; some are large. But, whatever their size, suppliers choose an organizational form that is appropriate for their business: Aircraft and Oil rigs are produced by large corporations; dental services are provided by individual professionals or private partnerships.

The initial sections of this chapter are devoted to the study of organizational forms and their operation. We then examine why individuals choose to invest in firms, and propose that such investment provides individual investors with a means both to earning a return on their savings and to managing the risk associated with investing.

Understanding the way firms and capital markets function is crucial to seeing not simply how our microeconomic models operate, but to understanding our economic history and how different forms of social and economic institutions interact. For example, seventeenth century Amsterdam had a thriving bourgeoisie, well-developed financial markets, and investors with savings. This environment facilitated the channeling of investors' funds to firms specializing in trade and nautical conquest. The tiny Dutch state was then the source of the world's leading explorers and traders, and had colonies stretching to Indonesia. The result was economic growth and prosperity.

In contrast, for much of the twentieth century, the Soviet Union dominated a huge territory covering much of Asia and Europe. But capital markets were non-existent, independent firms were stifled, and economic decline ultimately ensued. Much of the enormous difference in the respective patterns of economic development can be explained by the fact that one state fostered firms and capital markets, while the other did not. In terms of our production possibility frontier: One set of institutional arrangements was conducive to expanding the possibilities; the other was not.

Businesses, or firms, have several different forms. At the smallest scale, a business takes the form of a \terminology{sole proprietor} or sole trader who is the exclusive owner. A sole trader gets all of the revenues from the firm and incurs all of the costs. Hence he may make profits or be personally liable for the losses. In the latter case his business or even personal assets may be confiscated to cover debts. Personal bankruptcy may result.

\begin{DefBox}
\textbf{Sole proprietor} is the single owner of a business.
\end{DefBox}

If a business is to grow, \terminology{partners} may be required. Such partners can inject money in exchange for a share of future profits. Firms where trust is involved, such as legal or accounting firms, typically adopt this structure. A firm is given credibility when customers see that partners invest their own wealth in it. 

\begin{DefBox}
\textbf{Partnership}: a business owned jointly by two or more individuals, who share in the profits and are jointly responsible for losses.
\end{DefBox}

If a business is to grow to a significant size it will generally need cash and thus partners. Firms that provide legal services or dental services rely primarily on human expertise, and therefore they need relatively little physical capital. Hence their cash start-up needs tend to be modest. But firms that produce aircraft need vast amounts of money to construct assembly facilities, and undertake research and development. Such businesses form corporations -- also known as companies. 

Large organizations have several inherent advantages over small organizations when a high output level is required. Specialization in particular tasks leads to increased efficiency for production workers. At the same time, non-production workers can perform a multitude of different tasks. If a large corporation decided to contract out every task involved in bringing its product to market, the costs of such agreements could be prohibitively high. In addition, synergies can arise from teamwork. New ideas and better work flow are more likely to materialize when individuals work in close proximity than when working as isolated units, no matter how efficient they may be individually. A key aspect of such large organizations is that \textit{they have a legal identity separate from the managers and owners}.

\begin{DefBox}
\textbf{Corporation or company} is an organization with a legal identity separate from its owners that produces and trades.
\end{DefBox}

The owners of a corporation are known as its \terminology{shareholders}, and their object is usually to make profits. There also exist non- profit corporations whose objective may be philanthropic. Since our focus is upon markets, we will generally assume that profits form the objective of a typical corporation. The profits that accrue to a corporation may be paid to the shareholders in the form of a \terminology{dividend}, or retained in the corporation for future use. When large profits (or losses) accrue the value of the corporation increases (or decreases), and this is reflected in the value of each share of the company. If the value of each share in the company increases (decreases) there is a \terminology{capital gain (loss)} to the owners of the shares -- the shareholders.

\begin{DefBox}
\textbf{Shareholders} invest in corporations and therefore are the owners. They have limited liability personally if the firm incurs losses.

\textbf{Dividends} are payments made from after-tax profits to company shareholders.

\textbf{Capital gains (losses)} arise from the ownership of a corporation when an individual sells a share at a price higher (lower) than when the share was purchased.
\end{DefBox}

A key difference between a company and a partnership is that a company involves \terminology{limited liability}, whereas a partnership does not. Limited liability means that the liability of the company is limited to the value of the company's assets. Shareholders cannot be further liable for any wrongdoing on the part of the company. Accordingly, partnerships and sole traders frequently insure themselves and their operations. For example, all specialist doctors carry malpractice insurance.

\begin{DefBox}
\textbf{Limited liability} means that the liability of the company is limited to the value of the company's assets.
\end{DefBox}

%Table 7.1
\begin{table}[H]
\begin{center}
\begin{tabu} to 30em {|X[5,l]X[0.25,c]X[1,r]|} \hline 
\rowcolor{rowcolour}	Net income (after tax) from operations	&	\$	&	1,978 m 		\\
						Preferred dividends						&	\$	&	73 m 			\\
\rowcolor{rowcolour}	Income available to shareholders		& \$	&	1,905 m 		\\
						Number of common stocks					&	\$	&	502 m 			\\
\rowcolor{rowcolour}	Earnings per common share				&	\$	&	3.79 			\\
						Dividends per common share				&	\$	&	2.80 			\\
\rowcolor{rowcolour}											&			&				\\
						Market capitalization  (June 2008)		&	\$	&	23.7b			\\
\rowcolor{rowcolour}	Stock price  (June 2008)				&	\$	&	48.0    		\\ \hline 
\end{tabu}
\end{center}
\caption{Bank of Montreal income statement, 2008 \label{table:BMO2008}}
\end{table}

The objective of most suppliers or producers is to make profit. This is done by using capital, labour, and human expertise to produce a good, to supply a service, or to act as an intermediary. Corporations are required to produce an annual income statement that accurately describes the operation of the firm. An example is given in Table~\ref{table:BMO2008}.

``Net income from operations'' represents after-tax profits. Of this amount, \$73 million was paid to holders of a special class of shares, and \$1.905 million was available for either reinvesting in the company or payment of dividends. There were 502 million shares outstanding in the company at the end of 2008. This means that if the \$1,905 million were allocated over all such shares, \$3.79 could go to each one. But corporations normally retain a substantial part of such earnings in order to finance investment for coming years. This component is called the \terminology{retained earnings}. The remainder is distributed to the owners/shareholders in the form of dividends. In this case, of the \$3.79 earnings per share, \$2.80 was distributed to shareholders and the remainder was retained inside the company. The final two entries in the table indicate the total value of the company's stock was \$23.7 billion, and each unit of stock was valued at approximately \$48 in June 2008. 

Such information is publicly available for a vast number of corporations at the `finance' section of major search engines such as Google or Yahoo.

\begin{DefBox}
\textbf{Retained earnings} are the profits retained by a company for reinvestment and not distributed as dividends.
\end{DefBox}

In Canada, the corporate sector as a whole tends to hold on to more than half of after-tax profits in the form of retained earnings. However there exists considerable variety in the Behaviour of corporations, and each firm tends to have a known and established pattern of how it allocates it profits between dividends and retained earnings. In the Table~\ref{table:BMO2008} example, three quarters of profits were distributed; yet some corporations have a no-dividend policy. In these latter cases the benefit to investing in a firm must come in the form of capital gain to the owners of the shares.