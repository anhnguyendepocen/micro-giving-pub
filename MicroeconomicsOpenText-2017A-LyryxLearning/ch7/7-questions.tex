\newpage
\section*{Exercises for Chapter~\ref{chap:firminvestorcapital}}

\begin{enumialphparenastyle}

% Solutions file for exercises opened
\Opensolutionfile{solutions}[solutions/ch7ex]

\begin{ex}\label{ex:ch7ex1}
Estimate the average dollar outcome of each of the following games. Each dollar outcome has a probability of one-quarter. Then compute the average utility associated with each game, given the utility values that are associated with each dollar outcome.
\begin{enumerate}
	\item	$\$12,000, U=109.5; \$11,000, U=104.9; \$9,000, U=94.9; \$8,000, U=89.4.$
	\item	$\$32,000, U=178.9; \$31,000, U=176.1; \$29,000, U=170.3; \$28,000, U=167.3.$
	\item	$\$24,000, U=154.9; \$22,000, U=148.3; \$18,000, U=134.2; \$16,000, U=126.5.$
\end{enumerate}
\begin{sol}
	The dollar outcomes for the three games are: \$10,000, \$30,000 and \$20,000. The average utilities are: 99.7, 173.2 and 141.
	
\end{sol}
\end{ex}

\begin{ex}\label{ex:ch7ex2}
You see an advertisement for life insurance for everyone 50 years of age and older. No medical examination is required. If you are a healthy 52-year old, do you think you will get a good deal from this company?
\begin{sol}
	If there is no medical exam then it is probable that less healthy individuals will avail of it. Knowing this, the firm should choose its	benefit/payout structure to reflect a high cost clientele. It will have lower payouts and/or higher premiums. Therefore a healthy individual would likely not obtain favourable insurance terms.
	
\end{sol}
\end{ex}

\begin{ex}\label{ex:ch7ex3}
In which of the following are risks being pooled, and in which would risks likely be spread by insurance companies?
\begin{enumerate}
	\item	Insurance against Alberta's Bow River Valley flooding.
	\item	Life insurance.
	\item	Insurance for the voice of Avril Lavigne or Celine Dion.
	\item	Insuring the voices of the lead vocalists in Metallica, Black Eyed Peas, Incubus, Evanescence, Green Day, and Jurassic Five.
\end{enumerate}
\begin{sol}
\begin{enumerate}
	\item	Spreading.
	\item	Pooled.
	\item	Spreading.
	\item	Spreading and pooling.
\end{enumerate}
\end{sol}
\end{ex}

\begin{ex}\label{ex:ch7ex4}
\begin{enumerate}
	\item	Plot the following three utility functions that relate utility $U$ to wealth $W$, for values of wealth in the range 1\dots 50, using a spreadsheet tool such as Excel: $U_A=2W-0.01W^2$; $U_B=2W$; $U_C=2W+0.02W^2$. 
	\item	State whether each utility function displays risk neutrality, risk aversion or risk love.
	\item	Judging from the shapes of the functions you have plotted, do they display increasing, constant or diminishing marginal utility?
\end{enumerate}
\begin{sol}
\begin{enumerate}
	\item	See below.
	\item	Utility A is risk averse; Utility B is rish neutral; Utility C is risk loving.
	\item	A displays diminishing $MU$, B displays constant $MU$, C displays increasing $MU$.
\end{enumerate}
\begin{center}
	\begin{tikzpicture}[background color=figurebkgdcolour,use background]
	\begin{axis}[
	axis line style=thick,
	every tick label/.append style={font=\footnotesize},
	ymajorgrids,
	grid style={dotted},
	every node near coord/.append style={font=\scriptsize},
	xticklabel style={rotate=90,anchor=east,/pgf/number format/1000 sep=},
	scaled y ticks=false,
	yticklabel style={/pgf/number format/fixed,/pgf/number format/1000 sep = \thinspace},
	xmin=0,xmax=60,ymin=0,ymax=160,
	y=1cm/25,
	x=1cm/8,
	x label style={at={(axis description cs:0.5,-0.05)},anchor=north},
	xlabel={Wealth},
	ylabel={Utility},
	legend pos=north west
	]
	\addplot[datasetcolourone,ultra thick,domain=1:60] {2*x-0.01*x^2};
	\addlegendentry{$U_A$}
	\addplot[datasetcolourtwo,ultra thick,domain=1:60] {2*x};
	\addlegendentry{$U_B$}
	\addplot[datasetcolourthree,ultra thick,domain=1:60] {2*x+0.02*x^2};
	\addlegendentry{$U_C$}
	\end{axis}
	\end{tikzpicture}
\end{center}
\end{sol}
\end{ex}

\begin{ex}\label{ex:ch7ex5}
Use the data in Table~\ref{table:riskyincome} to compute the average utility that each participant gets from his or her income in the different situations under the following assumption about utility: with no pooling of risk, the utilities are either from obtaining \$5000 or from obtaining zero. But, with risk pooling, there is also the possibility of getting the utility associated with an income of \$2500. Suppose that the utility from getting a zero income is zero, the utility from \$2500 is 50, and the utility from \$5000 is 70.7.
\begin{enumerate}
	\item	How much more utility will each individual get, on average, when sharing his or her income with their partner relative to when not sharing?
\end{enumerate}
\begin{sol}
	Alone the average utility is 35.35; pooled the average utility is 42.7.
	
\end{sol}
\end{ex}

\begin{ex}\label{ex:ch7ex6}
In the example developed in Table~\ref{table:investmentstratriskyasset}, suppose there are four identical firms with the same return pattern, and the investor again has \$200 to invest.
\begin{enumerate}
	\item	Compute the portfolio variance associated with a strategy of investing \$50 in each stock. To do this you can use a property of the variance of a portfolio: when the stock prices move independently, the variance of the portfolio of stocks is the sum of the variance of each stock. So compute the variance of each stock, where \$50 is invested in each, and the variance of the portfolio will be four times this amount.
\end{enumerate}
\begin{sol}
	If each asset has a 10\% return with probability 1/2 and zero with probability 1/2, then if \$50 is invested in such an asset the variance is $1/2(55-52.5)^2+1/2(50-52.5)^2=6.25$. With 4 such assets each having the same variance then the variance of the portfolio is 25 when the returns on each asset are independent of the returns on the others.
	
\end{sol}
\end{ex}

\begin{ex}\label{ex:ch7ex7}
A worker has a utility of income function defined by $U=\sqrt{Y}$.
\begin{enumerate}
	\item	Plot this utility function for values of income in the range 1\dots 36. 
	\item	Suppose the individual this time period has $Y=16$, and he has a 50\% chance of seeing his income increase or decrease. If it decreases we know it will fall to \$1. If it increases, by how much would income have to increase to leave his expected utility equal to the level he attains when he gets an income of \$16 with certainty?
\end{enumerate}
\begin{sol}
\begin{enumerate}
	\item	See below.
	\item	If income falls to \$1 then utility from that outcome is the square root of 1. Hence we need to figure out $x$ such that $0.5\times 1+0.5\times\sqrt(x)=4$. It follows that $x=70$. You can check that outcomes of 1 and 70 with equal probability yield an expected utility of 4.
\end{enumerate}
\begin{center}
	\begin{tikzpicture}[background color=figurebkgdcolour,use background]
	\begin{axis}[
	axis line style=thick,
	every tick label/.append style={font=\footnotesize},
	ymajorgrids,
	grid style={dotted},
	every node near coord/.append style={font=\scriptsize},
	xticklabel style={rotate=90,anchor=east,/pgf/number format/1000 sep=},
	scaled y ticks=false,
	yticklabel style={/pgf/number format/fixed,/pgf/number format/1000 sep = \thinspace},
	xmin=0,xmax=40,ymin=0,ymax=7,
	y=0.75cm/0.75,
	x=1.5cm/8,
	x label style={at={(axis description cs:0.5,-0.05)},anchor=north},
	xlabel={Income},
	ylabel={Utility},
	]
	\addplot[datasetcolourone,ultra thick,domain=1:40] {sqrt(x)};
	\end{axis}
	\end{tikzpicture}
\end{center}
\end{sol}
\end{ex}

\begin{ex}\label{ex:ch7ex8}
Once again consider a worker who has a utility function $U=\sqrt{Y}$. In a good week he earns \$25 and in a bad week he earns nothing. Good and bad weeks each have probabilities of 50\%.
\begin{enumerate}
	\item	What is his average or expected utility in numerical terms?
	\item	Suppose the government enters the picture and requires him to contribute to an unemployment insurance scheme. He must pay \$9 every time he has a good week. In return the government pays him \$9 whenever he has a bad week. Compute his average or expected utility with the new insurance scheme in place.
\end{enumerate}
\begin{sol}
\begin{enumerate}
	\item	His expected utility is $0.5\times 0+0.5\times\sqrt{25}=2.5$.
	\item	Expected utility becomes $0.5\sqrt{9}+0.5\times\sqrt{16}=3.5$.
\end{enumerate}
\end{sol}
\end{ex}

\begin{ex}\label{ex:ch7ex9}
Consider the individual in Exercise~\ref{ex:ch7ex8} once again. He earns either zero or \$25 with equal probability in every time period. There is no employment insurance scheme.
\begin{enumerate}
	\item	If this individual thinks about the fact that he may be unemployed in any time period, how much should he save every (good) time period in order to have the maximum average or expected utility over time?
\end{enumerate}
\begin{sol}
	He should smooth his income completely and save 12.5 each good time period.
\end{sol}
\end{ex}

\begin{ex}\label{ex:ch7ex10}
Suppose an individual has utility that increases with the log of his wealth: $U=\ln (W)$, where $W$ is a number that denotes wealth. Compute the value of $U$ for values of $W$ running from 1\dots 20 (not including zero). Then plot the graph of the function and determine if it displays increasing, constant or decreasing marginal utility.
\begin{sol}
	See below. Clearly this displays diminishing marginal utility.
	\begin{center}
	\begin{tikzpicture}[background color=figurebkgdcolour,use background]
		\begin{axis}[
		axis line style=thick,
		every tick label/.append style={font=\footnotesize},
		ymajorgrids,
		grid style={dotted},
		every node near coord/.append style={font=\scriptsize},
		xticklabel style={rotate=90,anchor=east,/pgf/number format/1000 sep=},
		scaled y ticks=false,
		yticklabel style={/pgf/number format/fixed,/pgf/number format/1000 sep = \thinspace},
		xmin=0,xmax=25,ymin=0,ymax=3.5,
		y=1.5cm/0.75,
		x=2.5cm/8,
		x label style={at={(axis description cs:0.5,-0.05)},anchor=north},
		xlabel={Wealth},
		ylabel={Utility},
		]
		\addplot[datasetcolourone,ultra thick,domain=1:25] {ln(x)};
		\end{axis}
	\end{tikzpicture}
	\end{center}
\end{sol}
\end{ex}

% Closes solutions file for this chapter
\Closesolutionfile{solutions}

\end{enumialphparenastyle}