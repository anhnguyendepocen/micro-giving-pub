\section{Government-to-individual transfers} \label{sec:transfers}

Many Canadians take pride in Canada's extensive `social safety net' that aims to protect individuals from misfortune and the reduction of income in old age. Others believe it is too generous. While it is more supportive that the safety net in the US, the Canadian safety net is no more protective than the nets of the developed economies of the European Union. The extent of such support depends in large measure upon the degree to which governments are willing to impose, and individuals are willing to pay, higher or lower tax rates. The major elements of this umbrella of programs are the following.

The \terminology{Canada and Quebec Pension Plans} (C/QPP) are funded from the contributions of workers and their employers. Contributions form 9.9\% of an individual's earnings up to a maximum of approximately \$50,000. The Canada and Quebec components of the plan operate very similarly, but are managed separately. Contributions to the plans from workers and their employers are, for the most part, paid immediately to retired workers. Part of the contributions is invested in a fund.

The objective of the plans is to ensure that some income is saved for retirement. Many individuals are not very good at planning -- they constantly postpone the decision to save, so the state steps in and requires them to save. An individual contributing throughout a full-time working lifecycle can expect an annual pension of about \$12,000. The objective of the plan is to provide a minimum level of retirement income, not an income that will see individuals live in great comfort. Nonetheless, the C/QPP plans have contributed greatly to the reduction of poverty among the elderly since their introduction in the mid-sixties.

The aging of the baby-boom generation -- that very large cohort born in the late forties through to the early sixties -- means that the percentage of the population in the post-65 age group will increase radically, beginning about 2015. To meet this changing demographic, the federal and provincial governments reshaped the plans in the late nineties -- primarily by increasing contributions, in order to build up a fund that will support the aged in the following decades.

\terminology{Old Age Security} (OAS), the \terminology{Guaranteed Income Supplement} (GIS) and the \terminology{Spousal Allowance} (SPA) together form the second support leg for the retired. OAS is a payment made automatically to individuals once they attain the age of 65. The GIS is an additional payment made only to those on very low incomes -- for example, individuals who have little income from their C/QPP or private pension plans. The SPA, which is payable to the spouse or survivor of an OAS recipient, accounts for a small part of the sums disbursed.

The payments for these plans come from the general tax revenues of the federal government. Unlike the C/QPP, the benefits received are not related to the contributions that an individual makes over the working lifecycle. This program has also had a substantial impact on poverty reduction among the elderly.

\terminology{Employment Insurance} (EI) and \terminology{Social Assistance} (SA) are designed to support, respectively, the unemployed and those with no other source of income. Welfare is the common term used to describe SA. Expenditures on EI and SA are strongly cyclical.  At the trough of an economic cycle the real value of expenditures on these programmes greatly exceeds expenditures at the peak of the cycle. Unemployment in Canada rose above 8\% in 2009, and payments to the unemployed and those on welfare reflected this dire state. The strongly cyclical pattern of the cost of these programs reflects the importance of a healthy job market: macro-economic conditions have a major impact on social programme expenditures. 

EI is funded by contributions from employees and their employers. As of 2012 employees pay 1.8\% of their earnings and their employer pays 2.5\%. EI is called an insurance programme, but in reality it is much more than that. Certain groups systematically use the program more than others -- those in seasonal jobs, those in rural areas and those in the Atlantic Provinces, for example. Accordingly, using the terminology of Chapter~\ref{chap:firminvestorcapital}, it is not a universally `fair' insurance programme. Benefits payable to unemployed individuals may also depend on their family size, in addition to their work history. While most payments go in the form of `regular' benefits to unemployed individuals, the EI program also sponsors employee retraining, family benefits that cover maternity and paternity leave, and some other specific target programs for the unemployed. 

Social Assistance is provided to individuals who are in serious need of financial support -- having no income and few assts. Provincial governments administer SA, although the cost of the programme is partly covered by federal transfers through the Canada Social Transfer. The nineteen nineties witnessed a substantial tightening of regulations virtually across the whole of Canada. Access to SA benefits is now more difficult, and benefits have fallen in real terms since the late nineteen eighties.

Welfare dependence peaked in Canada in 1994, when 3.1 million individuals were dependent upon support. As of 2012 the total is a little more than one half of this, on account of more stringent access conditions, reduced benefit levels and an improved job market. Some groups in Canada, particularly the National Council of Welfare, believe that benefits should be higher. Other groups believe that making welfare too generous provides young individuals with the wrong incentives in life, and may lead them to neglect schooling and skill development. 

\terminology{Workers Compensation} supports workers injured on the job. Worker/employer contributions and general tax revenue form the sources of program revenue, and the mix varies from province-to-province. In contrast to the swings in expenditures that characterize SA and EI since the early nineties, expenditures on Worker's Compensation have remained relatively constant.

The major remaining pillar in Canada's social safety net is the \terminology{Canada Child Tax Benefit }(CCTB). In contrast to the other programmes, this is a tax credit, and payments are designed so that they reach families with children who have low to middle incomes. The program has evolved and been enriched over the last two decades, partly with the objective of reducing poverty among households with children, and partly with a view to helping parents receiving SA to transit back to the labour market. The program currently distributes in excess of \$10 billion in benefits.

\begin{ApplicationBox}{Government debts, deficits and transfer \label{app:govdebtdeftran}}
Canada's expenditure and tax policies in the nineteen seventies and eighties led to the accumulation of large government debts, as a result of running fiscal deficits most of those years. By the mid-nineties the combined federal and provincial debt reached 100\% of GDP, with the federal debt accounting for two thirds of this amount. This ratio was perilously high: interest payments absorbed a large fraction of annual revenues, which in turn limited the ability of the government to embark on new programs or enrich existing ones. Canada's debt rating on international financial markets declined.  
In 1995, Finance Minister Paul Martin addressed this problem, and over the following years program spending was pared back. Ultimately, the economy expanded and by the end of the decade the annual deficits at the federal level were eliminated.

\bigskip
As of 2011 the ratio of combined federal and provincial debts stands at just below 70\% of GDP. This ratio would have been lower had Canada not encountered the same economic downturn as the rest of the world in 2008.
\end{ApplicationBox}