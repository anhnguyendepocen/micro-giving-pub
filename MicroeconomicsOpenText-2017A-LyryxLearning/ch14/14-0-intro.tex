\chapter{Government} \label{chap:government}

\begin{topics}
\textbf{In this chapter we will explore:}
\begin{description}
\item [~\ref{sec:ch14sec1}] Market failure and the role of government
\item [~\ref{sec:ch14sec2}] Fiscal federalism in Canada
\item [~\ref{sec:ch14sec3}] Federal-provincial relations and powers
\item [~\ref{sec:transfers}] Redistribution to individuals
\item [~\ref{sec:regulation}] Regulatory activity and competition policy
\end{description}
\end{topics}

Governments have a profound impact on economies. The economies of Scandinavia are very different from those in North America. North and South Korea are night and day, even though they were identical several decades ago. Canada and Argentina were very similar in the early decades of the twentieth century. Both had abundant space, natural resources and migrants. Today Canada is one of the most prosperous economies in the world, while Argentina struggles as a middle-income economy.

Governments, therefore, are not peripheral to the marketplace. They can facilitate or hinder the operation of markets. They can ignore poverty or implement policies to support those on low income and those who are incapacitated. Governments can treat the economy as their fiefdoms, as has been the case for decades in many underdeveloped economies. By assuming the role of warlords, local governors inhibit economic development, because the fruits of investment and labour are subject to capture by the ruling power elite. 

In Canada we take for granted the existence of a generally benign government that serves the economy, rather than one which expects the economy to serve it. The separation of powers, the existence of a constitution, property rights, a police force, and a free press are all crucial ingredients in the mix that ferments economic development and a healthy society.

The analysis of government is worthy not just of a full course, but a full program of study. Accordingly, our objective in this chapter must be limited. We begin by describing the various ways in which markets are inadequate and what government can do to remedy these deficiencies. Next we describe the size and scope of government in Canada, and define the sources of government revenues. On the expenditure side, we emphasize the redistributive and transfer roles that are played by Canadian governments. Tax revenues, particularly at the federal level, go predominantly to transfers rather than the provision of goods and services. Finally we examine how governments seek to control, limit and generally influence the marketplace: How do governments foster the operation of markets in Canada? How do they attempt to limit monopolies and cartels? How do they attempt to encourage the entry of new producers and generally promote a market structure that is conducive to competition, economic growth and consumer well-being?