\section{Federal-provincial fiscal relations}\label{sec:ch14sec3}

The federal government transfers revenue to the provinces using three main programs: Equalization, the Canada Social Transfer and the Canada Health Transfer. Each of these has a different objective. Equalization aims to reduce fiscal disparities among the provinces; The Canada Social Transfer (CST) is for educational and Social Assistance (`welfare') expenditures; The Canada Health Transfer (CHT) performs the same function for health. 

\subsection*{Equalization}

\begin{quote}
\textit{``Parliament and the Government of Canada are committed to the principle of making equalization payments to ensure that provincial governments have sufficient revenues to provide reasonably comparable levels of public service at reasonably comparable levels of taxation''.}
\end{quote}

This statement, from Section 36(2) of the Constitution Act of 1982, defines the purpose of Equalization. Equalization payments are unconditional -- receiving provinces are free to spend the funds on public services according to their own priorities, or even use the revenue to reduce their provincial taxes. Payments are calculated according to a formula that ensures those provinces with revenue-raising ability, or fiscal capacity, below a threshold or `standard' receive payments from the federal government to bring their capacity up to that standard. 

Equalization has gone through very many changes in the several decades of its existence. Its current status reflects the 2006 recommendations of an Expert Panel set up by the federal government. The fiscal capacity of a province is measured by its ability to raise revenues from five major sources:  personal and business income taxes, sales taxes, property taxes, and natural resources. This ability is then compared to the ability of all of the provinces combined to raise revenue; if a difference or shortfall exists, the federal government transfers revenue accordingly, with the amount determined by both the population of the province and the magnitude of its per-person shortfall.  

The program currently transfers about \$14b per annum. The recipiency status of some provinces varies from year to year. Variation in energy prices and energy-based government revenues are the principal cause of this. British Columbia, Alberta, Saskatchewan and Ontario tend to receive little or zero. Manitoba, Quebec and the Atlantic Provinces have been the major recipient provinces. Quebec receives the largest single amount -- more than half the total allocation, on account of its population size. Details are available at the Federal government's department of Finance web site.

\subsection*{The Canada Social Transfer and the Canada Health Transfer}

The CST is a block transfer to provinces in support of post-secondary education, Social Assistance and social services more generally. The CST came into effect in 2004. Prior to that date it was integrated with the health component of federal transfers in a program entitled the Canada Health and Social Transfer (CHST). The objective of the separation was to increase the transparency and accountability of federal support for health while continuing to provide funding for other objectives. The CHT is the other part of the unbundled CHST: it provides funding to the provinces for their health expenditures.

The CST and CHT funding comes in two parts: a cash transfer and tax transfer. A tax transfer essentially provides the same support as a cash transfer of equal value; it just comes in a different form. In 1977 the federal government agreed with provincial and territorial governments to reduce federal personal and corporate tax rates in order to permit the provincial governments to increase the corresponding provincial rates. The net effect was that the federal government got less tax revenue and the provinces got more. And to this day, the federal and provincial governments keep a record of the implied tax transfers that arise from this thirty-five-year-old agreement. This is the tax transfer component of the CST and the CHT. 

The CST support is allocated to provinces and territories on an equal per-capita basis to ensure equal support for all Canadians regardless of their place of residence. The CHT is distributed likewise, and it requires the provinces to abide by the federally-legislated \textit{Canada Health Act}, which demands that provincial health coverage be comprehensive, universal, portable, accessible and publicly administered. 

Approximately \$40b is currently transferred under the CHT and \$20b under the CST. Health care and health expenditures are a core issue of policy at all levels of government on account of the envisaged growth in these expenditures that will inevitably accompany the aging of the baby-boomers. 