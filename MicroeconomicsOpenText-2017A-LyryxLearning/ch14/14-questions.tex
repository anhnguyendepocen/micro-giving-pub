\newpage
\section*{Exercises for Chapter~\ref{chap:government}}

\begin{enumialphparenastyle}

% Solutions file for exercises opened
\Opensolutionfile{solutions}[solutions/ch14ex]

\begin{ex}\label{ex:ch14ex1}
An economy is composed of two individuals, whose demands for a public good -- street lighting -- are given by $P=12-(1/2)Q$ and $P=8-(1/3)Q$.
\begin{enumerate}
	\item	Graph these demands on a diagram.
	\item	Derive the total demand for this public good by summing the demands vertically, and write down a resulting equation for this demand curve.
	\item	Let the marginal cost of providing the good be \$5 per unit. Find the efficient supply of the public good in this economy -- where the marginal cost equals the total value of a marginal unit.
\end{enumerate}
\begin{sol}
\begin{enumerate}
	\item	See figure below.
	\item	The total demand for the public good has a vertical intercept of 20 and a horizontal intercept of 24. The form of the equation is therefore $P=20-(5/6)Q$.
	\item	Equate the $MC$ of \$5 to the total demand curve to obtain $Q=18$. This is the `optimal' output -- where the cost of the last unit produced equals the value placed on it by both individuals. At this quantity the individual valuations (the price that each is willing to pay) are obtained from the individual demand curves. Substituting $Q=18$ into each yields \$3 and \$2.
\end{enumerate}
\begin{center}
\begin{tikzpicture}[background color=figurebkgdcolour,use background,xscale=0.3,yscale=0.3]
	\draw [thick] (0,25) node (yaxis) [mynode1,above] {\$} |- (30,0) node (xaxis) [mynode1,right] {Quantity};
	\draw [ultra thick,demandcolour,name path=TD] (0,20) node [mynode,left,black] {20} -- node [mynode,above right,black,pos=0.2] {Total demand} (24,0) node [mynode,below,black] {24};
	\draw [ultra thick,demandcolour!50,name path=ID1] (0,12) node [mynode,left,black] {12} -- (24,0);
	\draw [ultra thick,demandcolour!50,name path=ID2] (0,8) node [mynode,left,black] {8} -- (24,0);
	\draw [ultra thick,supplycolour,name path=S] (0,5) -- +(30,0);
	\path [name path=arrowline] (7,0) -- +(0,25);
	\draw [name intersections={of=arrowline and ID1, by=i1}]
		[<-,thick,shorten <=1mm] (i1) -- +(10,5) coordinate (IDcoord) node [mynode,right] {Individual demands};
	\draw [name intersections={of=arrowline and ID2, by=i2}]
		[<-,thick,shorten <=1mm] (i2) -- (IDcoord);
\end{tikzpicture}
\end{center}
\end{sol}
\end{ex}

\begin{ex}\label{ex:ch14ex2}
In Exercise~\ref{ex:ch14ex1}, suppose a new citizen joins the economy, and her demand for the public good is given by $P=10-(5/12)Q$.
\begin{enumerate}
	\item	Derive the new demand for the public good on the part of the whole economy and compute the new optimal level of supply, given that the $MC$ remains unchanged.
	\item	Calculate the total value to the consumers of the amount supplied at this efficient output level. 
	\item	Compute the net value to society of that choice -- the total value minus the total cost.
\end{enumerate}
\begin{sol}
\begin{enumerate}
	\item	The demand curve in the economy becomes $P=30-(5/4)Q$. Equating this to the $MC=5$ yields $Q=20$.
	\item	The answer here is the area under the demand curve up to an output of 20 units, which equals \$350.
	\item	The net value to society is \$350 minus the supply cost of \$100=\$250.
\end{enumerate}	
\end{sol}
\end{ex}

\begin{ex}\label{ex:ch14ex3}
An industry that is characterized by a decreasing cost structure has a demand curve given by $P=100-Q$ and the marginal revenue curve by $MR=100-2Q$. The marginal cost is $MC=4$, and average cost is $AC=4+188/Q$.
\begin{enumerate}
	\item	Graph this cost and demand structure.
	\item	Calculate the efficient output and the monopoly output for the industry.
	\item	What price would the monopolist charge if he were unregulated, and what would be his profit per unit?
\end{enumerate}
\begin{sol}
\begin{enumerate}
	\item	The demand, $MR$, and $MC$ all have straightforward shapes. The $ATC$ curve falls from a value of 192 where $Q=1$, to a value of \$4 when $Q$ becomes very large. For example when $Q=4$, $ATC=\$51$; when $Q=94$, $ATC=\$6$; etc. This function curves downwards and approaches a value of \$4 asymptotically.
	\item	The efficient output is where the $MC=P$, as given by the demand curve. Hence equating demand to $MC$ yields $Q=96$. He would maximize profit by producing where $MC=MR$, which occurs at $Q=48$.
	\item	He would choose a price of \$52 from the demand equation at an output of 48 units. At this output the $ATC$ is $(4+188/48)$. Hence profit is $\$(52-(4+188/48))\times 48=\$2,116$.
\end{enumerate}
\end{sol}
\end{ex}

\begin{ex}\label{ex:ch14ex4}
Instead of having a monopoly the government decides to regulate this supplier in the interests of the consumer.
\begin{enumerate}
	\item	What price and output would emerge if the supplier were regulated so that his allowable price equalled average cost? 
	\item	Compute the value of the deadweight loss associated with having an unregulated monopoly relative to having a regulated monopoly where a price is permitted that covers $ATC$.
\end{enumerate}
\begin{sol}
\begin{enumerate}
	\item	Equating the $ATC$ to the demand curve yields $100-Q=4+188/Q$. The solution is $Q=94$.
	\item	The deadweight loss when acting as a monopoly is $0.5\times 48\times 48=\$1,152$. When regulated, the DWL is $0.5\times 2\times 2=\$2$.
\end{enumerate}
\end{sol}
\end{ex}

\begin{ex}\label{ex:ch14ex5}
As an alternative to regulating the supplier such that price covers average total cost, suppose that a two part tariff were used to generate revenue. This scheme involves charging the $MC$ for each unit that is purchased and in addition charging each buyer in the market a fixed cost that is independent of the amount he purchases. If an efficient output is supplied in the market, estimate the total revenue to be obtained from the component covering a price per unit of the good supplied, and the component covering fixed cost.
\begin{sol}
	An efficient output is where $P=MC$, that is $Q=96$. At this output he charges a price of \$4. Hence his loss per unit is the difference between price and $ATC$ which is 188/96. Since he produces 96 units, then charging a price of just \$4 leads to a revenue shortfall of $96\times (188/96)=\$188$. This amount would have to be spread as a charge over the number of buyers in the market as a fixed cost associated with purchasing. In essence each buyer would have to pay a certain entry fee just to purchase the good.
	
\end{sol}
\end{ex}

% Closes solutions file for this chapter
\Closesolutionfile{solutions}

\end{enumialphparenastyle}