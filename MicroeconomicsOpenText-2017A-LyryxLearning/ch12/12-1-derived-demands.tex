\section{Labour -- a derived demand}\label{sec:ch12sec1}

Labour markets differ somewhat from goods and services markets: goods and services are purchased and consumed directly by the buyers. In contrast, labour and capital are used as inputs in producing those goods and services. Prices in product markets reflect the value consumers place upon them and the supply conditions governing them. The wage is the price that equilibrates the supply and demand for a given type of labour; it reflects the value of the product that emerges from the use of capital and labour. The demand for labour (and capital) is thus a \terminology{derived demand} -- the value of labour to the employer derives from the value of the end product in the marketplace for goods and services. It is not a final demand. 

\begin{DefBox}
\textbf{Demand for labour}: a derived demand, reflecting the demand for the output of final goods and services.
\end{DefBox}

We must distinguish between the long run and the short run in our analysis of factor markets. On the \textit{supply side} certain factors of production are fixed in the short run. For example, the supply of radiologists can be increased only over a period of years. While one hospital may be able to attract radiologists from another hospital to meet a shortage, this does not increase the supply in the economy as a whole.  

On the \textit{demand side} there is the conventional difference between the short and long run: in the short run some of a firm's factors of production, such as capital, are fixed and therefore the demand for labour differs from when all factors are variable -- the long run.

\subsection*{Demand in the short run}

Table~\ref{table:srprodlabourdemand} contains information from the example developed in Chapter~\ref{chap:prodcost}. It can be used to illustrate how a firm reacts in the short run to a change in an input price.  Such a response constitutes a demand function -- a schedule relating the quantity demanded for an input to different input prices. The output produced by the various numbers of workers yields a marginal product curve, whose values are stated in column 3. The marginal product of labour, $MP_L$, as developed in Chapter~\ref{chap:prodcost}, is the additional output resulting from one more worker being employed, while holding constant the other (fixed) factors. But what is the \textit{dollar value} to the firm of an additional worker? It is the additional \textit{value of output} resulting from the additional employee -- the price of the output times the worker's marginal contribution to output, his $MP$. We term this the \terminology{value of the marginal product}.

\begin{Table}{26em}{Short-run production and labour demand \label{table:srprodlabourdemand}}{\centering Each unit of labour costs \$1000; output sells at a fixed price of \$70 per unit.}
\begin{tabu} to \linewidth {|X[1,c]X[1,c]X[1,c]X[1.5,c]X[1.5,c]|} \hline 
\rowcolor{rowcolour}	\textbf{Workers} & \textbf{Output} & \textbf{$MP_L$} & \textbf{$VMP_L=MP_L\times P$} & \textbf{Marginal profit = ($VMP_L-$wage) } \\[-0.05em]
\rowcolor{rowcolour}\textbf{(1)}	&	\textbf{(2)}	&	\textbf{(3)}	&	\textbf{(4)}	&	\textbf{(5)}	\\
						0 & 0	&		&		&			\\
\rowcolor{rowcolour}	1 & 15	& 15	& 1050	& 150		\\
						2 & 40	& 25	& 1750	& 750		\\
\rowcolor{rowcolour}	3 & 70	& 30	& 2100	& 1100		\\ 
						4 & 110 & 40	& 2800	& 1800		\\
\rowcolor{rowcolour}	5 & 145 & 35	& 2450	& 1450		\\
						6 & 175 & 30	& 2100	& 1100		\\
\rowcolor{rowcolour}	7 & 200 & 25	& 1750	& 750		\\
						8 & 220 & 20	& 1400	& 400		\\
\rowcolor{rowcolour}	9 & 235 & 15	& 1050	& 50		\\
						10& 240 & 5		& 350	& negative	\\ \hline 
\end{tabu}
\end{Table}

\begin{DefBox}
The \textbf{value of the marginal product} is the marginal product multiplied by the price of the good produced.
\end{DefBox}

In this example the $MP_L$ first rises and then falls. With each unit of output selling for \$70 the value of the marginal product of labour ($VMP_L$) is given in column 4. The first worker produces 15 units each week, and since each unit sells for a price of \$70, then his net value to the firm is \$1,500. A second worker produces 25 units, so his weekly value to the firm is \$1,750, and so forth. If the weekly wage of each worker is \$1,000 then the firm can estimate its marginal profit from hiring each additional worker. This is given in the final column. 

It is profitable to hire more workers as long as the cost of an extra worker is less than the $VMP_L$. The equilibrium amount of labour to employ is therefore 9 units in this example. If the firm were to hire one more worker the contribution of that worker to its profit would be negative, and if it hired one worker less it would forego the opportunity to make an additional profit of \$50 on the 9th unit.

\begin{quote}
\textit{Profit maximizing hiring rule:}
\begin{itemize}
\item \textit{If the $VMP_L$ of next worker $>$ wage, hire more labour.}
\item \textit{If the $VMP_L$ $<$ wage, hire less labour.}
\end{itemize}
\end{quote}

To this point we have determined the profit maximizing amount of labour to employ when the output price and the wage are given. However, a demand function for labour reflects the demand for labour at many different wage rates. Accordingly, suppose the wage rate is \$1,500 per week rather than \$1,000. The optimal amount of labour to employ in this case is determined in exactly the same manner: employ the amount of labour where its contribution is marginally profitable. Clearly the optimal amount to employ is 7 units: the value of the seventh worker to the firm is \$1,750 and the value of the eighth worker is \$1,400. Hence it would not be profitable to employ the eighth, because his marginal contribution to profit would be negative. Following the same procedure we could determine the optimal labour to employ at any wage. It is evident that the $VMP_L$ function is the demand for labour function because it determines the most profitable amount of labour to employ at any wage.

The optimal amount of labour to hire is illustrated in Figure~\ref{fig:demandforlabour}. The wage and $VMP_L$ curves come from Table~\ref{table:srprodlabourdemand}. The $VMP_L$ curve has an upward sloping segment, reflecting increasing productivity, and then a regular downward slope as developed in Chapter~\ref{chap:prodcost}. At employment levels where the $VMP_L$ is greater than the wage additional labour should be employed. But when the $VMP_L$ falls below the wage rate employment should stop. If labour is divisible into very small units, the optimal employment decision is where the $MP_L$ function intersects the wage line.

% Figure 12.1
\begin{FigureBox}{1}{1}{25em}{The demand for labour \label{fig:demandforlabour}}{The optimal hiring decision is defined by the condition that the value of the $MP_L$ is greater than or equal to the wage paid.}
\begin{axis}[
	axis line style=thick,
	every tick label/.append style={font=\footnotesize},
	every node near coord/.append style={font=\scriptsize},
	xticklabel style={anchor=north,/pgf/number format/1000 sep=},
	scaled y ticks=false,
	x=1cm/1,
	yticklabel style={/pgf/number format/fixed,/pgf/number format/1000 sep = \thinspace},
	xmin=0,xmax=11,ymin=0,ymax=3100,
	xlabel={Labour},
	ylabel={Wage (\$)},
]
\addplot[thick,mark=none] coordinates {
	(0,1500)
	(11,1500)
};
\addplot[thick,mark=none] coordinates {
	(0,1000)
	(11,1000)
};
\addplot[dashed,thick,mark=none] coordinates {
	(6,0)
	(6,1500)
};
\addplot[dashed,thick,mark=none] coordinates {
	(8,0)
	(8,1400)
};
\addplot[dashed,thick,mark=none] coordinates {
	(9,0)
	(9,1050)
};
\addplot[ultra thick,vmpcolour,mark=none] coordinates { % when price is $70
	(1,1050)
	(2,1750)
	(3,2100)
	(4,2800)
	(5,2450)
	(6,2100)
	(7,1750)
	(8,1400)
	(9,1050)
	(10,350)
} node [black,mynode,pos=0.5,above right] {$VMP_L$ at $P=\$70$};
\addplot[ultra thick,vmpcolour!50,mark=none] coordinates { % when price is $50
	(1,750)
	(2,1250)
	(3,1500)
	(4,2000)
	(5,1750)
	(6,1500)
	(7,1250)
	(8,1000)
	(9,750)
	(10,250)
} node [black,mynode,pos=0,below right] {$VMP_L$ at $P=\$50$};
\end{axis}
\end{FigureBox}

Figure~\ref{fig:demandforlabour} also illustrates what happens to hiring when the output price changes. Consider a reduction in its price to \$50 from \$70. The profit impact of such a change is negative because the value of each worker's output has declined. Accordingly the demand curve must reflect this by shifting inward, as in the figure. At various wage rates, less labour is now demanded.

In this example the firm is a perfect competitor in the output market -- the price of the good being produced is fixed. Where the firm is not a perfect competitor it faces a declining $MR$ function. In this case the value of the $MP_L$ is the product of $MR$ and $MP_L$ rather than $P$ and $MP_L$. To distinguish the different output markets we use the term \terminology{marginal \textit{revenue} product of labour} ($MRP_L$) when the demand for the output slopes downward. But the optimizing principle remains the same: the firm should calculate the value of each additional unit of labour, and hire up to the point where the additional revenue produced by the worker exceeds or equals the additional cost of that worker.

\begin{DefBox}
The \textbf{marginal revenue product of labour} is the additional revenue generated by hiring one more unit of labour where the marginal revenue declines.
\end{DefBox}
 
\subsection*{Demand in the Long Run}

In Chapter~\ref{chap:prodcost} we proposed that firms choose their factors of production in accordance with cost-minimizing principles. In producing a specific output, firms will choose the least-cost combination of labour and plant size. But how is this choice affected when the price of labour or capital changes? Adjusting to such changes may require a long period of time, because such changes will usually require an adjustment in the optimal amount of capital to use. If one factor becomes more expensive, the firm will likely change the mix of capital and labour away from that factor.

This behaviour is to be seen everywhere, and explains why we have fundamental differences in production techniques in different economies. For example, Canadian farmers face high wages relative to the rental on a harvester, while in South Asia labour is cheap and abundant relative to capital. Consequently harvesting in Canada is capital intensive whereas in Asia it is done with lots of sickle-wielding labour.

In the short run a higher wage increases costs, but the firm is constrained in its choice of inputs by a fixed plant size. In the long run, a wage increase will induce the firm to use relatively more capital than when labour was less expensive in producing a given output. But despite the new choice of inputs, a rise in the cost of any input must increase the total cost of producing any output. This higher cost of production must be reflected in a shift in the supply curve and therefore be reflected in a change in the price of the final good being produced.

Thus a change in the price of any factor has two impacts on firms: in the first place they will \textit{substitute} away from the factor whose price increases; second, since the cost structure for the whole industry increases, the supply curve in the market for the good must decline -- less will be supplied at any output price. With a downward sloping demand, this shift in supply must increase the price of the good and reduce the amount sold. This second effect can be called an \textit{output effect}. 

To conclude: the adjustment responses of firms to changes in input prices are twofold: Firms adjust their combinations of labour and capital in accordance with the prices of each.  But increases in input prices result in higher prices for final goods, and these higher prices reduce the demand in the marketplace for the good being produced. Such changes in turn reduce the demand for inputs.

\subsection*{Monopsony}

Some firms may have to pay a higher wage in order to employ more workers. Think of Hydro Quebec building a dam in Northern Quebec. Not every hydraulic engineer would be equally happy working there as in Montreal. Some engineers may demand only a small wage premium to work in the North, but others will demand a high premium. If so, Hydro Quebec must pay a higher wage to attract more workers - it faces an upward sloping supply of labour curve. Hydro Quebec is the sole buyer in this particular market and is called a \terminology{monopsonist} -- a single buyer. But our general optimizing principle still holds, even if we have different names for the various functions: hire any factor of production up to the point where the cost of an additional unit equals the value generated for the firm by that extra worker. The essential difference here is that when a firm faces an upward sloping labour supply it will have to pay more to attract additional workers and \textit{also pay more to its existing workers}. This will impact the firm's willingness to hire additional workers. 

\begin{DefBox}
A \textbf{monopsonist} is the sole buyer of a good or service and faces an upward-sloping supply curve.
\end{DefBox}