\section{Capital services}\label{sec:ch12sec6}

\subsection*{Demand}

The analysis of the demand for the services of capital parallels closely that of labour demand: the rental rate for capital replaces the wage rate and capital services replace the hours of labour. It is important to keep in mind the distinction we drew above between capital services on the one hand and the amount of capital on the other. Capital services are produced by capital assets, just as work is produced by humans. Terms that are analogous to the marginal product of labour emerge naturally: the \terminology{marginal product of capital} ($MP_K$) is the output produced by one additional unit of capital services, with other inputs held constant. The \terminology{value of this marginal product} ($VMP_K$) is its value in the market place. It is the $MP_K$ multiplied by the price of output.

The $MP_K$ must eventually decline with a fixed amount of other factors of production. So, if the price of output is fixed for the firm, it follows that the $VMP_K$ must also decline. We could pursue an analysis of the short run demand for capital services, assuming labour was fixed, that would completely mirror the short run demand for labour that we have already developed.  But this would not add any new insights, so we move on to the supply side.

\begin{DefBox}
The \textbf{marginal product of capital} is the output produced by one additional unit of capital services, with all other inputs being held constant.

The \textbf{value of the marginal product of capital} is the marginal product of capital multiplied by the price of the output it produces.
\end{DefBox}

\subsection*{Supply}

We can grasp the key features of the market for capital by recognizing that the \textit{flow} of capital services is determined by the capital \textit{stock}: more capital means more services. The analysis of supply is complex because we must distinguish between the long run and the short run, and also between the supply to an industry and the supply in the whole economy.

In the \textit{short run} the total supply of capital assets, and therefore services, is fixed to the \textit{economy}, since new production capacity cannot come on stream overnight: the short run supply of services is therefore vertical. In contrast, \textit{a particular industry} in the short run faces a positively sloped supply: by offering a higher rental rate for trucks, one industry can bid them away from others.

The \textit{long run} is a period of sufficient length to permit an addition to the capital stock. A supplier of capital, or capital services, must estimate the likely return he will get on the equipment he is contemplating having built. To illustrate, suppose an earthmover costs \$100,000, and the annual maintenance and depreciation costs are \$10,000. In addition the interest rate is 5\%, and therefore the cost of borrowing the money to invest in this machinery is \$5,000 per annum. It follows that the annual cost of owning such a machine is \$15,000. If this entrepreneur is to undertake the investment he must therefore earn at least this amount annually, and this is what is termed the \terminology{required rental}. We can think of it as the opportunity cost of ownership.

\begin{DefBox}
The \textbf{required rental} covers the \textbf{sum of maintenance, depreciation and interest costs}.
\end{DefBox}

In the long run, capital services in any sector of the economy must earn the required rental. If they earn more, entrepreneurs will be induced to build or purchase additional capital goods; if they earn less, owners of capital will allow machines to depreciate, or move the machines to other sectors of the economy. Figure~\ref{fig:supplycapitalservices} depicts the long run supply of capital \textit{to the economy}, $S_0$. It is upward sloping, reflecting the fact that higher returns to capital (rentals) induce a greater supply. In this figure the rental rate in the economy is given by the value $R_0$. This can be thought of as the typical return on capital in the whole economy.

% Figure 12.7
\begin{FigureBox}{0.3}{0.25}{25em}{The supply of capital services to the economy \label{fig:supplycapitalservices}}{In the short run the stock of capital, and therefore the supply of capital services, is fixed. In the long run a higher return will induce suppliers to produce more capital goods, or capital goods may enter the economy from other economies. $R_0$ is the current return to capital in the economy -- given by where the demand for capital (or its services) intersects the given supply curve in the short run.}
% supply lines
\draw [supplycolour,ultra thick,name path=S0] (0,1) -- (30,22) node [black,mynode,below right,pos=0.8] {$S_0$ (Long-run supply)};
\draw [supplycolour,ultra thick,name path=SF] (13.188,0) -- (13.188,24) node [black,mynode,right,pos=0.8] {Fixed short-\\run supply};
% axes
\draw [thick, -] (0,25) node [mynode1,above] {Rental\\Rate} |- (30,0) node [mynode1,right] {Capital\\Services};
% intersection of supply lines
\draw [name intersections={of=S0 and SF, by=R0}];
% node for R_0
\node [mynode,below right] at (R0) {$R_0$};
\end{FigureBox}

What are the mechanics of an upward-sloping supply of capital in the economy? If the return to capital is high capital might enter the economy from abroad. Alternatively, keeping in mind that capital goods ultimately must be produced, entrepreneurs in the economy may decide that it is more profitable to produce capital goods than consumer goods if the returns warrant it.

The point $R_0$ is the current return to capital in the economy -- given by where the demand for capital (or its services) intersects the given supply curve in the short run.

\subsection*{Dynamics in the capital market}

Figure~\ref{fig:capitaladj} illustrates the market for capital services for a \textit{particular industry}. The long-run equilibrium is at $E_0$ where the supply intersects the industry demand $D_0$. Demand is derived from the firms' $VMP_K$ schedules, and we have assumed that the supply to this particular industry is infinitely elastic. $K_0$ units of capital services are traded at the rental rate $R_0$. This rate must correspond to the rate of return obtained in the economy at large in equilibrium: if returns in this sector were lower than the returns available elsewhere we would expect capital to migrate to those sectors where the return is higher; conversely if returns were higher.

% Figure 12.8
\begin{FigureBox}{0.25}{0.25}{25em}{Capital adjustment in a small industry \label{fig:capitaladj}}{From the equilibrium $E_0$, a drop in the demand for capital services in a specific industry from $D_0$ to $D_1$ drives the return on its capital $K_0$ down to $R_1$. This is lower than the economy-wide return $R_0$. No new investment takes place in the sector; capital depreciates and the stock ultimately falls to $K_1$, where the return again equals $R_0$.}
% supply lines
\draw [supplycolour,ultra thick,name path=smallS] (8,0) node [black,mynode,below] {$K_1$} -- (8,10.1);
\draw [supplycolour,ultra thick,name path=srS] (13,0) node [black,mynode,below] {$K_0$} -- (13,20) node [black,mynode,right] {Short-run supply};
% demand lines
\draw [demandcolour,ultra thick,name path=D0] (0,23) -- node [black,mynode,right,pos=0.1] {$D_0$} (23,0);
\draw [demandcolour,ultra thick,name path=D1] (0,18) -- node [black,mynode,right,pos=0.1] {$D_1$} (18,0);
% axes
\draw [thick] (0,25) node (yaxis) [mynode1,above] {Rental\\Rate} |- (25,0) node (xaxis) [mynode1,right] {Capital\\Services};
% intersection of demand and supply lines
\draw [name intersections={of=D1 and smallS, by=E1},name intersections={of=D0 and srS, by=E0},name intersections={of=D1 and srS, by=R1K0}]
	[dotted,thick] (yaxis |- E1) node [mynode,left] {$R_0$} -- (E1) node [mynode,above right] {$E_1$} -- (E0) node [mynode,above right] {$E_0$} -- ([xshift=-4cm]xaxis |- E0)
	[dotted,thick] (yaxis |- R1K0) node [mynode,left] {$R_1$} -- (R1K0);
% arrow between E0 and R1K0
\draw [->,thick,shorten >=0.5mm,shorten <=2mm] ([xshift=0.4cm]E0) -- ([xshift=0.4cm]R1K0);
% arrow between R1K0 and E1
\draw [->,thick,shorten >=1.5mm,shorten <=2mm] ([xshift=0.25cm,yshift=0.25cm]R1K0) -- ([xshift=0.25cm,yshift=0.25cm]E1);
\end{FigureBox}

Suppose now there is a slowdown in this industry that reduces the demand for capital services from $D_0$ to $D_1$, yielding a new equilibrium $E_1$. However, \textit{this new equilibrium cannot be attained immediately}: while firms may be able to unload some of their capital, such as trucks, much of their capital is fixed in place. Firms cannot readily offload buildings or machinery designed for a specific purpose. Instead capital depreciates over time from $K_0$ to $K_1$ and gradually attains the new equilibrium $E_1$, where once again the sector is earning the return available in the economy at large, $R_0$.

\subsection*{The price of capital assets}

When a higher stream of rental earnings is anticipated, buyers are willing to pay a higher purchase price for capital assets; but with lower anticipated streams they find it profitable to demand capital goods only if their price is lower: there is a downward sloping demand. The upward sloping supply and downward sloping demand together determine the quantity and price of capital goods in the economy. In turn this determines the flow of capital services.

What happens to the price of capital assets when an industry faces a decline in its demand for capital services, as in Figure~\ref{fig:capitaladj}? In the short run the return on the fixed capital services falls to $R_1$. The capital that is in use \textit{in this sector of the economy} is now less valuable on account of the lower return it is generating. If capital elsewhere in the economy can still earn a return of $R_0$, then capital in this one sector will be allowed to depreciate. This depreciation continues until capital services in the industry become sufficiently scarce that the rental rate returns to its original level. At this point the present value of future rentals matches the price of capital goods in the whole economy, and the industry begins to replace its depreciating capital once again. A new LR equilibrium in the industry is attained at $K_1$.