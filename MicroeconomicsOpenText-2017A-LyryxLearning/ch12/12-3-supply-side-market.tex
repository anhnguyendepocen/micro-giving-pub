\section{The supply side of the market}\label{sec:ch12sec3}

Most prime-age individuals work, but some do not. The decision to join the \terminology{labour force} is called the \terminology{participation decision}. Of those who do participate in the labour force, some individuals work full time, others work part time, and yet others cannot find a job. The \terminology{unemployment rate} is the fraction of the labour force actively seeking employment that is not employed.

\begin{DefBox}
The \textbf{participation rate} for the economy is the fraction of the population in the working age group that joins the labour force.

The \textbf{labour force} is that part of the population either employed or seeking employment.

The \textbf{unemployment rate} is the fraction of the labour force actively seeking employment that is not employed.
\end{DefBox}

The individual participation rate is higher for men than for women in most economies. But the trends in their participation in Canada have been very distinct in recent decades: the participation rate for women has increased steadily, while the participation rate for men has fallen. The former trend reflects changes in social customs and changes in household productivity. Women today are more highly educated, and their role in society and the economy is viewed very differently than in the nineteen fifties and sixties. Family planning enables women to better plan their careers over the life cycle. Female participation has also increased on account of the rise in household productivity and the development of service industries designed to support home life. Almost every home now has an array of appliances that greatly economize on time in the running of a household.  In addition, the economy has seen the development of a market for daycare and the development of home cleaning and maintenance services.  

In contrast, male participation rates have been declining on account of an increasing fraction of the male labour force retiring before the traditional age of 65.

What determines the participation decision of individuals? First, the wage rate that an individual can earn in the market is crucial. If that wage is low, then the individual may be more efficient in producing home services directly, rather than going into the labour market, earning a modest income and having to pay for home services.  Second, there are fixed costs associated with working. A decision to work means that the individual must have work clothing, must undertake the costs of travel to work, and pay for daycare if there are children in the family.  Third the participation decision depends upon non-labour income. Does the individual in question have a partner who earns a substantial amount? Does she have investment income?

Let us now turn to the supply decision for an individual who has decided to work. We propose that the supply curve is positively sloped, indicating that as the wage increases, the individual wishes to supply more labour. From the point $e_0$ on the supply function in Figure~\ref{fig:indlaboursupply}, let the wage increase from $W_0$ to $W_1$.

% Figure 12.3
\begin{FigureBox}{0.3}{0.3}{25em}{Individual labour supply \label{fig:indlaboursupply}}{A wage increase from $W_0$ to $W_1$ induces the individual to \emph{substitute} away from leisure, which is now more expensive, and work \emph{more}. But the higher wage also means the individual can work fewer hours for a given standard of living; therefore the income effect induces \emph{fewer hours}. On balance the substitution effect tends to dominate and the supply curve therefore slopes upward.}
% Supply curve
\draw [supplycolour,ultra thick,domain=270:360,name path=S] plot ({5+10*cos(\x)},{15+10*sin(\x)}) node [mynode,above,black] {$S$};
% axes
\draw [thick, -] (0,20) node (yaxis) [mynode1,above] {Wage\\Rate} |- (20,0) node (xaxis) [right] {Hours};
% paths for e0 and e1
\path [name path=e0line] (10,0) -- +(0,20);
\path [name path=e1line] (14.5,0) -- +(0,20);
% intersection of S with paths
\draw [name intersections={of=S and e0line, by=e0},name intersections={of=S and e1line, by=e1}]
	[dotted,thick] (yaxis |- e0) node [mynode,left] {$W_0$} -- (e0) node [mynode,above left] {$e_0$} -- (xaxis -| e0) node [mynode,below] {$H_0$}
	[dotted,thick] (yaxis |- e1) node [mynode,left] {$W_1$} -- (e1) node [mynode,above left] {$e_1$} -- (xaxis -| e1) node [mynode,below] {$H_1$};
\end{FigureBox}

The individual offers more labour, $H_1$, at the higher wage.  What is the economic intuition behind the higher labour supply? Like much of choice theory there are two impacts associated with the higher wage.  First the higher wage makes leisure more expensive relative to working. That midweek game of golf has become more expensive in terms of what the individual could earn. So the individual should \textit{substitute} away from the more expensive 'good', leisure, towards labour.  But at the same time, in order to purchase a given bundle of goods, the individual can work fewer hours at the higher wage. This is a type of \textit{income effect} and induces the individual to work less.  The fact that we draw the labour supply curve with a positive slope means that we believe the substitution effect is the more important of the two.

\subsection*{Elasticity of the supply of labour}

Suppose an industry is small relative to the whole economy and employs workers with common skills. These industries tend to pay the `going wage'. For example, a very large number of students are willing to work at the going rate for telemarketing firms. This means that the supply curve of such labour is in effect horizontal from the standpoint of the telemarketing industry. 

But some industries may not be small relative to the pool of labour they employ. And in order to get more labour to work for them they may have to offer higher wages to induce workers away from other sectors of the economy. Consider the behaviour of two related sectors in housing -- new construction and home restoration. New home builders may have to offer higher wages to employ more plumbers and carpenters, because these workers will move from the renovation sector only if wages are higher. In this case the new housing industry's labour supply curve slopes upwards.

In the long run, the industry's supply curve is more elastic than in the short run. When a sectoral expansion bids up the wages of information technology (IT) workers, more school leavers are likely to develop IT skills. In the longer run the resulting increased supply will moderate the short-run wage increases.

\begin{ApplicationBox}{Labour supply policy and economic incentives \label{app:laboursupplypolicy}}
Governments can have a significant influence on the amount of labour supplied in the market place. The wage rate that the individual actually faces is the net-of-income-tax wage. Since the government can alter the income tax rate, it can therefore impact the amount of labour supplied. The elasticity of the labour supply is critical in determining the extent to which labour will react to a change in its wage rate. 
\end{ApplicationBox}