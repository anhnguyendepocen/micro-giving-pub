\section{The market for capital}\label{sec:ch12sec5}

The stock of \terminology{physical capital} includes assembly-line machinery, rail lines, dwellings, consumer durables, school buildings and so forth. It is the stock of produced goods used as inputs to the production of other goods and services.

\begin{DefBox}
\textbf{Physical capital} is the stock of produced goods that are inputs to the production of other goods and services.
\end{DefBox}

Physical capital is distinct from land in that the former is produced, whereas land is not. These in turn differ from \textit{financial wealth}, which is not an input to production. We add to the capital stock by undertaking investment. But, because capital depreciates, investment in new capital goods is required merely to stand still. \terminology{Depreciation} accounts for the difference between \terminology{gross and net investment}.

\begin{DefBox}
\textbf{Gross investment} is the production of new capital goods and the improvement of existing capital goods.

\textbf{Net investment} is gross investment minus depreciation of the existing capital stock.

\textbf{Depreciation} is the annual change in the value of a physical asset.
\end{DefBox}

Since capital is a \terminology{stock} of productive assets we must distinguish between the value of services that \terminology{flow} from capital and the value of capital assets themselves.

\begin{DefBox}
A \textbf{stock} is the quantity of an asset at a point in time.

A \textbf{flow} is the stream of services an asset provides during a period of time.
\end{DefBox}

When a car is rented it provides the driver with a service; the car is the asset, or stock of capital, and the driving, or ability to move from place to place, is the service that flows from the use of the asset. When a photocopier is leased it provides a stream of services to a printing company. The copier is the asset; it represents a stock of physical capital. The printed products result from the service the copier provides per unit of time. 

The \terminology{price of an asset} is what a purchaser pays for the asset. The owner then obtains the future stream of capital services it provides. Buying a car for \$30,000 entitles the owner to a stream of future transport services. We use the term \terminology{rental rate} to define the cost of the services from capital, to distinguish this cost from the cost of purchasing the asset -- which is its price. The cost of using capital services is the rental rate for capital. The price of an asset is the financial amount for which the asset can be purchased.

\begin{DefBox}
\textbf{Capital services} are the production inputs generated by capital assets.

The \textbf{rental rate} is the cost of using capital services.

The \textbf{price of an asset} is the financial sum for which the asset can be purchased.
\end{DefBox}

But what determines the \textit{price} of a productive asset? The price must reflect the value of future services that the capital provides. But we cannot simply add up these future values, because a dollar today is more valuable than a dollar several years from now. The key to valuing an asset lies in understanding how to compute the \textit{present value} of a future income stream.

\subsection*{Present values and discounting}

Rentals are paid on an annual or per period basis. In order to relate rentals to the price of an asset it is necessary to determine the value in one period of the stream of rentals that an asset will earn over its life. In other words we need to find the \textit{present value} of the stream of rental payments. This is done through \textit{discounting}. 

Imagine that you are promised the sum of \$105 exactly one year from now. How would you value this today? The value today of \$105 received a year from now is less than \$105, because if you had this amount today you could invest it at the going rate of interest and end up with more than \$105. For example, if the rate of interest is 5\% (= 0.05), then \$105 next year is equivalent to \$100 today, because by investing the \$100 today we would obtain an additional \$5 at the end of the year. The amount returned to the investor is thus (one times) the original amount plus the interest rate times the amount. So, to obtain the next period value of a sum of money that we have today, we multiply this sum by one plus the rate of interest: 

\begin{equation*}
\text{Value next period}=\text{value this period}\times(1+\text{interest rate})
\end{equation*}

By the same reasoning, to obtain the `today' value of a sum obtained one year from now, that sum must be divided by one plus the interest rate. Dividing both sides of the above relation by $(1+\text{interest rate})$, and using the letter $i$ to denote the interest rate: 

\begin{equation*}
\frac{\text{Value next period}}{(1+i)}=\text{value this period}
\end{equation*}

To push the example further: suppose you have \$100 today and you want to find its value two years from now. In this case, the value of the \$100 one year from now could be reinvested in total (the \$100 plus the \$5) to yield that sum multiplied by one plus the interest rate again:

\begin{align*}
\text{Value in two years}&=\text{value today}\times(1+i)\times(1+i)	\\
&=\text{value today}\times(1+i)^2.
\end{align*}

Correspondingly, the value today of a sum received in two years' time is that sum divided by $(1+i)^2$. More generally the value today of any sum received $t$ periods into the future is that sum divided by $(1+i)^t$.

\begin{equation*}
\text{Value today}=\frac{(\text{value in year }t)}{(1 + i)^t}.
\end{equation*}

Note two features of this relationship. First, if the interest rate is high, the value today of future sums is smaller than if the interest rate is low. Second, sums received far in the future are worth much less than sums received in the near future: the denominator increases the larger is $t$.

Consider the example in Table~\ref{table:presentvalueasset} below: a machine earns \$8,000 for each of two years. At the end of the second period the machine is almost 'dead' and has a scrap value of \$2,000. The interest rate is 10\%. The present value of each year's income stream is given in the final column. The first payment is received in one year's time, and therefore its value today is $\$8,000/(1.10)=\$7,272.70$. The value of the second year's income is $\$8,000/(1.10)^2=\$6,611.58$.  Finally the scrap value is $\$2,000/(1.10)^2=\$1,652.90$. The present value of this future income stream is simply the sum of these amounts is \$15,537.18.

\begin{table}[H]
\begin{center}
\begin{tabu} to \linewidth {|X[1,c]X[1,c]X[1,c]X[1,c]|} \hline 
\rowcolor{rowcolour}	\textbf{Year} & \textbf{Annual earnings of machine \$}	& \textbf{Scrap value \$}	& \textbf{Discounted value \$}	\\[0.25em]
1 	& 8,000	&		& 7,272.70	\\
\rowcolor{rowcolour}	2	& 8,000	& 2,000	& 6,611.58 + 1,652.90	\\
Asset value in initial year &		&		& 15,537.18	\\
\multicolumn{4}{|c|}{\cellcolor{rowcolour}Interest rate = 10\%} \\ \hline 
\end{tabu}
\end{center}
\caption{Present value of an asset in dollars \label{table:presentvalueasset}}
\end{table}

\begin{DefBox}
The \textbf{present value of a stream of future earnings} is the sum of each year's earnings divided by one plus the interest rate raised to the appropriate power.
\end{DefBox}

The value today of the services this machine offers should determine the price a buyer is willing to pay for the machine. If a potential buyer can purchase the machine for less than the present discounted value of the machine's future earnings a potential buyer should purchase it. Conversely if the asking price for the machine exceeds the present value of the earnings, the potential buyer should avoid the purchase.