\section{Labour-market equilibrium and mobility}\label{sec:ch12sec4}

The fact that labour is a derived demand is what distinguishes the labour market's dynamics from the goods-market dynamics. Let us investigate these dynamics with the help of Figure~\ref{fig:eqindustrylabour}; it contains supply and demand functions for one particular industry -- the cement industry let us assume.

% Figure 12.4
\begin{FigureBox}{0.3}{0.25}{25em}{Equilibrium in an industry labour market \label{fig:eqindustrylabour}}{A fall in the \emph{price of the good} produced in a particular industry reduces the value of the $MP_L$. Demand for labour thus falls from $D_0$ to $D_1$ and a new equilibrium $E_1$ results.\\Alternatively, from $E_0$, an increase in wages in another sector of the economy induces some labour to move to that sector. This is represented by the shift of $S_0$ to $S_1$ and the new equilibrium $E_2$.}
% demand lines
\draw [demandcolour,ultra thick,name path=D1] (0,15) -- (15,0) node [black,mynode,above right] {$D_1$};
\draw [demandcolour,ultra thick,name path=D0] (0,20) -- (27,0) node [black,mynode,above right] {$D_0$};
% supply lines
\draw [supplycolour,ultra thick,name path=S0] (0,1) -- (30,22) node [black,mynode,above] {$S_0$};
\draw [supplycolour,ultra thick,name path=S1] (0,3) -- (25,25) node [black,mynode,above] {$S_1$};
\draw [thick, -] (0,25) node (yaxis) [mynode1,above] {Wage\\Rate} |- (30,0) node (xaxis) [right] {Labour};
% intersection of supply and demand lines
\draw [name intersections={of=S0 and D0, by=E0},name intersections={of=S0 and D1, by=E1},name intersections={of=S1 and D0, by=E2}]
	[dotted,thick] (yaxis |- E0) node [mynode,left] {$W_0$} -- (E0) node [mynode,above] {$E_0$} -- (xaxis -| E0) node [mynode,below] {$L_0$}
	[dotted,thick] (yaxis |- E1) node [mynode,left] {$W_1$} -- (E1) node [mynode,above] {$E_1$} -- (xaxis -| E1) node [mynode,below] {$L_1$}
	[dotted,thick] (yaxis |- E2) node [mynode,left] {$W_2$} -- (E2) node [mynode,above] {$E_2$} -- (xaxis -| E2) node [mynode,below] {$L_2$};
% paths to create arrows between supply and demand lines
\path [name path=supplyarrowline] (0,20) -- +(30,0);
\path [name path=demandarrowline] (2,0) -- +(0,25);
% arrows between supply and demand lines
\draw [name intersections={of=S0 and supplyarrowline, by=s0},name intersections={of=S1 and supplyarrowline, by=s1}]
	[->,thick,shorten >=1mm,shorten <=1mm] (s0) -- (s1);
\draw [name intersections={of=D0 and demandarrowline, by=d0},name intersections={of=D1 and demandarrowline, by=d1}]
	[->,thick,shorten >=1mm,shorten <=1mm] (d0) -- (d1);
\end{FigureBox}


In Figure~\ref{fig:demandforlabour} we illustrated the impact on the demand for labour of a decline in the price of the output produced. In the current example, suppose that the demand for cement declines as a result of a slowdown in the construction sector, which in turn results in a fall in the cement price. The impact of this price fall is to reduce the output value of each worker in the cement producing industry, because their output now yields a lower price.  This decline in the $VMP_L$ is represented in Figure~\ref{fig:eqindustrylabour} as a shift from $D_0$ to $D_1$. The new $VMP_L$ curve ($D_1$) results in the new equilibrium $E_1$. 

As a second example: suppose the demand for labour in some other sectors of the economy increases and its wage in those sectors rises correspondingly, this is reflected in the cement sector as a backward shift in the supply of labour (some \textit{other price} has changed and results in a \textit{shift} in the labour supply curve). In Figure~\ref{fig:eqindustrylabour} supply shifts from $S_0$ to $S_1$ and the equilibrium goes from $E_0$ to $E_2$.

How large are these impacts likely to be? That will depend upon how mobile labour is between sectors: spillover effects will be smaller if labour is less mobile. This brings us naturally to the concepts of \terminology{transfer earnings} and \terminology{rent}.

Consider the case of a performing violinist whose wage is \$80,000. If, as a best alternative, she can earn \$60,000 as a music teacher then her rent is \$20,000 and her transfer earnings \$60,000: her rent is the excess she currently earns above the best alternative. Another violinist in the same orchestra, earning the same amount, who could earn \$65,000 as a teacher has rent of \$15,000. The alternative is called the \terminology{reservation wage}. The violinists should not work in the orchestra unless they earn at least what they can earn in the next best alternative.

\begin{DefBox}
\textbf{Transfer earnings} are the amount that an individual can earn in the next highest paying alternative job.

\textbf{Rent} is the excess remuneration an individual currently receives above the next best alternative. This alternative is the \textbf{reservation wage}.
\end{DefBox}

These concepts are illustrated in Figure~\ref{fig:transferearningrent} for a large number of individuals. In this illustration, different individuals are willing to work for different amounts, but all are paid the same wage $W_0$. The market labour supply curve by definition defines the wage for which each individual is willing to work. Thus the rent earned by labour in this market is the sum of the excess of the wage over each individual's transfer earnings -- the area $W_0E_0$A. This area is also what we called producer or supplier surplus in Chapter~\ref{chap:welfare}.

% Figure 12.5
\begin{FigureBox}{0.3}{0.25}{25em}{Transfer earnings and rent \label{fig:transferearningrent}}{Rent is the excess of earnings over reservation wages. Each individual earns $W_0$ and is willing to work for the amount defined by the labour supply curve. Hence rent is $W_0E_0$A and transfer earnings OA$E_0L_0$. Rent is thus the term for supplier surplus in this market.}
% demand line
\draw [demandcolour,ultra thick,name path=D] (0,20) -- (27,0) node [black,mynode,above right,pos=0.95] {$D$};
% supply line
\draw [supplycolour,ultra thick,name path=S] (0,1) node [black,mynode,left] {A} -- (30,22) node [black,mynode,above] {$S$};
% axes
\draw [thick] (0,25) node (yaxis) [mynode1,above] {Wage\\Rate} -- (0,0) node [mynode,below left] {O} -- (30,0) node (xaxis) [right] {Labour};
% intersection of supply and demand line
\draw [name intersections={of=S and D, by=E0}]
	[dotted,thick] (yaxis |- E0) node [mynode,left] {$W_0$} -- node [mynode,below=0.25cm and 0cm,pos=0.4] {Rent} (E0) node [mynode,above] {$E_0$} -- node [mynode,left=0cm and 0.5cm,pos=0.6] {Transfer\\earnings} (xaxis -| E0) node [mynode,below] {$L_0$};
\end{FigureBox}

\subsection*{Free labour markets?}

Real-world labour markets are characterized by trade unions, minimum wage laws, benefit regulations, severance packages, parental leave, sick-day allowances and so forth. So can we really claim that markets work in the way we have described them -- essentially as involving individual agents demanding and supplying labour? But labour markets are not completely 'free' in the conventional sense. The issue is whether these interventions, that are largely designed to protect workers, have a large or small impact on the market. It is often claimed that the reason why unemployment rates are generally higher in European economies than in Canada and the US is that labour markets are less subject to controls in the latter economies than the former.

\begin{ApplicationBox}{Are high salaries killing professional sports? \label{app:highsalarysport}}
It is often said that the agents of professional players are killing their sport by demanding unreasonable salaries. Frequently the major leagues are threatened with strikes, even though players are paid millions each year. In fact, wages are high because the derived demand is high. Fans are willing to pay high ticket prices, and television rights generate huge revenues. Combined, these revenues not only make ownership profitable, but increase the demand for the top players.

\bigskip
If this is so why do some teams incur financial losses? In fact very few teams make losses: cries of poverty on the part of owners are more frequently part of the bargaining process. Occasionally teams are located in the wrong city and they should therefore either exit the industry or move the franchise to another market. 
\end{ApplicationBox}

The impact of these measures on wages and employment levels can be determined with the help of the analysis that we have developed in this chapter. Let us see how in the case of unions.

In Figure~\ref{fig:marketinterventions} the equilibrium in a given labour market is at $E_0$, assuming that the workers are not unionized. If unionization increases the wage paid above $W_0$ to $W_1$ then fewer workers will be employed. But how big will this reduction be?  Clearly it depends on the elasticity of demand. If demand is inelastic the impact will be small and conversely if demand is elastic. If the minimum wage is substantially higher than the equilibrium wage, an inelastic demand will ensure that employment effects are small while cost increases could be large. In Chapter~\ref{chap:elasticities} we saw that the dollar value of expenditure on a good increases when the price rises if the demand is inelastic. In the current example the `good' is labour. Hence the less elastic is labour the more employers will pay in response to a rise in the wage rate. A case which has stirred great interest is described in Application Box~\ref{app:davidcardminwage}.

% Figure 12.6
\begin{FigureBox}{0.3}{0.25}{25em}{Market interventions \label{fig:marketinterventions}}{$E_0$ is the equilibrium in the absence of a union. If the presence of a union forces the wage to $W_1$ fewer workers are employed. The magnitude of the decline from $L_0$ to $L_1$ depends on the elasticity of demand for labour. The excess supply at the wage $W_1$ is (F-$E_1$).}
% Demand line
\draw [demandcolour,ultra thick,name path=D] (0,20) -- (27,0) node [black,mynode,above right,pos=0.95] {$D$};
% Supply line
\draw [supplycolour,ultra thick,name path=S] (0,1) -- (30,22) node [black,mynode,above] {$S$};
% axes
\draw [thick, -] (0,25) node (yaxis) [mynode1,above] {Wage\\Rate} |- (30,0) node (xaxis) [right] {Labour};
% intersection of demand and supply
\draw [name intersections={of=S and D, by=E0}]
	[dotted,thick] (yaxis |- E0) node [mynode,left] {$W_0$} -- (E0) node [mynode,above] {$E_0$} -- (xaxis -| E0) node [mynode,below] {$L_0$};
% path for W1
\path [name path=W1line] (0,15) -- +(30,0);
% intersection of supply and demand lines with W1line
\draw [name intersections={of=W1line and S, by=F},name intersections={of=W1line and D, by=E1}]
	[dotted,thick] (yaxis |- E1) node [mynode,left] {$W_1$} -| (xaxis -| E1) node [mynode,below] {$L_1$}
	[dotted,thick] (E1) node [mynode,above] {$E_1$} -- (F) node [mynode,above] {F};
\end{FigureBox}

\begin{ApplicationBox}{David Card on minimum wage \label{app:davidcardminwage}}
David Card is a famous Canadian-born labour economist who has worked at Princeton and Berkeley. He is a winner of the prestigious Clark medal, an award made annually to an outstanding economist under the age of forty. Among his many contributions to the discipline, is a study of the impact of minimum wage laws on the employment of fast-food workers. With Alan Krueger as his co-researcher, Card examined the impact of the 1992 increase in the minimum wage in New Jersey and contrasted the impact on employment changes with neighbouring Pennsylvania, which did not experience an increase. They found virtually no difference in employment patterns between the two states. This research generated so much interest that it led to a special conference. Most economists now believe that modest changes in the level of the minimum wage have a small impact on employment levels.
\end{ApplicationBox}