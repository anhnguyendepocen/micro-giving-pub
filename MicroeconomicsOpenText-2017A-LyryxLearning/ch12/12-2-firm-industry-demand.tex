\section{Firm versus industry demand}\label{sec:ch12sec2}

The demand for labour within an industry is obtained from the sum of the demands by each individual firm. It is analogous to the goods market, but there is a subtle and crucial difference.

In Figure~\ref{fig:industrydemandlabour} we plot the price of labour -- the wage rate -- against the amount of employment. Each individual firm has a downward sloping labour demand -- the $VMP_L$ curve, indicating that a firm purchases more labour at a lower wage rate. For example, if the wage falls from $W_0$ to $W_1$ the demand by an individual firm increases from $e_0$ to $e_1$. If all $N$ firms in this industry are similar the total industry-wide demand increases correspondingly. However, the ultimate demand increase is not quite $N$ times the individual increase. This is because if all firms wish to supply more in the marketplace, consumers will purchase more only if the price is reduced somewhat.

% Figure 12.2.
\begin{FigureBox}{0.3}{0.25}{25em}{The industry demand for labour \label{fig:industrydemandlabour}}{The industry demand for labour is an adjusted sum of individual firm demands. A fall in the wage induces more employment \emph{and therefore more output}. At the industry level this additional output can be sold only if the price falls, which in turn reduces the value of the $MP_L$. This moderates the increase in the amount of labour employed, from $E^1$ to $E_1$.}
% Demand lines
\draw [demandcolour,ultra thick,name path=Dindiv] (3,20) node [mynode,right,black] {Demand for labour\\by one firm$=MP_L$} -- (12,3);
\draw [demandcolour,ultra thick,name path=Dsum] (15,20) -- (24,3) node [mynode,right,black] {Sum of\\individual demands};
% Industry demand line
\draw [inddemandcolour,ultra thick,name path=IndD] (17,20) node [mynode,right,black] {Industry demand} -- (20,3);
% axes
\draw [thick, -] (0,25) node (yaxis) [mynode1,above] {Wage\\Rate} |- (30,0) node (xaxis) [right] {Employment};
% intersection of IndD and Dsum
\draw [name intersections={of=IndD and Dsum, by=E0}]
	[dotted,thick] (yaxis |- E0) node [mynode,left] {$W_0$} -- (E0) node [mynode,above right] {$E_0$} -- ([xshift=-6cm]xaxis |- E0);
% path for W1 line, as well as path for W0line
\path [name path=W1line] (0,7) -- +(30,0);
\path [name path=W0line] (yaxis |- E0) -- +(30,0);
% intersection of W1line with demand lines, as well as intersection of W0line with demand Dindiv
\draw [name intersections={of=Dindiv and W1line, by=e1},name intersections={of=IndD and W1line, by=E1},name intersections={of=Dsum and W1line, by=Eprime}]
	[dotted,thick] (yaxis |- e1) node [mynode,left] {$W_1$} -- (e1) node [mynode,above right] {$e_1$} -- (E1) node [mynode,above left] {$E_1$} -- (Eprime) node [mynode,above right] {$E^1$} -- ([xshift=-6cm]xaxis |- Eprime);
\end{FigureBox}

How does such a decline in the output price impact each firm? Clearly it reduces the value of the $MP_L$, because additional units of output must be sold at a reduced price. This second-round impact of the fall in price for the good being produced therefore moderates the amount of labour demanded by each firm. In geometric terms this means that if we were to sum all of the firm-level labour demands horizontally, the amount of labour corresponding to the lower wage rate overstates the true ultimate increase in labour demanded. This is illustrated in Figure~\ref{fig:industrydemandlabour} by contrasting the actual total labour demand with the horizontal sum of the individual labour demands.