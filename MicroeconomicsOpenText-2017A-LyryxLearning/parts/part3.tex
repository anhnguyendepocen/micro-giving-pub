\cleardoublepage
%\newpage
\thispagestyle{empty}
\vspace{30mm}
\addcontentsline{toc}{part}{\normalfont \textbf{Part Three: Decision Making by Consumer and Producers}}
{\color{parttextcolour}\fontsize{1.25cm}{3em}\selectfont\textbf{Part Three}} \\ \\
{\color{parttextcolour}\huge Decision Making by Consumer and Producers}

\vspace{10mm}
{\color{partlinecolour}\rule{25em}{2pt}}
\vspace{10mm}

{\large\color{parttextcolour}
~\ref{chap:individualchoice}. Individual choice

~\ref{chap:firminvestorcapital}. Firms, investors and capital markets

~\ref{chap:prodcost}. Producer choice}

\vspace{10mm}

{\normalfont Individuals and producers make choices when they interact with one another. We begin by supposing that individuals act in a manner that is at least in their self-interest -- they transact in markets because they derive satisfaction or utility from doing so. Individuals may also be altruistic. But in each case we suppose they plan their actions in a way that is consistent with a goal. Utility may be measurable or only comparable, and we derive the tools of choice optimization using these two alternative notions of utility in Chapter~\ref{chap:individualchoice}.

Chapter~\ref{chap:firminvestorcapital} furnishes a link between buyers and sellers, between savers and investors, between households and firms.

Like individual buyers, sellers and producers also act with a plan to reach a goal. They are interested in making a profit from their enterprise and produce goods and services at the least cost that is consistent with available production technologies. Cost minimization and profit maximization in the short run and the long run are explored in Chapter~\ref{chap:prodcost}.}

