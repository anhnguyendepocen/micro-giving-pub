\cleardoublepage
%\newpage
\thispagestyle{empty}
\vspace{30mm}
\addcontentsline{toc}{part}{\normalfont \textbf{Part Four: Market Structures}}
{\color{parttextcolour}\fontsize{1.25cm}{3em}\selectfont\textbf{Part Four}} \\ \\
{\color{parttextcolour}\huge Market Structures}

\vspace{10mm}
{\color{partlinecolour}\rule{25em}{2pt}}
\vspace{10mm}

{\large\color{parttextcolour}
~\ref{chap:perfectcompetition}. Perfect competition

~\ref{chap:monopoly}. Monopoly

~\ref{chap:imperfectcompetition}. Imperfect competition}

\vspace{10mm}

{\normalfont Markets are all around us and they come in many forms. Some are on-line, others are physical. Some involves goods such as food and vehicles; others involve health provision or financial advice. Markets differ also by the degree of competition associated with each. For example the wholesale egg market is very homogeneous in that the product has minimal variation. The restaurant market offers food, but product variation is high -- restaurants specialize in foods with specific ethnic or geographic origins for example. So an Italian restaurant differs from a Thai restaurant even though they both serve food. 

In contrast to the egg and restaurant market, which tend to have very many suppliers and furnish products that are strongly related, many other markets are characterized by just a few suppliers or in some cases just one. For example, passenger train services may have just a single supplier, and this supplier is therefore a monopolist. Pharmaceuticals tend to be supplied by a limited number of large international corporations plus a group of generic drug manufacturers. Passenger aircraft used by international airlines are all produced by just a handful of builders.

In this part we examine the reasons why markets take on different forms and display a variety of patterns of behaviour. We delve into the working of each market structure to understand why these markets retain their structure.}