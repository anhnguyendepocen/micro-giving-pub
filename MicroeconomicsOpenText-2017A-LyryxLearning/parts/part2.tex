\cleardoublepage
%\newpage
\thispagestyle{empty}
\vspace{30mm}
\addcontentsline{toc}{part}{\normalfont \textbf{Part Two: Responsiveness and the Value of Markets}}
{\color{parttextcolour}\fontsize{1.25cm}{3em}\selectfont\textbf{Part Two}} \\ \\
{\color{parttextcolour}\huge Responsiveness and the Value of Markets}

\vspace{10mm}
{\color{partlinecolour}\rule{25em}{2pt}}
\vspace{10mm}

{\large\color{parttextcolour}
~\ref{chap:elasticities}. Elasticities

~\ref{chap:welfare}. Welfare economics, externalities and non-classical markets}

\vspace{10mm}

{\normalfont The degree to which individuals or firms, or any economic agent, responds to incentives is important to ascertain for pricing and policy purposes: if prices change, to what degree will consumers respond in their purchases? How will markets respond to taxes? Chapter~\ref{chap:elasticities} explores and develops the concept of elasticity, which is the word economists use to define responsiveness. A meaningful metric that is applicable to virtually any market or incentive means that behaviours can be compared in different environments. 

In Chapter~\ref{chap:welfare} we explore how markets allocate resources and how the well-being of society's members is impacted by uncontrolled and controlled markets. A central development of this chapter is that markets are very useful environments, but may need to be controlled in many circumstances.}

