\section{Fixed costs and sunk costs}\label{sec:ch8sec5}

The distinction between fixed and variable costs is important for producers who encounter difficulty in making a profit. If a producer has committed himself to setting up a plant, then he has made a decision to incur a fixed cost. Having done this, he must now decide on a production strategy that will maximize profit. However, the price that consumers are willing to pay may be sufficient to yield a profit. So, if Black Diamond Snowboards cannot make a profit, should it shut down? The answer is that if it can cover its variable costs, having already incurred its fixed costs, it should stay in production. By covering the variable cost of its operation, Black Diamond is at least earning some return. A \terminology{sunk cost} is a fixed cost that has already been incurred and cannot be recovered. But if the pressures of the marketplace are so great that the total costs cannot be covered in the longer run, then this is not a profitable business and the firm should close its doors. 

Is a fixed cost always a sunk cost? Any production that involves capital will incur a fixed cost -- one not varying with the amount produced. Such capital can be financed in several ways however: it might be financed on a very short term lease basis, or it might have been purchased by the entrepreneur. If it is leased on a month to month basis, an entrepreneur who can only cover variable costs can exit the industry quickly -- by not renewing the lease on the capital. But an individual who has actually purchased equipment that cannot readily be resold has essentially sunk money into the fixed cost component of their production. This entrepreneur should continue to produce as long as she can cover variable costs.

\begin{DefBox}
\textbf{Sunk cost} is a fixed cost that has already been incurred and cannot be recovered, even by producing a zero output.
\end{DefBox}

In facing a possible shut down, we must also keep in mind the opportunity costs of running the business. The owner pays himself a salary, and ultimately he must recognize that the survival of the business should not depend upon his drawing a salary that is less than his opportunity cost. If he underpays himself in order to avoid shutting down, he might be better off in the long run to close the business and earn his opportunity cost elsewhere in the marketplace. 

\subsection*{R \& D as a sunk cost}

Sunk costs in the modern era are frequently in the form of research and development costs, not the cost of building a plant or purchasing machinery. The prototypical example is the pharmaceutical industry, where it is becoming progressively more challenging to make new drug breakthroughs -- both because the `easier' breakthroughs have already been made, and because it is necessary to meet tighter safety conditions attached to new drugs. Research frequently leads to drugs that are not sufficiently effective to meet their goal. As a consequence, the pharmaceutical sector regularly writes off hundreds of millions of dollars of lost sunk costs -- research and development that did not yield sufficient fruit. This is a perfect example of a branch of a large firm essentially `closing down': anticipated revenues are not sufficient to cover total costs.