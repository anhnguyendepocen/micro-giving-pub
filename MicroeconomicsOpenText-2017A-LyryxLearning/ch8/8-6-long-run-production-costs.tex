\section{Long run production and costs}\label{sec:ch8sec6}

The snowboard manufacturer we portray produces a relatively low level of output; in reality, millions of snowboards are produced each year in the global market. Black Diamond Snowboards may have hoped to get a start by going after a local market---the ``free-ride'' teenagers at Mont Sainte Anne in Quebec or at Fernie in British Columbia. If this business takes off, the owner must increase production, take the business out of his garage and set up a larger-scale operation. But how will this affect his cost structure? Will he be able to produce boards at a lower cost than when he was producing a very limited number of boards each season? Real-world experience would indicate yes. 

Production costs almost always decline when the \textit{scale} of the operation initially increases. We refer to this phenomenon simply as \terminology{economies of scale}. There are several reasons why scale economies are encountered. One is that production flows can be organized in a more efficient manner when more is being produced. Another is that the opportunity to make greater use of task specialization presents itself; for example, Black Diamond Snowboards may be able to subdivide tasks within the laminating and packaging stations. If scale economies do define the real world, then a bigger plant---one that is geared to produce a higher level of output---should have an average total cost curve that is ``lower'' than the cost curve corresponding to the smaller scale of operation we considered in the example above.

\subsection*{Average costs in the long run}

Figure~\ref{fig:longshortavg} illustrates a possible relationship between the $ATC$ curves for three different scales of operation. $ATC_1$ is the average total cost curve associated with a small-sized plant. $ATC_2$ is associated with a somewhat larger plant, and so forth. Obviously, the cost curves to the right correspond to larger production facilities, given that output is measured on the horizontal axis. If there are economies associated with a larger scale of operation, then the average costs associated with producing larger outputs in a larger plant should be lower than the average costs associated with lower outputs in a smaller plant, assuming that the plants are producing the output levels they were designed to produce. For this reason, the cost curve $ATC_2$ has a segment that is lower than the lowest segment on $ATC_1$. However, in Figure~\ref{fig:longshortavg} the cost curve $ATC_3$ has moved upwards. What behaviours are implied here?

% Figure 8.5
\begin{FigureBox}{0.3}{0.25}{25em}{Long run and short run average costs \label{fig:longshortavg}}{The long-run $ATC$ curve, $LATC$, is the lower envelope of all short-run $ATC$ curves. It defines the least cost per unit of output when all inputs are variable. Minimum efficient scale is that output level at which the $LATC$ is a minimum, indicating that further increases in the scale of production will not reduce unit costs.}
\draw [latccolour,ultra thick,-]
	(3,13) to [out=315,in=225] (22,13) node [black,mynode,below right] {$LATC$}
	(3.5,13) to [out=290,in=225] (8.5,13) node [black,mynode,above] {$ATC_1$};
\draw [atccolour,ultra thick,-]
	(10,10.25) to [out=315,in=225] (15,10.25) node [black,mynode,above] {$ATC_2$}
	(16.5,13) to [out=315,in=250] (21.5,13) node [black,mynode,above] {$ATC_3$};
\draw [thick, -] (0,20) node [above] {Cost (\$)} |- (25,0) node [right] {Quantity};
\draw [->,thick,shorten >=0.5mm] (12.5,5) node [mynode,below] {Minimum efficient\\scale ($MES$)}-- (12.5,9);
\node [mynode,below] at (3,9) {Region of $IRS$};
\node [mynode,below] at (22,9) {Region of $DRS$};
\end{FigureBox}

We propose that, beyond some large scale of operation, it becomes increasingly difficult to reap further cost reductions from specialization, organizational economies, or marketing economies. At such a point, the scale economies are effectively exhausted, and larger plant sizes no longer give rise to lower ATC curves. Figure~\ref{fig:longshortavg} implies that once we get to a production output that plant size 2 is geared to produce, there are few further scale economies available. And, once we move to producing a very high output (which requires a plant size corresponding to $ATC_3$), we are actually encountering \terminology{diseconomies of scale}. We stated above that scale economies reduce unit costs. Correspondingly, diseconomies of scale imply that unit costs increase as a result of the firm's becoming too large: Perhaps co-ordination difficulties have set in at the very high output levels, or quality-control monitoring costs have risen.

The terms \terminology{increasing}, \terminology{constant}, and \terminology{decreasing returns to scale} underlie the concepts of scale economies and diseconomies: Increasing returns to scale (IRS) implies that, when all inputs are increased by a given proportion, output increases more than proportionately. Constant returns to scale (CRS) implies that output increases in direct proportion to an equal proportionate increase in all inputs. Decreasing returns to scale (DRS) implies that an equal proportionate increase in all inputs leads to a less than proportionate increase in output.

\begin{DefBox}
\textbf{Increasing returns to scale} implies that, when all inputs are increased by a given proportion, output increases more than proportionately.

\textbf{Constant returns to scale} implies that output increases in direct proportion to an equal proportionate increase in all inputs.

\textbf{Decreasing returns to scale} implies that an equal proportionate increase in all inputs leads to a less than proportionate increase in output.
\end{DefBox}

These are pure production function relationships, but, if the prices of inputs are fixed for producers, they translate directly into the various cost structures illustrated in Figure~\ref{fig:longshortavg}. For example, if a doubling of capital and Labour in BDS allows for better production flows than when in the smaller plant, and therefore yields more than double the output, this implies that the cost per snowboard produced must fall in the new plant. In contrast, if a doubling of inputs leads to less than a doubling of outputs then the cost per snowboard must rise. Between these extremes, there is a range of relatively constant unit costs, corresponding to where the production relation is subject to constant returns to scale. In Figure~\ref{fig:longshortavg}, the falling unit costs output region has increasing returns to scale, the region that has relatively constant unit costs has constant returns to scale, and the increasing cost region has decreasing returns to scale.

\subsection*{Sectors with increasing returns}

Increasing returns to scale characterize businesses with large initial costs and relatively low costs of producing each unit of output. Computer chip manufacturers, pharmaceutical manufacturers, even brewers all appear to benefit from scale economies. In the beer market, brewing, bottling and shipping are all low cost operations relative to the capital cost of setting up a brewery. Consequently we observe surprisingly few breweries in any given producer, even in large land mass economies such as Canada or the US. 

Large firms frequently encounter decreasing returns to scale. Figure~\ref{fig:longshortavg} contains a curve that forms an envelope around the bottom of the various short-run average cost curves. This envelope is the \terminology{long run average total cost} ($LATC$) curve, because it defines average cost as we move from one plant size to another. Remember that in the long run both labour and capital are variable, and, as we move from one short-run average cost curve to another, that is exactly what happens---all factors of production are variable. Hence, the collection of short-run cost curves in Figure~\ref{fig:longshortavg} provides the ingredients for a long-run average total cost curve\footnote{Note that the long-run average total cost is not the collection of minimum points from each short-run average cost curve. The envelope of the short-run curves will pick up mainly points that are not at the minimum, as you will see if you try to draw the outcome. The intuition behind the definition is this: With increasing returns to scale, it may be better to build a plant size that operates with some spare capacity than to build one that is geared to producing a smaller output level. In building the larger plant, we can take greater advantage of the scale economies, and it may prove less costly to produce in such a plant than to produce with a smaller plant that has less unused capacity and does not exploit the underlying scale economies. Conversely, in the presence of decreasing returns to scale, it may be less costly to produce output in a plant that is used ``overtime'' than to use a larger plant that suffers from scale diseconomies.}. 

\begin{ApplicationBox}{Decreasing returns to scale \label{app:decretscale}}
The CEO of Hewlett Packard announced in October 2012 that the company would reduce its labour force by 29,000 workers (out of a total of 350,000). The problem was that communications within the company were so bad as to increase costs. In addition the company was producing an excessive product variety -- 2,100 variants of laser printer.
\end{ApplicationBox}

\begin{equation*}
LATC=\frac{LTC}{Q}
\end{equation*}

\begin{DefBox}
\textbf{Long-run average total cost} is the lower envelope of all the short-run ATC curves.
\end{DefBox}

The particular range of output on the LATC where it begins to flatten out is called the range of \terminology{minimum efficient scale}. This is an important concept in industrial policy, as we shall see in later chapters. At such an output level, the producer has expanded sufficiently to take advantage of virtually all the scale economies available.

\begin{DefBox}
\textbf{Minimum efficient scale} defines a threshold size of operation such that scale economies are almost exhausted.
\end{DefBox}

\subsection*{Marginal costs in the long run}

Just as there is a relationship between the marginal and average cost curves in the short run, there is a complementary relationship between the two in the long run also. In the long run all costs are variable. Hence, the \terminology{long run marginal cost} ($LMC$) is defined as the increment in cost associated with producing one more unit of output when all inputs are adjusted in a cost minimizing manner. 

\begin{equation*}
LMC=\frac{\Delta LTC}{\Delta Q}
\end{equation*}

\begin{DefBox}
\textbf{Long run marginal cost} is the increment in cost associated with producing one more unit of output when all inputs are adjusted in a cost minimizing manner.
\end{DefBox}

Figure~\ref{fig:LMCLAC} illustrates a $LMC$ curve where there are initially declining marginal costs, then constant marginal costs and finally increasing marginal costs.  The associated data are presented in Table~\ref{table:lrcreturntoscale}. This example has been constructed with three linear segments to the $LMC$ curve in order to show the relationship between returns to scale and long run marginal costs: the $LATC$ still has a smooth U shape, and intersects the $LMC$ at the minimum of the $LATC$. The logic for this occurrence is precisely as in the short run case: whenever the marginal cost is less than the average cost the latter must fall, and conversely when the marginal cost is greater than the average cost. 

% Figure 8.6
\begin{FigureBox}{1}{1}{25em}{LMC and LAC with returns to scale \label{fig:LMCLAC}}{\centering The $LMC$ cuts the $LAC$ at the minimum of the $LAC$.}
\begin{axis}[
	axis line style=thick,
	every tick label/.append style={font=\footnotesize},
	every node near coord/.append style={font=\scriptsize},
	xticklabel style={anchor=north,/pgf/number format/1000 sep=},
	scaled y ticks=false,
	x=1cm/250,
	yticklabel style={/pgf/number format/fixed,/pgf/number format/1000 sep = \thinspace},
	xmin=0,xmax=2250,ymin=0,ymax=90,
	xlabel={Output},
	ylabel={Cost (\$)},
]
\addplot[dashed,lmccolour,ultra thick,mark=none] coordinates { % Long run marginal
	(0,80)
	(100,72)
	(200,64)
	(300,56)
	(400,48)
	(500,40)
	(600,40)
	(700,40)
	(800,40)
	(900,40)
	(1000,40)
	(1100,44)
	(1200,48)
	(1300,52)
	(1400,56)
	(1500,60)
	(1600,64)
	(1700,68)
	(1800,72)
	(1900,76)
	(2000,80)
} node [black,mynode,pos=0.9,above left] {Long run\\marginal};
\addplot[latccolour,ultra thick,mark=none] coordinates { % Long run average
	(0,80)
	(100,76)
	(200,72)
	(300,68)
	(400,64)
	(500,60)
	(600,56.67)
	(700,54.29)
	(800,52.5)
	(900,51.11)
	(1000,50)
	(1100,49.27)
	(1200,49)
	(1300,49.08)
	(1400,49.43)
	(1500,50)
	(1600,50.75)
	(1700,51.65)
	(1800,52.67)
	(1900,53.79)
	(2000,55)
} node [black,mynode,pos=0.9,below] {Long run\\average};
\end{axis}
\end{FigureBox}


% Table 8.3
\begin{table}[H]
\begin{center}
\begin{tabu} to \linewidth {|X[1,c]X[1,c]X[1,c]X[1.25,c]|} \hline 
\rowcolor{rowcolour}	\textbf{Output} & \textbf{$LMC$} & \textbf{$LAC$} & \textbf{Scale} \\
						0 	& 80 & 80.00 &  \multirow{13}{7em}{\textit{Increasing returns to scale}}	\\
\cellcolor{rowcolour}	100 & \cellcolor{rowcolour}72 & \cellcolor{rowcolour}76.00 &	\\
						200 & 64 & 72.00 &	\\
\cellcolor{rowcolour}	300 & \cellcolor{rowcolour}56 & \cellcolor{rowcolour}68.00 &	\\
						400 & 48 & 64.00 &	\\
\cellcolor{rowcolour}	500	& \cellcolor{rowcolour}40 & \cellcolor{rowcolour}60.00 &	\\
						600 & 40 & 56.67 &	\\
\cellcolor{rowcolour}	700 & \cellcolor{rowcolour}40 & \cellcolor{rowcolour}54.29 &	\\
						800 & 40 & 52.50 &	\\
\cellcolor{rowcolour}	900 & \cellcolor{rowcolour}40 & \cellcolor{rowcolour}51.11 &	\\
						1000 & 40 & 50.00 &	\\
\cellcolor{rowcolour}	1100 & \cellcolor{rowcolour}44 & \cellcolor{rowcolour}49.27 &	\\
						1200 & 48 & 49.00 &	\\
\rowcolor{rowcolour}	1300 & 52 & 49.08 &	\\[-0.1em]
						1400 & 56 & 49.43 &	\cellcolor{rowcolour}\\[-0.1em]
\rowcolor{rowcolour}	1500 & 60 & 50.00 &	\\[-0.1em]
						1600 & 64 & 50.75 &	\cellcolor{rowcolour}\\[-0.1em]
\rowcolor{rowcolour}	1700 & 68 & 51.65 &	\\[-0.1em]
						1800 & 72 & 52.67 &	\cellcolor{rowcolour}\\[-0.1em]
\rowcolor{rowcolour}	1900 & 76 & 53.79 &	\\[-0.1em]
						2000 & 80 & 55.00 &	\multirow{-8}{7em}{\cellcolor{rowcolour}\textit{Decreasing returns to scale}}\\	\hline
\end{tabu}
\end{center}
\caption{Long run costs and returns to scale \label{table:lrcreturntoscale}}
\end{table}