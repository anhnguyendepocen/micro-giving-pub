\section{The time frame}\label{sec:ch8sec2}

We distinguish initially between the \terminology{short run} and the \terminology{long run}. When discussing technological change, we use the term \terminology{very long run}. These concepts have little to do with clocks or calendars. Rather, they are defined by the degree of flexibility an entrepreneur or manager has in his production process. A key decision variable is capital.

A customary assumption is that a producer can hire more labour immediately, if necessary, either by taking on new workers (since there are usually some who are unemployed and looking for work), or by getting the existing workers to work longer hours. But getting new capital is more time consuming: the entrepreneur may have to place an order for new machinery, which will involve a production and delivery time lag. Or he may have to move to a more spacious location in order to accommodate the added capital. Whether this calendar time is one week, one month, or one year is of no concern to us. We define the long run as a period of sufficient length to enable the entrepreneur to adjust his capital stock, whereas in the short run at least one factor of production is fixed.

\begin{DefBox}
\textbf{Short run}: a period during which at least one factor of production is fixed. If capital is fixed, then more output is produced by using additional labour.

\textbf{Long run}: a period of time that is sufficient to enable all factors of production to be adjusted.

\textbf{Very long run}: a period sufficiently long for new technology to develop.
\end{DefBox}