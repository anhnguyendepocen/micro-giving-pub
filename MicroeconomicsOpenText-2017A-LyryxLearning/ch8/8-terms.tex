\newpage
\markboth{Key Terms}{Key Terms}
	\addcontentsline{toc}{section}{Key terms}
	\section*{\textsc{Key Terms}}
\begin{keyterms}
\textbf{Production function}: a technological relationship that specifies how much output can be produced with specific amounts of inputs.

\textbf{Technological efficiency} means that the maximum output is produced with the given set of inputs.

\textbf{Economic efficiency} defines a production structure that produces output at least cost.

\textbf{Short run}: a period during which at least one factor of production is fixed. If capital is fixed, then more output is produced by using additional labour.

\textbf{Long run}: a period of time that is sufficient to enable all factors of production to be adjusted.

\textbf{Very long run}: a period sufficiently long for new technology to develop.

\textbf{Total product} is the relationship between total output produced and the number of workers employed, for a given amount of capital.

\textbf{Marginal product of labour} is the addition to output produced by each additional worker. It is also the slope of the total product curve.

\textbf{Law of diminishing returns}: when increments of a variable factor (labour) are added to a fixed amount of another factor (capital), the marginal product of the variable factor must eventually decline.

\textbf{Average product of labour} is the number of units of output produced per unit of labour at different levels of employment.

\textbf{Fixed costs} are costs that are independent of the level of output.

\textbf{Variable costs} are related to the output produced.

\textbf{Total cost} is the sum of fixed cost and variable cost.

\textbf{Average fixed cost} is the total fixed cost per unit of output.

\textbf{Average variable cost} is the total variable cost per unit of output.

\textbf{Average total cost} is the sum of all costs per unit of output.

\textbf{Marginal cost} of production is the cost of producing each additional unit of output.

\textbf{Sunk cost} is a fixed cost that has already been incurred and cannot be recovered, even by producing a zero output.

\textbf{Increasing returns to scale} implies that, when all inputs are increased by a given proportion, output increases more than proportionately.

\textbf{Constant returns to scale} implies that output increases in direct proportion to an equal proportionate increase in all inputs.

\textbf{Decreasing returns to scale} implies that an equal proportionate increase in all inputs leads to a less than proportionate increase in output.

\textbf{Long-run average total cost} is the lower envelope of all the short-run ATC curves.

\textbf{Minimum efficient scale} defines a threshold size of operation such that scale economies are almost exhausted.

\textbf{Long run marginal cost} is the increment in cost associated with producing one more unit of output when all inputs are adjusted in a cost minimizing manner.

\textbf{Technological change} represents innovation that can reduce the cost of production or bring new products on line.

\textbf{Globalization} is the tendency for international markets to be ever more integrated.

\textbf{Cluster}: a group of firms producing similar products, or engaged in similar research.

\textbf{Learning by doing} can reduce costs. A longer history of production enables firms to accumulate knowledge and thereby implement more efficient production processes.

\textbf{Economies of scope} occur if the unit cost of producing particular products is less when combined with the production of other products than when produced alone.
\end{keyterms}