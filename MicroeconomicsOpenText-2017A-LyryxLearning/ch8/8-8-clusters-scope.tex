\section{Clusters, learning by doing, scope economies}\label{sec:ch8sec8}
 
\subsection*{Clusters}

The phenomenon of a grouping of firms that specialize in producing related products is called a \terminology{cluster}. For example, Ottawa has more than its share of software development firms; Montreal has a disproportionate share of Canada's pharmaceutical producers and electronic game developers; Calgary has its `oil patch'; Hollywood has movies; Toronto is Canada's financial capital. Provincial capitals have most of their province's bureaucracy. Clusters give rise to externalities, frequently in the form of ideas that flow between firms, which in turn result in cost reductions and new products.

\begin{DefBox}
\textbf{Cluster}: a group of firms producing similar products, or engaged in similar research.
\end{DefBox}

The most famous example of clustering is Silicone Valley in California, the original high-tech cluster. The presence of a large group of firms with a common focus serves as a signal to workers with the complementary skill set that they are in demand in such a region. Furthermore, if these clusters are research oriented, as they frequently are, then knowledge spillovers benefit virtually all of the contiguous firms; when workers change employers, they bring their previously-learned skills with them; on social occasions, friends may chat about their work and interests and share ideas. This is a positive externality.

\subsection*{Learning by doing}

Learning from production-related experiences frequently reduces costs: The accumulation of knowledge that is associated with having produced a large volume of output over a considerable time period enables managers to implement more efficient production methods and avoid errors. We give the term \terminology{learning by doing} to this accumulation of knowledge. 

Examples abound, but the best known may be the continual improvement in the capacity of computer chips, whose efficiency has doubled about every eighteen months for several decades -- a phenomenon known as Moore's Law. Past experience is key: learning can be a facilitator in economic growth on a national scale. Japan, Korea and Taiwan learned from the West in the seventies and eighties. The phenomenal economic growth of China and India today is made possible by existing knowledge.

\begin{DefBox}
\textbf{Learning by doing} can reduce costs. A longer history of production enables firms to accumulate knowledge and thereby implement more efficient production processes.
\end{DefBox}

\subsection*{Economies of scope}

\terminology{Economies of scope} refers to the possibility that it may be less expensive to produce and sell a line of related goods than to produce just one product from such a line. Scope economies, therefore, define the returns or cost reductions associated with broadening a firm's product range. 

Corporations like Proctor and Gamble do not produce a single product in their health line; rather, they produce first aid, dental care, and baby care products. Cable companies offer their customers TV, high-speed Internet, and voice-over telephone services either individually or packaged.

\begin{DefBox}
\textbf{Economies of scope} occur if the unit cost of producing particular products is less when combined with the production of other products than when produced alone.
\end{DefBox}