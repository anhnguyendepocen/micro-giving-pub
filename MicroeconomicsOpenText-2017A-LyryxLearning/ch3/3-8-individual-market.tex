\section{Individual and market functions}\label{sec:ch3sec8}

Markets are made up of many individual participants on the demand and supply side. The supply and demand functions that we have worked with in this chapter are those for the total of all participants on each side of the market. But how do we arrive at such market functions when the economy is composed of individuals? We can illustrate how, with the help of Figure~\ref{fig:individualdemand}.

% Figure 3.9
\begin{FigureBox}{0.6}{0.3}{25em}{Summing individual demands \label{fig:individualdemand}}{At $P_1$ individual A purchases $Q_{A1}$ and B purchases $Q_{B1}$. The total demand is the sum of these individual demands at this price ($Q_1$). At $P_2$ individual demands are summed to $Q_2$. Since the points $Q_1$ and $Q_2$ define the demands of the market participants it follows that market demand is the horizontal sum of these curves.}
	\draw [demandcolour,ultra thick,name path=demandA] (0,12) -- coordinate [pos=0.4] (A) (6,0);
	\draw [demandcolour,ultra thick,name path=demandB] (0,12) -- coordinate [pos=0.25] (B) (9,0);
	\draw [demandcolour,ultra thick,name path=demandAB] (0,12) -- (15,0) node [black,mynode,midway,above right] {$D_{market}$=sum of A \& B demands};
	\draw [thick, -] (0,15) node (yaxis) [above] {Price} |- (18,0) node (xaxis) [right] {Quantity};
	% horizontal paths for P_1 and P_2
	\path [name path=P1line] (0,8) -- +(15,0);
	\path [name path=P2line] (0,3.5) -- +(15,0);
	% intersection of demand lines with P1line
	\draw [name intersections={of=demandA and P1line,by=QB1},name intersections={of=demandB and P1line,by=QA1},name intersections={of=demandAB and P1line,by=Q1}]
	[dotted,thick] (yaxis |- QB1) node [mynode,left] {$P_1$} -| (xaxis -| Q1) node [mynode,below] {$Q_1$}
	[dotted,thick] (QB1) -- (xaxis -| QB1) node [mynode,below] {$Q_{B1}$}
	[dotted,thick] (QA1) -- (xaxis -| QA1) node [mynode,below] {$Q_{A1}$};
	% intersection of demand lines with P2line
	\draw [name intersections={of=demandA and P2line,by=QB2},name intersections={of=demandB and P2line,by=QA2},name intersections={of=demandAB and P2line,by=Q2}]
	[dotted,thick] (yaxis |- QB2) node [mynode,left] {$P_2$} -| (xaxis -| Q2) node [mynode,below] {$Q_2$}
	[dotted,thick] (QB2) -- (xaxis -| QB2) node [mynode,below] {$Q_{B2}$}
	[dotted,thick] (QA2) -- (xaxis -| QA2) node [mynode,below] {$Q_{A2}$};
	% arrows to demandA and demandB
	\draw [<-,thick,shorten <=1mm,shorten >=1mm] (A) to[out=15,in=250] +(5,5) node [mynode,right=0cm and -0.2cm] {Individual B's demand};
	\draw [<-,thick,shorten <=1mm,shorten >=1mm] (B) to[out=15,in=250] +(5,5) node [mynode,right=0cm and -0.2cm] {Individual A's demand};
\end{FigureBox}

To concentrate on the essentials, imagine that there are just two buyers of gasoline in the economy. A has a bigger car than B, so his demand is greater. To simplify, let the two demands have the same intercept on the vertical axis. The curves $D_A$ and $D_B$ indicate how much gasoline A and B, respectively, will buy at each price. The market demand indicates how much they buy \textit{together} at any price. Accordingly, at $P_1$, A and B purchase the quantities $Q_{A1}$ and $Q_{B1}$ respectively. At a price $P_2$, they purchase $Q_{A2}$ and $Q_{B2}$. The \terminology{market demand} is therefore the horizontal sum of the individual demands at these prices. In the figure this is defined by $D_{market}$.

\begin{DefBox}
\textbf{Market demand}: the horizontal sum of individual demands.
\end{DefBox}

\subsection*{Conclusion}

We have covered a lot of ground in this chapter. It is intended to open up the vista of economics to the new student in the discipline. Economics is powerful and challenging, and the ideas we have developed here will serve as conceptual foundations for our exploration of the subject. Our next chapter deals with measurement and responsiveness.