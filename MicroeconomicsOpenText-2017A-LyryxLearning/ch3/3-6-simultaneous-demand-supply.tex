\section{Simultaneous supply and demand impacts}\label{sec:ch3sec6}

In the real world, demand and supply frequently shift at the same time. We present two very real such cases in Figures~\ref{fig:demandsupplyshift} and \ref{fig:housedemandsupply}.

Figure~\ref{fig:demandsupplyshift} is a development of the gas market already discussed. In 2003/04 the price of oil sat at about \$30 US per barrel. By the end of the decade it had climbed to \$100 per barrel. During the same period the price of natural gas dropped from the \$8 per unit range to \$3 per unit. A major factor in generating this decline was the development of new `fracking' technologies -- the retrieval of gas from shale formations. These technologies are not widespread in Canada due to concerns about their environmental impact, but have been adopted on a large scale in the US. Cheaper production has led to a substantial shift in supply, at the same time as users were demanding more gas due to the rising price of oil. Figure~\ref{fig:demandsupplyshift} illustrates a simultaneous shift in both functions, with the dominant impact coming from the supply side.

Our second example comes from data on a small Montreal municipality. Vertical curves define the supply side of the market. Such vertical curves mean that a fixed number of homeowners decide to put their homes on the market, and these suppliers just take whatever price results in the market. In this example, fewer houses were offered for sale in 2002 (less than 50) than in 1997 (more than 70). 

During this time period household incomes increased substantially and, also, mortgage rates fell. Both of these developments shifted the demand curve upward/outward: buyers were willing to pay more for housing in 2002 than in 1997. The higher price in 2002 was therefore due to \textit{both} demand and supply side shifts in the marketplace. 

% Figure 3.4
\begin{FigureBox}{0.4}{0.3}{25em}{Simultaneous demand and supply shifts \label{fig:demandsupplyshift}}{The outward shift in supply dominates the outward shift in demand, leading to a new equilibrium $E_1$ at a lower price and higher quantity.}
\draw [demandcolour,ultra thick,name path=demand0] (0,10) -- (10,0);
\draw [demandcolour,ultra thick,name path=demand1] (0,13) -- (13,0);
\draw [supplycolour,ultra thick,name path=supply0] (0.5,4.25) -- (10,9);
\draw [supplycolour,ultra thick,name path=supply1] (2,2) -- (12,7);
\draw [thick, -] (0,15) node (yaxis) [above] {Price} |- (15,0) node (xaxis) [right] {Quantity};
% intersection of demand0 and supply0
\draw [name intersections={of=supply0 and demand0,by=E0}]
	[dotted,thick] (yaxis |- E0) node [mynode,left] {$P_0$} -- (E0) node [mynode,above] {$E_0$} -- (xaxis -| E0) node [mynode,below] {$Q_0$};
% intersection of demand1 and supply1
\draw [name intersections={of=supply1 and demand1,by=E1}]
	[dotted,thick] (yaxis |- E1) node [mynode,left] {$P_1$} -- (E1) node [mynode,above] {$E_1$} -- (xaxis -| E1) node [mynode,below] {$Q_1$};
% arrow between old and new demand
\path [name path=demandline] (0,8) -- (7,13);
\draw [name intersections={of=demandline and demand0,by=D0},name intersections={of=demandline and demand1,by=D1}]
	[->,thick,shorten >=1mm,shorten <=1mm] (D0) -- (D1);
% arrow between old and new supply
\path [name path=supplyline] (5,12) -- (12,5);
\draw [name intersections={of=supplyline and supply0,by=S0},name intersections={of=supplyline and supply1,by=S1}]
	[->,thick,shorten >=1mm,shorten <=1mm] (S0) -- (S1);
\end{FigureBox}


% Figure 3.5
\begin{FigureBox}{0.3}{0.25}{25em}{A model of the housing market with shifts in demand and supply \label{fig:housedemandsupply}}{The vertical supply denotes a fixed number of houses supplied each year. Demand was stronger in 2002 than in 1997 both on account of higher incomes and lower mortgage rates. Thus the higher price in 2002 is due to both a reduction in supply and an increase in demand.}
\draw [demandcolour,ultra thick,name path=demand2002] (5,18) node [black,mynode,left] {$D_{2002}$} -- (17,12);
\draw [demandcolour,ultra thick,name path=demand1997] (5,11.25) node [black,mynode,left] {$D_{1997}$} -- (16,8.5);
\draw [supplycolour,ultra thick,name path=supply2002] (9,0) -- (9,19) node [black,mynode,above] {$S_{2002}$};
\draw [supplycolour,ultra thick,name path=supply1997] (14,0) -- (14,19) node [black,mynode,above] {$S_{1997}$};
\draw [thick, -] (0,20) node (yaxis) [mynode1,above] {Price in\\ \$000} |- (20,0) node (xaxis) [right] {Quantity};
% intersection of supply and demand lines
\draw [name intersections={of=demand2002 and supply2002,by=E2002},name intersections={of=demand1997 and supply1997,by=E1997}]
	[dotted,thick] (yaxis |- E2002) -- (E2002) node [mynode,above right] {$E_{2002}$}
	[dotted,thick] (yaxis |- E1997) -- (E1997) node [mynode,above right] {$E_{1997}$};
% axis markers
\draw [thick] (5,0) node [mynode,below] {25} -- +(0,0.2) -- +(0,-0.2);
\draw [thick] (10,0) node [mynode,below] {50} -- +(0,0.2) -- +(0,-0.2);
\draw [thick] (15,0) node [mynode,below] {75} -- +(0,0.2) -- +(0,-0.2);
\draw [thick] (0,5) node [mynode,left] {100} -- +(0.2,0) -- +(-0.2,0);
\draw [thick] (0,10) node [mynode,left] {200} -- +(0.2,0) -- +(-0.2,0);
\draw [thick] (0,15) node [mynode,left] {300} -- +(0.2,0) -- +(-0.2,0);
\end{FigureBox}