\section{Demand and supply curves}\label{sec:ch3sec3}

The \terminology{demand curve} is a graphical expression of the relationship between price and quantity demanded, holding other things constant. Figure~\ref{fig:sdeq} measures price on the vertical axis and quantity on the horizontal axis. The curve $D$ represents the data from the first two columns of Table~\ref{table:dsnaturalgas}. Each combination of price and quantity demanded lies on the curve. In this case the curve is \textit{linear}---it is a straight line. The demand curve slopes downward (technically we say that its slope is negative), reflecting the fact that buyers wish to purchase more when the price is less.

\begin{DefBox}
The \textbf{demand curve} is a graphical expression of the relationship between price and quantity demanded, with other influences remaining unchanged.
\end{DefBox}

The \terminology{supply curve} is a graphical representation shows the relation between price and quantity supplied, holding other things constant. The supply curve $S$ in Figure~\ref{fig:sdeq} is based on the data from columns 1 and 3 in Table~\ref{table:dsnaturalgas}. It, too, is linear, but has a positive slope indicating that suppliers wish to supply more at higher prices.

\begin{DefBox}
The \textbf{supply curve} is a graphical expression of the relationship between price and quantity supplied, with other influences remaining unchanged.
\end{DefBox}

% figure 3.1
\begin{FigureBox}{0.4}{0.3}{25em}{Supply, demand, equilibrium \label{fig:sdeq}}{}
\draw [demandcolour,ultra thick,name path=demand] (0,10) node [black,mynode,left] {10} -- node [mynode,black,above right,pos=0.25] {Demand} (10,0) node [black,mynode,below] {10};
\draw [supplycolour,ultra thick,name path=supply] (0,1) node [black,mynode,left] {1} -- (10,6) node [black,mynode,above] {Supply};
\draw [thick, -] (0,15) node (yaxis) [above] {Price} |- (15,0) node (xaxis) [right] {Quantity};
% intersection of demand and supply and dotted line
\draw [name intersections={of=demand and supply, by=E}]
	[dotted,thick] (yaxis |- E) node [mynode,left] {4} -- (E) node [mynode,above] {$E_0$} -- (xaxis -| E) node [mynode,below] {6};
\end{FigureBox}

The demand and supply curves intersect at point $E_0$, corresponding to a price of \$4 which, as illustrated above, is the equilibrium price for this market. At any price below this the horizontal distance between the supply and demand curves represents excess demand, because demand exceeds supply. Conversely, at any price above \$4 there is an excess supply that is again measured by the horizontal distance between the two curves. Market forces tend to eliminate excess demand and excess supply as we explained above.

\subsection*{Computing the market equilibrium}

It is not difficult to represent the supply and demand functions underlying Table~\ref{table:dsnaturalgas} in their mathematical form:

\begin{equation*}
\text{Demand: }P=10-Q
\end{equation*}

\begin{equation*}
\text{Supply: }P=1+(1/2)Q
\end{equation*}

In the previous chapter we stated that a straight line is represented completely by the intercept and slope. Let us first verify that these equations do, indeed, represent the data in Table~\ref{table:dsnaturalgas}. On the demand side, we see that a zero quantity is demanded at a price of \$10, and this is therefore the intercept with the price (vertical) axis. To see this just set $P=10$ in the demand equation. As for the slope, each unit change in quantity demanded (measured in thousands) is associated with a \$1 change in price. For instance, when the price is increased by \$2, the quantity demanded declines by 2 units. In reverse, if the price is lowered by \$2, the quantity demanded increases by 2 units. Since the price is on the vertical axis, it follows that the slope is given by $-\$1/1=-1$. It is negative because an increase in quantity demanded is associated with a decrease in price.

On the supply side, column 3 in Table~\ref{table:dsnaturalgas} indicates that at a quantity of zero the price is \$1. Therefore, \$1 is the price intercept. As for the slope, each 2-unit change in quantity is associated with a change in price of \$1. Consequently, the slope is given by $\$1/2=1/2$. In this case the slope is positive, since both the price and quantity move in the same direction.

We have now obtained the two defining characteristics of the demand and supply curves, which enable us to write them as above. Next we must find where they intersect -- the market equilibrium. Since, at their intersection point, the price on the demand curve equals the price on the supply curve, and the quantity demanded equals the quantity supplied, this unique price-quantity combination is obtained by equating the two curves:

\begin{equation*}
D=S\Rightarrow 10-Q=1+(1/2)Q\Rightarrow 10-1=Q+(1/2)Q\Rightarrow 9=1.5Q\footnote{The $\Rightarrow$ symbol is used in mathematics to denote ``implication''. For example, $A\Rightarrow B$ translates to ``If $A$, then $B$.''}
\end{equation*}

Therefore,

\begin{equation*}
Q=9/1.5=6
\end{equation*}

The \textit{equilibrium solution} for $Q$ is therefore 6 units. What about an equilibrium price? It is obtained by inserting the equilibrium $Q$ value into \textit{either the supply or the demand function}. \textit{Either} function can be used because, where $Q=6$, the supply and demand functions intersect -- they have equal $P$ values:

\begin{equation*}
\text{Demand price at $Q$=6: }P=10-1\times 6 =10-6=4
\end{equation*}
\begin{equation*}
\text{Supply price at $Q$=6: }P=1+1/2\times 6=1+3=4
\end{equation*}

We have just solved a mathematical model of a particular market! It was not so difficult, but the method is very powerful and we will use it many times in the text.

In the demand and supply equations above the price appeared on the left hand side and quantity on the right. Normally this format implies a causation running from the right to the left hand side variable, while in economic markets we normally think of the quantity demanded and supplied depending upon the price in the market place. But the supply and demand equations can be rearranged so that quantity appears on the left and price on the right. For example the demand equation can be rewritten as follows:

\begin{equation*}
P=10-Q\Rightarrow  Q=10-P,
\end{equation*}
\begin{equation*}
\text{or: }Q=10-P\Rightarrow Q=10-P.
\end{equation*}

Writing the demand curve this way illustrates that the quantity intercept is 10 -- the quantity demanded when the price becomes zero. The supply curve can be rearranged similarly.