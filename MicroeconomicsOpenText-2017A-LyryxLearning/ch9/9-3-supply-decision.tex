\section{The firm's supply decision}\label{sec:ch9sec3}

The concept of \terminology{marginal revenue} is key to analyzing the supply decision of an individual firm. We have used marginal analysis at several points to date. In consumer theory, we saw how consumers balance the utility per dollar \textit{at the margin} in allocating their budget. Marginal revenue is the additional revenue accruing to the firm from the sale of one more unit of output.

\begin{DefBox}
\textbf{Marginal revenue} is the additional revenue accruing to the firm resulting from the sale of one more unit of output.
\end{DefBox}

In perfect competition, a firm's marginal revenue ($MR$) is the price of the good. Since the price is constant for the individual supplier, each additional unit sold at the price $P$ brings in the same additional revenue. Therefore, $P=MR$. This equality holds in no other market structure, as we shall see in the following chapters.

\subsection*{Supply in the short run}

Each firm's plant size is fixed in the \terminology{short run}, so too is the number of firms in an industry. In the \terminology{long run}, each individual firm can change its scale of operation, and at the same time new firms can enter or existing firms can leave the industry.

\begin{DefBox}
The \textbf{short run} is a period during which the number of firms and their plant sizes are fixed.

The \textbf{long run} is a sufficiently long period of time to permit entry and exit and for firms to change their plant size.
\end{DefBox}

Perfectly competitive suppliers face the choice of how much to produce at the going market price: that is, the amount that will maximize their profit. We abstract for the moment on how the price in the marketplace is determined. We shall see later in this chapter that it emerges as the value corresponding to the intersection of the supply and demand curves for the whole market -- much as described in Chapter~\ref{chap:classical}. 

The firm's $MC$ curve is critical in defining the optimal amount to supply at any price. In Figure~\ref{fig:optoutput}, $MC$ is the firm's marginal cost curve in the short run. At the price $P_0$ the optimal amount to supply is $Q_0$, the amount determined by the intersection of the $MC$ and the demand. To see why, imagine that the producer chose to supply the quantity $Q_1$. Such an output would leave the opportunity for further profit untapped. By producing one additional unit, the supplier would get $P_0$ in additional revenue and incur a smaller additional cost in producing those units. In fact, on every unit between $Q_1$ and $Q_0$ he can make a profit, because the $MR$ exceeds the associated cost, $MC$. By the same argument, it makes no sense to increase output beyond $Q_0$, to $Q_2$ for example, because the cost of such additional units of output, $MC$, exceeds the revenue from them. \textit{The $MC$ therefore defines an optimal supply response}.

% Application box 9.1
\begin{ApplicationBox}{The law of one price \label{app:lawoneprice}}
If information does not flow then prices in different parts of a market may differ and potential entrants may not know to enter a profitable market.

\bigskip
Consider the fishermen off the coast of Kerala, India in the late 1990s. Their market was studied by Robert Jensen, a development economist. Prior to 1997, fishermen tended to bring their fish to their home market or port. This was cheaper than venturing to other ports, particularly if there was no certainty regarding price. For the most part, prices were high in some local markets and low in others -- depending upon the daily catch. Frequently fish was thrown away in low-price markets even though it might have found a favourable price in another village's fish market.

\bigskip
This all changed with the advent of cell phones. Rather than head automatically to their home port, fishermen began to phone several different markets in the hope of finding a good price for their efforts. They began to form agreements with buyers before even bringing their catch to port. Economist Jensen observed a major decline in price variation between the markets that he surveyed. In effect the `law of one price' came into being for sardines as a result of the introduction of cheap technology and the relatively free flow of information. 
\end{ApplicationBox}

But how low can the price go before supply becomes unprofitable? To understand the supply decision further, in Figure~\ref{fig:shortrunsupply} the firm's $AVC$ and $ATC$ curves have been added to Figure~\ref{fig:optoutput}.

% Figure 9.2
\begin{FigureBox}{0.3}{0.25}{25em}{Short-run supply for the competitive firm \label{fig:shortrunsupply}}{A price below $P_1$ does not cover variable costs, so the firm should shut down. Between prices $P_1$ and $P_3$, the producer can cover variable, but not total, costs and therefore should produce in the short run if fixed costs are `sunk'. In the long-run the firm must close if the price does not reach $P_3$. Profits are made if the price exceeds $P_3$. The short-run supply curve is the quantity supplied at each price. It is therefore the $MC$ curve above $P_1$.}
% MC curve
\draw [dashed,mccolour,ultra thick,domain=3:18,name path=MC] plot (\x, {0.25*pow(1.25,\x)+4}) node [mynode,right,black] {$MC$};
% AVC curve
\draw [avccolour,ultra thick,domain=3:16,name path=AVC] plot (\x, {-1*sqrt(16-(\x-10)*(\x-10)/4)+10.32831}) node [black,mynode,right] {$AVC$};
% ATC curve
\draw [atccolour,ultra thick,domain=8:22,name path=ATC] plot (\x, {-1*sqrt(16-(\x-15)*(\x-15)/4)+15.1054}) node [black,mynode,right] {$ATC$};
% axes
\draw [thick, -] (0,20) node (yaxis) [above] {Price} |- (25,0) node (xaxis) [right] {Quantity};
% intersection of MC with AVC and ATC
\draw [name intersections={of=MC and AVC, by=P1},name intersections={of=MC and ATC, by=P3}]
	[dotted,thick] (yaxis |- P1) node [mynode,left] {$P_1$} -- (P1)
	[dotted,thick] (yaxis |- P3) node [mynode,left] {$P_3$} -- (P3);
% paths for P2 and P4
\path [name path=P2line] (0,9) -- +(25,0);
\path [name path=P4line] (0,14) -- +(25,0);
% intersection of paths with MC
\draw [name intersections={of=MC and P2line, by=P2},name intersections={of=MC and P4line, by=P4}]
	[dotted,thick] (yaxis |- P2) node [mynode,left] {$P_2$} -- (P2)
	[dotted,thick] (yaxis |- P4) node [mynode,left] {$P_4$} -- (P4);
% arrow for Shut down point
\draw [<-,thick,shorten <=1mm,shorten >=-2mm] (P1) -- +(2,-2) node [mynode,below right] {Shut-down\\point};
% arrow for Break even point
\draw [<-,thick,shorten <=2mm] ([xshift=5pt]P3) -- +(4,4) node [mynode,right] {Break-even\\point};
\end{FigureBox}

First, note that any price below $P_3$, which corresponds to the minimum of the $ATC$ curve, yields no profit, since it does not enable the producer to cover all of his costs. This price is therefore called the \terminology{break-even price}. Second, any price below $P_1$, which corresponds to the minimum of the $AVC$, does not even enable the producer to cover variable costs. What about a price such as $P_2$, that lies between these? The answer is that, if the supplier has already incurred some fixed costs, he should continue to produce, provided he can cover his variable cost. But in the long run he must cover all of his costs, fixed and variable. Therefore, if the price falls below $P_1$, he should shut down, even in the short run. This price is therefore called the \terminology{shut-down price}. If a price at least equal to $P_3$ cannot be sustained in the long run, he should leave the industry. But at a price such as $P_2$ he can cover variable costs and therefore should continue to produce in the short run. The firm's \terminology{short-run supply curve} is, therefore, that portion of the $MC$ curve above the minimum of the $AVC$. 

To illustrate this more concretely, let's go back to the example of our snowboard producer, and imagine that he is producing in a perfectly competitive marketplace. How should he behave in response to different prices? Table~\ref{table:profitmaxsr} reproduces the data from Table~\ref{table:snowprodcost}.

\begin{Table}{25em}{Profit maximization in the short run \label{table:profitmaxsr}}{\centering Output Price=\$70; Wage=\$1,000; Fixed Cost=\$3,000}\footnotesize
\begin{tabu} to \linewidth {|X[0.7,c]X[0.7,c]X[1,c]X[1,c]X[1,c]X[0.9,c]X[0.7,c]X[1.9,c]|} \hline 
\rowcolor{rowcolour}	\textbf{Labour} & \textbf{Output} & \textbf{Total} & \textbf{Average} & \textbf{Average} & \textbf{Marginal} & \textbf{Total} & \textbf{Production} \\[-0.4em]
\rowcolor{rowcolour}	&	&	\textbf{Revenue \$}	&	\textbf{Variable}	&	\textbf{Total Cost}	&	\textbf{Cost \$}	&	\textbf{Cost \$}	&	 \textbf{Decision}	\\[-0.4em]
\rowcolor{rowcolour}	&	&	&	\textbf{Cost}	&	\textbf{\$}	&	&	&	\\
\textbf{L} & \textbf{Q} & \textbf{TR} & \textbf{AVC} & \textbf{ATC} & \textbf{MC} & \textbf{TC} &  \\
\cellcolor{rowcolour}0 & \cellcolor{rowcolour}0 & \cellcolor{rowcolour} & \cellcolor{rowcolour} & \cellcolor{rowcolour} & \cellcolor{rowcolour} & \cellcolor{rowcolour} & \multirow{6}{10em}{No production where $P$ $<$ min $AVC$} \\
1 & 15 & 1,050 & 66.7 & 266.7 & 266.7 & 4,000 &  \\
\cellcolor{rowcolour}2 & \cellcolor{rowcolour}40 & \cellcolor{rowcolour}2,800 & \cellcolor{rowcolour}50.0 & \cellcolor{rowcolour}125.0 & \cellcolor{rowcolour}40.0 & \cellcolor{rowcolour}5,000 &  \\
3 & 70 & 4,900 & 42.9 & 85.7 & 33.3 & 6,000 &  \\
\cellcolor{rowcolour}4 & \cellcolor{rowcolour}110 & \cellcolor{rowcolour}7,700 & \cellcolor{rowcolour}36.4 & \cellcolor{rowcolour}63.6 & \cellcolor{rowcolour}25.0 & \cellcolor{rowcolour}7,000 &  \\
5 & 145 & 10,150 & 34.5 & 55.2 & 28.6 & 8,000 &  \\
\rowcolor{rowcolour}	6 & 175 & 12,250 & 34.3 & 51.4 & 33.3 & 9,000 & Min $AVC$ \\
7 & 200 & 14,000 & 35.0 &  50.0 &  40.0 & 10,000 & Price covers $AVC$, not $ATC$ \\[0.1em] 
\rowcolor{rowcolour}	8 & 220 & 15,400 & 36.4 &  50.0 &  50.0 & 11,000 &  \\
9 & 235 & 16,450 & 38.3 &  51.1 &  66.7 & 12,000 & \cellcolor{rowcolour} \\
\rowcolor{rowcolour}	10 & 240 & 16,800 & 41.7 &  54.2 & 200.0 & 13,000 &	\multirow{-3}{10em}{Profit where $P$ $>$ min $ATC$, and supply where $P=MC$}\\	\hline
\end{tabu}
\end{Table}

\begin{DefBox}
The \textbf{shut-down price} corresponds to the minimum value of the $AVC$ curve.

The \textbf{break-even price} corresponds to the minimum of the $ATC$ curve.

The firm's \textbf{short-run supply curve} is that portion of the $MC$ curve above the minimum of the $AVC$.
\end{DefBox}

Suppose that the price is \$70. How many boards should he produce? The answer is defined by the behaviour of the $MC$ curve. For any output less than or equal to 235, the $MC$ is less than the price. For example, at $L=9$ and $Q=235$, the $MC$ is \$66.7. At this output level, he makes a profit on the marginal unit produced, because the $MC$ is less than the revenue he gets from selling it. 

But, at outputs above this, he registers a loss on the marginal units because the $MC$ exceeds the revenue. For example, at $L=10$ and $Q=240$, the $MC$ is \$200. Clearly, 235 snowboards is the optimum. To produce more would generate a loss on each additional unit, because the additional cost would exceed the additional revenue. Furthermore, to produce fewer snowboards would mean not availing of the potential for profit on additional boards.

His profit is based on the difference between revenue per unit and cost per unit at this output: $(P-ATC)$. Since the $ATC$ for the 235 units produced by the nine workers is \$51.1, his profit margin is \$70 - \$51.1 = \$18.9 per board, and total profit is therefore 235 $\times$ \$18.9 = \$4441.5.

Let us establish two other key outputs and prices for the producer. First, the shut-down point is the minimum of his $AVC$ curve. Table~\ref{table:profitmaxsr} tells us that the price must be at least \$34.3 for him to be willing to supply \textit{any} output, since that is the value of the $AVC$ at its minimum. Second, the minimum of his $ATC$ is at \$50. Accordingly, provided the price exceeds \$50, he will cover both variable and fixed costs and make a maximum profit when he chooses an output where $P=MC$, above $P=\$50$. Finally we can specify the short run supply curve for Black Diamond Snowboards: it is the segment of the $MC$ curve in Figure~\ref{fig:AMCcurve} above the $AVC$ curve.

Given that we have developed the individual firm's supply curve, the next task is to develop the \underbar{industry} supply curve.

\subsection*{Industry supply in the short run}

In Chapter~\ref{chap:classical} it was demonstrated that individual demands can be aggregated into an industry demand by summing them horizontally. The industry supply is obtained in exactly the same manner---by summing the firms' supply quantities across all firms in the industry. 

To illustrate, imagine we have many firms, possibly operating at different scales of output and therefore having different short-run $MC$ curves. The $MC$ curves of two of these firms are illustrated in Figure~\ref{fig:industrysupply}. The $MC$ of A is below the $MC$ of B; therefore, B likely has a smaller scale of plant than A. Consider first the supply decisions in the price range $P_1$ to $P_2$. At any price between these limits, only firm A will supply output -- firm B does not cover its $AVC$ in this price range. Therefore, the joint contribution to industry supply of firms A and B is given by the $MC$ curve of firm A. But once a price of $P_2$ is attained, firm B is now willing to supply. The $S_{sum}$ schedule is the horizontal addition of their supply quantities. Adding the supplies of every firm in the industry in this way yields the \terminology{industry supply}.

% Figure 9.3
\begin{FigureBox}{0.3}{0.25}{25em}{Deriving industry supply \label{fig:industrysupply}}{At any price below $P_1$ production is unprofitable and supply is therefore zero for both firms. At prices between $P_1$ and $P_2$ firm A is willing to supply, but not firm B. Consequently the market supply comes only from A. At prices above $P_2$ both firms are willing to supply. Therefore the market supply is the horizontal sum of each firm's supply.}
\draw [dotted,thick]
	(0,6) node [mynode,left] {$P_1$} -- (5,6)
	(0,8) node [mynode,left] {$P_2$} -- (8,8);
% Market supply
\draw [marketsupplycolour!60,ultra thick]
(5,6) to [out=15,in=233] (8,8) -- (12,8) to [out=15,in=270] (17,16) node [black,below right,mynode,pos=0.5] {$S_{sum}=S_A+S_B$=market supply};
% individual supply
\draw [supplycolour,dashed,ultra thick]
	(4,8) to [out=15,in=270] (7,15) node [black,above,rotate=45,mynode] {$S_B=MC_B$}
	(5,6) to [out=15,in=270] (10,16) node [black,above,rotate=45,mynode] {$S_A=MC_A$};
% axes
\draw [thick, -] (0,20) node (yaxis) [above] {Price} |- (25,0) node (xaxis) [right] {Quantity};
\end{FigureBox}

\begin{DefBox}
\textbf{Industry supply (short run)} in perfect competition is the horizontal sum of all firms' supply curves.

\textbf{Short run equilibrium} in perfect competition occurs when each firm maximizes profit by producing a quantity where $P=MC$.
\end{DefBox}

\subsection*{Industry equilibrium}

Consider next the industry equilibrium. Since the industry supply is the sum of the individual supplies, and the industry demand curve is the sum of individual demands, an equilibrium price and quantity $(P_E,Q_E)$ are defined by the intersection of these industry-level curves, as in Figure~\ref{fig:marketeq}. Here, each firm takes $P_E$ as given (it is so small that it cannot influence the going price), and supplies an amount determined by the intersection of this price with its $MC$ curve. The sum of such quantities is therefore $Q_E$.

% Figure 9.4
\begin{FigureBox}{0.3}{0.25}{25em}{Market equilibrium \label{fig:marketeq}}{The market supply curve $S$ is the sum of each firm's supply or $MC$ curve above the shut-down price. $D$ is the sum of individual demands. The market equilibrium price and quantity are defined by $P_E$ and $Q_E$.}
% supply curve
\draw [supplycolour,ultra thick,domain=3:18,name path=S] plot (\x, {0.25*pow(1.25,\x)+4}) node [black,mynode,above right] {$S$=Sum of firm\\$MC$ curves};
% demand curve
\draw [demandcolour,ultra thick,domain=5:20,name path=D] plot (\x, {0.5*pow(1.25,-1*\x+20)+4}) node [black,mynode,right] {$D$=Sum of\\individual demands};
% axes
\draw [thick, -] (0,20) node (yaxis) [above] {Price} |- (25,0) node (xaxis) [right] {Quantity};
% intersection of demand and supply
\draw [name intersections={of=S and D, by=E}]
	[dotted,thick] (yaxis |- E) node [mynode,left] {$P_E$} -| (xaxis -| E) node [mynode,below] {$Q_E$};
\end{FigureBox}