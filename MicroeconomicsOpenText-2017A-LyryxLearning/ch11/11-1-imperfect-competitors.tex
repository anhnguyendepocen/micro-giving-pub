\section{Imperfect competitors}\label{sec:ch11sec1}

The preceding chapters have explored extreme forms of supply. The monopolist is the sole supplier and possesses as much market power as possible. In contrast the perfect competitor is insignificant in the marketplace and has no market power whatsoever. He simply accepts the price for his product that is determined in the market by the forces of supply and demand. These are very useful paradigms to explore, but the real world for the most part lies between these extremes. We observe that there are a handful of dominant brewers in Canada who supply more than three quarters of the market, and they are accompanied by numerous micro brewers that form the fringe of the brewing business. Likewise we have a small number of air carriers and one of them controls half of the national market. 

In this chapter we will explore the relationship between firm behaviour and the size of the firm relative to the whole sector. 

Rather than defining imperfect competitors in terms of the number of firms in the sector, or the share of total sales going to a small number of suppliers, we can differentiate imperfect competition from perfect competition by the characteristics of the demand curves they all face.  We know that a perfect competitor faces a perfectly elastic demand at the existing market price; this is the only market structure to have this characteristic. In all other market structures suppliers effectively face a downward-sloping demand. This means that they have some influence on the price of the good, and also that if they change the price they charge, they can expect demand to reflect this in a predictable manner. So we will lump all other market structures, including the extreme of monopoly, under the title of \terminology{imperfect competition}, and firms in such markets face downward sloping demands.

\begin{DefBox}
\textbf{Imperfectly competitive firms} face a downward-sloping demand curve, and their output price reflects the quantity sold.
\end{DefBox}

The demand curve for the firm and industry coincide for the monopolist, but not for other imperfectly competitive firms. It is convenient to categorize the producing sectors of the economy as either having a relatively small number of participants, or having a large number. The former market structures are called \terminology{oligopolistic}, and the latter are called \terminology{monopolistically competitive}. The word \textit{oligopoly} comes from the Greek word \textit{oligos} meaning few, and \textit{polein} meaning to sell.

\begin{DefBox}
\textbf{Oligopoly} defines an industry with a small number of suppliers.

\textbf{Monopolistic competition} defines a market with many sellers of products that have similar characteristics. Monopolistically competitive firms can exert only a small influence on the whole market.
\end{DefBox}

The home appliance industry is an oligopoly. The prices of \textit{KitchenAid} appliances depend not only on its own output and sales, but also on the prices of \textit{Whirlpool}, \textit{Maytag} and \textit{Bosch}. If a firm has just two main producers it is called a duopoly. \textit{Canadian National} and \textit{Canadian Pacific} are the only two major rail freight carriers in Canada. In contrast, the local Italian restaurant is a monopolistic competitor. Its output is a package of distinctive menu choices, personal service, and convenience for local customers. It can charge a slightly higher price than the out-of-neighbourhood restaurant. But if its prices are too high local shoppers will travel elsewhere. Many markets are defined by producers who supply similar but not identical products. Canada's universities all provide degrees, but the institutions differ one from another in their programs, their balance of in-class and on-line courses, their student activities, whether they are science based or liberal arts based, whether they have cooperative programs or not, and so forth. While universities are not in the business of making profit, they certainly wish to attract students, and one way of doing this is to differentiate themselves from other institutions. The profit-oriented world of commerce likewise seeks to increase its market share by distinguishing its product line.

\begin{DefBox}
\textbf{Duopoly} defines a market or sector with just two firms.
\end{DefBox}

These distinctions are not completely airtight. For example, if a sole domestic producer is subject to international competition it cannot act in the way we described in the previous chapter -- it has potential or actual competition. \textit{Bombardier} may be Canada's sole aircraft manufacturer, but it is not a monopolist, even in Canada. It could best be described as being part of an international oligopoly in mid-sized aircraft. Likewise, it is frequently difficult to delineate the boundary of a given market. For example, is \textit{Canada Post} a monopoly in mail delivery, or an oligopolist in hard-copy communication? We can never fully remove these ambiguities.

\subsection*{The role of cost structures}

A critical determinant of market structure is the way in which demand and cost interact to determine the likely number of market participants in a given sector or market. Structure also evolves over the long run: Time is required for entry and exit.

Figure~\ref{fig:demandcostmarketstructure} shows the demand curve $D$ for the output of an industry in the long run. Suppose, initially, that all firms and potential entrants face the long-run average cost curve $LATC_1$. At the price $P_1$, free entry and exit means that each firm produces $q_1$. With the demand curve $D$, industry output is $Q_1$. The number of firms in the industry is $N_1$ (=$Q_1/q_1$). If $q_1$, the minimum average cost output on $LATC_1$, is small relative to $D$, then $N_1$ is large. This outcome might be perfect competition -- $N$ virtually infinite -- or monopolistic competition -- $N$ large with slightly differentiated products produced by each firm.

% Figure 11.1
\begin{FigureBox}{0.3}{0.25}{25em}{Demand, costs and market structure \label{fig:demandcostmarketstructure}}{With a cost structure defined by $LATC_1$ this market has space for many firms -- perfect or monopolistic competition, each producing approximately $q_1$. If costs correspond to $LATC_2$, where scale economies are substantial, there may be space for just one producer. The intermediate case, $LATC_3$, can give rise to oligopoly, with each firm producing more than $q_1$ but less than a monopolist. These curves encounter their MES at very different output levels.}
% Demand line
\draw [demandcolour,ultra thick,name path=D] (1,24) -- (30,8) node [black,mynode,midway,above] {$D$};
% LATC curves
\draw [latccolour,ultra thick,name path=LATC1] (1,16) node [black,mynode,right] {$LATC_1$} to [out=-80,in=250] (7,16);
\draw [latccolour,ultra thick,name path=LATC2] (1,15) to [out=-45,in=180] (24,8) node [black,mynode,right] {$LATC_2$};
\draw [latccolour,ultra thick,name path=LATC3] (8,16) node [black,mynode,right] {$LATC_3$} to [out=-80,in=190] (20,12);
% axes
\draw [thick, -] (0,25) node (yaxis) [above] {\$} |- (30,0) node (xaxis) [right] {Quantity};
% path to minimum of LATC1
\path [name path=q1line] (4,0) -- +(0,25);
% intersection of q1line with LATC1
\draw [name intersections={of=q1line and LATC1, by=P1q1}]
[dotted,thick] (P1q1) -- (xaxis -| P1q1) node [mynode,below] {$q_1$};
% path to create dotted line intersecting with D
\path [name path=P1line] (yaxis |- P1q1) -- +(30,0);
% intersection of P1line with D
\draw [name intersections={of=P1line and D, by=P1Q1}]
[dotted,thick] (yaxis |- P1Q1) node [mynode,left] {$P_1$} -| (xaxis -| P1Q1) node [mynode,below] {$Q_1$};
\end{FigureBox}

Instead, suppose that the production structure in the industry is such that the long-run average cost curve is $LATC_2$. Here, scale economies are vast, relative to the market size. At the lowest point on this cost curve, output is large relative to the demand curve $D$. If this one firm were to act like a monopolist it would produce an output where $MR=MC$ in the long run and set a price such that the chosen output is sold. Given the scale economies, there may be no scope for another firm to enter this market, because such a firm would have to produce a very high output to compete with the existing producer. This situation is what we previously called a ``natural'' monopolist.

Finally, the cost structure might involve curves of the type $LATC_3$, which would give rise to the possibility of several producers, rather than one or very many. This results in oligopoly.

It is clear that one \textit{crucial determinant of market structure is minimum efficient scale relative to the size of the total market} as shown by the demand curve. The larger the minimum efficient scale relative to market size, the smaller is the number of producers in the industry.