\section{Entry, exit \& potential competition}\label{sec:ch11sec7}

At this point we inquire about the potential impact of new firms -- firms who might enter the industry if conditions were sufficiently enticing -- meaning high profits. Indeed we could ask if firms will necessarily enter a market such as the one described in the duopoly game above, if there is `freedom to enter and exit'. 

A simple way to illustrate the possibilities is to change that illustration very slightly: in addition to the constant marginal cost of producing the good, suppose there is a fixed cost associated with setting up production. This might comprise the research and development costs necessary to launch production, and such costs are incurred by each firm that participates in the market. 

In the foregoing example each of the two firms that competed for profit using Cournot behaviour earned a profit of \$36. Obviously then, if the fixed costs are less than this number, each firm continues to make a profit net of fixed costs, but if fixed costs are greater than \$36 both firms cannot be profitable. So let us assume fixed costs are low enough such that each firm is profitable after all costs are considered. In addition there are no legal constraints preventing entry. Will new firms enter and compete for the market? The answer is `not necessarily', even if the potential new firm has the same marginal and fixed costs: by entering, the market will now be split three ways and that may squeeze the profit margins of all three suppliers to such an extent that the operating margins are no longer sufficient to cover the fixed costs. If, at the time of entry, profit margins are low (fixed costs are almost large enough to swallow the operating margin of \$36) the addition of a third firm may put every firm into a state of negative profit. Hence no entry will occur, because the potential new entrant recognizes she cannot make a profit.

To complete the discussion, let's ask what would happen if fixed costs were greater than the operating profit of the two incumbents in the example as developed above. For example the government decided to levy a charge on firms that sell tobacco (as was done in the US in 2009). This charge effectively increases fixed costs. What are the dynamics of this situation? The answer here is that it may still be profitable for \textit{one} firm to make a profit if the other exits. This is because the exit of one firm will result in a monopoly. Not only would the single producer produce more than each duopolist (and thereby reduce average costs as a result of scale economies), he may also be able to charge a higher price in the absence of a competitor.

The economic reasoning behind the foregoing results is that there are returns to scale in this industry. As illustrated in Chapter~\ref{chap:monopoly}, if the fixed costs are large relative to marginal costs, then the average cost continually declines, and it may be necessary to produce a large output in order to reduce average costs sufficiently to be competitive. The problem with having a third firm considering entry to a profitable duopoly is that each firm would be constrained to produce a reduced output, and such an output may be too small to enable them make a profit given the demand conditions. Clearly then, there is a scale economies reason for entry not materializing; the incumbent firms need not take any preemptive measures to prevent entry.

\subsection*{Deterring entry}

In contrast to this case consider now a monopolist who actively pursues a strategy of entry deterrence. This means that an already-operating supplier adopts a strategy that will deter new entrants. What follows is a non-technical description of how a monopolist might strategically erect a barrier to entry.

Devil's Glen is the sole ski resort accessible to the residents of local Angel's Valley. The resort is profitable, but the owners fear that an adjoining hill, owned by a potential competitor, may be developed. Such a development would reduce Devil's Glen's profits radically. In the face of this threat, can Devil's Glen develop a strategy to protect its profit, or at least a large part of its profit?

One such strategy is for Devil's Glen to pre-commit itself to a specific course of action that is additionally costly to itself. In particular, it could invest in a new ski lift that would generate excess capacity, or that it might not even use. Such a strategy would indeed be costly, but it would undoubtedly send a signal to its potential competitor that Devil's Glen could bring on this spare capacity, reduce lineup times, and thereby make it very difficult for a new resort operator to make a profit.

A key component of this strategy is that the incumbent invests ahead of time - and inflicts a cost on itself. The incumbent does not simply say ``I will build another lift if you enter the business.'' Such a policy does not carry the same degree of credibility as actually incurring the cost of construction ahead of time.

This capacity commitment is an example of strategic entry deterrence. Of course, there is no guarantee that such a move will always work; it might be just too costly to pre-empt entry by putting spare capacity in place. An alternative strategy might be to permit entry and share the market.

Spare capacity is not the only pre-commitment available to incumbents. Anything with the character of a sunk cost may work. Advertising to build brand loyalty is a further example. So is product proliferation. Suppose one cereal manufacturer has a strong grip on the market. A potential competitor is contemplating entry. So the incumbent decides to introduce a new line of cereals in order to prevent the entry. The incumbent thereby commits himself to an additional cost in the form of product development. And this development cost is sunk. The incumbent reduces his profit today in order to prevent entry, and maintain his profit tomorrow. 

The threats associated with the incumbent's behaviour become more credible since the incumbent incurs costs up front. To be completely credible the potential competitor must be convinced that the incumbent will actually follow through with the implication of building the additional capacity. Hence in the case of the ski hill owner who has added capacity, the potential competitor would have to be convinced that the incumbent would actually use that additional capacity if the competitor were to enter.

\subsection*{Conclusion}

Monopoly and perfect competition are interesting paradigms; but few markets resemble them in the real world. In this chapter we addressed some of the complexities that define the world around us: it is characterized by strategic planning, entry deterrence, differentiated products and so forth. 

Entry and exit are critical to competitive markets. Frequently entry is blocked because of scale economies -- an example of a natural or unintended entry barrier. In addition, incumbents can formulate strategies to limit entry, for example by investing in additional capacity or in the development of new products. Such strategies, where effective, are called credible threats. 

We also stress that firms act strategically -- particularly whenever there are just a few participants present. Before taking any action, these firms make conjectures about how their competitors will react, and incorporate such reactions into their planning. Competition between suppliers can frequently be defined in terms of a game, and such games usually have an equilibrium outcome. The Cournot duopoly model that we developed is a game between two competitors in which an equilibrium market output is determined from a pair of reaction functions.

Scale economies are critical. Large development costs or set up costs mean that the market will generally support just a limited number of producers. In turn this implies that potential new (small-scale) firms cannot benefit from the scale economies and will not survive competition from large-scale suppliers. 

Finally, product differentiation is critical. If small differences exist between products produced in markets where there is free entry we get a monopolistically competitive market. In these markets long run profits are `normal' and firms operate with some excess capacity. It is not possible to act strategically in this kind of market.