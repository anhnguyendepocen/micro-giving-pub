\section{Comparing market structures}\label{sec:ch11sec6}

We are now in a position to make a general statement about the relative output levels that will be produced in several different market structures when marginal costs are constant. For ease of exposition we use the foregoing numerical example, where the market demand is given by $P=24-Q$, and $MC=6$.

In the duopoly case we saw that the output for each firm $i$ (assumed identical) was $Q_i=6$.

With monopoly, profit is maximized where $MC=MR$. Since $MR=24-2Q$, the monopoly output is obtained from the equation $6=24-2Q$; that is $Q=9$.

If these two firms acted in the same way as perfect competitors (we cannot call this perfect competition because there are just two firms), we know from Chapter~\ref{chap:perfectcompetition} that the market solution is where $P=MC$. Consequently, if we equate the demand and $MC$ functions the solution for the market outcome is $Q=18$ (equating $24-Q=6$ yields $Q=18$).

In this case of constant marginal costs, the above solutions indicate that we can formulate a general rule regarding the output of firms that depends on the number of firms $N$.

\begin{equation*}
\text{Individual firm output}=\frac{\text{market output under competitive behaviour}}{(N + 1)}.
\end{equation*}

For example, in the duopoly case, substituting $N=2$ and competitive output of 18 yields a firm output of 6. In the monopoly case $N=1$ and hence output is 9.