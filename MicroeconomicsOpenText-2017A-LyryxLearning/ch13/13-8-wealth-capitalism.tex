\section{Wealth and capitalism}\label{sec:ch13sec8}

In an immensely insightful and popular study of capital accumulation, from both a historical and contemporary perspective, Thomas Piketty draws our attention to the enormous inequality in the distribution of wealth and explores what the future may hold in his book \textit{Capital in the Twenty-First Century}.

The distribution of wealth is universally more unequal than the distribution of incomes or earnings. Gini coefficients in the neighbourhood of 0.8 are commonplace in developed economies. In terms of shares of the wealth pie, such a magnitude may imply that the top 1\% of wealth holders own one third of all of an economy's wealth, that the top decile may own two-thirds of all wealth, and that the remaining one third is held by the `bottom 90\%'. And within this bottom 90\%, virtually all of the remaining wealth is held by the 40\% of the population below the top decile, leaving only a few percent of all wealth to the bottom 50\% of the population.

% Figure 13.5
\begin{FigureBox}{1}{1}{25em}{Wealth inequality in Europe and the US, 1810-2010 \label{fig:wealthshareuseurope}}{\centering\textit{Source}: \url{http://piketty.pse.ens.fr/fr/capital21c}}
\begin{axis}[
	axis line style=thick,
	every tick label/.append style={font=\footnotesize},
	extra y ticks={0.1,0.3,0.5,0.7,0.9},
	grid style={dotted},
	xmajorgrids,
	ymajorgrids,
	every node near coord/.append style={font=\scriptsize},
	xticklabel style={rotate=90,anchor=east,/pgf/number format/1000 sep=},
	scaled y ticks=false,
	yticklabel=\pgfmathparse{100*\tick}\pgfmathprintnumber{\pgfmathresult}\,\%,
	yticklabel style={/pgf/number format/fixed,/pgf/number format/1000 sep = \thinspace},
	xmin=1810,xmax=2010,ymin=0,ymax=1,
	y=0.8cm/0.1,
	x=0.32cm/5,
	x label style={at={(axis description cs:0.5,-0.05)},anchor=north},
	y label style={at={(axis description cs:0.02,0.5)},anchor=south},
	xlabel={Year},
	ylabel={Share of top decile or percentile in total wealth},
	legend entries={Top 10\% wealth share: Europe,Top 10\% wealth share: United States,Top 1\% wealth share: Europe,Top 1\% wealth share: United States},
	legend style={font=\footnotesize,at={(axis cs:1860,0.02)},anchor=south west},
]
\addplot[thick,mark=square,mark options=datasetcolourone] table { % Top 10\% wealth share: Europe
	X			Y
	1810	0.8222626056
	1870	0.8535888259
	1910	0.8954871102
	1920	0.861154263
	1930	0.8283620878
	1950	0.7536215669
	1960	0.6822420862
	1970	0.6026666667
	1980	0.5928092114
	1990	0.6088738645
	2000	0.6280333333
	2010	0.6389
};
\addplot[thick,mark=square*,mark options=datasetcolourtwo] table { % Top 10\% wealth share: United States
	X			Y
	1810	0.58
	1870	0.71
	1910	0.8112975137
	1920	0.7972690161
	1930	0.7340596342
	1940	0.6638948644
	1950	0.6566553524
	1960	0.67
	1970	0.6418200114
	1980	0.672
	1990	0.687
	2000	0.6965
	2010	0.715
};
\addplot[thick,mark=o,mark options=datasetcolourone] table { % Top 1\% wealth share: Europe
	X			Y
	1810	0.5213105044
	1870	0.5624027565
	1910	0.6353315066
	1920	0.5467479571
	1930	0.483737867
	1950	0.3780441862
	1960	0.2973546921
	1970	0.2076666667
	1980	0.2040662782
	1990	0.2173544448
	2000	0.2366
	2010	0.2437
};
\addplot[thick,mark=*,mark options=datasetcolourtwo] table { % Top 1\% wealth share: United States
	X			Y
	1810	0.25
	1870	0.32
	1910	0.4512975137
	1920	0.4372690161
	1930	0.3740596342
	1940	0.3038948644
	1950	0.2966553524
	1960	0.314
	1970	0.2818200114
	1980	0.301
	1990	0.329
	2000	0.3305
	2010	0.338
};
\end{axis}
\end{FigureBox}

While such an unequal holding pattern may appear shockingly unjust, Piketty informs us that current wealth inequality is not as great in most economies as it was about 1900. Figure~\ref{fig:wealthshareuseurope} (Piketty 10.6) above is borrowed from his book. Wealth was more unequally distributed in Old World Europe than New World America a century ago, but this relativity has since been reversed. A great transformation in the wealth holding pattern of societies in the twentieth century took the form of the emergence of a `patrimonial middle class', by which he means the emergence of substantial wealth holdings on the part of that 40\% of the population below the top decile. This development is noteworthy, but Piketty warns that the top percentiles of the wealth distribution may be on their way to controlling a share of all wealth close to their share in the early years of the twentieth century. He illustrates that two elements are critical to this prediction; first is the rate of growth in the economy relative to the return on capital, the second is inheritances.

To illustrate the importance of the rate of economic growth, imagine an economy with very low growth, and where the owners of capital obtain an annual return of say 5\%. If the owners merely maintain their capital intact and consume the remainder of this 5\%, then the pattern of wealth holding will continue to be stable. However, if the holders of wealth can reinvest from their return an amount more than is necessary to replace depreciation then their wealth will grow. And if labour income in this economy is static then wealth holders will secure a larger share of the economic pie. In contrast, if economic growth is significant, then labour income may grow in line with income from capital and inequality may remain stable. This summarizes Piketty's famous $(r-g)$ law -- inequality depends upon the difference between the return on wealth and the growth rate of the economy. This potential for an ever-expanding degree of inequality is magnified when the stock of capital in the economy is large.

A second element in his theory concerns inheritances. That is to say, do individuals leave large or small inheritances when they die, and how concentrated are such inheritances? If individual wealth accumulation patterns are generated by a desire to save for retirement and old age -- during which time they decumulate by spending their assets, such motivation should result in small bequests being left to following generations. In contrast, if individuals who are in a position to do so save and accumulate, not just for their old age, but because they have dynastic preferences, of if they take pleasure simply from the ownership of wealth, or even if they are very cautious about running down their wealth in their old age, then we will see substantial inheritances passed on to the sons and daughters of these individuals, thereby exacerbating the inequality of wealth holding in the economy.

Piketty shows that in fact individuals who save substantial amounts tend to leave large bequests; that is they do not save purely for life-cycle motives. In modern economies the annual amount of bequests and gifts from parents to children amounts to as much as 15\% of annual GDP. This may grow in future decades, and since wealth is highly concentrated, these bequests in turn are concentrated among a small number of the following generation -- inequality is transmitted from one generation to the next.

As a final observation, if we consider the distribution of income and wealth together, particularly at the very top end, we can see readily that a growing concentration of income among the top 1\% must translate itself into greater wealth inequality through the savings on their part, particularly if the individuals among the top 1\% of earners are also those with substantial amounts of wealth. To compound matters, if individuals who inherit wealth also tend to inherit more human capital from their parents than others, the concentration of income and wealth may become yet stronger.

The study of distributional issues in economics has probably received too little attention in the modern era. Yet it is vitally important both in terms of the well-being of the individuals who constitute an economy and in terms of adherence to social norms. Given that utility declines with additions to income and wealth, transfers from those at the top to those at the bottom have the potential to increase total utility in the economy. Furthermore, an economy in which justice is seen to prevail---in the form of avoiding excessive inequality---is more likely to achieve a higher degree of social coherence than one where inequality is large.