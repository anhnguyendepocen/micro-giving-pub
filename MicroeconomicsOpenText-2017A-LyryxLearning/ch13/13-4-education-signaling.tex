\section{Education as signalling}\label{sec:ch13sec4}

An alternative view of education springs from the theory of signalling. This is a provocative theory that proposes education may be worthwhile, even if it generates little additional skills! The theory recognizes that people possess different abilities. However, firms cannot easily recognize the more productive workers without actually hiring them and finding out ex-post -- sometimes a costly process. \terminology{Signalling theory} says that, in pursuing more education, people who know they are more capable send a signal to potential employers that they are the more capable workers. Education therefore \terminology{screens} out the high-productivity workers. Firms pay more to the educated workers, because firms know that the high-ability workers are those with the additional education.

\begin{DefBox}
\textbf{Signalling} is the decision to undertake an action in order to reveal information.

\textbf{Screening} is the process of obtaining information by observing differences in behaviour.
\end{DefBox}

To be effective, the process must separate the lower-ability individuals from the higher-ability individuals. Why don't lower-ability workers go to university and pretend they are of the high-ability type? Primarily because that strategy could backfire: such individuals are less likely to succeed at school and there are costs associated with school in the form of school fees, books and foregone earnings. While they may have lower innate skills, they are likely smart enough to recognize a bad bet!

On balance economists believe that further education does indeed add to productivity, although there may be element of screening present: an engineering degree (we should hope) increases an individual's understanding of mechanical forces in addition to telling a potential employer that the student is smart.

Finally, it should be evident that if education raises productivity, it is also good for society and the economy at large.