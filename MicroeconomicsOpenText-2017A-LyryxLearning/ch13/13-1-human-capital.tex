\section{Human capital}\label{sec:ch13sec1}

\terminology{Human capital}, HK, is the stock of expertise or ability accumulated by a worker that determines future productivity and earnings. It depends on many different attributes - education, experience, intelligence, interpersonal skills etc. While individuals look upon human capital as a determinant of their own earnings, it also impacts strongly upon the productivity of the economy at large, and is therefore a vital force in determining long-run growth. Canada's productivity growth has lagged behind that of the US and several other developed economies since the first oil shock in the seventies. At the same time Canada has been investing heavily in human capital in recent decades, and this suggests that future productivity and earnings will benefit accordingly.

\begin{DefBox}
\textbf{Human capital} is the stock of expertise accumulated by a worker that determines future productivity and earning.
\end{DefBox}

Three features of Canada's recent human capital accumulation are noteworthy. First, Canada's enrolment rate in post-secondary education now exceeds the US rate, and that of virtually every economy in the world. Second is the fact that the number of women in third-level institutions far exceeds the number of men. Almost 60\% of university students are women. Third, international testing of high-school students sees Canadian students perform well, which indicates that the quality of the Canadian educational system appears to be high. These are remarkably positive aspects of a system that frequently comes under criticism.

Let us now try to understand individuals' motivation for embarking on the accumulation of human capital, and in the process see why different groups earn different amounts. We start by analyzing the role of education and then turn to on-the-job training.