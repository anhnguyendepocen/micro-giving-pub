\section{Education returns and quality}\label{sec:ch13sec5}

How can we be sure that further education really does generate the returns, in the form of higher future incomes, to justify the investment? For many years econometricians proposed that an extra year of schooling might offer a return in the region of 10\% -- quite a favourable return in comparison with what is frequently earned on physical capital. Doubters then asked if the econometric estimation might be subject to bias - what if the additional earnings of those with more education are simply attributable to the fact that it is the smarter, or innately more capable, individuals who both earn more and who have more schooling? And since we cannot observe who has more innate ability, how can we be sure that it is the education itself, rather than just differences in ability, that generate the extra income?  

This is a classical problem in inference: does correlation imply causation? The short answer to this question is that education economists are convinced that the time invested in additional schooling does indeed produce additional rewards, even if it is equally true that individuals who are innately smarter who choose to invest in that way. Furthermore, it appears that the returns to graduate education are higher than the returns to undergraduate education.

What can be said of the quality of different educational systems? Are educational institutions in different countries equally good at producing knowledgeable students? Or, viewed another way: has a grade nine student in Canada the same skill set as a grade nine student in France or Hong Kong? An answer to this question is presented in Table~\ref{table:sciencereading}, which contains two sets of results from the Program for International Student Assessment (PISA) -- an international survey of 15-year old student abilities in mathematics, science and literacy. This particular table presents the results on science and reading skills for 30 economies. The results indicate that Canadian students perform near the top of the international skill ladder in both dimensions.

\begin{Table}{25em}{Mean score in science and reading scales \label{table:sciencereading}}{``Science Competencies for Tomorrow's World'', Volume 1, PISA 2006, OECD 2007, pages 58 \& 298.}
\begin{tabu} to \linewidth {|X[2,l]X[1,c]X[1,c]X[2,l]X[1,c]X[1,c]|} \hline 
\rowcolor{rowcolour}	& \textbf{Science} & \textbf{Reading} &  & \textbf{Science} & \textbf{Reading} \\
						\textbf{Australia}	& 527 & 513 & \textbf{Korea	}		& 522 & 556 \\
\rowcolor{rowcolour}	\textbf{Austria}	& 511 & 490 & \textbf{Luxembourg}	& 486 & 479 \\ 
						\textbf{Belgium}	& 510 & 501 & \textbf{Mexico}		& 410 & 410 \\
\rowcolor{rowcolour}	\textbf{Canada}		& 534 & 527 & \textbf{Netherlands}	& 525 & 507 \\
						\textbf{Czech Rep}	& 513 & 483 & \textbf{New Zealand}	& 530 & 521 \\ 
\rowcolor{rowcolour}	\textbf{Denmark}	& 496 & 494 & \textbf{Norway}		& 487 & 484 \\
						\textbf{Finland}	& 563 & 547 & \textbf{Poland}		& 498 & 508 \\
\rowcolor{rowcolour}	\textbf{France}		& 495 & 488 & \textbf{Portugal}		& 474 & 472 \\
						\textbf{Germany}	& 516 & 495 & \textbf{Slovak Rep.}	& 488 & 466 \\
\rowcolor{rowcolour}	\textbf{Greece}		& 473 & 460 & \textbf{Spain}		& 488 & 461 \\
						\textbf{Hungary}	& 504 & 482 & \textbf{Sweden}		& 503 & 507 \\
\rowcolor{rowcolour}	\textbf{Iceland} 	& 491 & 484 & \textbf{Switzerland}	& 512 & 499 \\
						\textbf{Ireland} 	& 508 & 517 & \textbf{Turkey}		& 424 & 447 \\
\rowcolor{rowcolour}	\textbf{Italy} 		& 475 & 469 & \textbf{US}			& 489 & N/A \\ 
						\textbf{Japan} 		& 531 & 498 & \textbf{UK}			& 515 & 495 \\ \hline  
\end{tabu}
\end{Table}

An interesting paradox arises at this point: If productivity growth in Canada has been relatively low in recent decades, why is this so if Canada produces many well-educated high-skill workers? The answer may be that there is a considerable time lag before high participation rates in third-level education and high quality make themselves felt on the national stage in the form of elevated productivity. While past productivity performance may be somewhat disappointing, the evidence suggests a better future.