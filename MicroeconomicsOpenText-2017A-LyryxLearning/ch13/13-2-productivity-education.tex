\section{Productivity and education}\label{sec:ch13sec2}

Human capital is the result of past investment that raises future incomes. A critical choice for individuals is to decide upon exactly how much additional human capital to accumulate. The cost of investing in another year of school is the \textit{direct cost}, such as school fees, plus the \textit{indirect, or opportunity, cost}, which can be measured by the foregone earnings during that extra year. The benefit of the additional investment is that the future flow of earnings is augmented. Consequently, wage differentials should reflect different degrees of education-dependent productivity.

\subsection*{Age-earnings profiles}

Figure~\ref{fig:earningseducationlevel} illustrates two typical \terminology{age-earnings profiles} for individuals with different levels of education. These profiles define the typical pattern of earnings over time, and are usually derived by examining averages across individuals in surveys. Two aspects are clear: people with more education not only earn more, but the spread tends to grow with time. Less educated, healthy young individuals who work hard may earn a living wage but, unlike their more educated counterparts, they cannot look forward to a wage that rises substantially over time. More highly-educated individuals go into jobs and occupations that take a longer time to master: lawyers, doctors and most professionals not only undertake more schooling than truck drivers, they also spend many years learning on the job, building up a clientele and accumulating expertise.

% Figure 13.1
\begin{FigureBox}{0.15}{0.25}{25em}{Age-Earnings profiles by education level \label{fig:earningseducationlevel}}{Individuals with a higher level of education earn more than individuals with a `standard' level of education. In addition, the differential grows over time.}
\draw [dotted,thick]
	(18,0) node [mynode,below] {18 (High school grad)} -- +(0,25)
	(65,0) node [mynode,below] {65 (Retirement)} -- +(0,25);
\draw [datasetcolourthree,ultra thick] (18,5) -- (25,7) -- (35,8.5) -- (45,9.5) node [black,mynode,above left] {High school\\education} -- (55,10) -- (65,10.25);
\draw [datasetcolourthree!50,ultra thick] (18,8) -- (25,12) -- (35,15) -- (45,17) node [black,mynode,above left] {Third level\\education} -- (55,18) -- (65,18.5);
\draw [thick, -] (0,25) node [mynode1,above] {Average\\earnings\\in \$} -- (0,0) -- (70,0) node [mynode1,right] {Age};
\end{FigureBox}

\begin{DefBox}
\textbf{Age-earnings profiles} define the pattern of earnings over time for individuals with different characteristics.
\end{DefBox}

\subsection*{The education premium}

Individuals with different education levels earn different wages. The \terminology{education premium} is the difference in earnings between the more and less highly educated. Consider Figure~\ref{fig:edskillpremium} which contains supply and demand functions with a twist. We imagine that there are two types of labour: one with a high level of human capital, the other with a lower level. The vertical axis measures the wage \textit{premium} of the high-education group, and the horizontal axis measures the \textit{fraction of the total labour force that is of the high-skill type}. $D$ is the \textit{relative} demand for the high skill workers. It defines the premium that demanders are willing to pay to the higher skill group, depending upon the makeup of their work force: for example, if they were to employ all or virtually all high-skill workers, they would be willing to pay a smaller premium than if they were to employ a small fraction of high-skill and a large fraction of low-skill workers.

% Figure 13.2
\begin{FigureBox}{0.3}{0.25}{25em}{The education/skill premium \label{fig:edskillpremium}}{A shift in demand increases the wage premium in the short run from $E_0$ to $E_1$ by more than in the long run (to $E_2$). In the short run, the percentage of the labour force ($S_S$) that is highly skilled is fixed. In the long run it ($S_L$) is variable and responds to the wage premium.}
% demand lines
\draw [demandcolour,ultra thick,name path=D] (0,15) -- (15,0) node [black,mynode,above right,pos=0.95] {$D$};
\draw [demandcolour,ultra thick,name path=D1] (0,20) -- (27,0) node [black,mynode,above right,pos=0.95] {$D_1$};
% supply lines
\draw [supplycolour,ultra thick,name path=SS] (8.2353,0) -- (8.2353,24) node [black,mynode,above] {$S_S$};
\draw [supplycolour,ultra thick,name path=SL] (0,1) -- (30,22) node [black,mynode,above] {$S_L$};
% axes
\draw [thick] (0,25) node (yaxis) [mynode1,above] {Wage\\Rate} -- (0,0) node [mynode,below left] {0} -- (30,0) node [mynode1,right] {Fraction of labour\\force with high skill\\(max=1)};
% intersection of supply and demand lines
\draw [name intersections={of=SS and D, by=E0},name intersections={of=SS and D1, by=E1},name intersections={of=SL and D1, by=E2}]
	[dotted,thick] (yaxis |- E0) node [mynode,left] {$W_{p0}$} -- (E0) node [mynode,right=0cm and 0.1cm] {$E_0$}
	[dotted,thick] (yaxis |- E1) node [mynode,left] {$W_{p1}$} -- (E1) node [mynode,above right] {$E_1$}
	[dotted,thick] (yaxis |- E2) node [mynode,left] {$W_{p2}$} -- (E2) node [mynode,right=0cm and 0.1cm] {$E_2$};
\end{FigureBox}

\begin{DefBox}
\textbf{Education premium}: the difference in earnings between the more and less highly educated.
\end{DefBox}

In the short run the make-up of the labour force is fixed, and this is reflected in the vertical supply curve $S_s$. The equilibrium is at $E_0$, and $W_{p0}$ is the premium, or excess, paid to the higher-skill worker over the lower-skill worker. In the long run it is possible for the economy to change the composition of its labour supply: if the wage premium increases, more individuals will find it profitable to train as high-skill workers. That is to say, the fraction of the total that is high skill increases. It follows that the long run supply curve slopes upwards.

So what happens when there is an increase in the demand for high-skill workers relative to low-skill workers? The demand curve shifts upward to $D_1$, and the new equilibrium is at $E_1$. The supply mix is fixed in the short run. But over time, some individuals who might have been just indifferent between educating themselves more and going into the market place with lower skill levels now find it worthwhile to pursue further education. The higher anticipated returns to the additional human capital now exceed the additional costs of more schooling, whereas before the premium increase these additional costs and benefits were in balance. In Figure~\ref{fig:edskillpremium} the new short-run equilibrium at $E_1$, has a corresponding wage premium of $W_{p1}$. In the long run, after additional supply has reached the market, the increased premium is moderated to $W_{p2}$ at the equilibrium $E_2$.  

\begin{ApplicationBox}{How big is the education premium? \label{app:educationpremium}}
Many studies have attempted to compute the earnings value of additional education in Canada beyond a completed high-school diploma. A typical finding is that a completed university bachelor’s degree generates an additional 40\% in earnings for men on average in the modern era and perhaps even more for women. Individuals who do not complete high school earn at least 10\% less than high school graduates. The picture in the US economy is broadly similar; the education premium has been rising there for several decades. The education premium also appears to have grown in recent decades.

\bigskip
Most economists believe the reason for a growing premium is that the technological change experienced in the post-war period is complementary to high education and high skills in general. Furthermore, the premium has increased despite an enormous increase in the relative supply of highly skilled workers. In essence the growth in supply has not kept up with the growth in demand. 
\end{ApplicationBox}

\subsection*{Are students credit-constrained or culture-constrained?}

The foregoing analysis assumes that students and potential students make rational decisions on the costs and benefits of further education and act accordingly. It also assumes implicitly that individuals can borrow the funds necessary to build their human capital: if the additional returns to further education are worthwhile, individuals should borrow the money to make the investment, just as entrepreneurs do with physical capital.

However, there is a key difference in the credit markets. If an entrepreneur fails in her business venture the lender will have a claim on the physical capital. But a bank cannot repossess a human being who drops out of school without having accumulated the intended human capital. Accordingly, the traditional lending institutions are frequently reluctant to lend the amount that students might like to borrow - students are credit constrained. The sons and daughters of affluent families therefore find it easier to attend university, because they are more likely to have a supply of funds domestically. Governments customarily step into the breach and supply loans and bursaries to students who have limited resources. While funding frequently presents an obstacle to attending a third-level institution, a stronger determinant of attendance is the education of the parents, as detailed in Application Box~\ref{app:parentedcanada}.

\begin{ApplicationBox}{Parental education and  university attendance in Canada \label{app:parentedcanada}}
The biggest single determinant of university attendance in the modern era is parental education. A recent study* of who goes to university examined the level of parental education of young people `in transition' -- at the end of their high school -- for the years 1991 and 2000. 

\bigskip
For the year 2000 they found that, if a parent had not completed high school, there was just a 12\% chance that their son would attend university and an 18\% chance that a daughter would attend.  In contrast, for parents who themselves had completed a university degree, the probability that a son would also attend university was 53\% and for a daughter 62\%. Hence, the probability of a child attending university was roughly four times higher if the parent came from the top educational category rather the bottom category! Furthermore the authors found that this probability gap opened wider between 1991 and 2000.

\bigskip
*Finnie, R., C. Laporte and E. Lascelles. ``Family Background and Access to Post-Secondary Education: What Happened in the Nineties?'' Statistics Canada Research Paper, Catalogue number 11F0019MIE-226, 2004
\end{ApplicationBox}