\section{Rationality}\label{sec:ch6sec1}

A critical behavioural assumption in economics is that agents operate in a way that is oriented towards achieving a goal. This can be interpreted to mean that individuals and firms maximize their personal well-being and/or their profits. These players may have other goals in addition: philanthropy and the well-being of others are consistent with individual optimization.

If individuals are to achieve their goals then they must act in a manner that will get them to their objective; broadly, they must act in a rational manner. The theory of individual maximization that we will develop in this chapter is based on that premise or assumption. In assuming individuals are rational we need not assume that they have every piece of information available to them that might be relevant for a specific decision or choice. Nor need we assume that they have super computers in their brain when they evaluate alternative possible strategies.

What we do need to assume, however, is that individuals act in a manner that is consistent with obtaining a given objective. The modern theory of behavioural economics and behavioural psychology examines decision making in a wide range of circumstances and has uncovered many fascinating behaviours -- some of which are developed in Application Box~\ref{app:rationalimpulse} below.

% Application Box 6.1
\begin{ApplicationBox}{Rationality and impulse \label{app:rationalimpulse}}
A number of informative and popular books on decision making have appeared recently. Their central theme is that our decision processes should not be viewed solely as a rational computer -- operating in one single mode only, and unmoved by our emotions or history. Psychologists now know that our brains have at least two decision modes, and these are developed by economics Nobel Prize winner Daniel Kahneman in his book ``Thinking, Fast and Slow''. One part of our brain operates in a rational goal-oriented forward-looking manner (the `slow' part), another is motivated by immediate gratification (the `fast' part). Decisions that we observe in the world about us reflect these different mechanisms. 

\bigskip
Richard Thaler, a Chicago economist and his legal colleague Cass Sunstein, have developed a role for public policy in their book entitled ``Nudge''. They too argue that individuals do not inevitably operate in their own best long-term interests, and as a consequence individuals frequently require a \emph{nudge} by government to make the long-term choice rather than the short-term choice. For example, when individuals begin a new job, they might be automatically enrolled in the company pension plan and be given the freedom to opt out, rather than not be enrolled and given the choice to opt in. Such policies are deemed to be `soft paternalism'. They are paternalistic for the obvious reason -- another organism is directing, but they are also soft in that they are not binding. In the utility maximizing framework that economists use to model behaviour, they are assuming that individuals are operating in their rational mode rather than their impulsive mode.   
\end{ApplicationBox}

We indicated in Chapter~\ref{chap:intro} that as social scientists, we require a \textit{reliable model} of behaviour, that is, \textit{a way of describing the essentials of choice that is consistent with everyday observations on individual behaviour patterns}. In this chapter, our aim is to understand more fully the behavioural forces that drive the demand side of the economy.

Economists analyze individual decision making using two different, yet complementary, approaches -- utility analysis and indifference analysis. We begin by portraying individuals as maximizing their \textit{measurable utility} (sometimes called \textit{cardinal utility}); then progress to indifference analysis, where a weaker assumption is made on the ability of individuals to measure their satisfaction. In this second instance we do not assume that individuals can measure their utility numerically, only that they can say if one collection of goods and services yields them greater satisfaction than another group. This ranking of choices corresponds to what is sometimes called \textit{ordinal utility} -- because individuals can \textit{order} groups of goods and services in ascending order of satisfaction. In each case individuals are perceived as rational maximizers or optimizers: They allocate their income so as to choose the outcome that will make them as well off as possible.