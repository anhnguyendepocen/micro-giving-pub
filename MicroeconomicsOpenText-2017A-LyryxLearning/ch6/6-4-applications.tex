\section{Applications of indifference analysis}\label{sec:ch6sec4}

\subsection*{Price impacts: complements and substitutes}

The nature of complements and substitutes, defined in Chapter~\ref{chap:elasticities}, can be further understood with the help of Figure~\ref{fig:incomepriceadj}. The new equilibrium $E_2$ has been drawn so that the increase in the price of jazz results in more snowboarding---the quantity of $S$ increases to $S_2$ from $S_0$. These goods are substitutes in this picture, because snowboarding \textit{increases} in response to an \textit{increase} in the price of jazz. If the new equilibrium $E_2$ were at a point yielding a lower level of $S$ than $S_0$, we would conclude that they were complements.  

\subsection*{Cross-price elasticities}

Continuing with the same price increase in jazz, we could compute the \textit{percentage} change in the quantity of snowboarding demanded as a result of the \textit{percentage} change in the jazz price. In this example, the result would be a positive elasticity value, because the quantity change in snowboarding and the price change in jazz are both in the same direction, each being positive.


\subsection*{Income impacts: normal and inferior goods}

We know from Chapter~\ref{chap:elasticities} that the quantity demanded of a \textit{normal good} increases in response to an income increase, whereas the quantity demanded of an \textit{inferior good} declines. Clearly, both jazz and boarding are normal goods, as illustrated in Figure~\ref{fig:incomepriceadj}, because more of each one is demanded in response to the income increase from $I_0$ to $I_1$. It would challenge the imagination to think that either of these goods might be inferior. But if $J$ were to denote junky (inferior) goods, and $S$ super goods, we could envisage an equilibrium $E_1$ to the northwest of $E_0$ in response to an income increase, along the constraint $I_1$; less $J$ and more $S$ would be consumed in response to the income increase.

\subsection*{Income and substitution effects}

It is very useful, as we shall see presently, to decompose a price change, which moves the consumer along his demand curve, into two components---an income effect and a substitution effect. Figure~\ref{fig:incomesubeffect} illustrates the impact of an increase in the price of jazz: The equilibrium moves from $E_1$ to $E_2$ as a result of the budget constraint's changing from $I_1$ to $I_2$. This price increase reduces the demand for jazz in two ways. First, jazz becomes \textit{relatively more expensive} compared with other goods, and this is reflected in the steeper price line. Second, the price increase reduces the real purchasing power of the consumer's income.

% Figure 6.11 (called 6.9 in original text)
\begin{FigureBox}{0.3}{0.25}{25em}{Income and substitution effects \label{fig:incomesubeffect}}{The equilibrium $E_1$ is disturbed by an increase in $P_j$ that rotates $I_1$ to $I_2$. The new equilibrium is $E_2$. The substitution effect is from $E_1$ to $E'$ -- to a point on the initial indifference curve where the slope is the same as the slope of $I_2$. From $E'$ to $E_2$ is the income effect; it is equivalent to a parallel shift in the budget constraint $I'$ to $I_2$.}
% thick black lines tangent to the indifference curves
\draw [thick,name path=I1] (0,13.55) -- (24,2.32023) node [mynode,above right] {$I_1$};
\draw [thick,name path=I2] (0,13.55) -- (13,0) node [mynode,above right] {$I_2$};
\draw [thick,name path=Iprime] (0,19.55) -- (19,0) node [mynode,above right] {$I'$};
% indifference curves
\draw [indiffcolour,ultra thick,name path=U2] (4,13) node [black,mynode,above left] {$U_2$} to [out=270,in=180] (13,4);
\draw [indiffcolour,ultra thick,name path=U1] (7,16) node [black,mynode,above left] {$U_1$} to [out=270,in=180] (16,7);
% axes
\draw [thick, -] (0,20) node (yaxis) [above] {Snowboarding} |- (25,0) node (xaxis) [right] {Jazz};
% intersection of I lines with indiff curves
\draw [name intersections={of=U1 and I1, by=E1},name intersections={of=U2 and I2, by=E2},name intersections={of=U1 and Iprime, by=Eprime}]
	[dotted,thick] (E1) node [mynode,above right] {$E_1$} -- (xaxis -| E1) node [mynode,below] {$J_1$}
	[dotted,thick] (E2) node [mynode,above right] {$E_2$} -- (xaxis -| E2) node [mynode,below] {$J_2$}
	[dotted,thick] (Eprime) node [mynode,above right] {$E'$} -- (xaxis -| Eprime) node [mynode,below] {$J'$};
\end{FigureBox}

Consider an experiment in which we could give the consumer just enough income, at the new price, to attain the same level of utility as was originally attained, $U_1$. The budget line that permits this is parallel to the constraint $I_2$---the post price-change line---and tangent to $U_1$. Call this imaginary budget line $I'$, and let it result in a tangency at $E'$. The reduction in jazz \textit{that is due solely to the change in the relative prices, and that could keep the consumer at the indifference level $U_1$}, is called the \terminology{substitution effect} of the price change. This is defined in the figure as the distance from $J_1$ to $J'$.

\begin{DefBox}
The \textbf{substitution effect} of a price change is the response of demand to a relative price change that maintains the consumer on the initial indifference curve.
\end{DefBox}

The remaining impact of the price change on the demand for jazz---the move from $J'$ to $J_2$---is the \terminology{income effect}. The reason is that this move is associated with a \textit{parallel} inward shift of the budget constraint. It is therefore an income effect.

\begin{DefBox}
The \textbf{income effect} of a price change is the response of demand to the change in real income that moves the individual from the initial level to a new level of utility.
\end{DefBox}

Income and substitution effects are frequently thought of as being obscure. But this is not so. They are also very useful in understanding policy issues, such as subsidies and taxation. Let us turn to an application.

\subsection*{Policy: income transfers and price subsidies}

Government policies that improve the purchasing power of low-income households come in two main forms: pure income transfers and price subsidies. \textit{Social Assistance} payments (``welfare'') or \textit{Employment Insurance} benefits, for example, provide an increase in income to the needy. Subsidies, on the other hand, enable individuals to purchase particular goods or services at a lower price---for example, rent or daycare subsidies.

In contrast to taxes, which \textit{reduce} the purchasing power of the consumer, subsidies and income transfers \textit{increase} purchasing power. The impact of an income transfer, compared with a pure price subsidy, can be analyzed using Figure~\ref{fig:incometransfer} and ~\ref{fig:pricesubsidy}.

% Figure 6.12 (called 6.10a in original text)
\begin{FigureBox}{0.3}{0.25}{25em}{Income transfer \label{fig:incometransfer}}{An increase in income due to a government transfer shifts the budget constraint from $I_1$ to $I_2$. This parallel shift increases the quantity consumed of the target good (daycare) \textit{and} other goods, unless one is inferior.}
% thick black lines tangent to the indifference curves
\draw [thick,name path=I1] (0,13.55) -- (13,0) node [mynode,above right] {$I_1$};
\draw [thick,name path=I2] (0,19.55) -- (19,0) node [mynode,above right] {$I_2$};
% indifference curves
\draw [indiffcolour,ultra thick,name path=U1] (4,13) node [black,mynode,above left] {$U_1$} to [out=270,in=180] (13,4);
\draw [indiffcolour,ultra thick,name path=U2] (7,16) node [black,mynode,above left] {$U_2$} to [out=270,in=180] (16,7);
% axes
\draw [thick, -] (0,20) node (yaxis) [above] {Other goods} |- (25,0) node (xaxis) [right] {Daycare};
% intersection of black I lines with indiff curves
\draw [name intersections={of=I1 and U1, by=E1},name intersections={of=I2 and U2, by=E2}]
	[dotted,thick] (yaxis |- E1) -- (E1) node [mynode,below left] {$E_1$} -- (xaxis -| E1)
	[dotted,thick] (yaxis |- E2) -- (E2) node [mynode,above right] {$E_2$} -- (xaxis -| E2);
\end{FigureBox}

% Figure 6.13 (called 6.10b in original text)
\begin{FigureBox}{0.3}{0.25}{25em}{Price subsidy \label{fig:pricesubsidy}}{A subsidy to the targeted good, by reducing its price, rotates the budget constraint from $I_1$ to $I_2$. This induces the consumer to direct expenditure more towards daycare and less towards other goods than an income transfer that does not change the relative prices.}
% thick black lines tangent to the indiff curves
\draw [thick,name path=I2] (0,13.55) -- (24,2.32023) node [mynode,above right] {$I_2$};
\draw [thick,name path=I1] (0,13.55) -- (13,0) node [mynode,above right] {$I_1$};
% indifference curves
\draw [indiffcolour,ultra thick,name path=U1] (4,13) node [black,mynode,above left] {$U_1$} to [out=270,in=180] (13,4);
\draw [indiffcolour,ultra thick,name path=U2] (7,16) node [black,mynode,above left] {$U_2$} to [out=270,in=180] (16,7);
% axes
\draw [thick, -] (0,20) node (yaxis) [above] {Other goods} |- (25,0) node (xaxis) [right] {Daycare};
% intersection of I lines with indifference curves
\draw [name intersections={of=I1 and U1, by=E1},name intersections={of=I2 and U2, by=E2}]
	[dotted,thick] (yaxis |- E1) -- (E1) node [mynode,below left] {$E_1$} -- (xaxis -| E1)
	[dotted,thick] (yaxis |- E2) -- (E2) node [mynode,below left] {$E_2$} -- (xaxis -| E2);
\end{FigureBox}

In Figure~\ref{fig:incometransfer}, an \textit{income transfer} increases income from $I_1$ to $I_2$. The new equilibrium at $E_2$ reflects an increase in utility, and an increase in the consumption of \textit{both} daycare and other goods.

Suppose now that a government program administrator decides that, while helping this individual to purchase more daycare accords with the intent of the transfer, she does not intend that government money should be used to purchase other goods. She therefore decides that a daycare \textit{subsidy} program might better meet this objective than a pure income transfer.

A daycare subsidy reduces the price of daycare and therefore \textit{rotates the budget constraint outwards around the intercept on the vertical axis}. At the equilibrium in Figure~\ref{fig:pricesubsidy}, purchases of other goods change very little, and therefore most of the additional purchasing power is allocated to daycare.

The different program outcomes can be understood in terms of income and substitution effects. The pure income transfer policy described in Figure~\ref{fig:incometransfer} contains only income effects, whereas the policy in Figure~\ref{fig:pricesubsidy} has a substitution effect in addition. Since the relative prices have been altered, the subsidy policy necessarily induces a substitution towards the good whose price has fallen. In this example, where there is a minimal change in ``other good'' consumption following the daycare subsidy, the substitution effect towards more daycare almost offsets the income effect that increases these other purchases.

Let us take the example one stage further. From the initial equilibrium $E_1$ in Figure~\ref{fig:pricesubsidy}, suppose that, instead of a subsidy that took the individual to $E_2$, we gave an income transfer \textit{that enabled the consumer to purchase the combination $E_2$}. Such a transfer is represented in Figure~\ref{fig:subsidytransfercomp} by a parallel outward shift of the budget constraint from $I_1$ to $I'_1$, going through the point $E_2$. This budget constraint is very close to the one we used in establishing the substitution effect earlier. We now have a subsidy policy and an alternative income transfer policy, each permitting the same consumption bundle ($E_2$). The interesting aspect of this pair of possibilities is that the income transfer will enable the consumer to attain a higher level of satisfaction---for example, at point $E'$---and will also induce her to consume more of the good on the vertical axis.

% Figure 6.14 (called 6.10c in original text)
\begin{FigureBox}{0.3}{0.25}{25em}{Subsidy-transfer comparison \label{fig:subsidytransfercomp}}{A price subsidy to the targeted good induces the individual to move from $E_1$ to $E_2$, facing a budget constraint $I_2$. An income transfer that permits him to consume $E_2$ is given by $I'_1$; but it also permits him to attain a higher level of satisfaction, denoted by $E'$ on the indifference curve $U_3$.}
% black lines tangent to indiff curves
\draw [thick,name path=Iprime] (1.48661,19) -- (19.7154,0) node [mynode,above right] {$I_1'$};
\draw [thick,name path=I2] (0,13.55) -- (24,2.32023) node [mynode,above right] {$I_2$};
\draw [thick,name path=I1] (0,13.55) -- (13,0) node [mynode,above right] {$I_1$};
% indifference curves
\draw [indiffcolour,ultra thick,name path=U1] (4,13) node [black,mynode,above left=0cm and -0.25cm] {$U_1$} to [out=270,in=180] (13,4);
\draw [indiffcolour,ultra thick,name path=U2] (7,16) node [black,mynode,above left=0cm and -0.2cm] {$U_2$} to [out=270,in=180] (16,7);
\draw [indiffcolour,ultra thick,name path=U3] (7.45,16.4) node [black,mynode,above right=0cm and -0.2cm] {$U_3$} to [out=270,in=180] (16.45,7.4);
% axes
\draw [thick, -] (0,20) node (yaxis) [above] {Other goods} |- (25,0) node (xaxis) [right] {Daycare};
% intersection of black lines with indiff curves
\draw [name intersections={of=I1 and U1, by=E1},name intersections={of=I2 and U2, by=E2},name intersections={of=Iprime and U3, by=Eprime}]
	[dotted,thick] (yaxis |- E1) -- (E1) node [mynode,below left] {$E_1$} -- (xaxis -| E1)
	[dotted,thick] (yaxis |- E2) -- (E2) node [mynode,below left] {$E_2$} -- (xaxis -| E2);
\node [mynode,right=0cm and 0.25cm] at (Eprime) {$E'$};
\end{FigureBox}

\begin{ApplicationBox}{Daycare subsidies in Quebec \label{app:daycarequebec}}
The Quebec provincial government subsidizes daycare very heavily. In the network of daycares that are part of the government-sponsored ``Centres de la petite enfance'', parents can place their children in daycare for less than \$10 per day. This policy of providing daycare at a fraction of the actual cost was originally intended to enable parents on lower incomes to work in the marketplace, without having to spend too large a share of their earnings on daycare. However, the system was not limited to those households on low income. This policy is exactly the one described in Figure~\ref{fig:subsidytransfercomp}.

\bigskip
The consequences of such subsidization were predictable: extreme excess demand, to such an extent that children are frequently placed on waiting lists for daycare places not long after they are born. Annual subsidy costs amount to almost \$2 billion per year. Universality means that many low-income parents are not able to obtain subsidized daycare places for their children, because middle- and higher-income families get many of the limited number of such spaces. The Quebec government, like many others, also subsidizes the cost of daycare for families using the private daycare sector, through income tax credits.
\end{ApplicationBox}

\subsection*{The price of giving}

Imagine now that the good on the horizontal axis is charitable donations, rather than daycare, and the government decides that for every dollar given the individual will see a reduction in their income tax of 50 cents. This is equivalent to cutting the `price' of donations in half, because a donation of one dollar now costs the individual half of that amount. Graphically the budget constraint rotates outward with the vertical intercept unchanged. Since donations now cost less the individual has increased spending power as a result of the price reduction for donations. The price reduction is designed to increase the attractiveness of donations to the utility maximizing consumer.
