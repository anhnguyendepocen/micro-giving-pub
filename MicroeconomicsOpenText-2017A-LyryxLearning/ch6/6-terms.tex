\newpage
\markboth{Key Terms}{Key Terms}
	\addcontentsline{toc}{section}{Key terms}
	\section*{\textsc{Key Terms}}
\begin{keyterms}
\textbf{Cardinal utility} is a measurable concept of satisfaction.

\textbf{Total utility} is a measure of the total satisfaction derived from consuming a given amount of goods and services.

\textbf{Marginal utility} is the addition to total utility created when one more unit of a good or service is consumed.

\textbf{Diminishing marginal utility} implies that the addition to total utility from each extra unit of a good or service consumed is declining.

\textbf{Consumer equilibrium} occurs when marginal utility per dollar spent on the last unit of each good is equal.

\textbf{Law of demand} states that, other things being equal, more of a good is demanded the lower is its price.

\textbf{Ordinal utility} assumes that individuals can rank commodity bundles in accordance with the level of satisfaction associated with each bundle.

\textbf{Budget constraint} defines all bundles of goods that the consumer can afford with a given budget.

\textbf{Affordable set} of goods and services for the consumer is bounded by the budget line from above; the \textbf{non-affordable set} lies strictly above the budget line.

\textbf{Indifference curve} defines combinations of goods and services that yield the same level of satisfaction to the consumer.

\textbf{Indifference map} is a set of indifference curves, where curves further from the origin denote a higher level of satisfaction.

\textbf{Marginal rate of substitution} is the slope of the indifference curve. It defines the amount of one good the consumer is willing to sacrifice in order to obtain a given increment of the other, while maintaining utility unchanged.

\textbf{Diminishing marginal rate of substitution} reflects a higher marginal value being associated with smaller quantities of any good consumed.

\textbf{Consumer optimum} occurs where the chosen consumption bundle is a point such that the price ratio equals the marginal rate of substitution.

\textbf{Substitution effect} of a price change is the adjustment of demand to a relative price change alone, and maintains the consumer on the initial indifference curve.

\textbf{Income effect} of a price change is the adjustment of demand to the change in real income alone.
\end{keyterms}