\section{What's it all about?}\label{sec:ch1sec1}

\subsection*{The big issues}

Economics is the study of human behaviour. Since it uses scientific methods it is called a social science. We study human behaviour to better understand and improve our world. During his acceptance speech, a recent Nobel Laureate in Economics suggested:

\begin{quote}
\textit{Economics, at its best, is a set of ideas and methods for the improvement of society. It is not, as so often seems the case today, a set of ideological rules for asserting why we cannot face the challenges of stagnation, job loss and widening inequality.}

Christopher Sims, Nobel Laureate in Economics 2011
\end{quote}

This is an elegant definition of economics and serves as a timely caution about the perils of ideology. Economics evolves continuously as current observations and experience provide new evidence about economic behaviour and relationships. Inference and policy recommendations based on earlier theories, observations and institutional structures require constant analysis and updating if they are to furnish valuable responses to changing conditions and problems. 

Much of today's developed world faces severe challenges as a result of the financial crisis that began in 2008. Unemployment rates among young people are at historically high levels in several economies, government balance sheets are in disarray, and inequality is on the rise. In addition to the challenges posed by this severe economic cycle, the world simultaneously faces structural upheaval: overpopulation, climate change, political instability and globalization challenge us to understand and modify our behaviour. 

These challenges do not imply that our world is deteriorating. Literacy rates have been rising dramatically in the developing world for decades; child mortality has plummeted; family size is a fraction of what it was 50 years ago; prosperity is on the rise in much of Asia; life expectancy is increasing universally and deaths through wars are in a state of long term decline.

These developments, good and bad, have a universal character and affect billions of individuals. They involve an understanding of economies as large organisms with interactive components. The study of economies as large interactive systems is called \terminology{macroeconomics}. Technically, macroeconomics approaches the economy as a complete system with feedback effects among sectors that determine national economic performance. Feedbacks within the system mean we cannot aggregate from observations on one household or business to the economy as a whole. Application Box~\ref{app:paradoxofthrift} gives an example.

\begin{DefBox}
\textbf{Macroeconomics}: the study of the economy as system in which feedbacks among sectors determine national output, employment and prices.
\end{DefBox}

\subsection*{Individual behaviours}

Individual behaviour underlies much of our social and economic interactions. Some individual behaviours are motivated by self-interest, others are socially motivated. The Arab Spring of 2011 was sparked by individual actions in North Africa that subsequently became mass movements. These movements were aimed at improving society at large. Globalization, with its search for ever less costly production sources in Asia and elsewhere, is in part the result of cost-reducing and profit-maximizing behaviour on the part of developed-world entrepreneurs, and in part attributable to governments opening their economies up to the forces of competition, in the hope that living standards will improve across the board. The increasing income share that accrues to the top one percent of our population in North America and elsewhere is primarily the result of individual self-interest. 

At the level of the person or organization, economic actions form the subject matter of microeconomics. Formally, \terminology{microeconomics} is the study of individual behaviour in the context of scarcity.

\begin{DefBox}
\textbf{Microeconomics}: the study of individual behaviour in the context of scarcity.
\end{DefBox}

Individual economic decisions need not be world-changing events, or motivated by a search for profit. For example, economics is also about how we choose to spend our time and money. There are quite a few options to choose from: sleep, work, study, food, shelter, transportation, entertainment, recreation and so forth. Because both time and income are limited we cannot do all things all the time. Many choices are routine or are driven by necessity. You have to eat and you need a place to live. If you have a job you have committed some of your time to work, or if you are a student some of your time is committed to lectures and study. There is more flexibility in other choices. Critically, microeconomics seeks to understand and explain how we make choices and how those choices affect our behaviour in the workplace and society. 

A critical element in making choices is that there exists a \textit{scarcity} of time, or income or productive resources. Decisions are invariably subject to limits or constraints, and it is these constraints that make decisions both challenging and scientific.  

Microeconomics also concerns business choices. How does a business use its funds and management skill to produce goods and services?  The individual business operator or firm has to decide what to produce, how to produce it, how to sell it and in many cases, how to price it. To make and sell pizza, for example, the pizza parlour needs, in addition to a source of pizza ingredients, a store location (land), a pizza oven (capital), a cook and a sales person (labour). Payments for the use of these inputs generate income to those supplying them. If revenue from the sale of pizzas is greater than the costs of production, the business earns a profit for the owner. A business fails if it cannot cover its costs.

In these micro-level behaviours the decision makers have a common goal: to do as well as he or she can, \textit{given the constraints imposed by the operating environment}. The individual wants to mix work and leisure in a way that makes her as happy or contented as possible. The entrepreneur aims at making a profit. These actors, or agents as we sometimes call them, are \textit{maximizing}. Such maximizing behaviour is a central theme in this book and in economics at large.

\begin{ApplicationBox}{The paradox of thrift \label{app:paradoxofthrift}}
Finance Minister Jim Flaherty and Bank of Canada Governor Mark Carney in 2011 urged Canadian households to increase their savings in order to reduce their record high debt-to-income ratio. On an individual level this makes obvious sense. If you could save more and spend less you could pay down the balances on credit cards, your line of credit, mortgage and other debts.

\bigskip
But one household's spending is another household's income. For the economy as a system, an increase in households' saving from say 5 percent of income to 10 percent reduces spending accordingly. But lower spending by all households will reduce the purchases of goods and services produced in the economy, and therefore has the potential to reduce national incomes. Furthermore, with lower income the trouble some debt-to-income ratio will not fall, as originally intended. Hence, while higher saving may work for one household in isolation, higher saving by all households may not. The interactions and feed backs in the economic system create a `\textbf{paradox of thrift}'.

\bigskip
The paradox can also create problems for government finances and debt. Following the recession that began in 2008/09, many European economies with high debt loads cut spending and increased taxes to in order to balance their fiscal accounts. But this fiscal austerity reduced the national incomes on which government tax revenues are based, making deficit and debt problems even more problematic. Feedback effects, within and across economies, meant that European Union members could not all cut deficits and debt simultaneously. \end{ApplicationBox}

\subsection*{Markets and government}

Markets play a key role in coordinating the choices of individuals with the decisions of business. In modern market economies goods and services are supplied by both business and government. Hence we call them \terminology{mixed economies}. Some products or services are available to those who wish to buy them and have the necessary income - as in cases like coffee and wireless services. Other services are provided to all people through government programs like law enforcement and health care.

\begin{DefBox}
\textbf{Mixed economy}: goods and services are supplied both by private suppliers and government.
\end{DefBox}

Markets offer the choice of a wide range of goods and services at various prices. Individuals can use their incomes to decide the pattern of expenditures and the bundle of goods and services they prefer. Businesses sell goods and services in the expectation that the market price will cover costs and yield a profit. 

The market also allows for specialization and separation between production and use. Rather than each individual growing her own food, for example, she can sell her time or labour to employers in return for income. That income can then support her desired purchases. If businesses can produce food more cheaply than individuals the individual obviously gains from using the market -- by both having the food to consume, and additional income with which to buy other goods and services. Economics seeks to explain how markets and specialization might yield such gains for individuals and society.

We will represent individuals and firms by envisaging that they have explicit objectives -- to maximize their happiness or profit. However, this does not imply that individuals and firms are concerned only with such objectives. On the contrary, much of microeconomics and macroeconomics focuses upon the role of government: how it manages the economy through fiscal and monetary policy, how it redistributes through the tax-transfer system, how it supplies information to buyers and sets safety standards for products. 

Since governments perform all of these socially-enhancing functions, in large measure governments reflect the social ethos of voters. So, while these voters may be maximizing at the individual level in their everyday lives, and our models of human behaviour in microeconomics certainly emphasize this optimization, economics does not see individuals and corporations as being devoid of civic virtue or compassion, nor does it assume that only market-based activity is important. Governments play a central role in modern economies, to the point where they account for more than one third of all economic activity in the modern mixed economy. 

While governments supply goods and services in many spheres, governments are fundamental to the just and efficient functioning of society and the economy at large. The provision of law and order, through our legal system broadly defined, must be seen as more than simply accounting for some percentage our national economic activity. Such provision supports the whole private sector of the economy. Without a legal system that enforces contracts and respects property rights the private sector of the economy would diminish dramatically as a result of corruption, uncertainty and insecurity. It is the lack of such a secure environment in many of the world's economies that inhibits their growth and prosperity.

Let us consider now the methods of economics, methods that are common to science-based disciplines.
