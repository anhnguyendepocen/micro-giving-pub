\section*{Exercises for Chapter~\ref{chap:intro}}

\begin{enumialphparenastyle}

% Solutions file for exercises opened
\Opensolutionfile{solutions}[solutions/ch1ex]

\begin{ex}\label{ex:ch1ex1}
An economy has 100 workers. Each one can produce four cakes or three shirts, regardless of the number of other individuals producing each good. Assuming all workers are employed, draw the $PPF$ for this economy, with cakes on the vertical axis and shirts on the horizontal axis.
\begin{enumerate}
	\item	How many cakes can be produced in this economy when all the workers are cooking?
	\item	How many shirts can be produced in this economy when all the workers are sewing? 
	\item	Join these points with a straight line; this is the $PPF$.
	\item	Label the inefficient and unattainable regions on the diagram.
\end{enumerate}
\begin{sol}
\begin{enumerate}
	\item	If all 100 workers make cakes their output is $100\times 4=400$.
	\item	If all workers make shirts their output is $100\times 3=300$.
	\item	The diagram shows the $PPF$ for this economy.
	\item	As illustrated in the diagram.
\end{enumerate}
\begin{center}
	\begin{tikzpicture}[background color=figurebkgdcolour,use background,xscale=0.3,yscale=0.3]
	\draw [thick] (0,20) node (yaxis) [mynode1,above] {Cakes} |- (20,0) node (xaxis) [mynode1,right] {Shirts};
	\draw [ultra thick,ppfcolourone] (0,16) node [mynode,left,black] {400} -- node [mynode,above right,pos=0.8,black] {$PPF$} node [mynode,midway,below left=1em and 0em,black] {inefficient} node [mynode,midway,above right=1em and 1.5em,black] {unattainable} (12,0) node [mynode,below,black] {300};
	\end{tikzpicture}
\end{center}
\end{sol}
\end{ex}

\begin{ex}\label{ex:ch1ex2}
In the table below are listed a series of points that define an economy's production possibility frontier for goods $Y$ and $X$.
\begin{center}
\begin{tabu} to \linewidth {|X[1,c]X[1,c]X[1,c]X[1,c]X[1,c]X[1,c]X[1,c]X[1,c]X[1,c]X[1,c]X[1,c]X[1,c]|}\hline
\rowcolor{rowcolour}	$Y$	&	1000	&	900		&	800		&	700		&	600		&	500		&	400		&	300		&	200		&	100		&	0			\\
$X$	&	0			&	1600	&	2500	&	3300	&	4000	&	4600	&	5100	&	5500	&	5750	&	5900	&	6000	\\	\hline
\end{tabu}
\end{center}
\begin{enumerate}
	\item	Plot these points to scale, on graph paper, or with the help of a spreadsheet.
	\item	Given the shape of this $PPF$ is the economy made up of individuals who are similar or different in their production capabilities?
	\item	What is the opportunity cost of producing 100 more $Y$ at the combination $(X=5500,Y=300)$.
	\item	Suppose next there is technological change so that at every output level of good $Y$ the economy can produce 20 percent more $X$. Compute the co-ordinates for the new economy and plot the new $PPF$.
\end{enumerate}
\begin{sol}
\begin{enumerate}
	\item	The $PPF$ is curved outwards with intercepts of 1000 on the Thinkpod axis and 6000 on the iPad axis. Each point on the $PPF$ shows one combination of outputs.
	\item	Different.
	\item	400 $X$.
	\item	The new $PPF$ in the diagram has the same Thinkpod intercept, 1000, but a new iPad intercept of 7200.
\end{enumerate}
\begin{center}
	\begin{tikzpicture}[background color=figurebkgdcolour,use background,xscale=0.3,yscale=0.3]
	\draw [thick] (0,20) node (yaxis) [mynode1,above] {Thinkpods} |- (30,0) node (xaxis) [mynode1,right] {iPads};
	\draw [ultra thick,dashed,ppfcolourtwo,name path=ppf2] (0,18) to[out=0,in=105] (28,0) node [mynode,below,black] {7200};
	\draw [ultra thick,ppfcolourone,name path=ppf1] (0,18) node [mynode,left,black] {1000} to[out=0,in=105] (24,0) node [mynode,below,black] {6000};
	\path [name path=combo] (0,10.8) -- +(30,0);
	\path [name path=arrow] (0,3) -- +(30,0);
	\draw [name intersections={of=combo and ppf1, by=E}]
	[dotted,thick] (yaxis |- E) node [mynode,left] {600} -| (xaxis -| E) node [mynode,below] {4000};
	\draw [name intersections={of=arrow and ppf1, by=i1},name intersections={of=arrow and ppf2, by=i2}]
	[->,thick,shorten >=1mm,shorten <=1mm] (i1) -- (i2);
	\end{tikzpicture}
\end{center}
\end{sol}
\end{ex}

\begin{ex}\label{ex:ch1ex3}
Using the $PPF$ that you have graphed using the data in Exercise~\ref{ex:ch1ex2}, determine if the following combinations are attainable or not: $(X=3000,Y=720)$, $(X=4800,Y=480)$.
\begin{sol}
	By examining the opportunity cost in the region where the combinations are defined, and by assuming a linear trade-off between each set of combinations, it can be seen that the first combination in the table is feasible, but not the second combination.
	
\end{sol}
\end{ex}

\begin{ex}\label{ex:ch1ex4}
You and your partner are highly efficient people. You can earn \$50 per hour in the workplace; your partner can earn \$60 per hour.
\begin{enumerate}
	\item	What is the opportunity cost of one hour of leisure for you?
	\item What is the opportunity cost of one hour of leisure for your partner?
	\item Now draw the $PPF$ for yourself where hours of leisure is on the horizontal axis and income in dollars is on the vertical axis. You can assume that you have 12 hours of time each day to allocate to work (income generation) or leisure.
	\item	Draw the $PPF$ for your partner.
	\item	If there is no domestic cleaning service in your area, which of you should do the housework, assuming that you are equally efficient at housework?
\end{enumerate}
\begin{sol}
\begin{enumerate}
	\item	\$50.
	\item	\$60.
	\item	See diagram.
	\item	See diagram.
	\item	The person with the lower wage.
\end{enumerate}
\begin{center}
	\begin{tikzpicture}[background color=figurebkgdcolour,use background,xscale=0.3,yscale=0.3]
	\draw [thick] (0,20) node (yaxis) [mynode1,above] {\$ Inc} |- (20,0) node (xaxis) [mynode1,right] {Leisure};
	\draw [ultra thick,ppfcolourone] (0,10) node [mynode,left,black] {\$600} -- (15,0);
	\draw [ultra thick,ppfcolourtwo] (0,15) node [mynode,left,black] {\$720} -- (15,0) node [mynode,below,black] {12};
	\end{tikzpicture}
\end{center}
\end{sol}
\end{ex}

\begin{ex}\label{ex:ch1ex5}
Louis and Carrie Anne are students who have set up a summer business in their neighbourhood. They cut lawns and clean cars. Louis is particularly efficient at cutting the grass -- he requires one hour to cut a typical lawn, while Carrie Anne needs one and one half hours. In contrast, Carrie Anne can wash a car in a half hour, while Louis requires three quarters of an hour.
\begin{enumerate}
	\item If they decide to specialize in the tasks, who should cut the grass and who should wash cars?
	\item If they each work a twelve hour day, how many lawns can they cut and how many cars can they wash if they specialize in performing the work?
\end{enumerate}
\begin{sol}
\begin{enumerate}
	\item	Louis has an advantage in cutting the grass while Carrie Anne should wash cars.
	\item	If they each work a twelve-hour day, between them they can cut 12 lawns and wash 24 cars.
\end{enumerate}
\end{sol}
\end{ex}

\begin{ex}\label{ex:ch1ex6}
In Exercise~\ref{ex:ch1ex5}, illustrate the $PPF$ for each individual where lawns are on the horizontal axis and car washes on the vertical axis. Carefully label the intercepts. Then construct the economy-wide $PPF$ using this information.
\begin{sol}
	Following the method described in the text:
	\begin{center}
	\begin{tikzpicture}[background color=figurebkgdcolour,use background,xscale=0.3,yscale=0.15]
		\draw [thick] (0,40) node (yaxis) [mynode1,above] {Cars} |- (25,0) node (xaxis) [mynode1,right] {Lawns};
		\draw [ultra thick,ppfcolourone] (0,24) node [mynode,left,black] {24} -- node [mynode,above right,pos=0.25,black] {C.A.} (8,0) node [mynode,below,black] {8};
		\draw [ultra thick,ppfcolourtwo] (0,15) node [mynode,left,black] {15} -- node [mynode,above right,pos=0.75,black] {Louis} (12,0) node [mynode,below,black] {12};
		\draw [ultra thick,ppfcolourthree] (0,39) node [mynode,left,black] {39} -- (12,24) node [mynode,above right,black] {12 lawns, 24 cars} -- (20,0) node [mynode,below,black] {20};
	\end{tikzpicture}
	\end{center}
\end{sol}
\end{ex}

\begin{ex}\label{ex:ch1ex7}
Continuing with the same data set, suppose Carrie Anne's productivity improves so that she can now cut grass as efficiently as Louis; that is, she can cut grass in one hour, and can still wash a car in one half of an hour.
\begin{enumerate}
	\item	In a new diagram draw the $PPF$ for each individual.
	\item	In this case does specialization matter if they are to be as productive as possible as a team?
	\item	Draw the new $PPF$ for the whole economy, labelling the intercepts and kink point coordinates.
\end{enumerate}
\begin{sol}
\begin{enumerate}
	\item	Carrie Anne's lawn intercept is now 12 rather than 8.
	\item	Yes, specialization still matters because C.A. is more efficient at cars.
	\item	The new coordinates will be 39 on the vertical axis, 24 on the horizontal axis and the kink point is the same.
\end{enumerate}
\end{sol}
\end{ex}

\begin{ex}\label{ex:ch1ex8}
Using the economy-wide $PPF$ you have constructed in Exercise~\ref{ex:ch1ex7}, consider the impact of technological change in the economy. The tools used by Louis and Carrie Anne to cut grass and wash cars increase the efficiency of each worker by a whopping 25\%. Illustrate graphically how this impacts the aggregate $PPF$ and compute the three new sets of coordinates.
\begin{sol}
	C.A.'s intercepts are now 30 cars and 15 lawns; Louis' intercepts are 18.75 cars and 15 lawns; the economy-wide $PPF$ car coordinate is thus 48.75, the lawn coordinate is 30, and the kink point is 15 lawns and 30 cars.
	
\end{sol}
\end{ex}

\begin{ex}\label{ex:ch1ex9}
Going back to the simple $PPF$ plotted for Exercise~\ref{ex:ch1ex1} where each of 100 workers can produce either four cakes or three shirts, suppose a recession reduces demand for the outputs to 220 cakes and 129 shirts.
\begin{enumerate}
	\item	Plot this combination of outputs in the diagram that also shows the $PPF$.
	\item	How many workers are needed to produce this output of cakes and shirts?
	\item	What percentage of the 100 worker labour force is unemployed?
\end{enumerate}
\begin{sol}
\begin{enumerate}
	\item	220 cakes requires 55 workers, the remaining 45 workers can produce 135 shirts. Hence this combination lies inside the $PPF$ described in Exercise~\ref{ex:ch1ex1}.
	\item	98 workers.
	\item	2\%.
\end{enumerate}
\end{sol}
\end{ex}

% Closes solutions file for this chapter
\Closesolutionfile{solutions}

\end{enumialphparenastyle}