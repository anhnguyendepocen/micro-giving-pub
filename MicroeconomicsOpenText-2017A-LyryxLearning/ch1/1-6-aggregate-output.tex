\section{Aggregate output, growth and business cycles}\label{sec:ch1sec6}

The $PPF$ can also be used to illustrate three aspects of macroeconomics: the level of a nation's output, the growth of national and per capita output over time, and short run business-cycle fluctuations in national output and employment.

\subsection*{Aggregate output}

An economy's capacity to produce goods and services depends on its endowment of resources and the productivity of those resources. The two-person, two-product examples in the previous section reflect this.

The \terminology{productivity of labour}, defined as output per worker or per hour, depends on:

\begin{itemize}
\item  Skill, knowledge and experience of the labour force;
\item  \terminology{Capital stock}: buildings, machinery, and equipment, and software the labour force has to work with; and
\item  Current technological trends in the labour force and the capital stock.
\end{itemize}

\begin{DefBox}
The \textbf{productivity of labour} is the output of goods and services per worker.

An economy's \textbf{capital stock} is the buildings, machinery, equipment and software used in producing goods and services.
\end{DefBox}

The economy's output, which we define by $Y$, can be defined as the output per worker times the number of workers; hence, we can write:

\begin{equation*}
Y=(\text{number of workers employed})\times(\text{output per worker}).
\end{equation*}

When the employment of labour corresponds to `full employment' in the sense that everyone willing to work at current wage rates and normal hours of work is working, the economy's actual output is also its capacity output $Y_c$. We also term this capacity output as \terminology{full employment output}:

\begin{DefBox}
\textbf{Full employment output} $Y_c=(\text{number of workers at full employment})\times(\text{output per worker})$.
\end{DefBox}

Suppose the economy is operating with full employment of resources producing outputs of two types: goods and services. In Figure~\ref{fig:growthandppf}, $PPF_0$ shows the different combinations of goods and services that the economy could produce in a particular year using all its labour, capital and the best technology available at the time. 

An aggregate economy produces a large variety of outputs in two broad categories. Goods are the products of the agriculture, forestry, mining, manufacturing and construction industries.  Services are provided by the wholesale and retail trade, transportation, hospitality, finance, health care, legal and other service sectors.  As in the two-product examples used earlier, the shape of the $PPF$ illustrates the opportunity cost of increasing the output of either product type.

Point $X_0$ on $PPF_0$ shows one possible structure of capacity output. This combination may reflect the pattern of demand and hence expenditures in this economy.  Output structures differ among economies with different income levels. High-income economies spend more on services than goods and produce higher ratios of services to goods. Middle income countries produce lower ratios of services to goods, and low income countries much lower ratios of services to goods.  Different countries also have different $PPF$s and different output structures, depending on their labour forces, labour productivity and expenditure patterns.

\subsection*{Economic growth}

Three things contribute to growth in the economy. The labour supply grows as the population expands; the stock of capital grows as spending by business on new offices, factories, machinery and equipment expands; and labour-force productivity grows as a result of experience, the development of scientific knowledge combined with product and process innovations, and advances in the technology of production.  Combined, these developments expand capacity output. In Figure~\ref{fig:growthandppf} economic growth shifts the $PPF$ out from $PPF_0$ to $PPF_1$.

% Tikz ch1fig5
\begin{FigureBox}{0.25}{0.25}{25em}{Growth and the PPF \label{fig:growthandppf}}{Economic growth or an increase in the available resources can be envisioned as an outward shift in the $PPF$ from $PPF_0$ to $PPF_1$. With $PPF_1$ the economy can produce more in both sectors than with $PPF_0$.}
\draw [ppfcolourthree,ultra thick,name path=ppf0] (0,16) node [black,mynode,above right] {a} node [black,mynode,left] {$S_{max}$} to [out=0,in=90] (16,0) node [black,mynode,above right] {b} node [black,mynode,below left=0cm and -0.40cm] {$G_{max}$};
\draw [ppfcolourthree,ultra thick,name path=ppf1] (0,18) node [black,mynode,above right] {A} node [black,mynode,left] {$S'_{max}$} to [out=0,in=90] (18,0) node [black,mynode,above right] {B} node [black,mynode,below right=0cm and -0.40cm] {$G'_{max}$};
\draw [thick, -] (0,20) node (yaxis) [above] {Services} |- (25,0) node (xaxis) [right] {Goods};
% path to intersect with ppf0 and ppf1 to create points X_0 and X_1
\path [name path=line1] (0,0) -- (20,20);
% intersection of line1 with pff0 and ppf1 to create dotted lines
\draw [name intersections={of=line1 and ppf0, by=X0},name intersections={of=line1 and ppf1, by=X1}]
	[dotted,thick] (yaxis |- X0) node [mynode,left] {$S_0$} -- (X0) node [mynode,below left] {$X_0$} -- (xaxis -| X0) node [mynode,below] {$G_0$}
	[dotted,thick] (yaxis |- X1) node [mynode,left] {$S_1$} -- (X1) node [mynode,above right] {$X_1$} -- (xaxis -| X1) node [mynode,below] {$G_1$};
% path to create arrow from ppf0 to ppf1
\path [name path=line2] (0,5) -- (10,20);
% intersection of line2 with ppf0 and ppf1 to create arrow
\draw [name intersections={of=line2 and ppf0, by=P0},name intersections={of=line2 and ppf1, by=P1}]
	[->,thick,shorten >=1mm,shorten <=1mm] (P0) node [mynode,below left] {$PPF_0$} -- (P1) node [mynode,above right] {$PPF_1$};
\end{FigureBox}

This basic description of economic growth covers the key sources of growth in total output. Economies differ in their rates of overall economic growth as a result of different rates of growth in labour force, in capital stock, and improvements in the technology. But improvements in standards of living require more than growth in total output. Increases in output \textit{per worker} and \textit{per person} are necessary.  Sustained increases in living standards require sustained growth in labour productivity based on advances in the technologies used in production. 

\subsection*{Recessions and booms}

The objective of economic policy is to ensure that the economy operates on or near the $PPF$ -- it would use its resources to capacity and have minimal unemployment. However, economic conditions are seldom tranquil for long periods of time. Unpredictable changes in business expectations of future profits, in consumer confidence, in financial markets, in commodity and energy prices, in output and incomes in major trading partners, in government policy and many other events disrupt patterns of expenditure and output. Some of these changes disturb the level of total expenditure and thus the demand for total output. Others disturb the conditions of production and thus the economy's production capacity. Whatever the exact cause, the economy may be pushed off its current $PPF$. If expenditures on goods and services decline the economy may experience a \terminology{recession}. Output would fall short of capacity output and unemployment would rise. Alternatively, times of rapidly growing expenditure and output may result in an economic \terminology{boom}: output and employment expand beyond capacity levels.

\begin{DefBox}
An \textbf{economic recession} occurs when output falls below the economy's capacity output.

A \textbf{boom} is a period of high growth that raises output above normal capacity output. 
\end{DefBox}

Recent history provides examples. Following the U.S financial crisis in 2008-09 many industrial countries were pushed into recessions. Expenditure on new residential construction collapsed for lack of income and secure financing, as did business investment, spending and exports. Lower expenditures reduced producers' revenues, forcing cuts in output and employment and reducing household incomes. Lower incomes led to further cutbacks in spending. In Canada in 2009 aggregate output declined by 2.9 percent, employment declined by 1.6 percent and the unemployment rate rose from 6.1 percent in 2008 to 8.3 percent. Although economic growth recovered, that growth had not been strong enough to restore the economy to capacity output at the end of 2011. The unemployment rate fell to 7.4 but did not return to its pre-recession value.  

An economy in a recession is operating inside its $PPF$. The fall in output from X to Z in Figure~\ref{fig:growthrecessions} illustrates the effect of a recession. Expenditures on goods and services have declined. Output is less than capacity output, unemployment is up and some plant capacity is idle. Labour income and business profits are lower.  More people would like to work and business would like to produce and sell more output but it takes time for interdependent product, labour and financial markets in the economy to adjust and increase employment and output. Monetary and fiscal policy may be needed to stimulate demand, increase output and employment and move the economy back to capacity output and full employment. The development and implementation of such policies form the core of macroeconomics. 

Alternatively, an unexpected increase in demand for exports would increase output and employment. Higher employment and output would increase incomes and expenditure, and in the process spread the effects of higher output sales to other sectors of the economy. The economy would move outside its $PPF$ as at W in Figure~\ref{fig:growthrecessions} by using its resources more intensively than normal. Unemployment would fall and overtime work would increase. Extra production shifts would run plant and equipment for longer hours and work days than were planned when it was designed and installed. Output at this level may not be sustainable, because shortages of labour and materials along with excessive rates of equipment wear and tear would push costs and prices up. Again we will examine how the economy reacts to such a state in our macroeconomic analysis. 

% Tikz ch1fig6
\begin{FigureBox}{0.25}{0.25}{25em}{Booms and recessions \label{fig:growthrecessions}}{Economic recessions leave the economy below its normal capacity; the economy might be driven to a point such as Z. Economic expansions, or booms, may drive capacity above its normal level, to a point such as W.}
\draw [ppfcolourthree,ultra thick,name path=ppf] (0,16) node [black,mynode,left] {$S_{max}$} node [black,mynode,above right] {a} to [out=0,in=90] node [mynode,above right,black,pos=0.15] {$PPF$} (16,0) node [black,mynode,below] {$G_{max}$} node [black,mynode,above right] {b};
\draw [thick, -] (0,20) node (yaxis) [above] {Services} |- (25,0) node (xaxis) [right] {Goods};
% path to intersect with ppf curve
\path [name path=Xpath] (0,0) -- (20,20);
% intersection of Xpath with ppf
\draw [name intersections={of=Xpath and ppf, by=X}]
	[dotted,thick] (yaxis |- X) node [mynode,left] {$S_0$} -- (X) node [mynode,above] {X} -- (xaxis -| X) node [mynode,below] {$G_0$} % dotted line to X and axes
	[dotted,thick] ([yshift=-2.5cm]yaxis |- X) -- ([xshift=-2.5cm,yshift=-2.5cm]X) node [mynode,below left] {Z} -- ([xshift=-2.5cm]xaxis -| X) % dotted line to Z and axes
	[dotted,thick] ([yshift=2.5cm]yaxis |- X) -- ([xshift=2.5cm,yshift=2.5cm]X) node [mynode,above right] {W} -- ([xshift=2.5cm]xaxis -| X); % dotted line to W and axes
% arrow from X to Z
\draw [->,thick,shorten >=1mm,shorten <=1mm] (X) -- ([xshift=-2.5cm,yshift=-2.5cm]X);
% arrow from X to W
\draw [->,thick,shorten >=1mm,shorten <=1mm] (X) -- ([xshift=2.5cm,yshift=2.5cm]X);
\end{FigureBox}

Output and employment in the Canadian economy over the past twenty years fluctuated about growth trend in the way Figure~\ref{fig:growthrecessions} illustrates. For several years prior to 2008 the Canadian economy operated slightly above the economy's capacity; but once the recession arrived monetary and fiscal policy were used to fight it -- to bring the economy back from a point such as Z to a point such as X on the $PPF$. 

\subsection*{Macroeconomic models and policy}

The $PPF$ diagrams illustrate the main dimensions of macroeconomics: capacity output, growth in capacity output and business cycle fluctuations in actual output relative to capacity. But these diagrams do not offer explanations and analysis of macroeconomic activity. We need a macroeconomic \textit{model} to understand and evaluate the causes and consequences of business cycle fluctuations. As we shall see, these models are based on explanations of expenditure decisions by households and business, financial market conditions, production costs and producer pricing decisions at different levels of output. Models also capture the objectives fiscal and monetary policies and provide a framework for policy evaluation.  A full macroeconomic model integrates different sector behaviours and the feedback across sectors that can moderate or amplify the effects of changes in one sector on national output and employment. 

Similarly, an economic growth model provides explanations of the sources and patterns of economic growth. Demographics, labour market structures and institutions, household expenditure and saving decisions, business decisions to spend on new plant and equipment and on research and development,  government policies in support of education, research, patent protection, competition and international trade conditions interact in the growth process. They drive the growth in the size and productivity of the labour force, the growth in the capital stock, and the advances in technology that are the keys to growth in aggregate output and output per person.

\section*{Conclusion}

We have covered a lot of ground in this introductory chapter. It is intended to open up the vista of economics to the new student in the discipline. Economics is powerful and challenging, and the ideas we have developed here will serve as conceptual foundations for our exploration of the subject. Our next chapter deals with methods and models in greater detail.