\section{Opportunity cost and the market}\label{sec:ch1sec3}

Individuals face choices at every turn: In deciding to go to the hockey game tonight, you may have to forgo a concert; or you will have to forgo some leisure time this week order to earn additional income for the hockey game ticket. Indeed, there is no such thing as a free lunch, a free hockey game or a free concert. In economics we say that these limits or constraints reflect opportunity cost. The \terminology{opportunity cost} of a choice is what must be sacrificed when a choice is made. That cost may be financial; it may be measured in time, or simply the alternative foregone.

\begin{DefBox}
\textbf{Opportunity cost}:  what must be sacrificed when a choice is made.
\end{DefBox}

Opportunity costs play a determining role in markets. It is precisely because individuals and organizations have different opportunity costs that they enter into exchange agreements. If you are a skilled plumber and an unskilled gardener, while your neighbour is a skilled gardener and an unskilled plumber, then you and your neighbour not only have different capabilities, you also have different opportunity costs, and \textit{you could gain by trading your skills}. Here's why. Fixing a leaking pipe has a low opportunity cost for you in terms of time: you can do it quickly. But pruning your apple trees will be costly because you must first learn how to avoid killing them and this may require many hours. Your neighbour has exactly the same problem, with the tasks in reverse positions. In a sensible world you would fix your own pipes \textit{and} your neighbour's pipes, and she would ensure the health of the apple trees in both backyards.

If you reflect upon this `sensible' solution---one that involves each of you achieving your objectives while minimizing the time input---you will quickly realize that it resembles the solution provided by the marketplace. You may not have a gardener as a neighbour, so you buy the services of a gardener in the marketplace. Likewise, your immediate neighbour may not need a leaking pipe repaired, but many others in your neighbourhood do, so you sell your service to them. You each specialize in the performance of specific tasks as a result of having different opportunity costs or different efficiencies. Let us now develop a model of exchange to illustrate the advantages of specialization and trade, and hence the markets that facilitate these activities. This model is developed with the help of some two-dimensional graphics.