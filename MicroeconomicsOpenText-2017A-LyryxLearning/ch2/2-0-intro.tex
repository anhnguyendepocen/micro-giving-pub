\chapter{Theories, models and data} \label{chap:tmd}

\begin{topics}
\textbf{In this chapter we will explore:}
\begin{description}
\item [~\ref{sec:ch2sec1}] Economic theories and models
\item [~\ref{sec:ch2sec2}] Variables, data \& index numbers
\item [~\ref{sec:ch2sec23}] Testing, accepting, and rejecting models
\item [~\ref{sec:ch2sec4}] Diagrams and economic analysis
\item [~\ref{sec:ch2sec5}] Ethics, efficiency and beliefs
\end{description}
\end{topics}

Economists, like other scientists and social scientists are interested observers of behaviour and events. Economists are concerned primarily with the economic causes and consequences of what they observe. They want to understand the economics of an extensive range of human experience including: money, government finances, industrial production, household consumption, inequality in income distribution, war, monopoly power, professional and amateur sports, pollution, marriage, music, art and much more.

Economists approach these issues using economic theories and models. To present, explain, illustrate and evaluate their theories and models they have developed a set of techniques or tools. These involve verbal descriptions and explanations, diagrams, algebraic equations, data tables and charts and statistical tests of economic relationships.

This chapter covers these basic techniques of economic analysis.
