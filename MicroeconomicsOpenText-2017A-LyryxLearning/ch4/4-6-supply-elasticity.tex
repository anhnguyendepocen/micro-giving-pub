\section{Elasticity of supply}\label{sec:ch4sec6}

Now that we have developed the various dimensions of elasticity on the demand side, the analysis of elasticities on the supply side is straightforward. The \terminology{elasticity of supply} measures the responsiveness of the quantity supplied to a change in the price.

\begin{DefBox}
The \textbf{elasticity of supply} measures the responsiveness of quantity supplied to a change in the price.
\end{DefBox}

\begin{equation*}
\varepsilon_s=\frac{\%\Delta Q}{\%\Delta P}
\end{equation*}

The subscript $s$ denotes supply. This is exactly the same formula as for the demand curve, except that the quantities now come from a supply curve. Furthermore, and in contrast to the demand elasticity, the supply elasticity is generally a positive value because of the positive relationship between price and quantity supplied. The more elastic, or the more responsive, is supply to a given price change, the larger will be the elasticity value. In diagrammatic terms, this means that ``flatter'' supply curves have a greater elasticity than more ``vertical'' curves at a given price and quantity combination. Numerically the flatter curve has a larger value than the more vertical supply -- try drawing a supply diagram similar to Figure~\ref{fig:limitingcasepriceelasticity}. Technically, a completely vertical supply curve has a zero elasticity and a horizontal supply curve has an infinite elasticity -- just as in the demand cases. 

As always we keep in mind the danger of interpreting too much about the value of this elasticity from looking at the visual profiles of supply curves.