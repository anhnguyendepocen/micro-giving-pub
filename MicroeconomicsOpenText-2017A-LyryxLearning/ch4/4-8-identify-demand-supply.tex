\section{Identifying demand and supply elasticities}\label{sec:ch4sec8}

Elasticities are very useful pieces of evidence on economic behaviour. But we need to take care in making inferences from what we observe in market data. Upon observing price and expenditure changes in a given market, it is tempting to infer that we can immediately calculate a demand elasticity.  But should we be thinking about supply elasticities? Let us look at the information needed before rushing into calculations. 

In order to identify a demand elasticity we need to be sure that we have price and quantity values \textit{that lie on the same demand curve}. And if we do indeed observe several price and quantity pairs that reflect a market equilibrium on a demand curve, then it must be the case that those combinations are caused by a shifting supply curve. Consider Figure~\ref{fig:indentifyelast}. Suppose that we observe a series of prices and accompanying quantities traded in three consecutive months, and we plot these combinations to yield points A, B, C in panel (a) of the figure. If these points are market equilibria, \textit{and if they lie on the same demand curve}, it must be the case that the supply curve has shifted. That is, if we can draw a single demand curve through these points, as in panel (b), the only way that they each reflect demand conditions is for the supply curve to have shifted to create these points as equilibria in the market. 

% Figure 4.10
\begin{FigureBox}{0.4}{0.3}{25em}{Identifying elasticities \label{fig:indentifyelast}}{In order to establish that points such as A, B and C in Panel (a) lie on the same demand curve, we must know that the supply curve alone has shifted in such a way as to result in these equilibrium price-quantity combinations, as illustrated in Panel (b).}
\draw [supplycolour,ultra thick,-]
	(15,0) -- (18,14) node [black,mynode,right] {$S_a$}
	(16,0) -- (22.3,14) node [black,mynode,right] {$S_b$}
	(18,0) -- (24,8) node [black,mynode,right] {$S_c$};
\draw [demandcolour,ultra thick,-] (15,12) -- (24,0) node [mynode,above right,pos=0.95,black] {$D$};
\draw [thick, -]
	(0,15) node [above] {Price} |- (10,0) node [right] {Quantity}
	(15,15) node [above] {Price} |- (25,0) node [right] {Quantity};
\node [mynode1,below] at (5,-0.5) {(a)};
\node [mynode1,below] at (20,-0.5) {(b)};
\fill [black] (2,9.333) circle (5pt) node [mynode,right] {A};
\fill [black] (17,9.333) circle (5pt) node [mynode,right] {A};
\fill [black] (4,6.666) circle (5pt) node [mynode,right] {B};
\fill [black] (19,6.666) circle (5pt) node [mynode,right] {B};
\fill [black] (6,4) circle (5pt) node [mynode,right] {C};
\fill [black] (21,4) circle (5pt) node [mynode,right] {C};
\end{FigureBox}

Exactly the same logic holds if we can infer that market equilibrium points all lie on the same supply curve. In that case the demand curve must have shifted in order to be able to identify the points as belonging to the supply curve. 

This challenge is what we call the identification problem in econometrics. Frequently new combinations of price and quantity reflect shifts in both the supply curve and demand curve, and we need to call upon the econometricians to tell us what shifts are taking place in the market.
