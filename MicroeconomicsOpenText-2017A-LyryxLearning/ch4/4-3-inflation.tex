\section{The time horizon and inflation} \label{sec:inflation}

The price elasticity of demand is frequently lower in the short run than in the long run. For example, a rise in the price of home heating oil may ultimately induce consumers to switch to natural gas or electricity, but such a transition may require a considerable amount of time. Time is required for decision-making and investment in new heating equipment. A further example is the elasticity of demand for tobacco. Some adults who smoke may be seriously dependent and find quitting almost impossible. But if young smokers, who are not yet addicted, decide to quit on account of the higher price, then over a long period of time the percentage of the population that smokes will decline. The full impact may take decades! So when we talk of the short run and the long run, there is no simple rule for defining how long the long run actually is in terms of months or years. In some cases, adjustment may be complete in weeks, in other cases years.

In Chapter~\ref{chap:tmd} we distinguished between real and nominal variables. The former adjust for inflation; the latter do not. Suppose all nominal variables double in value: Every good and service costs twice as much, wage rates double, dividends and rent double, etc. This implies that whatever bundle of goods was previously affordable is still affordable. Nothing has really changed. Demand behaviour is unaltered by this doubling of all prices and all incomes. 

How do we reconcile this with the idea that own-price elasticities measure changes in quantity demanded as prices change? Keep in mind that elasticities measure the impact of changing one variable alone, \textit{holding constant all of the others}. But when all variables are changing simultaneously, it is incorrect to think that the impact on quantity of one price or income change is a true measure of responsiveness or elasticity. The \textit{price changes that go into measuring elasticities are therefore changes in relative prices}.