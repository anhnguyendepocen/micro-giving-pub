\newpage
\section*{Exercises for Chapter~\ref{chap:elasticities}}

\begin{enumialphparenastyle}

% Solutions file for exercises opened
\Opensolutionfile{solutions}[solutions/ch4ex]

\begin{ex}\label{ex:ch4ex1}
Consider the information in the table below that describes the demand for movie rentals from your on-line supplier Instant Flicks.
\begin{center}
\begin{tabu} to \linewidth {|X[1,c]X[1,c]X[1,c]X[1,c]|}	\hline
\rowcolor{rowcolour}	\textbf{Price per movie (\$)}	&	\textbf{Quantity demanded}	&	\textbf{Total revenue}	&	\textbf{Elasticity of demand}	\\
						2	&	1200	&	&	\\
\rowcolor{rowcolour}	3	&	1100	&	&	\\
						4	&	1000	&	&	\\
\rowcolor{rowcolour}	5	&	900		&	&	\\
						6	&	800		&	&	\\
\rowcolor{rowcolour}	7	&	700		&	&	\\
						8	&	600		&	&	\\	\hline
\end{tabu}
\end{center}
\begin{enumerate}
	\item	Either on graph paper or a spreadsheet, map out the demand curve.
	\item	In column 3, insert the total revenue generated at each price.
	\item	At what price is total revenue maximized?
	\item	In column 4, compute the elasticity of demand corresponding to each \$1 price reduction, using the average price and quantity at each state.
	\item	Do you see a connection between your answers in parts (c) and (d)?
\end{enumerate}
\begin{sol}
\begin{enumerate}
	\item	The intercepts for this straight line demand curve are $P=\$14$, $Q=1400$.
	\item	Total revenue is this product of price times quantity. Compute it!
	\item	At $P=\$7$, total revenue is \$4,900.
	\item	Elasticities, in descending order, are 0.22, 0.33, 0.47, 0.65, 0.87, 1.15.
	\item	Elasticity becomes greater than one in magnitude at one point where total revenue is maximized.
\end{enumerate}
\begin{center}
	\begin{tikzpicture}[background color=figurebkgdcolour,use background,xscale=0.3,yscale=0.25]
	\draw [thick] (0,20) node (yaxis) [mynode1,above] {$P$} |- (25,0) node (xaxis) [mynode1,right] {$Q$};
	\draw [ultra thick,demandcolour] (0,14) node [mynode,left,black] {14} -- coordinate [midway] (arrowpoint) node [mynode,black,above right,pos=0.2] {$\varepsilon>1$ at $P>\$7$} (23,0) node [mynode,below,black] {1400};
	\draw [<-,thick,shorten <=1mm] (arrowpoint) -- +(3,3) node [mynode,right] {$\varepsilon=1$ at the mid-point of\\the demand curve: $P=\$7$.};
	\end{tikzpicture}
\end{center}
\end{sol}
\end{ex}

\begin{ex}\label{ex:ch4ex2}
Your fruit stall has 100 ripe bananas that must be sold today. Your supply curve is therefore vertical. From past experience, you know that these 100 bananas will all be sold if the price is set at 40 cents per unit.
\begin{enumerate}
	\item	Draw a supply and demand diagram illustrating the market equilibrium price and quantity.
	\item	The demand elasticity is -0.5 at the equilibrium price. But you now discover that 10 of your bananas are rotten and cannot be sold. Draw the new supply curve and calculate the percentage price increase that will be associate with the new equilibrium, on the basis of your knowledge of the demand elasticity.
\end{enumerate}
\begin{sol}
\begin{enumerate}
	\item	The supply curve is vertical at a quantity of 100.
	\item	We are told $-0.5=\%\Delta Q/\%\Delta P$. The percentage change of quantity is $-10/95$; therefore the percentage change in price must be: $\%\Delta P=-(10/95)/-0.5=20/95=21\%$. The new price is therefore $0.4\times 1.21=0.48$.
\end{enumerate}
\begin{center}
	\begin{tikzpicture}[background color=figurebkgdcolour,use background,xscale=0.3,yscale=0.25]
	\draw [thick] (0,20) node (yaxis) [mynode1,above] {$P$} |- (25,0) node (xaxis) [mynode1,right] {$Q$};
	\draw [ultra thick,demandcolour,name path=D] (5,17) -- (23,5) node [mynode,right,black] {$D$};
	\draw [ultra thick,supplycolour,name path=S90] (9,0) -- +(0,20) node [mynode,above,black] {$S=90$};
	\draw [ultra thick,supplycolour,name path=S100] (14,0) -- +(0,20) node [mynode,above,black] {$S=100$};
	\draw [name intersections={of=S100 and D, by=E}]
	[dotted,thick] (yaxis |- E) node [mynode,left] {0.4} -- (E);
	\end{tikzpicture}
\end{center}
\end{sol}
\end{ex}

\begin{ex}\label{ex:ch4ex3}
University fees in the State of Nirvana have been frozen in real terms for 10 years. During this period enrolments increased by 20 percent.
\begin{enumerate}
	\item	Draw a supply curve and two demand curves to represent the two equilibria described. 
	\item	Can you estimate a price elasticity of demand for university education in this market?
	\item	In contrast, during the same time period fees in a neighbouring state increased by 60 percent and enrolments increased by 15 percent. Illustrate this situation in a diagram.
\end{enumerate}
\begin{sol}
\begin{enumerate}
	\item	Since the price is fixed the supply curve is horizontal. See figure below.
	\item	You cannot estimate a demand elasticity value since there has been no price change.
	\item	Here the (adjoining) horizontal supply curve shifts upwards by 60\%. If enrolment has increased the demand curve must also have shifted upwards. Draw an additional supply curve representing a 60\% upward shift, and find an intersection between the new demand and new supply such that the percentage increase in quantity is 15\% (this diagram is not included here).
\end{enumerate}
\begin{center}
	\begin{tikzpicture}[background color=figurebkgdcolour,use background,xscale=0.3,yscale=0.25]
	\draw [thick] (0,20) node (yaxis) [mynode1,above] {$P$} |- (25,0) node (xaxis) [mynode1,right] {$Q$};
	\draw [ultra thick,demandcolour,name path=D0] (3,19) -- (16,3);
	\draw [ultra thick,demandcolour,name path=D1] (10,19) -- (23,3);
	\draw [ultra thick,supplycolour,name path=S] (0,11) -- coordinate[midway] (A) +(24,0) node [mynode,right,black] {$S$};
	\draw [name intersections={of=S and D0, by=i0},name intersections={of=S and D1, by=i1}]
	[dotted,thick] (i0) -- (xaxis -| i0)
	[dotted,thick] (i1) -- (xaxis -| i1);
	\draw [<-,thick,shorten <=1mm] (A) -- +(5,5) node [mynode,right] {$\%\Delta Q=20\%$};
	\end{tikzpicture}
\end{center}
\end{sol}
\end{ex}

\begin{ex}\label{ex:ch4ex4}
Consider the demand curve defined by the information in the table below.
\begin{center}
\begin{tabu} to \linewidth {|X[1,c]X[1,c]X[1,c]X[1,c]|}	\hline
\rowcolor{rowcolour}	\textbf{Price of movies}	&	\textbf{Quantity demanded}	&	\textbf{Total revenue}	&	\textbf{Elasticity of demand}	\\
						2	&	200	&	&	\\
\rowcolor{rowcolour}	3	&	150	&	&	\\
						4	&	120	&	&	\\
\rowcolor{rowcolour}	5	&	100	&	&	\\	\hline
\end{tabu}
\end{center}
\begin{enumerate}
	\item	Plot the demand curve to scale and note that it is non-linear.
	\item	Compute the total revenue at each price.
	\item	Compute the arc elasticity of demand for each price segment.
\end{enumerate}
\begin{sol}
\begin{enumerate}
	\item	The demand curve is nonlinear.
	\item	Total revenue is price times quantity.
	\item	Elasticity values are 0.71, 0.78, and 0.82 respectively.
\end{enumerate}
\begin{center}
	\begin{tikzpicture}[background color=figurebkgdcolour,use background]
	\begin{axis}[
	axis line style=thick,
	every tick label/.append style={font=\footnotesize},
	ymajorgrids,
	grid style={dotted},
	every node near coord/.append style={font=\scriptsize},
	xticklabel style={rotate=90,anchor=east,/pgf/number format/1000 sep=},
	scaled y ticks=false,
	yticklabel style={/pgf/number format/fixed,/pgf/number format/1000 sep = \thinspace},
	xmin=0,xmax=250,ymin=0,ymax=6,
	y=1cm/1,
	x=1.5cm/50,
	x label style={at={(axis description cs:0.5,-0.05)},anchor=north},
	xlabel={Quantity Demanded},
	ylabel={Price of Movies},
	]
	\addplot[datasetcolourone,ultra thick,mark=none] table {
		X		Y
		200		2
		150		3
		120		4
		100		5
	};
	\end{axis}
	\end{tikzpicture}
\end{center}
\end{sol}
\end{ex}

\begin{ex}\label{ex:ch4ex5}
The demand curve for seats at the Drive-in Delight Theatre is given by $P=48-0.2Q$. The supply of seats is given by $Q=40$.
\begin{enumerate}
	\item	Plot the supply and demand curves to scale, and estimate the equilibrium price.
	\item	At this equilibrium point, calculate the elasticities of demand and supply.
	\item	The owner has additional space in his theatre, and is considering the installation of more seats. He then remembers from his days as an economics student that this addition might not necessarily increase his total revenue. If he hired you as a consultant, would you recommend to him that he install additional seats or that he take out some of the existing seats and install a popcorn concession instead?  [Hint: You can use your knowledge of the elasticities just estimated to answer this question.]
	\item	For this demand curve, over what range of prices is demand inelastic?
\end{enumerate}
\begin{sol}
The supply curve is vertical at $Q=40$. Substituting this quantity into the demand equation yields an equilibrium price of \$40.
\begin{enumerate}
	\item	The supply curve is vertical at $Q=40$. Substituting this quantity into the demand equation yields an equilibrium price of \$40.
	\item	The supply elasticity is zero and the demand elasticity is $-5.0$. The latter is obtained by noting that $\Delta P/\Delta Q=-0.2$, and $P=\$40$ at $Q=40$. Using the elasticity formula yields $-5.0$.
	\item	Since the elasticity value exceeds unity he should reduce the price and install more seats if his objective is to generate more revenue.
	\item	Above the price \$24, which is the mid-point on the demand curve, demand is elastic.
\end{enumerate}
\begin{center}
	\begin{tikzpicture}[background color=figurebkgdcolour,use background,xscale=0.3,yscale=0.25]
	\draw [thick] (0,20) node (yaxis) [mynode1,above] {$P$} |- (25,0) node (xaxis) [mynode1,right] {$Q$};
	\draw [ultra thick,demandcolour,name path=D] (0,12) node [mynode,left,black] {48} -- node [mynode,above right,black,pos=0.7] {$D$} (24,0) node [mynode,below,black] {240};
	\draw [ultra thick,supplycolour,name path=S] (4,0) -- +(0,19) node [mynode,above,black] {$S=40$};
	\draw [name intersections={of=S and D, by=E}]
	[dotted,thick] (yaxis |- E) node [mynode,left,black] {40} -- (E);
	\end{tikzpicture}
\end{center}
\end{sol}
\end{ex}

\begin{ex}\label{ex:ch4ex6}
Waterson Power Corporation's regulator has just allowed a rate increase from 9 to 11 cents per kilowatt hour of electricity. The short run demand elasticity is -0.6 and the long run demand elasticity is -1.2.
\begin{enumerate}
	\item	What will be the percentage reduction in power demanded in the short run?
	\item	What will be the percentage reduction in power demanded in the long run?
	\item	Will revenues increase or decrease in the short and long runs?
\end{enumerate}
\begin{sol}
\begin{enumerate}
	\item	There has been a 20\% increase in price. Feeding this into the elasticity formula yields $-0.6=\%\Delta Q/20\%$. Hence the percentage change (reduction) in quantity is 12\%.
	\item	Following the same reasoning as in part (a) the result is 24\%.
	\item	In the short run revenue rises since demand is inelastic (less than one in absolute value); in the long run it falls since demand is elastic (greater than one in absolute value).
\end{enumerate}
\end{sol}
\end{ex}

\begin{ex}\label{ex:ch4ex7}
Consider the own- and cross-price elasticity data in the table below.
\begin{center}
\begin{tabu} to \linewidth {X[1.5,c]X[1,c]|X[1,c]X[1,c]X[1,c]|}	\hhline{~~---}
	&	&	\multicolumn{3}{c|}{\cellcolor{rowcolour}\textbf{\% change in price}}	\\
	&	&	\multicolumn{1}{c}{\textbf{CDs}}	&	\textbf{Magazines}	&	\textbf{Cappuccinos}	\\	\hline
\multicolumn{1}{|c}{\cellcolor{rowcolour}}	&	\textbf{CDs}	&	\cellcolor{rowcolour}-0.25	&	\cellcolor{rowcolour}0.06	&	\cellcolor{rowcolour}0.01	\\[-0.25em]
\multicolumn{1}{|c}{\cellcolor{rowcolour}}	&	\textbf{Magazines}	&	\cellcolor{rowcolour}-0.13	&	\cellcolor{rowcolour}-1.20	&	\cellcolor{rowcolour}0.27	\\[-0.25em]
\multicolumn{1}{|c}{\multirow{-3}{*}{\cellcolor{rowcolour}\textbf{\% change in quantity}}}	&	\textbf{Cappuccinos}	&	\cellcolor{rowcolour}0.07	&	\cellcolor{rowcolour}0.41	&	\cellcolor{rowcolour}-0.85	\\	\hline
\end{tabu}
\end{center}
\begin{enumerate}
	\item	For which of the goods is demand elastic and for which is it inelastic?
	\item	What is the effect of an increase in the price of CDs on the purchase of magazines and cappuccinos? What does this suggest about the relationship between CDs and these other commodities; are they substitutes or complements?
	\item	In graphical terms, if the price of CDs or the price of cappuccinos increases, illustrate how the demand curve for magazines shifts.
\end{enumerate}
\begin{sol}
\begin{enumerate}
	\item	It is elastic for magazines and inelastic for CDs and Cappuccinos.
	\item	A reduction in magazines purchased and an increase in cappuccinos purchased. Magazines are complements and cappuccinos are substitutes for CDs.
	\item	The demand curve for magazines shifts down in response to an increase in the price of CDs and it increases in response to an increase in the price of cappuccinos.
\end{enumerate}
\end{sol}
\end{ex}

\begin{ex}\label{ex:ch4ex8}
You are responsible for running the Speedy Bus Company and have information about the elasticity of demand for bus travel: The own-price elasticity is -1.4 at the current price. A friend who works in the competing railway company also tells you that she has estimated the cross-price elasticity of train-travel demand with respect to the price of bus travel to be 1.7.
\begin{enumerate}
	\item	As an economic analyst, would you advocate an increase or decrease in the price of bus tickets if you wished to increase revenue for Speedy?
	\item	Would your price decision have any impact on train ridership?
\end{enumerate}
\begin{sol}
\begin{enumerate}
	\item	Reduce the price, because the elasticity is greater than one.
	\item	Yes, it would reduce train ridership because the positive cross-price elasticity indicates that these goods are substitutes.
\end{enumerate}
\end{sol}
\end{ex}

\begin{ex}\label{ex:ch4ex9}
A household's income and restaurant visits are observed at different points in time. The table below describes the pattern.
\begin{center}
\begin{tabu} to \linewidth {|X[1,c]X[1,c]X[1,c]|}	\hline
\rowcolor{rowcolour}	\textbf{Income (\$)}	&	\textbf{Restaurant visits}	&	\textbf{Income elasticity of demand}	\\
						16,000	&	10	&		\\
\rowcolor{rowcolour}	24,000	&	15	&		\\
						32,000	&	18	&		\\
\rowcolor{rowcolour}	40,000	&	20	&		\\
						48,000	&	22	&		\\
\rowcolor{rowcolour}	56,000	&	23	&		\\
						64,000	&	24	&		\\	\hline
\end{tabu}
\end{center}
\begin{enumerate}
	\item	Construct a scatter diagram showing quantity on the vertical axis and income on the horizontal axis.
	\item	Is there a positive or negative relationship between these variables?
	\item	Compute the income elasticity for each income increase, using midpoint values.
	\item	Are restaurant meals a normal or inferior good?
\end{enumerate}
\begin{sol}
\begin{enumerate}
	\item	Plot the scatter.
	\item	The scatter is a positively sloping group of points indicating a positive relationship.
	\item	The elasticities estimated at mid values are 1.0, 0.64, 0.47, 0.52, 0.29 and 0.32. For example: the first pair of points yields a $\%\Delta P=5/12.5$ and $\%\Delta Q=8,000/20,000$. Hence, $\%\Delta Q/\%\Delta P=(8,000/20,000)/(5/12.5)=1.0$.
	\item	They are normal goods because the income elasticity is positive.
\end{enumerate}
\end{sol}
\end{ex}

\begin{ex}\label{ex:ch4ex10}
Consider the following three supply curves: $P=2.25Q$; $P=2+2Q$; $P=6+1.5Q$.
\begin{enumerate}
	\item	Draw each of these supply curves to scale, and check that, at $P = \$18$, the quantity supplied in each case is the same.
	\item	Calculate the (point) supply elasticity for each curve at this price.
	\item	Now calculate the same elasticities at $P = \$12$.
	\item	One elasticity value should be unchanged. Which one?
\end{enumerate}
\begin{sol}
\begin{enumerate}
	\item	All three supply curves intersect at $P=\$18$ and $Q=8$.
	\item	The supply elasticities are 1.0, 1.125 and 1.5 respectively. These are obtained from substituting the equilibrium $P$ and $Q$	values into Equation~\ref{eq:priceelastdemand} Part (c) in the text, and noting that the slopes, $\Delta Q/\Delta P$, from each equation are 2.25, 2, and 1.5.
	\item	The elasticities are computed in the same way, once you have calculated the equilibrium quantity for each equation at this new
	price: 1.0, 1.2 and 2.0.
	\item	The supply curve through the origin always has a value of unity.
\end{enumerate}
\begin{center}
	\begin{tikzpicture}[background color=figurebkgdcolour,use background,xscale=0.5,yscale=0.25]
	\draw [thick] (0,25) node (yaxis) [mynode1,above] {$P$} |- (15,0) node (xaxis) [mynode1,right] {$Q$};
	\draw [ultra thick,supplycolour,name path=S1] (0,0) -- (10.667,24);
	\draw [ultra thick,dashed,supplycolour,name path=S2] (0,2) node [mynode,left,black] {2} -- (11,24);
	\draw [ultra thick,supplycolour,name path=S3] (0,6) node [mynode,left,black] {6} -- (12,24) node [mynode,right,black] {$P=6+1.5Q$};
	\path [name path=arrowpath] (2,0) -- +(0,25);
	\draw [name intersections={of=S1 and arrowpath, by=s1},name intersections={of=S2 and arrowpath, by=s2}]
	[<-,thick,shorten <=1mm] (s1) -- +(2,0) node [mynode,right] {$P=2.25Q$};
	\draw [<-,thick,shorten <=1mm] (s2) -- +(0,6) node [mynode,above] {$P=2+2Q$};
	\end{tikzpicture}
\end{center}
\end{sol}
\end{ex}

\begin{ex}\label{ex:ch4ex11}
The demand for bags of candy is given by $P=48-0.2Q$, and the supply by $P=Q$.
\begin{enumerate}
	\item	Illustrate the resulting market equilibrium in a diagram.
	\item	If the government now puts a \$12 tax on all such candy bags, illustrate on a diagram how the supply curve will change. 
	\item	Compute the new market equilibrium. 
	\item	Instead of the specific tax imposed in part (b), a percentage tax (ad valorem) equal to 30 percent is imposed. Illustrate how the supply curve would change. 
	\item	Compute the new equilibrium.
\end{enumerate}
\begin{sol}
\begin{enumerate}
	\item	The price intercept for the demand curve is 48 and the quantity intercept is 240. The supply curve goes through the origin with a slope of 1. The equilibrium price is \$40 and the equilibrium quantity is 40.
	\item	The supply curve shifts upwards everywhere by \$12.
	\item	The price will increase to \$42 and the quantity declines to 30.
	\item	The curve still goes through the origin but with a slope of 1.3 rather than 1.0 (not illustrated in the figure).
	\item	The equilibrium quantity is $Q=32$; corresponding price is $P=32\times 1.3=41.6$.
\end{enumerate}
\begin{center}
	\begin{tikzpicture}[background color=figurebkgdcolour,use background,xscale=0.3,yscale=0.25]
	\draw [thick] (0,20) node (yaxis) [mynode1,above] {$P$} |- (25,0) node (xaxis) [mynode1,right] {$Q$};
	\draw [ultra thick,demandcolour,name path=D] (0,12) node [mynode,left,black] {48} -- node [mynode,above right,black,pos=0.7] {$D$} (24,0) node [mynode,below,black] {240};
	\draw [ultra thick,supplycolour,name path=S] (0,0) -- (6,19);
	\draw [ultra thick,dashed,supplycolour,name path=S1] (0,3) -- (5,19) coordinate (arrowpoint);
	\draw [name intersections={of=S and D, by=E}]
	[dotted,thick] (yaxis |- E) node [mynode,left,black] {40} -- (E);
	\draw [<-,thick,shorten <=1mm] (arrowpoint) to[out=75,in=180] +(4,1) node [mynode,right] {New $S$ curve has shifted\\upwards by \$12 (part (b))};
	\end{tikzpicture}
\end{center}
\end{sol}
\end{ex}

\begin{ex}\label{ex:ch4ex12}
Consider the demand curve $P=100-2Q$. The supply curve is given by $P=30$.
\begin{enumerate}
	\item	Draw the supply and demand curves to scale and compute the equilibrium price and quantity in this market.
	\item	If the government imposes a tax of \$10 per unit, draw the new equilibrium and compute the new quantity traded and the amount of tax revenue generated. 
	\item	Is demand elastic or inelastic in this price range?
\end{enumerate}
\begin{sol}
\begin{enumerate}
	\item	The demand curve has a price intercept of 100 and a quantity intercept of 50. The supply curve is horizontal at a price of \$30. The equilibrium quantity is 35 units at this price.
	\item	The new, tax-inclusive, supply curve is horizontal at $P=\$40$ (not illustrated in the figure). The equilibrium price is \$40 and the	equilibrium quantity becomes 30. With 30 units sold, each generating a tax of \$10, total tax revenue is \$300.
	\item	Since the equilibrium is on the lower half of a linear demand curve the demand is inelastic.
\end{enumerate}
\begin{center}
	\begin{tikzpicture}[background color=figurebkgdcolour,use background,xscale=0.3,yscale=0.25]
	\draw [thick] (0,20) node (yaxis) [mynode1,above] {$P$} |- (25,0) node (xaxis) [mynode1,right] {$Q$};
	\draw [ultra thick,demandcolour,name path=D] (0,15) node [mynode,left,black] {100} -- node [mynode,above right,black,pos=0.3] {$D$} (20,0) node [mynode,below,black] {50};
	\draw [ultra thick,supplycolour,name path=S] (0,5) -- +(23,0) node [mynode,right,black] {$P=30$};
	\draw [name intersections={of=D and S, by=E}]
	[dotted,thick] (E) -- (xaxis -| E) node [mynode,below] {35};
	\end{tikzpicture}
\end{center}
\end{sol}
\end{ex}

\begin{ex}\label{ex:ch4ex13}
In Exercise~\ref{ex:ch4ex12}: As an alternative to shifting the supply curve, try shifting the demand curve to reflect the \$10 tax being imposed on the consumer.
\begin{enumerate}
	\item	Solve again for the price that the consumer pays, the price that the supplier receives and the tax revenue generated.
	\item	Compare your answers with the previous question; they should be the same.
\end{enumerate}
\begin{sol}
	As illustrated in the text, we could equally shift the demand curve down by \$10 to yield $P=90-2Q$. Equating this to $P=30$ yields $Q=30$ once again. The price of \$30 here is what goes to the supplier; the buyer must pay this plus the tax -- that is \$40.
	
\end{sol}
\end{ex}

\begin{ex}\label{ex:ch4ex14}
The supply of Henry's hamburgers is given by $P=2+0.5Q$; demand is given by $Q=20$.
\begin{enumerate}
	\item	Illustrate and compute the market equilibrium.
	\item	A specific tax of \$3 per unit is subsequently imposed and that shifts the supply curve to $P=5+0.5Q$. Solve for the equilibrium price and quantity after the tax. 
	\item	Who bears the burden of the tax in parts (a) and (b)?
\end{enumerate}
\begin{sol}
\begin{enumerate}
	\item	The supply and demand curves are illustrated below.
	\item	Solving the demand equations for $Q=20$ yields prices of \$12 and \$15 respectively.
	\item	The consumer bears the entire tax burden.
\end{enumerate}
\begin{center}
	\begin{tikzpicture}[background color=figurebkgdcolour,use background,xscale=0.2,yscale=0.25]
	\draw [thick] (0,25) node (yaxis) [mynode1,above] {$P$} |- (40,0) node (xaxis) [mynode1,right] {$Q$};
	\draw [ultra thick,supplycolour,name path=S,domain=0:35] plot (\x,{2+0.5*\x}) node [mynode,right,black] {$P=2+0.5Q$};
	\draw [ultra thick,supplycolour,name path=S1,domain=0:35] plot (\x,{5+0.5*\x}) node [mynode,right,black] {$P=5+0.5Q$};
	\draw [ultra thick,demandcolour,name path=Q] (20,0) -- +(0,24) node [mynode,above,black] {$Q=20$};
	\draw [name intersections={of=Q and S, by=12},name intersections={of=Q and S1, by=15}]
	[dotted,thick] (yaxis |- 12) node [mynode,left] {12} -- (12)
	[dotted,thick] (yaxis |- 15) node [mynode,left] {15} -- (15);
	\end{tikzpicture}
\end{center}
\end{sol}
\end{ex}

% Closes solutions file for this chapter
\Closesolutionfile{solutions}

\end{enumialphparenastyle}