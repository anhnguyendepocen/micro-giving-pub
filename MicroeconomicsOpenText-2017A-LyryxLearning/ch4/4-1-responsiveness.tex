\section{Price responsiveness of demand}\label{sec:ch4sec1}

Put yourself in the position of an entrepreneur. One of your many challenges is to price your product appropriately. You may be Michael Dell choosing a price for your latest computer, or the local restaurant owner pricing your table d'h\^{o}te, or you may be pricing your part-time snow-shoveling service. A key component of the pricing decision is to know how \textit{responsive} your market is to variations in your pricing. How we measure responsiveness is the subject matter of this chapter.

We begin by analyzing the responsiveness of consumers to price changes. For example, consumers tend not to buy much more or much less food in response to changes in the general price level of food. This is because food is a pretty basic item for our existence. In contrast, if the price of textbooks becomes higher, students may decide to search for a second-hand copy, or make do with lecture notes from their friends or downloads from the course web site. In the latter case students have ready alternatives to the new text book, and so their expenditure patterns can be expected to reflect these options, whereas it is hard to find alternatives to food. In the case of food consumers are not very responsive to price changes; in the case of textbooks they are. The word `elasticity' that appears in this chapter title is just another term for this concept of responsiveness. Elasticity has many different uses and interpretations, and indeed more than one way of being measured in any given situation. Let us start by developing a suitable numerical measure.

The slope of the demand curve suggests itself as one measure of responsiveness: If we lowered the price of a good by \$1, for example, how many more units would we sell? The difficulty with this measure is that it does not serve us well when comparing different products. One dollar may be a substantial part of the price of your morning coffee and croissant, but not very important if buying a computer or tablet. Accordingly, when goods and services are measured in different units (croissants versus tablets), or when their prices are very different, it is often best to use a \textit{percentage} change measure, which is \textit{unit-free}. 

The \terminology{price elasticity of demand} is measured as the percentage change in quantity demanded, divided by the percentage change in price. Although we introduce several other elasticity measures later, when economists speak of the demand elasticity they invariably mean the price elasticity of demand defined in this way.

\begin{DefBox}
The \textbf{price elasticity of demand} is measured as the percentage change in quantity demanded, divided by the percentage change in price.
\end{DefBox}

The price elasticity of demand can be written in different forms. We will use the Greek letter epsilon, $\varepsilon$, as a shorthand symbol, with a subscript $d$ to denote demand, and the capital delta, $\Delta$, to denote a change. Therefore, we can write


\begin{equation*}
\text{Price elasticity of demand}=\frac{\text{Percentage change in quantity demanded}}{\text{Percentage change in price}}
\end{equation*}
\begin{subequations} \label{eq:priceelastdemand}
\begin{align}
\varepsilon_d	&=\frac{\%\Delta Q}{\%\Delta P}	\\
						&=\frac{\Delta Q/Q}{\Delta P/P}	\\
						&=\frac{\Delta Q}{\Delta P}\times\frac{P}{Q}
\end{align}
\end{subequations}

Calculating the value of the elasticity is not difficult. If we are told that a 10 percent price increase reduces the quantity demanded by 20 percent, then the elasticity value is

\begin{equation*}
\varepsilon_d=\frac{\%\Delta Q}{\%\Delta P}=\frac{-20\%}{10\%}=-2
\end{equation*}

The negative sign denotes that price and quantity move in opposite directions, but for brevity the negative sign is often omitted.

Consider now the data in Table~\ref{table:gaselastrev} and the accompanying Figure~\ref{fig:elasticitylineardemand}. This data reflect the demand equation for natural gas that we introduced in Chapter~\ref{chap:classical}: $P=10-Q$. Note first that, when the price and quantity change, we must decide what \textit{reference price and quantity} to use in the percentage change calculation in Equation~\ref{eq:priceelastdemand}. We could use the initial or final price-quantity combination, or an average of the two. Each choice will yield a slightly different numerical value for the elasticity. The best convention is to \textit{use the midpoint of the price values and the corresponding midpoint of the quantity values}. This ensures that the elasticity value is the same regardless of whether we start at the higher price or the lower price. Using the subscript 1 to denote the initial value and 2 the final value:

\begin{equation*}
\text{Average quantity}=(Q_1+Q_2)/2
\end{equation*}
\begin{equation*}
\text{Average price}=(P_1+P_2)/2
\end{equation*}

% Table 4.1
\begin{table}[H]
\begin{center}
\begin{tabu} to \linewidth {|X[1,c]X[1,c]X[1,c]X[1,c]X[1,c]|} \hline 
\rowcolor{rowcolour}\textbf{Price (\$)} & \textbf{Quantity} & \textbf{Price elasticity} & \textbf{Price elasticity} & \textbf{Total} \\[-0.5em]
\rowcolor{rowcolour}	&	\textbf{demanded}	&	\textbf{(arc)}	&	\textbf{(point)}	&	\textbf{revenue (\$)}	\\[-0.5em]
\rowcolor{rowcolour}	&	\textbf{(thousands}	&	&	&	\\[-0.5em]
\rowcolor{rowcolour}	&	\textbf{of cu ft.)}	&	&	&	\\
						10.00	&	0 	& -9.0 	& $-\infty$ &  		\\ 
\rowcolor{rowcolour}	8.00	&	2 	& -2.33 & -4 		& 16 	\\ 
						6.00	&	4 	& -1.22 & -1.5 		& 24 	\\
\rowcolor{rowcolour}	5.00	&	5 	& -0.82 & -1 		& 25 	\\
						4.00	&	6 	& -0.43 & -0.67 	& 24 	\\ 
\rowcolor{rowcolour}	2.00	&	8 	& -0.11 & -0.25 	& 16 	\\
						0.00	&	10 	& 		& 0 		& 0 	\\ \hline 
\end{tabu}
\end{center}
\caption{The demand for natural gas: elasticities and revenue \label{table:gaselastrev}}
\end{table}

% Figure 4.1
\begin{FigureBox}{0.4}{0.3}{25em}{Elasticity variation with linear demand \label{fig:elasticitylineardemand}}{In the high-price region of the demand curve the elasticity takes on a high value. At the mid-point of a linear demand curve the elasticity takes on a value of one, and at lower prices the elasticity value continues to fall.}
% demand line
\draw [demandcolour,ultra thick,name path=demand] (0,10) node [black,mynode,left] {$P_0=10$} -- node [mynode,above right,pos=0.05,black] {$\varepsilon=-9$} node [mynode,above right, pos=0.2,black] {High elasticity range (elastic)} node [mynode,above right,pos=0.8,black] {Low elasticity range (inelastic)} node [mynode,above right,pos=0.95,black] {$\varepsilon=-0.11$} (10,0) node [black,mynode,below] {$Q_0$=10};
% axes
\draw [thick, -] (0,15) node (yaxis) [above] {Price} |- (15,0) node (xaxis) [right] {Quantity};
% paths to intersect with demand and create dotted lines
\path [name path=price8] (0,8) -- (10,8);
\path [name path=price5] (0,5) -- (10,5);
\path [name path=price2] (0,2) -- (10,2);
% intersection of paths and demand
\draw [name intersections={of=price8 and demand, by=HighE},name intersections={of=price5 and demand, by=MidE},name intersections={of=price2 and demand, by=LowE}]
	[dotted,thick] (yaxis |- HighE) node [mynode,left] {8} -| (xaxis -| HighE)
	[dotted,thick] (yaxis |- MidE) node [mynode,left] {5} -| (xaxis -| MidE) node [mynode,below] {5}
	[dotted,thick] (yaxis |- LowE) node [mynode,left] {2} -| (xaxis -| LowE);
% arrow pointing to MidE
\draw [<-,thick,shorten <=1mm] (MidE) -- +(2,2) node [mynode,right] {Mid point of $D$: $\varepsilon=-1$};
\end{FigureBox}


Using this rule, consider now the value of $\varepsilon_d$ when price drops from \$10.00 to \$8.00. The change in price is \$2.00 and the average price is therefore \$9.00 [= (\$10.00 + \$8.00)/2]. On the quantity side, demand goes from zero to 2 units (measured in thousands of cubic feet), and the average quantity demanded is therefore (0 + 2)/2 = 1. Putting these numbers into the formula yields:

\begin{equation*}
\varepsilon_d=\frac{\%\Delta Q}{\%\Delta P}=\frac{-(2/1)}{(2/9)}=-9
\end{equation*}

Note that the price has declined in this instance and thus $\Delta P$ is negative. Continuing down the table in this fashion yields the full set of elasticity values in the third column. 

The demand elasticity is said to be \textit{high} if it is a large negative number; the large number denotes a high degree of sensitivity. Conversely, the elasticity is \textit{low} if it is a small negative number. High and low refer to the size of the number, ignoring the negative sign. The term \terminology{arc elasticity} is also used to define what we have just measured, indicating that it defines consumer responsiveness over a segment or \textit{arc} of the demand curve.

\begin{DefBox}
The \textbf{arc elasticity of demand} defines consumer responsiveness over a segment or \textit{arc} of the demand curve.
\end{DefBox}

It is helpful to analyze this numerical example by means of the corresponding demand curve that is plotted in Figure~\ref{fig:elasticitylineardemand}. It is a straight-line demand curve; but, despite this, the elasticity is not constant. At high prices the elasticity is high; at low prices it is low. The intuition behind this pattern is as follows: When the price is high, a given price change represents a small \textit{percentage} change, whereas the resulting percentage quantity change will be large. The large percentage quantity change results from the fact that, at the high price, the quantity consumed is small, and, therefore, a small number goes into the denominator of the percentage quantity change. In contrast, when we move to a lower price range on the demand function, a given absolute price change is large in percentage terms, and the resulting quantity change is smaller in percentage terms. 

\subsection*{Extreme cases}

The elasticity decreases in going from high prices to low prices. This is true for most non-linear demand curves also. Two exceptions are when the demand curve is horizontal and when it is vertical. 

When the demand curve is vertical, no quantity change results from a change in price from $P_1$ to $P_2$, as illustrated in Figure~\ref{fig:limitingcasepriceelasticity}. Therefore, the numerator in Equation~\ref{eq:priceelastdemand} is zero, and the elasticity has a zero value.

% Figure 4.2
\begin{FigureBox}{0.4}{0.3}{25em}{Limiting cases of price elasticity \label{fig:limitingcasepriceelasticity}}{When the demand curve is vertical ($D_v$), the elasticity is zero: a change in price from $P_1$ to $P_2$ has no impact on the quantity demanded because the numerator in the elasticity formula has a zero value. When $D$ becomes more horizontal the elasticity becomes larger and larger at $Q_0$, eventually becoming infinite.}
% demand lines
\draw [demandcolour,ultra thick,name path=demandv] (7,0) node [black,mynode,below] {$Q_0$} -- (7,14) node [black,mynode,above] {$D_v$};
\draw [demandcolour,ultra thick,name path=demandh] (0,5) node [black,mynode,left] {$P_1$} -- (14,5) node [black,mynode,right] {$D_h$};
\draw [demandcolour,ultra thick,dashed,name path=demandprime] (0,6) -- (14,4) node [black,mynode,right] {$D'$};
% axes
\draw [thick, -] (0,15) node (yaxis) [above] {Price} |- (15,0) node (xaxis) [right] {Quantity};
% path for P_2 and intersection with D_v for dotted line
\path [name path=p2line] (0,10) -- (15,10);
\draw [name intersections={of=p2line and demandv, by=Q0}]
	[dotted,thick] (yaxis |- Q0) node [mynode,left] {$P_2$} -- (Q0);
% path to draw arrows to D_h and D'
\path [name path=Eline] (11,0) -- (11,15);
% arrows from Infinite elasticity and Large elasticity to D_h and D', respectively
\draw [name intersections={of=Eline and demandh, by=arrowh},name intersections={of=Eline and demandprime, by=arrowprime}]
	[<-,thick,shorten <=1mm,shorten >=-1mm] (arrowh) -- +(0,1) node [mynode,above] {Infinite\\elasticity};
\draw [<-,thick,shorten <=1mm,shorten >=-1mm] (arrowprime) -- +(0,-1) node [mynode,below] {Large\\elasticity};
% arrow from Zero elasticity to D_v line
\draw [<-,thick,shorten <=1mm] ([yshift=2cm]Q0) -- +(3,0) node [mynode,right] {Zero\\elasticity};
\end{FigureBox}

In the horizontal case, we say that the elasticity is \textit{infinite}, which means that any percentage price change brings forth an infinite quantity change! This case is also illustrated in Figure~\ref{fig:limitingcasepriceelasticity} using the demand curve $D_h$. As with the vertical demand curve, this is not immediately obvious. So consider a demand curve that is almost horizontal, such as $D'$ instead of $D_h$. In this instance, we can achieve large changes in quantity demanded by implementing very small price changes. In terms of Equation~\ref{eq:priceelastdemand}, the numerator is large and the denominator small, giving rise to a large elasticity. Now imagine that this demand curve becomes ever more elastic (horizontal). The same quantity response can be obtained with a smaller price change, and hence the elasticity is larger. Pursuing this idea, we can say that, \textit{as the demand curve becomes ever more elastic, the elasticity value tends towards infinity.}

\subsection*{Using information on the slope of the demand curve}

The elasticity formula, Equation~\ref{eq:priceelastdemand} part (c), indicates that we could also compute the elasticity values using information on the slope of the demand curve, $\Delta Q/\Delta P$, multiplied by the appropriate price-quantity ratio. (Note that, even though we put price on the vertical axis, the slope of the demand curve is $\Delta Q/\Delta P$, as explained in Chapter~\ref{chap:classical}; $\Delta P/\Delta Q$ is the inverse of this slope, or the slope of the inverse demand function.) Consider the price change from \$10.00 to \$8.00 again. Columns 1 and 2 indicate that $\Delta Q/\Delta P$ = -2/\$2.00, or by simply looking at the equation for the demand curve we can see that its slope is -1.  Choosing again the midpoint values for price and quantity yields $P/Q$ = \$9.00/1. Therefore the elasticity is 

\begin{equation*}
\varepsilon_d=(\Delta Q/\Delta P)\times(P/Q)=-1\times (9.00/1.00)=-9
\end{equation*}

Knowing the slope of the demand curve can be very useful in establishing elasticity values when the demand curve is not linear, or when price changes are miniscule, or when the curve intersects the axes. Let us consider each of these cases in turn.

A \textit{non-linear demand curve} is illustrated in Figure~\ref{fig:nonlineardemand}. If price increases from $P_0$ to $P_1$, then over that range we can approximate the slope by the ratio $(P_1-P_0)/(Q_1-Q_0)$. This is, essentially, an average slope over the range in question that can be used in the formula, in conjunction with an average price and quantity of these values. 

% Figure 4.3
\begin{FigureBox}{0.4}{0.3}{25em}{Non-linear demand curves \label{fig:nonlineardemand}}{When the demand curve is non-linear the slope changes with the price. Hence, equal price changes do not lead to equal quantity changes: The quantity change associated with a change in price from $P_0$ to $P_1$ is smaller than the change in quantity associated with the same change in price from $P_0$ to $P_2$.}
% demand curve
\draw [demandcolour,ultra thick,domain=1:15,name path=demand] plot (\x, {14/\x});
% axes
\draw [thick] (0,15) node (yaxis) [above] {Price} |- (15,0) node (xaxis) [right] {Quantity};
% paths to intersect with demand curve
\path [name path=p2line] (0,2) -- (15,2);
\path [name path=p0line] (0,4.5) -- (15,4.5);
\path [name path=p1line] (0,7) -- (15,7);
% intersection of paths with demand curve
\draw [name intersections={of=p2line and demand, by=C},name intersections={of=p0line and demand, by=A},name intersections={of=p1line and demand, by=B}]
	[dotted,thick] (yaxis |- C) node [mynode,left] {$P_2$} -- (C) node [mynode,above right] {C} -- (xaxis -| C) node [mynode,below] {$Q_2$}
	[dotted,thick] (yaxis |- A) node [mynode,left] {$P_0$} -- (A) node [mynode,above right] {A} -- (xaxis -| A) node [mynode,below] {$Q_0$}
	[dotted,thick] (yaxis |- B) node [mynode,left] {$P_1$} -- (B) node [mynode,above right] {B} -- (xaxis -| B) node [mynode,below] {$Q_1$};
\end{FigureBox}

When a price change is \textit{infinitesimally small} the resulting estimate is called a \terminology{point elasticity of demand}. This differs slightly from the elasticity in column 3 of Table~\ref{table:gaselastrev}. In that case, we computed the elasticity along different segments or \textit{arcs} of the demand function. In Table~\ref{table:gaselastrev}, the point elasticity \textit{at the point P = \$8.00} is 

\begin{equation*}
\varepsilon_d=(\Delta Q/\Delta P)\times(P/Q)=-1\times (8.00/2.00)=-4
\end{equation*}

The first term in this expression states that quantity changes by 1 unit for each \$1 change in price, and the second term states that the elasticity is being evaluated \textit{at the price-quantity combination $P=\$8$ and $Q=2$}.  The value of the point elasticity at each price value listed in Table~\ref{table:gaselastrev} is given in column 4. The arc elasticity values in column 3 span a price range, whereas the point elasticities correspond exactly to each price value. 

\begin{DefBox}
The \textbf{point elasticity of demand} is the elasticity computed at a particular point on the demand curve.
\end{DefBox}

This point elasticity formula can also be applied to the non-linear demand curve in Figure~\ref{fig:nonlineardemand}. If we wished to compute this elasticity exactly at $P_2$, we could draw a tangent to the function at C and evaluate its slope. This slope could then be used in conjunction with the price-quantity combination $(P_2, Q_2)$ to evaluate $\varepsilon_d$ at that point. 

Next, note that when a demand \textit{curve intersects the horizontal axis} the elasticity value is zero, \textit{regardless of the slope}. Using Figure~\ref{fig:elasticitylineardemand}, we can see that this is because the price in the $P/Q$ component of the elasticity formula equals zero at the intersection point $Q_0$. Hence $P/Q=0$ and the elasticity is therefore zero. Likewise, when approaching an \textit{intersection with the vertical axis}, defined by the point $P_0$ in Figure~\ref{fig:elasticitylineardemand}, the denominator in the $P/Q$ component becomes very small, making the $P/Q$ ratio very large. As we get ever closer to the vertical axis, this ratio becomes correspondingly larger, and therefore we say that \textit{the elasticity approaches infinity}.

\subsection*{Elastic and inelastic demands}

While the elasticity value falls as we move down the demand curve, an important dividing line occurs at the value of -1. This is illustrated in Table~\ref{table:gaselastrev}, and is a property of all straight-line demand curves. Disregarding the negative sign, demand is said to be \terminology{elastic} if the price elasticity is greater than unity, and \terminology{inelastic} if the value lies between unity and 0. It is \terminology{unit elastic} if the value is exactly one.

\begin{DefBox}
Demand is \textbf{elastic} if the price elasticity is greater than unity. It is \textbf{inelastic} if the value lies between unity and 0. It is \textbf{unit elastic} if the value is exactly one.
\end{DefBox}

Economists frequently talk of goods as having a ``high'' or ``low'' demand elasticity. What does this mean, given that the elasticity varies throughout the length of a demand curve? It signifies that, \textit{at the price usually charged}, the elasticity has a high or low value. For example, your weekly demand for coffee at Starbucks might be unresponsive to variations in price around the value of \$2.00, but if the price were \$4, you might be more responsive to price variations. Likewise, when we stated at the beginning of this chapter that the demand for food tends to be inelastic, we really mean that \textit{at the price we customarily face for food}, demand is inelastic.

\subsection*{Determinants of price elasticity}

Why is it that the price elasticities for some goods and services are high and for others low? One answer lies in \textit{tastes}: If a good or service is a basic necessity in one's life, then price variations have minimal effect and these products have a relatively inelastic demand. 

A second answer lies in the \textit{ease with which we can substitute} alternative goods or services for the product in question. The local music school may find that the demand for its instruction is responsive to the price charged for lessons if there are many independent music teachers who can be hired directly by the parents of aspiring musicians. If Apple had no serious competition, it could price the products higher than in the presence of Samsung, Google etc. The ease with which we can substitute other goods or services is a key determinant. It follows that a critical role for the marketing department in a firm is to convince buyers of the uniqueness of the firm's product.

Where \textit{product groups} are concerned, the price elasticity of demand for one product is necessarily higher than for the group as a whole: Suppose the price of one tablet brand alone falls. Buyers would be expected to substitute towards this product in large numbers -- its manufacturer would find demand to be highly responsive. But if \textit{all} brands are reduced in price, the increase in demand for any one will be more muted. In essence, the one tablet whose price falls has several close substitutes, but tablets in the aggregate do not. 

Finally, there is a \textit{time dimension} to responsiveness, and this is explored in Section~\ref{sec:inflation}.

\subsection*{Using price elasticities}

Knowledge of elasticity values is useful in calculating the price change required to eliminate a shortage or surplus. For example, shifts in the supply of agricultural products can create surpluses and shortages. Because of variations in weather conditions, crop yields cannot be forecast accurately. In addition, on account of the low elasticity of demand for such products, low crop yields can increase prices radically, and bumper harvests can have the opposite impact.

Consider Figure~\ref{fig:elastquantfluctuations}. Econometricians tell us that the demand for foodstuffs is inelastic, so let us operate in the lower (inelastic) part of this demand, $D$. A change in supply conditions (e.g.  a shortage of rain and a poorer harvest) shifts the supply from $S_1$ to $S_2$ with the consequence that the price increases from $P_1$ to $P_2$. In this illustration the price increase is substantial.  In contrast, with a relatively flat, or elastic, demand, $D'$, through the initial point A, the shift in the supply curve has a more moderate impact on the price (from $P_1$ to $P_3$), but a relatively larger impact on quantity traded. 

%Figure 4.4
\begin{FigureBox}{0.4}{0.3}{25em}{The impact of elasticity on quantity fluctuations \label{fig:elastquantfluctuations}}{In the lower part of the demand curve $D$, where demand is inelastic, e.g. point A, a shift in supply from $S_1$ to $S_2$ induces a large percentage increase in price, and a small percentage decrease in quantity demanded. In contrast, for the demand curve $D'$ that goes through the original equilibrium, the region A is now an \emph{elastic} region, and the impact of the supply shift is contrary: the \%$\Delta P$ is smaller and the \%$\Delta Q$ is larger.}
% demand lines
\draw [demandcolour,ultra thick,name path=demand] (4,15) node [black,mynode,left] {$D$} -- (10,0);
\draw [demandcolour,ultra thick,name path=demandprime] (0,7) -- (15,3.25) node [black,mynode,above] {$D'$};
% supply lines
\draw [supplycolour,ultra thick,name path=supplytwo] (1,0) -- (8.5,15) node [black,mynode,right] {$S_2$};
\draw [supplycolour,ultra thick,name path=supplyone] (5,0) -- (14,15) node [black,mynode,right] {$S_1$};
\draw [thick] (0,15) node (yaxis) [above] {Price} |- (15,0) node (xaxis) [right] {Quantity};
% intersection of demand and supplyone
\draw [name intersections={of=demand and supplyone, by=P1Q1}]
	[dotted,thick] (yaxis |- P1Q1) node [mynode,left] {$P_1$} -- (P1Q1) node [mynode,above=0.25em and 0em] {A} -- (xaxis -| P1Q1) node [mynode,below] {$Q_1$};
% intersection of demand and supplytwo
\draw [name intersections={of=demand and supplytwo, by=P2Q2}]
	[dotted,thick] (yaxis |- P2Q2) node [mynode,left] {$P_2$} -| (xaxis -| P2Q2) node [mynode,below] {$Q_2$};
% intersection of demandprime and supplytwo
\draw [name intersections={of=demandprime and supplytwo, by=P3Q3}]
	[dotted,thick] (yaxis |- P3Q3) node [mynode,left] {$P_3$} -| (xaxis -| P3Q3) node [mynode,below] {$Q_3$};
\end{FigureBox}

