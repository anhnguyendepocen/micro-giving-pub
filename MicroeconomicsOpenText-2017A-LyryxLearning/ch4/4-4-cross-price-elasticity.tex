\section{Cross-price elasticities}\label{sec:ch4sec4}

The price elasticity of demand tells us about consumer responses to price changes in different regions of the demand curve, holding constant all other influences. One of those influences is the price of other goods and services. A \terminology{cross-price elasticity} indicates how demand is influenced by changes in the prices of other products. 

\begin{DefBox}
The \textbf{cross-price elasticity of demand} is the percentage change in the quantity demanded of a product divided by the percentage change in the price of another.
\end{DefBox}

In mathematical form we write the cross price elasticity of the demand for $x$ due to a change in the price of $y$ as

\begin{equation*}
\varepsilon_{d(x,y)}=\frac{\%\Delta Q_x}{\%\Delta P_y}.
\end{equation*}

For example, if the price of cable-supply TV services declines, by how much will the demand for satellite-supply TV services change?  The cross-price elasticity may be positive or negative. When the price of movie theatre tickets rises, the demand for \textit{Home Box Office} movies rises, and vice versa. In this example, we are measuring the cross-price elasticity of demand for HBO movies with respect to the price of theatre tickets. These goods are clearly \textit{substitutable}, and this is reflected in a \textit{positive} value of this cross-price elasticity: The percentage change in video rentals is positive in response to the increase in movie theatre prices. The numerator and denominator in the equation above have the same sign.

Suppose that, in addition to going to fewer movies, we also eat less frequently in the restaurant beside the movie theatre. In this case, the cross-price elasticity relating the demand for restaurant meals to the price of movies is \textit{negative}---an increase in movie prices reduces the demand for meals. The numerator and denominator in the cross-price elasticity equation are opposite in sign. In this instance, the goods are \textit{complements}.