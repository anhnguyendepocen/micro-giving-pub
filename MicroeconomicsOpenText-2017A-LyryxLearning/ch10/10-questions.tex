\newpage
\section*{Exercises for Chapter~\ref{chap:monopoly}}

\begin{enumialphparenastyle}

% Solutions file for exercises opened
\Opensolutionfile{solutions}[solutions/ch10ex]

\begin{ex}\label{ex:ch10ex1}
Consider a monopolist with demand curve defined by $P=100-2Q$. The $MR$ curve is $MR=100-4Q$ and the marginal cost is $MC=10+Q$.
\begin{enumerate}
	\item Develop a diagram that illustrates this market.
	\item	Compute the profit-maximizing price and output combination.
\end{enumerate}
\begin{sol}
	This is a standard diagram for the monopolist. See the figure below. Equating $MC=MR$ yields $Q=18$, $P=\$64$.
	\begin{center}
	\begin{tikzpicture}[background color=figurebkgdcolour,use background,xscale=0.15,yscale=0.06]
		\draw [thick] (0,110) node (yaxis) [mynode1,above] {Price} |- (60,0) node (xaxis) [mynode1,right] {Quantity};
		\draw [ultra thick,budgetcolour,name path=G] (0,100) node [mynode,left,black] {100} -- node [mynode,above right,black,pos=0.85] {Demand: $P=100-2Q$} (50,0) node [mynode,below,black] {50};
		\draw [ultra thick,dashed,budgetcolour,name path=halfG] (0,100) -- node [mynode,above right,black,pos=0.85] {$MR=100-4Q$} (25,0) node [mynode,below,black] {25};
		\draw [ultra thick,supplycolour,name path=quota] (0,10) node [mynode,left,black] {10} -- (50,60) node [mynode,above,black] {$MC=10+Q$};
	\end{tikzpicture}
	\end{center}
\end{sol}
\end{ex}

\begin{ex}\label{ex:ch10ex2}
Imagine we have a monopolist who wants to maximize revenue rather than profit. She has the demand curve $P=72-Q$, with marginal revenue $MR=72-2Q$, and $MC=12$.
\begin{enumerate}
	\item	Graph the three functions.
	\item	Calculate the price she should charge in order to maximize revenue. [Hint: where the $MR=0$.]
	\item	Compare her total revenue with the revenue obtained under profit maximization.
	\item	How much profit will she make when maximizing total revenue?
\end{enumerate}
\begin{sol}
\begin{enumerate}
	\item	See below.
	\item	Total revenue is a maximum where $MR$ becomes zero. This is at $P=\$36$ and $Q=36$.
	\item	$TR$ under revenue maximization is $36\times 36=\$1,296$. Under profit maximization the optimal output is where $MC=MR$ -- an output $Q=30$. Using the demand curve, this output of 30 units will be sold at a price of \$42. Hence $TR=30\times 42=\$1,260$.
	\item	Profit here is \$864 since total cost is \$12 per each of the 36 units (\$432) and revenue is \$1,296.
\end{enumerate}
\begin{center}
\begin{tikzpicture}[background color=figurebkgdcolour,use background,xscale=0.12,yscale=0.08]
	\draw [thick] (0,75) node (yaxis) [mynode1,above] {Price} |- (75,0) node (xaxis) [mynode1,right] {Quantity};
	\draw [ultra thick,budgetcolour,name path=G] (0,72) node [mynode,left,black] {72} -- node [mynode,above right,black,pos=0.7] {Demand: $P=72-Q$} (72,0) node [mynode,below,black] {72};
	\draw [ultra thick,dashed,budgetcolour,name path=halfG] (0,72) -- node [mynode,above right,black,pos=0.7] {$MR=72-2Q$} (36,0) node [mynode,below,black] {36};
	\draw [ultra thick,supplycolour,name path=quota] (0,12) node [mynode,left,black] {12} -- +(70,0) node [mynode,right,black] {$MC=12$};
\end{tikzpicture}
\end{center}
\end{sol}
\end{ex}

\begin{ex}\label{ex:ch10ex3}
In this question you will see why we never consider a supply curve for a monopolist -- in the way that is done in perfect competition. So suppose the monopolist faces a demand curve $P=72-Q$ and a $MR$ curve $MR=72-2Q$. The Marginal cost is $MC=Q$.
\begin{enumerate}
	\item	Find the profit maximizing output and price. 
	\item	Now suppose that the demand curve shifts to become $P=60-(1/3)Q$ and thus $MR=60-(2/3)Q$. The $MC$ remains the same. Establish that the profit maximizing price is the same for this demand curve as the one in part (a). You have now shown that there is not a unique relationship between cost and demand for the monopolist because the same price in this example is consistent with profit maximization with two different demand curves.
\end{enumerate}
\begin{sol}
\begin{enumerate}
	\item	Equating $MC=MR$ yields $72-2Q=Q$. Therefore $Q=24$. Substituting this into the demand curve yields a price of $P=\$48$.
	\item	Equating $MC=MR$ yields $60-(2/3)Q=Q$. Solving yields $Q=36$. This quantity will sell at a price derived from the demand curve: $P=60-(1/3)\times(36)$; therefore $P=48$. Hence both demand curves yield an equilibrium price of \$48, but different quantities.
\end{enumerate}
\end{sol}
\end{ex}

\begin{ex}\label{ex:ch10ex4}
Suppose that the monopoly in Exercise~\ref{ex:ch10ex2} has a large number of plants. Consider what could happen if each of these plants became a separate firm, and acted competitively. In this perfectly competitive world you can assume that the $MC$ curve of the monopolist becomes the industry supply curve.
\begin{enumerate}
	\item	What output would be produced in the industry? 
	\item	What price would be charged in the marketplace?
	\item	Compute the gain to the economy in dollar terms as a result of the DWL being eliminated [Hint: it resembles the area ABF in Figure~\ref{fig:cartelindustry}].
\end{enumerate}
\begin{sol}
\begin{enumerate}
	\item	Where demand equals $MC$, we obtain $72-Q=12$. Therefore $Q=60$.
	\item	From the demand curve, if $Q=60$ then $P=\$12$.
	\item	The efficiency gain in going from a profit maximizing monopoly ($Q=30$) to perfect competition ($Q=60$) is given by the area under the demand curve and above the $MC$ curve between these output levels. This is $q/2\times 30\times 30=\$450$.
\end{enumerate}
\end{sol}
\end{ex}

\begin{ex}\label{ex:ch10ex5}
A monopolist faces a demand curve $P=64-2Q$ and $MR=64-4Q$. His marginal cost is $MC=16$.
\begin{enumerate}
	\item	Graph the three functions and compute the profit maximizing output and price.
	\item	Compute the efficient level of output (where $MC$=demand), and compute the DWL associated with producing the profit maximizing output rather than the efficient output.
	\item	Suppose the government gave the monopolist a subsidy of \$4 per unit produced. The $MC$ would be reduced accordingly to \$12 from \$16. Compute the profit maximizing output level and the deadweight loss associated with this new output. Explain intuitively why the DWL has changed.
\end{enumerate}
\begin{sol}
\begin{enumerate}
	\item	Setting $MC=MR$ yields $Q=12$ and from the demand curve, $P=\$40$. See the figure below.
	\item	Where $MC$ equals demand the output is $Q=24$. In moving from the output level $Q=24$ to $Q=12$, the DWL is the area bounded by the demand curve and the $MC$ between these output levels: $1/2\times 12\times 24=\$144$.
	\item	With the subsidy the monopolist's new $MC$ is $MC=12$. Equating the $MC$ to this $MR$ yields: $12=64-4Q$. Therefore the new profit maximizing level is $Q=13$. The new deadweight loss is the area below the demand curve and above the actual $MC$ curve between the outputs $Q=13$ and $Q=24$: $1/2\times (24-13)\times(38-16)=11\times 22=\$121$.
\end{enumerate}
\begin{center}
\begin{tikzpicture}[background color=figurebkgdcolour,use background,xscale=0.2,yscale=0.08]
	\draw [thick] (0,75) node (yaxis) [mynode1,above] {Price} |- (35,0) node (xaxis) [mynode1,right] {Quantity};
	\draw [ultra thick,budgetcolour,name path=G] (0,64) node [mynode,left,black] {64} -- node [mynode,above right,black,pos=0.95] {Demand: $P=64-2Q$} (32,0) node [mynode,below,black] {32};
	\draw [ultra thick,dashed,budgetcolour,name path=halfG] (0,64) -- node [mynode,above right,black,pos=0.95] {$MR=64-4Q$} (16,0) node [mynode,below,black] {16};
	\draw [ultra thick,supplycolour,name path=quota] (0,16) node [mynode,left,black] {16} -- +(35,0) node [mynode,right,black] {$MC=16$};
\end{tikzpicture}
\end{center}
\end{sol}
\end{ex}

\begin{ex}\label{ex:ch10ex6}
In the text example in Table~\ref{table:profitmaxmonopolist}, compute the profit that the monopolist would make if he were able to discriminate between every buyer and charge each buyer their reservation price.
\begin{sol}
	The buyers' reservation prices are given in row $P$. The cost of producing each unit is given in row $MC$. The profit on the first unit is therefore $\$(14-2)=\$12$; on the second unit is $\$(12-3)=\$9$, etc. On the fifth unit the additional profit is zero. Therefore, four units should be produced and sold. Total profit is $\$12+\$9+\$6+\$3=\$30$.
	
\end{sol}
\end{ex}

\begin{ex}\label{ex:ch10ex7}
A monopolist is able to discriminate perfectly among his consumers -- by charging a different price to each one. The market demand curve facing him is given by $P=72-Q$. His marginal cost is given by $MC=24$ and marginal revenue is $MR=72-2Q$.
\begin{enumerate}
	\item	In a diagram, illustrate the profit-maximizing equilibrium, where discrimination is not practiced.
	\item	Calculate the equilibrium output if he discriminates perfectly.
	\item	If he has no fixed cost beyond the marginal production cost of \$24 per unit, calculate his profit in each pricing scenario.
\end{enumerate}
\begin{sol}
\begin{enumerate}
	\item	See figure below. The profit maximizing outcome is $Q=24$ and $P=\$48$ -- obtained from $MC=MR$.
	\item	With perfect price discrimination the monopolist's revenue is the area under the demand curve. He should continue to produce and sell as long as the demand price greater than $MC$. Where the demand price equals $MC$ profit is maximized. This occurs at $P=\$24$, $Q=48$.
	\item	Profit is $TR-TC$ at $Q=24$. This is \$576. In (b) the profit is the area under the demand curve up to the output $Q=48$ minus the area under the $MC$ curve up to this same output. This is \$1152.
\end{enumerate}
\begin{center}
\begin{tikzpicture}[background color=figurebkgdcolour,use background,xscale=0.12,yscale=0.08]
	\draw [thick] (0,75) node (yaxis) [mynode1,above] {Price} |- (75,0) node (xaxis) [mynode1,right] {Quantity};
	\draw [ultra thick,budgetcolour,name path=G] (0,72) node [mynode,left,black] {72} -- node [mynode,above right,black,pos=0.9] {Demand: $P=72-Q$} (72,0) node [mynode,below,black] {72};
	\draw [ultra thick,dashed,budgetcolour,name path=halfG] (0,72) -- node [mynode,above right,black,pos=0.9] {$MR=72-2Q$} (36,0) node [mynode,below,black] {36};
	\draw [ultra thick,supplycolour,name path=quota] (0,24) node [mynode,left,black] {24} -- +(70,0) node [mynode,right,black] {$MC=24$};
\end{tikzpicture}
\end{center}
\end{sol}
\end{ex}

\begin{ex}\label{ex:ch10ex8}
A monopolist faces two distinct markets A and B for her product, and she is able to insure that resale is not possible. The demand curves in these markets are given by $P=20-(1/4)Q_A$ and $P=14-(1/4)Q_B$. The marginal cost is constant: $MC=4$. There are no fixed costs.
\begin{enumerate}
	\item	Calculate the profit maximizing price and quantity in each market.
	\item	Compute the total profit made as a result of this discriminatory pricing.
\end{enumerate}
\begin{sol}
\begin{enumerate}
	\item	Profit maximizing output is where $MC=MR$ in each market. The $MR$s are $MR_A=20-(1/2)Q_A$, $MR_B=14-(1/2)Q_B$. Equating each of these in turn to $MC=4$ yields $Q_A=32$ and $Q_B=20$. These outputs can be sold at a price obtained from their demand curves: $P_A=\$12$ and $P_B=\$9$.
	\item	Total profit is the sum of profit in each market: $(P_A\times Q_A-TC_A)+(P_B\times Q_B-TC_B)=\$256+\$100=\$356$.
\end{enumerate}
\end{sol}
\end{ex}

\begin{ex}\label{ex:ch10ex9}
A monopolist with a demand curve given by $P=240-2Q$ has a cost structure made up of a fixed cost of \$500 and a marginal production cost of $MC=40$.
\begin{enumerate}
	\item	When maximizing profit, how much profit will she make?
	\item	Suppose now that she can outsource some of her assembly work to a plant in Indonesia, and this reduces her marginal production cost to \$32 per unit, but also increases her fixed cost to \$750. Should she outsource?
\end{enumerate}
\begin{sol}
\begin{enumerate}
	\item	Profit maximizing output is where $MC=MR$: $Q=50$. Therefore, from the demand curve, $P=\$140$. $TR$ is thus \$7,000 and $TC=500+50\times 40=\$2,500$. Profit is thus \$4,500.
	\item	Here the profit maximizing outcome is obtained by setting the $MR$ curve equal to the new $MC$ curve: $240-4Q=32$. This yields $Q=52$, $P=\$136$. Profit is obtained as before -- total revenue minus the sum of the variable cost plus (higher) fixed cost. Total revenue is $52\times 136=\$7,072$ and total cost is $750+52\times 32=\$2,414$. Profit is therefore \$4,658. Yes, she should outsource.
\end{enumerate}
\begin{center}
\begin{tikzpicture}[background color=figurebkgdcolour,use background,xscale=0.12,yscale=0.08]
	\draw [thick] (0,75) node (yaxis) [mynode1,above] {Price} |- (75,0) node (xaxis) [mynode1,right] {Quantity};
	\draw [ultra thick,budgetcolour,name path=G] (0,72) node [mynode,left,black] {240} -- node [mynode,above right,black,pos=0.7] {Demand: $P=240-2Q$} (72,0) node [mynode,below,black] {120};
	\draw [ultra thick,dashed,budgetcolour,name path=halfG] (0,72) -- node [mynode,above right,black,pos=0.7] {$MR=240-4Q$} (36,0) node [mynode,below,black] {60};
	\draw [ultra thick,supplycolour,name path=quota] (0,12) node [mynode,left,black] {40} -- +(70,0) node [mynode,right,black] {$MC=40$};
\end{tikzpicture}
\end{center}
\end{sol}
\end{ex}

\begin{ex}\label{ex:ch10ex10}
In Exercise~\ref{ex:ch10ex9}, prior to being able to outsource, imagine that the supplier is concerned about entry, and must spend \$2,000 in lobbying to maintain her position as a monopolist.
\begin{enumerate}
	\item	Can the firm still make a profit?
	\item	What is the maximum amount the firm could afford to spend on lobbying with the objective of maintaining the monopoly position?
\end{enumerate}
\begin{sol}
\begin{enumerate}
	\item	Yes, this lobbying is a fixed cost and her profits are more than enough to cover it.
	\item	The total of her profits -- normal profits are included in the cost structure.
\end{enumerate}
\end{sol}
\end{ex}

\begin{ex}\label{ex:ch10ex11}
A concert organizer is preparing for the arrival of the Grateful Living band in his small town. He knows he has two types of concert goers: one group of 40 people, each willing to spend \$60 on the concert, and another group of 70 people, each willing to spend \$40. His total costs are purely fixed at \$3,500.
\begin{enumerate}
	\item	Draw the market demand curve faced by this monopolist.
	\item	Draw the $MR$ and $MC$ curves.
	\item	With two-price discrimination what will be the monopolist's profit?
	\item	If he must charge a single price for all tickets can he make a profit?
\end{enumerate}
\begin{sol}
\begin{enumerate}
	\item	The diagram here is equivalent to the one in Figure~\ref{fig:pricedismovie} in the text. The first segment, up to an output of 40 units, has a price of \$60; the second, from an output of 40 to 110, has a price of \$40.
	\item	The $MC$ curve runs along the horizontal axis -- after the fixed cost is incurred, the $MC$ is zero. The demand curve is the $MR$ curve here, composed of the two horizontal segments.
	\item	A price of \$60 can be charged to 40 buyers, and a price of \$40 charged to 70 buyers. Hence $TR=\$5,200$. Since $TC=\$3,500$, profit is \$1,700.
	\item	Yes; 110 buyers at \$40 each yields a $TR=\$4,400$. Subtract the $TC$ to yield a profit of \$900.
\end{enumerate}
\begin{center}
\begin{tikzpicture}[background color=figurebkgdcolour,use background]
	\begin{axis}[
	axis line style=thick,
	every tick label/.append style={font=\footnotesize},
	ymajorgrids,
	grid style={dotted},
	every node near coord/.append style={font=\scriptsize},
	xticklabel style={rotate=90,anchor=east,/pgf/number format/1000 sep=},
	scaled y ticks=false,
	yticklabel style={/pgf/number format/fixed,/pgf/number format/1000 sep = \thinspace},
	xmin=0,xmax=120,ymin=0,ymax=70,
	y=1cm/10,
	x=1cm/15,
	]
	\addplot[datasetcolourtwo,ultra thick] table {
		X	Y
		0	60
		40	60
		40	40
		110	40
	};
	\end{axis}
\end{tikzpicture}
\end{center}
\end{sol}
\end{ex}

% Closes solutions file for this chapter
\Closesolutionfile{solutions}

\end{enumialphparenastyle}