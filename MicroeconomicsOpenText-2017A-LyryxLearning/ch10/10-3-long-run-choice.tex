\section{Long run choices}\label{sec:ch10sec3}

Consider next the impact of a shift in demand upon the profit maximizing choice of this firm. A rightward shift in demand in Figure~\ref{fig:monopolyeq} also yields a new $MR$ curve. The firm therefore chooses a new level of output, using the same profit maximizing rule: set $MC=MR$. This output will be greater than the previous output, but again the price must be on an elastic portion of the new demand curve. If operating with the same plant size, the $MC$ and $ATC$ curves do not change and the new profit per unit is again read from the $ATC$ curve. The new equilibrium is again defined by where $MR=MC$.

By this stage the curious student will have asked: ``what happens to plant size in the long run?'' For example, is the monopolist in Figure~\ref{fig:monopolyeq} using the most appropriate plant size in the first place? Even if she is, should the monopolist consider adopting an expanded plant size in response to the shift in demand?

The answer is: in the long run the monopolist is free to choose whatever plant size is best. Her initial plant size might have been optimal for the demand she faced, but if it was, it is unlikely to be optimal for the larger scale of production associated with the demand shift. Accordingly, with the new demand curve, she must consider how much profit she could make using different plant sizes.

To illustrate one possibility, we will think of this firm as having constant returns to scale at all output ranges, as displayed in Figure~\ref{fig:monopolyplantsize}. The key characteristic of constant returns to scale is that a doubling of inputs leads to a doubling of output. Therefore, if the per-unit cost of inputs is fixed, a doubling of inputs (and therefore output) leads exactly to a doubling of costs. This implies that, when the firm varies its plant size \textit{and} its labour use, the cost of producing each additional unit must be constant. The long run marginal cost $LMC$ is therefore constant and equals the $ATC$ in the long run. 

% Figure 10.7 {called 10.5 in original text)
\begin{FigureBox}{0.3}{0.25}{25em}{The monopolist's choice of plant size \label{fig:monopolyplantsize}}{With constant returns to scale and constant prices per unit of labour and capital, a doubling of output involves exactly a doubling of costs. Thus, per unit costs, or average costs, are constant in the LR. Hence $LAC=LMC$, and each is constant.}
\draw [dashed,mccolour,ultra thick]
	(4,6) to [out=15,in=270] (9,14) node [black,mynode,above] {$MC_1$}
	(15,6) to [out=15,in=270] (20,14) node [black,mynode,above] {$MC_2$};
% AC curves
\draw [atccolour,ultra thick,domain=180:360] plot ({7+4*cos(\x)},{12+4*sin(\x)}) node [mynode,above,black] {$AC_1$};
\draw [atccolour,ultra thick,domain=180:360] plot ({18+4*cos(\x)},{12+4*sin(\x)}) node [mynode,above,black] {$AC_2$};
\draw [latccolour,ultra thick,-] (0,8) -- (25,8) node [black,mynode,below] {$LAC=LMC$};
\draw [thick, -] (0,20) node [above] {\$} |- (25,0) node [right] {Quantity};
\end{FigureBox}


Figure~\ref{fig:plantsizeLR} describes the market for this good. The optimal output and price are determined in the usual manner: set $MC=MR$. If the monopolist has plant size corresponding to $ATC_1$, the optimal output is $Q_1$ and should be sold at the price $P_1$.The key issue now is: given the demand conditions, could the monopolist make more profit by choosing a plant size that differs from the one corresponding to $ATC_1$? 

% Figure 10.8 (called 10.6 in original text)
\begin{FigureBox}{0.3}{0.25}{25em}{Plant size in the long run \label{fig:plantsizeLR}}{With demand conditions defined by $D$ and $MR$, the optimal plant size is one corresponding to the point where $MR=MC$ in the long run. Therefore $Q_2$ is the optimal output and the optimal plant size corresponds to $ATC_2$. If the current plant is defined by $ATC_1$, then optimal SR production is $Q_1$.}
% MC curves
\draw [dashed,mccolour,ultra thick,name path=MC1] (9,6) node [black,mynode,below] {$MC_1$} to [out=15,in=270] (14,14);
\draw [dashed,mccolour,ultra thick,name path=MC2] (15,6) node [black,mynode,below] {$MC_2$} to [out=15,in=270] (20,14);
% ATC curves
\draw [atccolour,ultra thick,domain=360:180,name path=ATC1] plot ({12+4*cos(\x)},{12+4*sin(\x)}) node [mynode,left,black] {$ATC_1$};
\draw [atccolour,ultra thick,domain=225:360,name path=ATC2] plot ({18+4*cos(\x)},{12+4*sin(\x)}) node [mynode,right,black] {$ATC_2$};
% LAC
\draw [latccolour,ultra thick,name path=LAC] (0,8) node [black,mynode,below right] {$LAC=LMC$} -- (25,8);
% Demand line
\draw [demandcolour,ultra thick,domain=0:23,name path=D] (0,20) -- (23,8.5) node [mynode,above right,black,pos=0.2] {$D$};
% MR line
\draw [dashed,mrcolour,ultra thick,domain=0:23,name path=MR] (0,20) -- (23,4.6667) node [mynode,below left,black,pos=0.2] {$MR$};
% axes
\draw [thick, -] (0,25) node (yaxis) [above] {\$} |- (25,0) node (xaxis) [right] {Quantity};
% intersection of MR and MC1
\draw [name intersections={of=MR and MC1, by=Q1}]
	[dotted,thick] (Q1) -- (xaxis -| Q1) node [mynode,below] {$Q_1$};
% intersection of MR and MC2
\draw [name intersections={of=MR and MC2, by=Q2}]
	[dotted,thick] (Q2) -- (xaxis -| Q2) node [mynode,below] {$Q_2$};
% paths to connect Q1 and Q2 to Demand line
\path [name path=Q1line] (xaxis -| Q1) -- +(0,25);
\path [name path=Q2line] (xaxis -| Q2) -- +(0,25);
% intersection of Demand line with Q1line and Q2line
\draw [name intersections={of=D and Q1line, by=P1},name intersections={of=D and Q2line, by=P2}]
	[dotted,thick] (yaxis |- P1) node [mynode,left] {$P_1$} -| (Q1)
	[dotted,thick] (yaxis |- P2) node [mynode,left] {$P_2$} -| (Q2);
\end{FigureBox}

In this instance the answer is a clear `yes'. Her $LMC$ curve is horizontal and so, by increasing output from $Q_1$ to $Q_2$ she earns a profit on each additional unit in that range, because the $MR$ curve lies above the $LMC$ curve.  In order to produce the output level $Q_2$ at least cost she must choose a plant size corresponding to $AC_2$. 