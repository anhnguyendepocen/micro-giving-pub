\section{Profit maximizing behaviour}\label{sec:ch10sec2}

We established in the previous chapter that, in deciding upon a profit-maximizing output, any firm should produce up to the point where the additional cost equals the additional revenue from a unit of output. What distinguishes the supply decision for a monopolist from the supply decision of the perfect competitor is that the monopolist faces a downward sloping demand. A monopolist is the sole supplier and therefore must meet the full market demand. This means that if more output is produced, the price must fall. What are the pricing and output implications of this? Let us use an example to illustrate.

\subsection*{Marginal revenue}

Table~\ref{table:profitmaxmonopolist} tabulates the price and quantity values for a demand curve in columns 1 and 2. Column three contains the sales revenue generated at each output. It is the product of price and quantity. Since the price denotes the revenue per unit, it is sometimes referred to as \terminology{average revenue}. The total revenue ($TR$) is plotted in Figure~\ref{fig:totalmarginalrevenue}. It reaches a maximum at \$32, where 4 units of output are produced. A greater output necessitates a lower price on every unit sold, and in this case revenue falls if the fifth unit is brought to the market. Even though the fifth unit sells for a positive price, the price on the other 4 units is now lower and the net effect is to reduce revenue. The pattern in Figure~\ref{fig:totalmarginalrevenue} corresponds exactly to what we examined in Chapter~\ref{chap:elasticities}: as price is lowered from the highest possible value of \$14 (where 1 unit is demanded) and the corresponding quantity increases, revenue rises, peaks, and ultimately falls as output increases. In Chapter~\ref{chap:elasticities} we explained that this maximum revenue point occurs where the price elasticity is unity (-1), at the mid-point of a linear demand curve. 

\begin{table}[H]
\begin{center}
\begin{tabu} to \linewidth {|X[1,c]X[1,c]X[1,c]X[1,c]X[1,c]X[1,c]X[1,c]|} \hline 
\rowcolor{rowcolour}	\textbf{Quantity ($Q$)} & \textbf{Price ($P$)} & \textbf{Total Revenue ($TR$)} & \textbf{Marginal Revenue ($MR$)} & \textbf{Marginal Cost ($MC$)} & \textbf{Total Cost ($TC$)} & \textbf{Profit} \\ 
						0 & 16 &  &  &  &  &  \\
\rowcolor{rowcolour}	1 & 14 & 14 & 14 & 2 & 2 & 12 \\ 
						2 & 12 & 24 & 10 & 3 & 5 & 19 \\
\rowcolor{rowcolour}	3 & 10 & 30 & 6 & 4 & 9 & 21 \\ 
						4 & 8 & 32 & 2 & 5 & 14 & 18 \\ 
\rowcolor{rowcolour}	5 & 6 & 30 & -2 & 6 & 20 & 10 \\ 
						6 & 4 & 24 & -6 & 7 & 27 & -3 \\
\rowcolor{rowcolour}	7 & 2 & 14 & -10 & 8 & 35 & -21 \\ \hline 
\end{tabu}
\end{center}
\caption{A profit maximizing monopolist \label{table:profitmaxmonopolist}}
\end{table}

% Figure 10.3 (called 10.2a in original text)
\begin{FigureBox}{0.3}{0.25}{25em}{Total revenue and marginal revenue \label{fig:totalmarginalrevenue}}{When the quantity sold increases total revenue/expenditure initially increases also. At a certain point, further sales require a price that not only increases quantity, but reduces revenue on units already being sold to such a degree that $TR$ declines -- where the demand elasticity equals $-1$ (the mid point of a linear demand curve). Here the mid point occurs at $Q=4$. Where the $TR$ is a maximum the $MR=0$.}
% Total revenue function
\draw [trcolour,ultra thick,domain=180:0,name path=TR] plot ({12.5+12.5*cos(\x)},{12.5*sin(\x)}) node [mynode,black,below] {8};
% axes
\draw [thick, -] (0,20) node (yaxis) [above] {Revenue} |- (25,0) node (xaxis) [right] {Quantity};
% paths to intersect TR curve at R=24 and R=32 (max)
\path [name path=R24] (0,10.8253) -- +(10,0);
\path [name path=R32] (0,12.5) -- +(25,0);
% intersection of TR curve and R24 and R32
\draw [name intersections={of=TR and R24, by=Q2},name intersections={of=TR and R32, by=Q4}]
	[dotted,thick] (yaxis |- Q2) node [mynode,left] {24} -| (xaxis -| Q2) node [mynode,below] {2}
	[dotted,thick] (yaxis |- Q4) node [mynode,left] {32} -| (xaxis -| Q4) node [mynode,below] {4};
% arrow pointing to maximum of TR curve
\draw [<-,thick,shorten <=1mm,shorten >=-1.5mm] (Q4) -- +(0,2) node [mynode,above] {$TR$ is a maximum, $MR=0$};
\end{FigureBox}

Related to the total revenue function is the \textbf{marginal revenue} function. It is the addition to total revenue due to the sale of one more unit of the commodity.  

\begin{DefBox}
\textbf{Marginal revenue} is the change in total revenue due to selling one more unit of the good. 

\textbf{Average revenue} is the price per unit sold.
\end{DefBox}

The $MR$ in this example is defined in the fourth column of Table~\ref{table:profitmaxmonopolist}. When the quantity sold increases from 1 unit to 2 units total revenue increases from \$14 to \$24. Therefore the marginal revenue associated with the second unit of output is \$10. When a third unit is sold $TR$ increases to \$30 and therefore the $MR$ of the third unit is \$6. As output increases the $MR$ declines and eventually becomes negative -- at the point where the $TR$ is a maximum: if $TR$ begins to decline then the additional revenue is by definition negative.

The $MR$ function is plotted in Figure~\ref{fig:monopolistprofitmax}. It becomes negative when output increases from 4 to 5 units.

% Figure 10.4 (called 10.2b in original text)
\begin{FigureBox}{1}{0.3}{25em}{Monopolist's profit maximizing output \label{fig:monopolistprofitmax}}{It is optimal for the monopolist to increase output as long as $MR$ exceeds $MC$. In this case $MR>MC$ for units 1, 2 and 3. But for the fourth unit $MC>MR$ and therefore the monopolist would reduce total profit by producing it. He should produce only 3 units of output.}
\draw [dashed,mccolour,ultra thick]
	(0,2) -- (1,2) -- (1,3) -- (2,3) -- (2,4) -- (3,4) -- (3,5) -- (4,5) -- (4,6) -- (5,6) node [black,mynode,right] {$MC$ function};
\draw [dashed,mrcolour,ultra thick]
	(0,14) -- (1,14) -- (1,10) -- (2,10) -- (2,6) -- (3,6) -- (3,2) -- (4,2) -- (4,-2) -- (5,-2) node [black,mynode,right] {$MR$ function};
\draw [thick, -]
	(0,16) node [above] {\$} |- (8,0) node [right] {Quantity}
	(0,0) node [mynode,left] {0} -- (0,-3);
\node [mynode,left] at (0,-2) {-2};
\node [mynode,left] at (0,2) {2};
\node [mynode,left] at (0,4) {4};
\node [mynode,left] at (0,6) {6};
\node [mynode,left] at (0,8) {8};
\node [mynode,left] at (0,10) {10};
\node [mynode,left] at (0,12) {12};
\node [mynode,left] at (0,14) {14};
\node [mynode,below] at (1,0) {1};
\node [mynode,below] at (3,0) {3};
\node [mynode,below] at (5,0) {5};
\node [mynode,below] at (7,0) {7};
\end{FigureBox}
 
\subsection*{The optimal output}

This producer has a marginal cost structure given in the fifth column of the table, and this too is plotted in Figure~\ref{fig:monopolistprofitmax}. Our profit maximizing rule from Chapter~\ref{chap:prodcost} states that it is optimal to produce a greater output as long as the additional revenue exceeds the additional cost of production on the next unit of output. In perfectly competitive markets the additional revenue is given by the fixed price for the individual producer, whereas for the monopolist the additional revenue is the marginal revenue. Consequently as long as $MR$ exceeds $MC$ for the next unit a greater output is profitable, but once $MC$ exceeds $MR$ the production of additional units should cease.

From Table~\ref{table:profitmaxmonopolist} and Figure~\ref{fig:monopolistprofitmax} it is clear that the optimal output is at 3 units. The third unit itself yields a profit of 2\$, the difference between $MR$ (\$6) and $MC$ (\$4).  A fourth unit however would reduce profit by \$3, because the $MR$ (\$2) is less than the $MC$ (\$5). What price should the producer charge? The price, as always, is given by the demand function. At a quantity sold of 3 units, the corresponding price is \$10, yielding total revenue of \$30. 

Profit is the difference between total revenue and total cost. In Chapter~\ref{chap:prodcost} we computed total cost as the average cost times the number of units produced. It can also be computed as the sum of costs associated with each unit produced: the first unit costs \$2, the second \$3 and the third \$4. The total cost of producing 3 units is the sum of these dollar values: \$9 = \$2 + \$3 + \$4. The profit-maximizing output therefore yields a profit of \$21 ($\$30-\$9$). 

When illustrating market behaviour it is convenient to describe behaviour by simple linear equations that give rise to straight line curves. So rather than using the step functions as above, let us confront the situation where the demand and cost functions can be represented by straight lines. Instead of dealing with whole units we can think of the good in question as being divisible: the product can be sold in fractions of units. 

Alternatively we can think of quantities as being measured in thousands or millions. In this case the step curves in our figures effectively become straight lines, as we illustrated in Chapter~\ref{chap:welfare}. So consider the following demand and marginal cost functions; they reflect the values in the above table.

\begin{align*}
\text{Demand: }P&=16-2Q\,\text{;}	\\
\text{Marginal cost: }MC&=1+1Q\,\text{.}
\end{align*}

With these two linear equations, which are displayed in Figure~\ref{fig:demandmargtotalrev}, we can illustrate the monopolist's optimal choice. It is straightforward to substitute values for $Q$ into each of these equations to arrive at the price and $MC$ values in Table~\ref{table:profitmaxmonopolist}. Of course a $MR$ curve is required to compute the profit maximizing output, and we draw upon our knowledge of elasticities developed in Chapter~\ref{chap:elasticities} for this purpose.

% Figure 10.5 (called 10.3 in original text)
\begin{FigureBox}{1}{0.25}{25em}{Demand, marginal revenue and total revenue \label{fig:demandmargtotalrev}}{For a linear demand curve the price elasticity has a value of unity ($-1$) at the mid point. Since $TR$ is maximized here it follows that no additional revenue can be generated from further sales; that is, the $MR$ is zero. Profit is maximized where $MR=MC$, where $Q=3$. The price at this output is read from the demand curve: $P=16-2\cdot 3=10$.}
% MC function
\draw [dashed,mccolour,ultra thick,name path=MC] (0,1) -- (7,8) node [black,mynode,right] {$MC=1+Q$};
% MR function
\draw [dashed,mrcolour,ultra thick,name path=MR] (0,16) -- coordinate [pos=0.25] (MRnamepoint) (4,0) coordinate (MidD);
% Demand function
\draw [demandcolour,ultra thick,name path=D] (0,16) node [black,mynode,left] {16} -- coordinate [pos=0.1] (Dnamepoint) (8,0) node [black,mynode,below] {8};
% axes
\draw [thick, -] (0,20) node (yaxis) [above] {Price} |- (8,0) node (xaxis) [right] {Quantity};
% intersection of MC and MR
\draw [name intersections={of=MC and MR, by=E}]
	[dotted,thick] (E) -- (xaxis -| E) node [mynode,below] {3};
% paths to intersect with demand line
\path [name path=Q4path] (MidD) -- +(0,20);
\path [name path=Q3path] (xaxis -| E) -- +(0,20);
% intersection of D with Q3path and Q4path
\draw [name intersections={of=D and Q4path, by=Q4},name intersections={of=D and Q3path, by=Q3}]
	[dotted,thick] (yaxis |- Q4) node [mynode,left] {8} -| (xaxis -| Q4) node [mynode,below] {4}
	[dotted,thick] (yaxis |- Q3) node [mynode,left] {10} -| (E);
% arrows to MRnamepoint and Dnamepoint
\draw [<-,thick,shorten <=1mm] (MRnamepoint) -- +(1,3) node [mynode,right] {$MR=16-4Q$};
\draw [<-,thick,shorten <=1mm] (Dnamepoint) -- +(1,3) node [mynode,right] {Demand: $P=16-2Q$};
% arrow to midpoint of demand
\draw [<-,thick,shorten <=1mm] (Q4) -- +(1,3) node [mynode,right] {Mid point of Demand};
\end{FigureBox}

From Chapter~\ref{chap:elasticities} we know that, when the demand curve is a straight line, the midpoint has an elasticity value of unity and also is the point where total revenue is greatest (see Figure~\ref{fig:elastquantfluctuations}). It follows that, since total revenue is greatest at the midpoint, this point must also define the output where the addition to revenue from further sales goes from positive to negative. That is, the $MR$ curve is zero at the midpoint of the linear demand curve. The $MR$ curve in Figure~\ref{fig:demandmargtotalrev} reflects this: the midpoint on the quantity axis is 4 units and therefore $MR$ must be zero at that point.

Furthermore, since the $MR$ intersects the quantity axis at a point half way to the horizontal intercept of the demand curve, it must have a slope that is twice the slope of the demand curve. Hence the $MR$ curve can be written as the demand curve with twice the slope: 

\begin{equation*}
MR=16-4Q.
\end{equation*}

The profit maximum is obtained from the intersection of the $MR$ and the $MC$. So let us equate these to obtain the optimal output:

\begin{equation*}
MR=MC\text{ implies: }16-4Q=1+1Q,
\end{equation*}

that is,  

\begin{equation*}
16-1=1Q+4Q\Rightarrow 15=5Q\Rightarrow Q = 3.
\end{equation*}

We have found the profit maximizing output. What price will the monopolist choose? This is obtained from the demand curve. The optimal, or profit maximizing, price is:     

\begin{equation*}
P=16-2\times 3\Rightarrow P=16-6\Rightarrow P=\$10.
\end{equation*}

Total revenue is therefore \$30 (3 $\times$ \$10). Total cost can be obtained from the average cost curve, which in this case is $ATC=1+Q/2$. At $Q=3$, average cost is \$2.5 (1 + 3/2). Thus total cost is \$7.5 and profit is \$22.5. Note that this value differs slightly from the value in Table~\ref{table:profitmaxmonopolist} and derived from Figure~\ref{fig:monopolistprofitmax}, as we would expect, because we use straight line functions rather than step functions and they are not identical\footnote{Total revenue and total cost can also be obtained by taking the areas under the $MR$ and $MC$ curves in Figure~\ref{fig:demandmargtotalrev} at an output of three units.}.

\subsection*{Demand elasticity and marginal revenue}

We have shown above that the $MR$ curve cuts the horizontal axis at a quantity where the elasticity of demand is unity. We know from Chapter~\ref{chap:elasticities} that demand is elastic at points on the demand curve above this unit-elastic point. Furthermore, since the intersection of $MR$ and $MC$ must be at a positive dollar value ($MC$ cannot be negative), then it must be the case that the \textit{profit maximizing price for a monopolist always lies on the elastic segment of the demand curve}.

\subsection*{A general graphical representation}

In Figure~\ref{fig:monopolyeq} we generalize the graphical representation of the monopoly profit maximizing output by allowing the $MC$ curve to be nonlinear, and by introducing the $ATC$ curve. The optimal output is at $Q_E$, where $MR=MC$, and the price $P_E$ sustains that output. With the average cost known, profit per unit is AB, and therefore total profit is this margin multiplied by the number of units sold, $Q_E$.

Total profit is therefore $P_E$AB$C_E$. 

Note that the monopolist may not always make a profit. Losses could result in Figure~\ref{fig:monopolyeq} if average costs were to rise with no change in demand conditions\footnote{This could occur with an increase in fixed costs and no change in variable costs.}. For example, if the $ATC$ curve shifted upwards so that it lay everywhere above the demand curve, and the $MC$ remained unchanged, then the output where $MC=MR$ is again $Q_E$. In this case losses would result. In the longer term the monopolist would have to either reduce costs or perhaps stimulate demand through advertising if she wanted to continue in operation.

% Figure 10.6 (called 10.4 in original text)
\begin{FigureBox}{1}{0.25}{25em}{The monopoly equilibrium \label{fig:monopolyeq}}{The profit maximizing output is $Q_E$, where $MC=MR$. This output can be sold at a price $P_E$. The cost per unit of $Q_E$ is read from the $ATC$ curve, and equals B. Per unit profit is therefore AB and total profit is $P_E$AB$C_E$.}
% MC curve
\draw [dashed,mccolour,ultra thick,name path=MC] (0.5,6) to [out=60,in=270] (3.5,18) node [black,mynode,above] {$MC$};
% ATC curve
\draw [atccolour,ultra thick,domain=1:4,name path=ATC] plot (\x, {16-4*sqrt(2.25-(\x-2)*(\x-3))}) node [mynode,black,above] {$ATC$};
% MR line
\draw [dashed,demandcolour,ultra thick,name path=MR] (0,16) -- (4,0) node [pos=0.9,black,mynode,above right] {$MR$};
% demand line
\draw [demandcolour,ultra thick,name path=D] (0,16) -- coordinate[midway] (MidD) (8,0) node [pos=0.9,black,mynode,above right] {$D$};
% axes
\draw [thick, -] (0,20) node (yaxis) [above] {\$} |- (8,0) node (xaxis) [right] {Quantity};
% intersection of MC and MR
\draw [name intersections={of=MC and MR, by=E}]
	[dotted,thick] (E) -- (xaxis -| E) node [mynode,below] {$Q_E$};
% path to intersect with ATC and D along same x value as point E
\path [name path=Eline] (xaxis -| E) -- +(0,20);
% intersection of Eline with ATC and D
\draw [name intersections={of=Eline and ATC, by=B},name intersections={of=Eline and D, by=A}]
	[dotted,thick] (yaxis |- A) node [mynode,left] {$P_E$} -- (A) node [mynode,above right=-0.1cm and 0cm] {A} -- (E)
	[dotted,thick] (yaxis |- B) node [mynode,left] {$C_E$} -- (B) node [mynode,above right=-0.1cm and 0cm] {B};
% arrow to midpoint of demand
\draw [<-,thick,shorten <=1mm] (MidD) -- +(1,3) node [mynode,right] {Mid point of Demand\\Demand elasticity$=-1$};
\end{FigureBox} 