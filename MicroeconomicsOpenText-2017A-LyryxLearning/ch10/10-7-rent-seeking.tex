\section{Rent-seeking}\label{sec:ch10sec7}

Citizens are usually appalled when they read of the lobbying and `bribing' of politicians. These practices are not limited to less developed economies; every capital city in the world has an army of lobbyists, seeking to influence legislators and regulators. Such individuals are in the business of \terminology{rent seeking}, which uses the economy's productive resources to redistribute profit or surplus to particular groups, rather than produce additional output and create new profit or surplus. For example, Canadian media owners  seek to reduce competition from their US-based competition by requesting the CRTC (the Canadian Radio-Television and Telecommunications Commission) to restrict the inflow of signals from US suppliers. In Virginia and Kentucky we find that state taxes on cigarettes are the lowest in the US -- because the tobacco leaf is grown in these states, and the tobacco industry makes major contributions to the campaigns of some political representatives. Pharmaceutical companies in Washington propose that drugs imported from their northern neighbour may be unsafe for the American consumer.  Influence peddling is a business the world over.

Economists argue that the possible corruption of the political process is not the only damage done by this type of behaviour. They emphasize that it also has a resource cost:  imagine that we could outlaw the lobbying business and put these lobbyists to work producing goods and services in the economy, instead of spending their time in the greasing of palms and making not-so-erudite presentations to government bodies.  Their purpose is to maintain as much market or quasi-monopoly power in the hands of their clients as possible, and to ensure that the fruits of this effort go to those same clients. If this practice could be curtailed then the time and resources involved could be redirected to other productive ends.

\begin{DefBox}
\textbf{Rent seeking} is an activity that uses productive resources to redistribute rather than create output and value.
\end{DefBox}

\subsection*{Who pays for rent seeking?}

Industries in which rent seeking is most prevalent tend to be those in which the potential for economic profits is greatest -- monopolies or near-monopolies. These, therefore, are the industries that allocate resources to the preservation of their protected status. We do not observe laundromat owners or shoe-repair businesses lobbying in Ottawa. 

Rent seeking is an additional cost that must be borne by the producer and increases her total costs. Such lobbying is therefore costly both to the monopolist and society at large -- not only does the existence of a monopoly cause resource inefficiencies, but in addition resources are used to preserve this inefficient market structure.