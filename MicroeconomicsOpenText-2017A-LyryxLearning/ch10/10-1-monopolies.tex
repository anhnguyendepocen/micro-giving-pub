\section{Monopolies}\label{sec:ch10sec1}

In analyzing perfect competition we emphasized the difference between the industry and the individual supplier. The individual supplier is an atomistic unit with no market power. In contrast, a monopolist has a great deal of market power, for the simple reason that a \terminology{monopolist} is the sole supplier of a particular product and so really \textit{is} the industry. The word monopoly, comes from the Greek words \textit{monos}, meanings one, and \textit{polein} meaning to sell. When there is just a single seller, our analysis need not distinguish between the industry and the individual firm. They are the same on the supply side.

Furthermore, the distinction between long run and short run is blurred, because a monopoly that continues to survive as a monopoly obviously sees no entry or exit. This is not to pretend that monopolized sectors of the economy do not `die' and reincarnate themselves in some other more competitive form; they do. But if a monopoly continues to exist, by definition there is no entry or exit in the industry. 

\begin{DefBox}
A \textbf{monopolist} is the sole supplier of an industry's output, and therefore the industry and the firm are one and the same.
\end{DefBox}

Economists' thinking on monopoly has evolved considerably in recent decades. In part this is due to the impact of the opening up of national borders, in part due to the impact of technological change. The world is now more open and more subject to change, and so too are the circumstances governing monopolies. There are three main reasons for the existence and continuance of monopolies: scale economies, national policy and successful prevention of entry.

\subsection*{Natural monopolies}

Traditionally monopolies were viewed as being `natural' in some sectors of the economy. This means that scale economies define some industries' production and cost structures up to very high output levels, and that the whole market might be supplied at least cost by a single firm.

Consider the situation depicted in Figure~\ref{fig:naturalmonopolist}. The long run $ATC$ curve declines indefinitely. There is no output level where average costs begin to increase. Imagine now having several firms, each producing with a plant size corresponding to the short run average cost curve $ATC_1$. In contrast, one larger firm, having a plant size corresponding to $ATC_2$ can produce several times the output at a lower unit cost -- and \textit{supply the complete market} in the process. Such a scenario is termed a \terminology{natural monopoly}.

% Figure 10.1 (called 10.1a in original text)
\begin{FigureBox}{0.3}{0.25}{25em}{A `natural' monopolist \label{fig:naturalmonopolist}}{When LR average costs continue to decline at very high output, one firm may be able to supply the industry at a lower unit cost than several firms. With a plant size corresponding to $ATC_2$, is a supplier can supply the whole market, several smaller firms, each with plan size corresponding to $ATC_1$, cannot compete with the larger firm on account of differential unit costs.}
% LATC curve
\draw [latccolour,ultra thick,name path=LATC] (2,14) to [out=-30,in=175] (24,8) node [black,mynode,right] {$LATC$};
% MC curves
\draw [dashed,mccolour,ultra thick,-]
	(3,7) to [out=15,in=270] (8.5,16.5) node [black,mynode,above] {$MC_1$}
	(17,6) to [out=15,in=270] (21,12) node [black,mynode,above] {$MC_2$};
% axes
\draw [thick, -] (0,20) node [above] {Cost (\$)} |- (25,0) node [right] {Quantity};
% paths to intersect LATC
\path [name path=oneline] (6,0) -- +(0,20);
\path [name path=twoline] (20,0) -- +(0,20);
% Intersection of LATC with oneline and twoline paths, and resulting ATC curves
\draw [name intersections={of=LATC and oneline, by=one},name intersections={of=LATC and twoline, by=two}]
	[atccolour,ultra thick] (3.5,14.5) to [out=-65,in=150] (one) to [out=-20,in=-100] (12,14) node [mynode,above,black] {$ATC_1$}
	[atccolour,ultra thick] (17,11) to [out=-65,in=170] (two) to [out=0,in=-100] (23,10) node [mynode,above,black] {$ATC_2$};
\end{FigureBox}

\begin{DefBox}
\textbf{Natural monopoly}: one where the $ATC$ of producing any output declines with the scale of operation.
\end{DefBox}

We used to think of \textit{Bell Canada} as a natural monopoly -- before the days of cell phones and `voice over internet protocol'. Many nations thought of their national passenger railway as a natural monopoly -- before the advent of Greyhound and the airplane. \textit{Canada Post} might have had the same beliefs about itself before the advent of \textit{FEDEX}, \textit{UPS} and other couriers. Thus there is no guarantee that even a self-perceived natural monopoly can continue to hold that status. Entry and new developments can compete it away.

In reality there are very few pure monopolies. \textit{Facebook} and \textit{Google} may be extraordinarily dominant in their markets -- as social media or search engines. But ultimately these firms are in the advertising business; their objective is to attract regular clients in order to push advertising. And the on-line or internet advertising business is so large that even these dominant companies get but a modest share of all such business.

\subsection*{National Policy}

A second reason for monopolies is national policy. Some governments are, or once were, proud to have a `national carrier' in the airline industry. The mail service is frequently a symbol of nationhood, and this has certainly been true of both Canada and the US. \textit{Canada Post} and the \textit{US Postal} system are national emblems that have historic significance. They were vehicles for integrating the provinces or states at various points in the federal lives of these countries.

The down side of such nationalist policies is that they can be costly to the taxpayer. Industries that are not subject to competition can become fat and uncompetitive: managers have insufficient incentives to curtail costs; unions realize the government is committed to sustain the monopoly and push for higher wages than under a more competitive structure. The result is that many government-sponsored monopolies must `go to the government trough' frequently to obtain financial life support. 

\subsection*{Maintaining barriers to entry}

A third reason that monopolistic companies continue to survive is that they are successful in preventing the entry of new firms and products. \textit{Patents and copyrights} are one vehicle for preserving the sole-supplier role, and are certainly necessary to encourage firms to undertake the research and development (R\&D) for new products.  

While copyright protection is legal, \textit{predatory pricing} is an illegal form of entry barrier, and we explore it more fully in Chapter~\ref{chap:government}. An example would be where an existing firm that sells nationally may deliberately undercut the price of a would-be local entrant to the industry.  Airlines with a national scope are frequently accused of posting low fares on flights in regional markets that a new carrier is trying to enter. 

Political lobbying is another means of maintaining monopolistic power. For example, the Canadian Wheat Board had fought successfully for decades to prevent independent farmers from marketing wheat. This Board lost it's monopoly status in August 2012, when the government of the day decided it was not beneficial to consumers or farmers in general. 

Critical networks also form a type of barrier, though not always a monopoly. The compression software WinZip benefits from the fact that it is used almost universally.  There are many free compression programs available on the internet. But WinZip can charge for its product because virtually everybody uses it. Free network products that no one uses are worth very little, even if they come at zero cost.

Many large corporations produce products that require a large up-front investment; this might be in the form of research and development, or the construction of costly production facilities. For example, Boeing or Airbus incurs billions of dollars in developing their planes; pharmaceuticals may have to invest a billion dollars to develop a new drug. However, once such an investment is complete, the cost of producing each unit of output may be constant. Such a phenomenon is displayed in Figure~\ref{fig:fixedconstmarginalcost}. In this case the average cost for a small number of units produced is high, but once the fixed cost is spread over an ever larger output, the average cost declines, and in the limit approaches the marginal cost. These production structures are common in today's global economy, and they give rise to markets characterized either by a single supplier or a small number of suppliers.

% Figure 10.2 (called 10.1b in original text)
\begin{FigureBox}{0.3}{0.25}{25em}{Fixed cost and constant marginal cost \label{fig:fixedconstmarginalcost}}{With a fixed cost of producing the first unit of output equal to F and a constant marginal cost thereafter, the long run average total cost, $LATC$, declines indefinitely and becomes asymptotic to the marginal cost curve.}
\draw [dotted,thick] (2,0) node [mynode,below] {1} -- (2,18);
\draw [latccolour,ultra thick,-]
	(0,18) node [black,mynode,left] {F} -- (2,18) to [out=-90,in=180] (25,4.5) node [black,mynode,above] {$LATC$};
\draw [dashed,mccolour,ultra thick,-]
	(2,4) -- (25,4) node [black,mynode,below] {Constant $MC$};
% axes
\draw [thick, -] (0,20) node [above] {Cost (\$)} |- (25,0) node [right] {Quantity};
\end{FigureBox}

This figure is also useful in understanding the role of patents. Suppose that \textit{Pharma A} spends one billion dollars in developing a new drug and has constant unit production costs thereafter, while \textit{Pharma B} avoids research and development and simply imitates \textit{Pharma A}'s product. Clearly \textit{Pharma B} would have a $LATC$ equal to its $LMC$, and would be able to undercut the initial developer of the drug. Such an outcome would discourage investment in new products and the economy at large would suffer as a consequence. 