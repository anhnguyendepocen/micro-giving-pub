\section{Trade in our daily lives}\label{sec:ch15sec1}

Virtually every economy in the modern world trades with other economies -- they are what we call `open' economies. Evidence of such openness is everywhere evident in our daily life. The world eats Canadian wheat; China exports manufactured goods to virtually anywhere we can think of; and Canadians take their holidays in Florida.

As consumers we value the choice and variety of products that trade offers. We benefit from lower prices than would prevail in a world of protectionism. At the same time there is a constant chorus of voices calling for protection from international competition: Manufacturers are threatened by production in Asia; dairy farmers cry out against the imports of poultry, beef, and dairy products; even the service sector is concerned about offshore competition from call centres and designers. In this world of competing views it is vital to understand how trade has the potential to improve the well-being of economies.

This chapter examines the theory of international trade, trade flows, and trade policy: who trades with whom, in what commodities, and why. In general, countries trade with one another because they can buy foreign products at a lower price than it costs to make them at home. International trade reflects specialization and exchange, which in turn improve living standards. It is cost differences between countries rather than technological differences that drive trade: In principle, Canada could supply Toronto with olives and oranges grown in Nunavut greenhouses. But it makes more sense to import them from Greece, Florida or Mexico

Trade between Canada and other countries differs from trade between provinces. By definition, international trade involves jumping a border, whereas most trade within Canada does not. Internal borders are present in some instances -- for example when it comes to recognizing professional qualifications acquired out of province. In the second instance, international trade may involve different currencies. When Canadians trade with Europeans the trade is accompanied by financial transactions involving Canadian dollars and Euros. A Canadian buyer of French wine pays in Canadian dollars, but the French vineyard worker is paid in euros. Exchange rates are one factor in determining national competitiveness in international markets. Evidently, not every international trade requires currency trades at the same time -- members of the European Union all use the Euro. Indeed a common currency was seen as a means of facilitating trade between member nations of the EU, and thus a means of integrating the constituent economies more effectively.