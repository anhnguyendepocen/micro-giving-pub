\section{Canada in the world economy}\label{sec:ch15sec2}

World trade has grown rapidly since the end of World War II, indicating that trade has become ever more important to national economies. Canada has been no exception. Canada signed the Free Trade Agreement with the US in 1989, and this agreement was expanded in 1994 when Mexico was included under the North America Free Trade Agreement (NAFTA). Imports and exports rose dramatically, from approximately one quarter to forty percent of GDP. Canada is now what is termed a very `open' economy -- one where trade forms a large fraction of total production. 

Smaller economies are typically more open than large economies---Belgium and the Netherlands depend upon trade more than the United States. This is because large economies have a sufficient variety of resources to supply much of an individual country's needs. The European Union is similar, in population terms, to the United States, but it is composed of many distinct economies. Some European economies are similar in size to individual American states. But trade between California and New York is not international, whereas trade between Italy and the Denmark is. 

Because our economy is increasingly open to international trade, events in other countries affect our daily lives much more than they did some decades ago. The conditions in international markets for basic commodities and energy affect all nations, both importers and exporters. For example, the prices of primary commodities on world markets increased dramatically in the latter part of the 2000s. Higher prices for grains, oil, and fertilizers on world markets brought enormous benefits to Canada, particular the Western provinces, which produce these commodities. If these primary commodity prices fall to where they were at the turn of the millennium, Canada will lose considerably.

The service sector accounts for more of our GDP than the manufacturing sector. As incomes grow, the demand for health, education, leisure, financial services, tourism etc. dominates the demand for physical products. Technically the income elasticity demand for the former group exceeds the income elasticity of demand for the latter. Internationally, while trade in services is growing rapidly, it still forms a relatively small part of total world trade. Trade in goods---merchandise trade---remains dominant, partly because many countries import goods, add some value, and re-export them. Even though the value added from such import--export activity may make just a small contribution to GDP, the gross flows of imports and exports can still be large relative to GDP. The transition from agriculture to manufacturing and then to services has been underway in developed economies for over a century. This transition has been facilitated in recent decades by the communications revolution and globalization. Globalization has seen a rapid shift in production from the developed to the developing world. 

Table~\ref{table:cdnmerchtradepatterns} shows the patterns of Canadian merchandise trade in 2008. The United States was and still is Canada's major trading partner, buying over 75 percent of exports and supplying more than 50 percent of Canadian imports. Table~\ref{table:cdngoodsexports} details exports by type. Although exports of resource-based products account for only about 40 percent of total exports, Canada is now viewed as a resource-based economy. This is in part because manufactured products account for almost 80 percent of U.S. and European exports but only about 60 percent of Canadian exports. Nevertheless, Canada has important export strength in machinery, equipment, and automotive products.

\begin{Table}{25em}{Canada's merchandise trade patterns 2008 \label{table:cdnmerchtradepatterns}}{1. OECD, excluding United States, Japan, United Kingdom, and other EU economies.\newline 2. Economies not included in the EU or the OECD.\newline \textit{Source:} Adapted from Statistics Canada CANSIM Database, \url{http://cansim2.statcan.gc.ca}, Tables 228-0001 and 228-0002.}
\begin{tabu} to \linewidth {|X[1.75,l]X[1,c]X[1,c]|}	\hline
\rowcolor{rowcolour}										&	Exports by destination	&	Imports by source	\\
					United States							&	76.9					&	54.2				\\
\rowcolor{rowcolour}Japan									&	2.2						&	3.8					\\
					United Kingdom							&	3.1						&	2.8					\\
\rowcolor{rowcolour}Other EU economies						&	5.2						&	9.3					\\
					Other OECD economies\textsuperscript{1}	&	4.3						&	9.1					\\
\rowcolor{rowcolour}Others\textsuperscript{2}				&	8.4						&	20.8				\\
					Total									&	100.0					&	100.0				\\	\hline
\end{tabu}
\end{Table}

\begin{Table}{25em}{Canadian goods exports 2012 \label{table:cdngoodsexports}}{\textit{Source:} Adapted from Statistics Canada CANSIM Database, \url{http://cansim2.statcan.gc.ca}, Table 376-0007.}
\begin{tabu} to 30em {|X[2.5,l]X[1,c]|}	\hline
\rowcolor{rowcolour}	Sector								&	Percentage of total	\\
						Agriculture and fishing				&	9.0					\\
\rowcolor{rowcolour}	Energy								&	24.5				\\
						Forestry							&	4.9					\\
\rowcolor{rowcolour}	Industrial goods and materials		&	25.5				\\
						Machinery and equipment				&	17.6				\\
\rowcolor{rowcolour}	Automotive products					&	12.9				\\
						Other consumer goods				&	3.6					\\
\rowcolor{rowcolour}	Total								&	100.0				\\	\hline
\end{tabu}
\end{Table}