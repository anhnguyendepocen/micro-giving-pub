\section{Institutions governing trade}\label{sec:ch15sec7}

In the nineteenth century, world trade grew rapidly, in part because the leading trading nation at the time---the United Kingdom---pursued a vigorous policy of free trade. In contrast, US tariffs averaged about 50 percent, although they had fallen to around 30 percent by the early 1920s. As the industrial economies went into the Great Depression of the late 1920s and 1930s, there was pressure to protect domestic jobs by keeping out imports. Tariffs in the United States returned to around 50 percent, and the United Kingdom abandoned the policy of free trade that had been pursued for nearly a century. The combination of world recession and increasing tariffs led to a disastrous slump in the volume of world trade, further exacerbated by World War II.

\subsection*{The WTO and GATT}

After World War II, there was a collective determination to see world trade restored. Bodies such as the International Monetary Fund and the World Bank were set up, and many countries signed the General Agreement on Tariffs and Trade (GATT), a commitment to reduce tariffs successively and dismantle trade restrictions.

Under successive rounds of GATT, tariffs fell steadily. By 1960, United States tariffs were only one-fifth their level at the outbreak of the War. In the United Kingdom, the system of wartime quotas on imports had been dismantled by the mid-1950s, after which tariffs were reduced by nearly half in the ensuing 25 years. Europe as a whole moved toward an enlarged European Union in which tariffs between member countries have been abolished. By the late 1980s, Canada's tariffs had been reduced to about one-quarter of their immediate post-World War II level.

The GATT Secretariat, now called the World Trade Organization (WTO), aims both to dismantle existing protection that reduces efficiency and to extend trade liberalization to more and more countries. Tariff levels throughout the world are now as low as they have ever been, and trade liberalization has been an engine of growth for many economies. The consequence has been a substantial growth in world trade.

\subsection*{NAFTA and the EU}

In North America, recent trade policy has led to a free trade area that covers the flow of trade between Canada, the United States, and Mexico. The Canada/United States free trade agreement (FTA) of 1989 expanded in 1994 to include Mexico in the North American Free Trade Agreement (NAFTA). The objective in both cases was to institute free trade between these countries in most goods and services. This meant the elimination of tariffs over a period of years and the reduction or removal of non-tariff barriers to trade, with a few exceptions in specific products and cultural industries. Evidence of the success of these agreements is reflected in the fact that Canadian exports have grown to more than 30 percent of GDP, and trade with the United States accounts for the lion's share of Canadian trade flows.

The European Union was formed after World War II, with the prime objective of bringing about a greater degree of \textit{political} integration in Europe. Two world wars had laid waste to their economies and social fabric. Closer economic ties and greater trade were seen as the means of achieving this integration. The Union was called the ``Common Market'' for much of its existence. The Union originally had six member states, but as of 2009 the number is 27, with several other candidate countries in the process of application, most notably Turkey. The European Union (EU) has a secretariat and parliament in Bruxelles. You can find more about the EU at \url{http://europa.eu}.