\section{Comparative advantage: the gains from trade}\label{sec:ch15sec3}

In the opening chapter of this text we emphasized the importance of opportunity cost and differing efficiencies in the production process as a means of generating benefits to individuals through trade in the marketplace. The simple example we developed illustrated that, where individuals differ in their efficiency levels, benefits can accrue to each individual as a result of specializing and trading. In that example it was assumed that individual A had an absolute advantage in producing one product and that individual Z had an absolute advantage in producing the second good. This set-up could equally well be applied to two economies that have different efficiencies and are considering trade, with the objective of increasing their consumption possibilities. Technically, we could replace Amanda and Zoe with Argentina and Zambia, and nothing in our analysis would have to change in order to illustrate that consumption gains could be attained by both Argentina and Zambia as a result of specialization and trade. 

\textit{Remember}: The opportunity cost of a good is the quantity of another good or service given up in order to have one more unit of the good in question. 

So, let us now consider two \underbar{economies} with differing production capabilities, as illustrated in Figures~\ref{fig:compadvprod} and~\ref{fig:compadvcons}. In this instance we will assume that one economy has an absolute advantage in both goods, but the degree of that advantage is greater in one good than the other. In international trade language, there exists a comparative advantage as well as an absolute advantage. It is frequently a surprise to students that this situation has the capacity to yield consumption advantages to both economies if they engage in trade, even though one is absolutely more efficient in producing both of the goods. This is what is called the \terminology{principle of comparative advantage}, and it states that even if one country has an absolute advantage in producing both goods, gains to specialization and trade still materialize, provided the opportunity cost of producing the goods differs between economies. This is a remarkable result, and much less intuitive than the principle of absolute advantage. We explore it with the help of the example developed in Figures~\ref{fig:compadvprod} and \ref{fig:compadvcons}.

\begin{DefBox}
\textbf{Principle of comparative advantage} states that even if one country has an absolute advantage in producing both goods, gains to specialization and trade still materialize, provided the opportunity cost of producing the goods differs between economies.
\end{DefBox}

The two economies considered are the US and Canada. Their production possibilities are defined by the $PPF$s in Figure~\ref{fig:compadvprod}. Canada can produce 5 units of $V$ or 35 units of $F$, or any combination defined by the line joining these points. The US can produce 8$V$ or 40$F$, or any combination defined by its $PPF$\footnote{Note that we are considering the $PPF$s to be straight lines rather than concave shapes. The result we illustrate here carries over to that case also, but it is simpler to illustrate with the linear $PPF$s.}. Let Canada initially consume 3$V$ and 14$F$, and the US consume 5$V$ and 15$F$. These combinations lie on their respective $PPF$s. The opportunity cost of a unit of $V$ in Canada is 7$F$ (the slope of Canada's $PPF$ is $5/35=1/7$). In the US the opportunity cost of one unit of $V$ is 5$F$ (slope is $8/40=1/5$). In this set-up the US is more efficient in producing $V$ than $F$ relative to Canada, as reflected by the opportunity costs. Hence we say that the \textit{US has a comparative advantage in the production of $V$} \underbar{and} that \textit{Canada has therefore a comparative advantage in producing $F$}. 

% Figure 15.1 (called 15.1a in original text)
\begin{FigureBox}{0.20}{0.5}{25em}{Comparative advantage -- production \label{fig:compadvprod}}{Canada specializes completely in Fish at (35,0), where it has a comparative advantage. Similarly, the US specializes in Vegetable at (0,8). They trade at a rate of 1:6. The US trades 3$V$ to Canada in return for 18$F$.}
\draw [ppfcolourthree,ultra thick,-]
	(0,5) node [black,mynode,left] {(0,5)} -- (35,0) node [black,mynode,below left,pos=0.8] {Canada $PPF$} node [black,mynode,below] {(35,0)\\Canada\\specializes}
	(0,8) node [black,mynode,left] {(0,8) US\\specializes} -- (40,0) node [black,mynode,above right,pos=0.8] {US $PPF$} node [black,mynode,below] {(40,0)};
\draw [fill] (15,5) circle [radius=0.1] node [mynode,above right] {(15,5) initial\\consumption};
\draw [fill] (14,3) circle [radius=0.1] node [mynode,below left] {(14,3) initial\\consumption};
\draw [thick, -] (0,10) node [mynode1,above] {Vegetable} -- (0,0) node [mynode,below left] {0} -- (50,0) node [mynode1,right] {Fish};
\end{FigureBox}

Prior to trade each economy is producing all of the goods it consumes. This no-trade state is termed \terminology{autarky}.

\begin{DefBox}
\textbf{Autarky} denotes the no-trade situation.
\end{DefBox}

\subsection*{The gains from trade}

We now permit each economy to specialize in producing where it has a comparative advantage. So Canada specializes completely by producing 35$F$ and the US produces 8$V$. Having done this the economies must now agree on the terms of trade. The \terminology{terms of trade} define the rate at which the two goods will trade post-specialization. Let us suppose that a bargaining process leads to agreement that one unit of $V$ will trade for six units of $F$. Such a trading rate, one that lies between the opportunity costs of each economy, benefits both economies. By specializing in $F$, Canada can now obtain an additional unit of $V$ by sacrificing six units of $F$, whereas pre trade it had to sacrifice seven units of $F$ for a unit of $V$. Technically, by specializing in $F$ and trading at a rate of 1:6 Canada's consumption possibilities have expanded and are given by the consumption possibility frontier ($CPF$) illustrated in Figure~\ref{fig:compadvcons}. The \terminology{consumption possibility frontier} defines what an economy can consume after production specialization and trade.

% Figure 15.2 (called 15.1b in original text)
\begin{FigureBox}{0.20}{0.5}{25em}{Comparative advantage -- consumption \label{fig:compadvcons}}{Post specialization the economies trade 1$V$ for 6$F$. Total production is 35$F$ plus 8$V$. Hence once consumption possibility would be (18,5) for the US and (17,3) for Canada. Here Canada exchanges 18$F$ in return for 3$V$.}
\draw [dashed,ultra thick,name path=cdnconpos] (0,5.8333) -- (35,0);
\draw [dashed,ultra thick,name path=usconpos] (0,8) -- (48,0);
\draw [ppfcolourthree,ultra thick,-]
	(0,5) node [black,mynode,left] {(0,5)} -- (35,0) node [black,mynode,below] {(35,0)\\Canada\\specializes}
	(0,8) node [black,mynode,left] {(0,8) US\\specializes} -- (40,0) node [black,mynode,below] {(40,0)};
\draw [fill] (18,5) circle [radius=0.1] node [mynode,above] {(18,5)};
\draw [fill] (17,3) circle [radius=0.1] node [mynode,above] {(17,3)};
\draw [thick, -] (0,10) node [mynode1,above] {Vegetable} -- (0,0) node [mynode,below left] {0} -- (50,0) node [mynode1,right] {Fish};
% path used to create arrows to dotted lines
\path [name path=arrowline] (25,0) -- +(0,10);
% intersection of arrowline path and consumption possibilities (dotted) lines
\draw [name intersections={of=arrowline and cdnconpos, by=I1},name intersections={of=arrowline and usconpos, by=I2}]
	[<-,thick,shorten >=1mm,shorten <=1mm] (I1) -- +(5,2) node [mynode,right] {Canada consumption\\possibilities};
\draw [<-,thick,shorten >=1mm,shorten <=1mm] (I2) -- +(5,2) node [mynode,right] {US consumption\\possibilities};
\end{FigureBox}

The US also experiences an improved set of consumption possibilities. By specializing in $V$ and trading at a rate of 1:6 its $CPF$ lies outside its $PPF$ and this enables it to consume more than in the pre-specialization state, where its $CPF$ was defined by its $PPF$. 

\begin{DefBox}
\textbf{Terms of trade} define the rate at which the goods trade internationally.

\textbf{Consumption possibility frontier} defines what an economy can consume after production specialization and trade.
\end{DefBox}

Evidently, the US and Canada CPFs are parallel since they trade with each other at the same rate: if Canada exports six units of $F$ for every unit of $V$ that it imports from the US, then the US must import the same six units of $F$ for each unit of $V$ it exports to Canada. To illustrate the gains numerically, let Canada import 3$V$ from the US in return for exporting 18$F$. Note that this is a trading rate of 1:6. Hence, Canada consumes 3$V$ and 17$F$ (Canada produced 35$F$ and exported 18$F$, leaving it with 17$F$). It follows that the US consumes 5$V$, having exported 3$V$ of the 8$V$ it produced, and obtained in return 18$F$ in imports. The new consumption bundles are illustrated in the figure: $(17F,3V)$ for Canada and $(18F,5V)$ for the US. Comparing these combinations with the pre-trade scenario, we see that each economy consumes more of one good and not less of the other. Hence the well-being of each has increased as a result of their increased consumption.

Comparative advantage constitutes a remarkable result. It indicates that gains to trade are to be reaped by an efficient economy, by trading with an economy that may be less efficient in producing \underbar{each} good. 

\subsection*{Comparative advantage and factor endowments}

A traditional statement of why comparative advantage arises is that economies have different endowments of the factors of production -- land, capital and labour endowments differ. A land endowment that facilitates the harvesting of grain (Saskatchewan) or the growing of fruit (California) may be innate to an economy. We say that wheat production is \textit{land intensive}, that aluminum production is \textit{power intensive}, that research and development is \textit{skill intensive}, that auto manufacture is \textit{capital intensive}. Consequently, if a country is well endowed with some particular factors of production, it is to be expected that it will specialize in producing goods that use those inputs. A relatively abundant supply or endowment of one factor of production tends to make the cost of using that factor relatively cheap: it is relatively less expensive to produce clothing in Hong Kong and wheat in Canada than the other way around. This explains why Canada's Prairies produce wheat, why Quebec produces aluminum, why Asia produces apparel. But endowments can evolve. 

How can we explain why Switzerland specializes in watches, precision instruments, and medical equipment, while Vietnam specializes in rice and tourism? Evidently, Switzerland made a decision to educate its population and invest in the capital required to produce these goods. It was not naturally endowed with these skills, in the same way that Greece is endowed with sun or Saskatchewan is endowed with fertile flat land. 

While we have demonstrated the principle of comparative advantage using a two-good example (since we are constrained by the geometry of two dimensions), the conclusions carry over to the case of many goods. Finally it is to be noted that the benefits to specialization that we proposed in Chapter~\ref{chap:intro} carry over to our everyday lives in the presence of comparative advantage: If one person in the household is more efficient at doing all household chores than another, there are still gains to specialization provided the efficiency differences are not all identical. This is the principle of comparative advantage at work in a microcosm.

\begin{ApplicationBox}{The one hundred mile diet}\label{app:ch15app1}
In 2005 two young British Columbians embarked on what has famously become known as the “one hundred mile diet”---a challenge to eat and drink only products grown within this distance of their home. They succeeded in doing this for a whole year, wrote a book on their experience and went on to produce a TV series. They were convinced that such a project is good for humanity, partly because they wrapped up ideas on organic farming and environmentally friendly practices in the same message.

\bigskip
Reflect now on the implications of this superficially attractive program: If North Americans were to espouse this diet, it would effectively result in the closing down of the mid-west of the Continent. From Saskatchewan to Kansas, we are endowed with grain-producing land that is the envy of the planet. But since most of this terrain is not within 100 miles of any big cities, these deluded advocates are proposing that we close up the production of grains and cereals exactly in those locations where such production is extraordinarily efficient. Should we sacrifice grains and cereals completely in this hemisphere, or just cultivate them on a hillside close to home, even if the resulting cultivation were to be more labour and fuel intensive? Should we produce olives in greenhouses in Edmonton rather than importing them from the Mediterranean, or simply stop eating them? Should we sacrifice wine and beer in North Battleford because insufficient grapes and hops are grown locally?

\bigskip
Would production in temperate climates really save more energy than the current practice of shipping vegetables and fruits from a distance---particularly when there are returns to scale associated with their distribution? The one hundred mile diet is based on precepts that are contrary to the norms of the gains from trade. In its extreme the philosophy proposes that food exports be halted and that the world's great natural endowments of land, water, and sun be allowed to lie fallow. Where would that leave a hungry world? 
\end{ApplicationBox}

\subsection*{The role of exchange rates}

What we have shown in the foregoing examples is that there exists a \textit{potential for gain}, in the presence of comparative advantage, not that gains are actually realized. Such gains depend upon markets, not economic planners, and countries usually trade with each other using different currencies which themselves have a market. The rate at which one currency trades for another is the exchange rate. Citizens of Canada buy U.S. goods if these goods sell more cheaply in Canada than Canadian-produced goods, not because some economist has told them about the principle of comparative advantage! In recent years the Canadian dollar has traded more or less at parity with the US dollar: one Canadian dollar could buy one US dollar. But in the late nineteen nineties a Canadian dollar could only purchase seventy US cents -- it was more costly for Canadians to buy American products with their Canadian dollars. When it is more costly for Canadians to purchase foreign products we buy less of them. Hence the \textit{actual} trade flow between economies depends upon both the efficiency with which goods are produced in the different economies and also the exchange rate between the economies. 