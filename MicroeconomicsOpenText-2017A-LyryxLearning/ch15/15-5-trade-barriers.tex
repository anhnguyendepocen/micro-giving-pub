\section{Trade barriers: tariffs, subsidies and quotas}\label{sec:ch15sec5}

A \terminology{tariff} is a tax on an imported product that is designed to limit trade in addition to generating tax revenue. It is a barrier to trade. There also exist \terminology{quotas}, which are quantitative restrictions on imports; other \terminology{non-tariff barriers}, such as product content requirements; and \terminology{subsidies}. By raising the domestic price of imports, a tariff helps domestic producers but hurts domestic consumers. Quotas and other non-tariff barriers have similar impacts.

\begin{DefBox}
A \textbf{tariff} is a tax on an imported product that is designed to limit trade in addition to generating tax revenue. It is a barrier to trade.

A \textbf{quota} is a quantitative limit on an imported product.

A \textbf{trade subsidy} to a domestic manufacturer reduces the domestic cost and limits imports.

\textbf{Non-tariff barriers}, such as product content requirements, limit the gains from trade.
\end{DefBox}

\begin{ApplicationBox}{Tariffs -- the national policy of J.A. MacDonald}\label{app:ch15app2}
In Canada, tariffs were the main source of government revenues, both before and after Confederation in 1867 and up to World War I. They provided “incidental protection” for domestic manufacturing. After the 1878 federal election, tariffs were an important part of the National Policy introduced by the government of Sir John A. MacDonald. The broad objective was to create a Canadian nation based on east-west trade and growth.

\bigskip
This National Policy had several dimensions. Initially, to support domestic manufacturing, it increased tariff protection on foreign manufactured goods, but lowered tariffs on raw materials and intermediate goods used in local manufacturing activity. The profitability of domestic manufacturing improved. But on a broader scale, tariff protection, railway promotion Western settlement, harbour development, and transport subsidies to support the export of Canadian products were intended to support national economic development. Although “reciprocity agreements” with the United States removed duties on commodities for a time, tariff protection for manufactures was maintained until the GATT negotiations of the post-World War II era. 
\end{ApplicationBox}

\subsection*{Tariffs}

Figure~\ref{fig:tarifftrade} describes how tariffs operate. We can think of this as the Canadian wine market---a market that is heavily taxed in Canada. The world price of Cabernet Sauvignon is \$10 per bottle, and this is shown by the horizontal world supply curve at that price. It is horizontal because our domestic market accounts for only a small part of the world demand for wine. International producers can supply us with any amount we wish to buy at the world price. The Canadian demand for this wine is given by the demand curve $D$, and Canadian suppliers have a supply curve given by $S$ (Canadian Cabernet is assumed to be of the same quality as the imported variety in this example). At a price of \$10, Canadian consumers wish to buy $Q_D$ litres, and domestic producers wish to supply $Q_S$ litres. The gap between domestic supply $Q_S$ and domestic demand $Q_D$ is filled by imports. This is the \textit{free trade equilibrium}.

% Figure 15.4 (called 15.3 in original text)
\begin{FigureBox}{0.25}{0.25}{25em}{Tariffs and trade \label{fig:tarifftrade}}{At a world price of \$10 the domestic quantity demanded is $Q_0$. Of this amount $Q_s$ is supplied by domestic producers and the remainder by foreign producers. A tariff increases the world price to \$12. This reduces demand to $Q'_0$; the domestic component of supply increases to $Q'_s$. Of the total loss in consumer surplus (LFGJ), tariff revenue equals EFHI, increased surplus for domestic suppliers equals LECJ, and the deadweight loss is therefore the sum of the triangular areas A and B.}
% Filled in triangles
\draw [fill,supplycolour!25] (7,10) -- (9,12) -- (9,10);
\draw [fill,demandcolour!25] (15,10) -- (15,12) -- (17,10);
% Domestic supply
\draw [supplycolour,ultra thick,domain=0:17,name path=DomSup] plot (\x, {3+\x}) node [black,mynode,above] {Domestic\\supply};
% 
\draw [demandcolour,ultra thick,name path=DomDem] (7,20) node [mynode,above,black] {Domestic\\demand} -- (25,2);
% World supplies
\draw [supplycolour,ultra thick,name path=WorldSup] (0,10) node [black,mynode,left] {$P$=\$10} node [black,mynode,above right] {J} -- (24,10) node [black,mynode,below right,pos=0.9] {World supply\\curve};
\draw [supplycolour,ultra thick,name path=WorldSupTariff] (0,12) node [black,mynode,left] {$P$=\$12} node [black,mynode,above right] {L} -- (24,12) node [black,mynode,above right,pos=0.9] {World supply curve\\including tariff};
% axes
\draw [thick, -] (0,23) node (yaxis) [mynode1,above] {Price} |- (25,0) node (xaxis) [mynode1,right] {Quantity};
% intersection of lines
\draw [name intersections={of=WorldSup and DomSup, by=C},name intersections={of=WorldSupTariff and DomSup, by=E},name intersections={of=WorldSup and DomDem, by=G},name intersections={of=WorldSupTariff and DomDem, by=F}];
% paths to create points I and H on WorldSup line
\path [name path=Iline] (xaxis -| E) -- +(0,23);
\path [name path=Hline] (xaxis -| F) -- +(0,23);
% intersection of Iline and Hline with WorldSup and dotted lines
\draw [name intersections={of=Iline and WorldSup, by=I},name intersections={of=Hline and WorldSup, by=H}]
	[dotted,thick] (C) node [mynode,above] {C} -- (xaxis -| C) node [mynode,below] {$Q_s$}
	[dotted,thick] (E) node [mynode,above] {E} -- (I) node [mynode,below right] {I} -- (xaxis -| I) node [mynode,below] {$Q_s'$}
	[dotted,thick] (F) node [mynode,above] {F} -- (H) node [mynode,below left] {H} -- (xaxis -| H) node [mynode,below] {$Q_0'$}
	[dotted,thick] (G) node [mynode,above] {G} -- (xaxis -| G) node [mynode,below] {$Q_0$};
% legend square
\draw [fill=white,thick] (22,18) rectangle (26,23);
\draw [fill=supplycolour!25,thick] (23,21) rectangle (24,22) node [mynode,midway,right] {\textbf{ A}};
\draw [fill=demandcolour!25,thick] (23,19) rectangle (24,20) node [mynode,midway,right] {\textbf{ B}};
\end{FigureBox}

If the government now imposes a 20 percent tariff on imported wines, foreign wine sells for \$12 a bottle, inclusive of the tariff. The tariff raises the domestic `tariff-inclusive' price above the world price, and this shifts the supply of this wine upwards. By raising wine prices in the domestic market, the tariff protects domestic producers by raising the domestic price at which imports become competitive. Those domestic suppliers who were previously not quite competitive at a global price of \$10 are now competitive. The total quantity demanded falls from $Q_D$ to $Q_D^{'}$ at the new equilibrium F. Domestic producers supply the amount $Q_S^{'}$ and imports fall to the amount $(Q_D^{'}-Q_S^{'})$. Reduced imports are partly displaced by domestic producers who can supply at prices between \$10 and \$12. Hence, imports fall both because total consumption falls and because domestic suppliers can displace some imports under the protective tariff.

Since the tariff is a type of tax, its impact in the market depends upon the elasticities of supply and demand, as illustrated Chapters~\ref{chap:elasticities} and \ref{chap:welfare}. The more elastic is the demand curve, the more a given tariff reduces imports. In contrast, if it is inelastic the quantity of imports declines less.

\subsection*{Costs and benefits of a tariff}

The costs of a tariff come from the higher price to consumers, but this is partly offset by the tariff revenue that goes to the government. This tariff revenue is a benefit and can be redistributed to consumers or spent on goods from which consumers derive a benefit. But there are also efficiency costs associated with tariffs---deadweight losses, as we call them. These are the real costs of the tariff, and they arise because the marginal cost of production does not equal the marginal benefit to the consumer. Let us see how these concepts apply with the help of Figure~\ref{fig:tarifftrade}.

Consumer surplus is the area under the demand curve and above the equilibrium market price. It represents the total amount consumers would have been willing to pay for the product but did not have to pay at the equilibrium price. It is a measure of consumer welfare. The tariff raises the market price and reduces this consumer surplus by the amount LFGJ. This area measures by how much domestic consumers are worse off as a result of the price increase caused by the tariff. But this is not the net loss for the whole domestic economy, because the government obtains some tax revenue and domestic producers get more revenue and profit.

Government revenue accrues from the domestic sales of imports. On imports of $(Q_D^{'}-Q_S^{'})$, tax revenue is EFHI. Then, domestic producers obtain an additional profit of LECJ---the excess of additional revenue over their cost per additional bottle. If we are not concerned about who gains and who loses, it is clear that there is a net loss to the domestic economy equal to the areas A and B.

The area B is the standard measure of deadweight loss. At the quantity $Q_D^{'}$, the cost of an additional bottle is less than the value placed on it by consumers; and, by not having those additional bottles supplied, consumers forgo a potential gain. The area A tells us that when supply by domestic higher-cost producers is increased, and supply of lower-cost foreign producers is reduced, the corresponding resources are not being used efficiently. The sum of the areas A and B is therefore the total deadweight loss of the tariff.

\subsection*{Production subsidies}

Figure~\ref{fig:subsidytrade} illustrates the effect of a subsidy to a domestic supplier. As in Figure~\ref{fig:tarifftrade}, the amount $Q_D$ is demanded in the free trade equilibrium and, of this, $Q_S$ is supplied domestically. With a subsidy per unit of output sold, the government can reduce the supply cost of the domestic supplier, thereby shifting the supply curve downward from $S$ to $S^{'}$. In this illustration, the total quantity demanded remains at $Q_D$, but the domestic share increases to $Q_S^{'}$.

% Figure 15.5 (called 15.4 in original text)
\begin{FigureBox}{0.25}{0.25}{25em}{Subsidies and trade \label{fig:subsidytrade}}{With a world supply price of $P$, a domestic supply curve $S$, and a domestic demand $D$, the amount $Q_0$ is purchased. Of this, $Q_s$ is supplied domestically and $(Q_0-Q_s)$ by foreign suppliers. A per-unit subsidy to domestic suppliers shifts their supply curve to $S'$, and increases their market share to $Q'_s$.}
% supply lines
\draw [supplycolour,ultra thick,domain=0:17,name path=DomSup] plot (\x, {5+\x}) node [black,mynode,above right] {$S$ -- domestic supply};
\draw [supplycolour,ultra thick,domain=0:19,name path=DomSupSub] plot (\x, {1+\x}) node [black,mynode,right] {$S'$ -- domestic supply\\with subsidy};
\draw [supplycolour,ultra thick,name path=WorldSup] (0,10) node [black,mynode,left] {$P$} -- (24,10) node [black,mynode,right] {World supply\\curve};
% demand line
\draw [demandcolour,ultra thick,domain=5:25,name path=DomDem] (5,22) node [mynode,above,black] {$D$ -- domestic\\demand} -- (25,2);
% axes
\draw [thick, -] (0,25) node (yaxis) [mynode1,above] {Price} |- (25,0) node (xaxis) [mynode1,right] {Quantity};
% intersection of WorldSup with other lines
\draw [name intersections={of=WorldSup and DomSup, by=Qs},name intersections={of=WorldSup and DomSupSub, by=Qsprime},name intersections={of=WorldSup and DomDem, by=Q0}]
	[dotted,thick] (Qs) -- (xaxis -| Qs) node [mynode,below] {$Q_s$}
	[dotted,thick] (Qsprime) -- (xaxis -| Qsprime) node [mynode,below] {$Q_s'$}
	[dotted,thick] (Q0) -- (xaxis -| Q0) node [mynode,below] {$Q_0$};
\end{FigureBox}

The new equilibrium represents a misallocation of resources. When domestic output increases from $Q_S$ to $Q_S^{'}$, a low-cost international producer is being replaced by a higher cost domestic supplier; the domestic supply curve $S$ lies above the international supply curve $P$ in this range of output.

Note that this example deals with a subsidy to domestic suppliers who are selling in the domestic market. It is not a subsidy to domestic producers who are selling in the international market -- an export subsidy.

\subsection*{Quotas}

A quota is a limit placed upon the amount of a good that can be imported. Consider Figure~\ref{fig:quotatrade}, where again there is a domestic supply curve coupled with a world price of $P$. Rather than imposing a tariff, the government imposes a quota that restricts imports to a physical amount denoted by the distance \textit{quota} on the quantity axis. The supply curve facing domestic consumers then has several segments to it. First it has the segment RC, reflecting the fact that domestic suppliers are competitive with world suppliers up to the amount C. Beyond this output, world suppliers can supply at a price of $P$, whereas domestic suppliers cannot compete at this price. Therefore the supply curve becomes horizontal, but only \textit{up to the amount permitted under the quota}---the quantity CU corresponding to \textit{quota}. Beyond this amount, international supply is not permitted and therefore additional amounts are supplied by the (higher cost) domestic suppliers. Hence the supply curve to domestic buyers becomes the supply curve from the domestic suppliers once again.

% Figure 15.6 (called 15.5 in original text)
\begin{FigureBox}{0.25}{0.25}{25em}{Quotas and trade \label{fig:quotatrade}}{At the world price $P$, plus a \emph{quota}, the supply curve becomes RCUV. This has three segments: (i) domestic suppliers who can supply below $P$; (ii) \emph{quota}; and (iii) domestic suppliers who can only supply at a price above $P$. The quota equilibrium is at T, with price $P_{dom}$ and quantity $Q'_D$; the free-trade equilibrium is at G. Of the amount $Q'_D$, \emph{quota} is supplied by foreign suppliers and the remainder by domestic suppliers. The quota increases the price in the domestic market.}
% Supply lines
\draw [supplycolour,ultra thick,name path=SupQuota] (0,5) node [black,mynode,left] {R} -- (5,10) coordinate (C) node [black,mynode,above left] {C} -- (9,10) coordinate (U) node [black,mynode,above left] {U} -- (21,22) node [black,mynode,above] {Supply curve to domestic\\market under quota=RCUV} node [black,mynode,below right] {V};
\draw [supplycolour,ultra thick,name path=WorldSup] (0,10) node [black,mynode,left] {$P$} -- (24,10) node [black,mynode,right] {World supply\\curve};
% demand line
\draw [demandcolour,ultra thick,domain=5:25,name path=DomDem] (5,22) node [mynode,above,black] {$D$ -- domestic\\demand} -- (25,2);
% axes
\draw [thick] (0,25) node (yaxis) [mynode1,above] {Price} |- (25,0) node (xaxis) [mynode1,right] {Quantity};
% intersection of WorldSup with DomDem, as well as intersection of DomDem and SupQuota
\draw [name intersections={of=WorldSup and DomDem, by=G},name intersections={of=SupQuota and DomDem, by=T}]
	[dotted,thick] (C |- T) node [mynode,above] {W} -- (xaxis -| C)
	[dotted,thick] (U |- T) node [mynode,above] {Y} -- (xaxis -| U)
	[dotted,thick] (yaxis |- T) node [mynode,left] {$P_{dom}$} -- (T) node [mynode,above] {T} -- (xaxis -| T) node [mynode,below] {$Q_D'$}
	[dotted,thick] (G) node [mynode,above] {G} -- (xaxis -| G) node [mynode,below] {$Q_D$};
% arrow for quota
\draw [<->,thick,shorten <=0.5mm,shorten >=0.5mm] ([yshift=2cm]C |- xaxis) -- node [mynode,below,midway] {\textit{quota}} ([yshift=2cm]U |- xaxis);
\end{FigureBox}

The resulting supply curve yields an equilibrium quantity $Q_D^{'}$. There are several features to note about this equilibrium. First, the quota pushes the domestic price above the world price because low-cost international suppliers are partially supplanted by higher-cost domestic suppliers. Second, if the quota is chosen `appropriately', the same domestic market price could exist under the quota as under the tariff in Figure~\ref{fig:tarifftrade}. Third, in contrast to the tariff case, the government obtains no tax revenue from the quotas. Fourth, there are inefficiencies associated with the equilibrium at $Q_D^{'}$: consumers lose the amount $\text{PGTP}_{\text{dom}}$ in surplus. Against this, suppliers gain $\text{PUTP}_{\text{dom}}$ (domestic suppliers gain the amount $\text{PCWP}_{\text{dom}}+\text{UYT}$, which represents the excess of market price over their ``willingness to supply'' price. Importers are limited by the quota but can still buy CU at price $P$, sell at $P_{\text{dom}}$, and therefore gain a surplus equal to CUYW). Therefore the deadweight loss of the quota is the area UTG---the difference between the loss in consumer surplus and the gain in supplier surplus.

\begin{ApplicationBox}{Cheese quota in Canada}\label{app:ch15app3}
In 1978 the federal government set a cheese import quota for Canada at just over 20,000 tonnes. This quota was implemented initially to protect the interests of domestic suppliers. Despite a strong growth in population and income in the intervening decades, the import quota has remained unchanged. The result is a price for cheese that is considerably higher than it would otherwise be. The quotas are owned by individuals and companies who have the right to import cheese. The quotas are also traded among importers, at a price. Importers wishing to import cheese beyond their available quota pay a tariff of about 250 percent. So, while the consumer is the undoubted loser in this game, who gains?

\bigskip
First the suppliers gain, as illustrated in Figure~\ref{fig:subsidytrade}. Canadian consumers are required to pay high-cost domestic producers who displace lower-cost producers from overseas. Second, the holders of the quotas gain. With the increase in demand for cheese that comes with higher incomes, the domestic price increases over time and this in turn makes an individual quota more valuable.
\end{ApplicationBox}